% This document is a brief guide to obtaining and building the 
% Torus code.
%
% Initial version by Dave Acreman, May 2008

\documentclass{article}

\usepackage{a4}

\begin{document}

\title{Building the Torus Radiative Transfer Code}
\author{David Acreman}

\maketitle

\section{Introduction}

This document describes how to build the Torus radiative transfer
code. It is assumed that you are using csh or tcsh as your UNIX
shell. The symbol \verb  \$ indicates your UNIX prompt.  

\section{Getting started}

\subsection{Obtaining the code}

The torus code is managed using CVS (Concurrent Version System). To
download the code from the remote repository type: 
%
\begin{verbatim}
$ cvs -d pinky:/h/th/CVS -q co torus 
\end{verbatim}
%
You will need to have an account on pinky to be able to do this. 

\subsection{Building Torus}

Torus is comes with a Makefile which can be used to 
build the code for a number of different systems. System specific
options (e.g. compiler flags) are specified in the Makefile so you
should not need to set these up, unless you are using a new system. 

The environment variable \verb SYSTEM  is used to tell the Makefile
which configuration to use. For example to use the ``zen''
configuration type 
\begin{verbatim}
$ setenv SYSTEM zen 
\end{verbatim}
Alternatively you can set this environment variable in your .cshrc
file so it is always correctly set. 

You can now go into the torus directory and build the code
%
\begin{verbatim}
$ cd torus
$ make
\end{verbatim}

\section{Building Torus with options}

\subsection{Make options}

Several options can be passed to the make process on the command
line, see table~\ref{table:make_options}
for a summary.

\begin{table}
  \begin{center}
    \begin{tabular}[|l|l|l|]{|l|l|l|}
\hline
    Option    & Description & Default \\
\hline
      pgplot  & link with pgplot libraries  & yes \\
      cfitsio & link with cfitsio libraries & yes \\
      debug   & switch on debugging flags   & no  \\
      profile & enable profiling            & no  \\
      static  & perform static linking      & no  \\
      trace   & use Intel trace collector   & no  \\
\hline
  \end{tabular}
\caption{Options which can be supplied to the make command when
  building torus.}
\label{table:make_options}
\end{center}
\end{table}

For example to build torus without linking to the pgplot libraries and
with debug flags enabled use
\begin{verbatim}
$ make pgplot=no debug=yes 
\end{verbatim}

You may wish to remove all files from a previous build and start from
scratch (for example if you have previously compiled with debug flags
and need to recompile without debug flags). The Makefile has an option
to clean up files from the previous build 
\begin{verbatim}
$ make clean
\end{verbatim}


\end{document}

