%% LyX 1.3 created this file.  For more info, see http://www.lyx.org/.
%% Do not edit unless you really know what you are doing.
\documentclass[oneside,english,useAMS, usenatbib]{mn2e}
\usepackage[T1]{fontenc}
\usepackage[latin1]{inputenc}
\setcounter{tocdepth}{3}
\usepackage{graphicx}

\makeatletter

%%%%%%%%%%%%%%%%%%%%%%%%%%%%%% LyX specific LaTeX commands.
%% Bold symbol macro for standard LaTeX users
\newcommand{\boldsymbol}[1]{\mbox{\boldmath $#1$}}

%% Because html converters don't know tabularnewline
\providecommand{\tabularnewline}{\\}

%%%%%%%%%%%%%%%%%%%%%%%%%%%%%% User specified LaTeX commands.
\usepackage{times}

\newcommand{\aap}{A\&A}
\newcommand{\aaps}{A\&AS}
\newcommand{\apj}{ApJ}
\newcommand{\apjl}{\apj}
\newcommand{\pasp}{PASP}
\newcommand{\aj}{AJ}
\newcommand{\mnras}{MNRAS}
\newcommand{\apjs}{ApJS}
\newcommand{\aapr}{A\&AR}
\newcommand{\pasj}{PASJ}

\newcommand{\kmps}{\mathrm{km~s^{-1}}}
\newcommand\ion[2]{#1$\,${\sc {#2}}}   % ion, i.e., CII = \ion{C}{ii}
\newcommand{\Kelvin}{\mathrm{K}}

\usepackage{babel}
\makeatother
\begin{document}
%==========================================

\begin{table*}

\label{tab:literatures}

\caption{Summary of known properties of the targers from literatures: a.~\citet{luhman:1999} (original list for Oph), b.~\citet{ardila:2000} (original list for Upper Sco objects),  c.~\citet{wilking:1999}, d.~\citet{muzerolle:2003}, and e.~\citet{kraus:2005}. }

\begin{center}

\begin{tabular}{lllcccc}
\hline 
Object&
Sp.&
$I$&
mass&
RV&
$v\sin i$&
Known multiple?\tabularnewline
&
&
&
$\left[\mathrm{M_{\sun}}\right]$&
$\left[\mathrm{km\, s^{-1}}\right]$&
$\left[\mathrm{km\, s^{-1}}\right]$&
\tabularnewline
\hline 
GY 5&
$\mathrm{M7^{c}}$&
$\sim16$&
$0.07^{\mathrm{d}}$&
$-6.3\pm1.9^{\mathrm{d}}$&
$16.8\pm2.7^{\mathrm{d}}$&
no\tabularnewline
GY 141&
$\mathrm{M8.5^{a}}$&
$\sim16$&
$0.02^{\mathrm{d}}$&
\ldots{}&
\ldots{}&
no\tabularnewline
GY 310&
$\mathrm{M8.5^{c}}$&
$\sim16$&
$0.08^{\mathrm{a,d}}$&
\ldots{}&
\ldots{}&
no\tabularnewline
USco 40&
$\mathrm{M5^{b}}$&
$14.3$&
$0.1^{\mathrm{b}}$&
\ldots{}&
\ldots{}&
no\tabularnewline
USco 53&
$\mathrm{M5^{b}}$&
$14.5$&
$0.1^{\mathrm{b}}$&
\ldots{}&
\ldots{}&
no\tabularnewline
USco 55&
$\mathrm{M5.5^{b}}$&
$14.6$&
$0.10+0.07^{\mathrm{e}}$&
\ldots{}&
\ldots{}&
$\mathrm{yes^{e}}$\tabularnewline
USco 66&
$\mathrm{M6^{b}}$&
$14.9$&
$0.07+0.07^{\mathrm{e}}$&
$-4.4\pm0.6^{\mathrm{d}}$&
\ldots{}&
$\mathrm{yes^{e}}$\tabularnewline
USco 67&
$\mathrm{M5.5^{b}}$&
$14.9$&
$0.10^{\mathrm{e}}$&
\ldots{}&
\ldots{}&
no\tabularnewline
USco 75&
$\mathrm{M6^{b}}$&
$15.1$&
$0.07^{\mathrm{e}}$&
$-5.6\pm1.1^{\mathrm{d}}$&
\ldots{}&
no\tabularnewline
USco 100&
$\mathrm{M7^{b}}$&
$15.6$&
$0.05^{\mathrm{e}}$&
$-8.9\pm0.6^{\mathrm{d}}$&
\ldots{}&
no\tabularnewline
USco 101&
$\mathrm{M5^{b}}$&
$15.6$&
$0.05^{\mathrm{b}}$&
\ldots{}&
\ldots{}&
no\tabularnewline
USco 104&
$\mathrm{M5^{b}}$&
$15.7$&
$0.05^{\mathrm{b}}$&
\ldots{}&
\ldots{}&
no\tabularnewline
USco 109&
$\mathrm{M6^{b}}$&
$16.1$&
$0.07+0.04^{\mathrm{e}}$&
$-3.8\pm0.7^{\mathrm{d}}$&
\ldots{}&
$\mathrm{yes^{e}}$\tabularnewline
USco 112&
$\mathrm{M5.5^{b}}$&
$16.1$&
$0.1^{\mathrm{e}}$&
\ldots{}&
\ldots{}&
no\tabularnewline
USco 121&
$\mathrm{M6^{b}}$&
$16.5$&
$0.02^{\mathrm{b}}$&
$-38.9\pm1.0^{\mathrm{d}}$&
\ldots{}&
no\tabularnewline
USco 128&
$\mathrm{M7^{b}}$&
$17.1$&
$0.05^{\mathrm{e}}$&
$-3.0\pm1.6^{\mathrm{d}}$&
\ldots{}&
no\tabularnewline
USco 130&
$\mathrm{M7.5^{e}}$&
$17.5$&
$0.04^{\mathrm{e}}$&
\ldots{}&
\ldots{}&
no\tabularnewline
USco 132&
$\mathrm{M7^{b}}$&
$17.6$&
$0.05^{\mathrm{e}}$&
$-8.2\pm1.1^{\mathrm{d}}$&
\ldots{}&
no\tabularnewline
\hline
\end{tabular}

\end{center}

\end{table*}

%==========================================

%==========================================

\begin{table*}

\label{tab:summary_results}

\caption{Summary of the observation, the heliocentric radial velocities ($\mathrm{RV}$) from two-epoch and the average rotational velocities ($v\sin i$). The uncetainties of relative radial velocties ($sigma_{\mathrm{RRV}}$) with respect to the template star LHS~049 and the average radial velocties ($\overline{\mathrm{RV}}$) are also given. The last column indicates whether a target is a candidate for multiplicity i.e.~the measured radial velocity changes from two different epoch is larger than $1\sigma_{\mathrm{RRV}}$ of each others (c.f.~Fig.~\ref{fig:radvel01}). Rank indeicates the likeliness of the candidecy base of the RV separation and the peak strength of the cross-correlation curves.}

\begin{center}

\begin{tabular}{lcrrrrrccc}
\hline 
Object&
Date&
MDJ-2453100&
RV&
$\sigma_{\mathrm{RRV}}$&
$\overline{\mathrm{RV}}$&
$v\sin i$&
candidate?&
rank&
known \tabularnewline
&
&
&
$\left[\mathrm{km\, s^{-1}}\right]$&
$\left[\mathrm{km\, s^{-1}}\right]$&
$\left[\mathrm{km\, s^{-1}}\right]$&
$\left[\mathrm{km\, s^{-1}}\right]$&
&
&
\tabularnewline
\hline 
GY 5&
2004-Apr-25&
20.22051&
$-6.14\pm0.84$&
0.68&
&
&
&
&
\tabularnewline
&
2004-May-08&
33.27776&
$-5.96\pm0.60$&
0.34&
$-6.05\pm1.03$&
$16.5\pm0.6$&
no&
&
\tabularnewline
GY 141&
2004-May-11&
36.16016&
$-4.39\pm0.60$&
0.34&
&
&
&
&
\tabularnewline
&
2004-May-18&
43.15707&
$-2.95\pm0.51$&
0.11&
$-3.67\pm0.79$&
$4.4\pm1.4$&
yes&
2&
\tabularnewline
GY 310&
2004-Apr-25&
20.32700&
$-4.83\pm0.74$&
0.54&
&
&
&
&
\tabularnewline
&
2004-May-10&
35.31139&
$-8.43\pm0.51$&
0.11&
$-6.63\pm0.90$&
$11.1\pm6.0$&
yes&
2&
\tabularnewline
USco 40&
2004-Apr-06&
01.25900&
$-7.15\pm0.74$&
0.54&
&
&
&
&
\tabularnewline
&
2004-May-08&
33.25324&
$-6.80\pm0.51$&
0.11&
$-6.98\pm0.90$&
$34.2\pm0.5$&
no&
&
\tabularnewline
USco 53&
2004-Apr-05&
00.39651&
$-7.27\pm0.93$&
1.21&
&
&
&
&
\tabularnewline
&
2004-May-03&
28.23079&
$-5.43\pm0.74$&
0.55&
$-6.35\pm1.19$&
$40.0\pm0.6$&
no&
&
\tabularnewline
USco 55&
2004-Apr-06&
01.33509&
$-5.39\pm0.50$&
0.02&
&
&
&
&
\tabularnewline
&
2004-May-03&
28.30546&
$-6.38\pm0.53$&
0.27&
$-6.38\pm0.73$&
$22.9\pm0.8$&
yes&
1&
X\tabularnewline
USco 66&
2004-Apr-06&
01.29007&
$-5.32\pm0.57$&
0.29&
&
&
&
&
\tabularnewline
&
2004-May-03&
28.28683&
$-6.41\pm0.65$&
0.42&
$-5.87\pm0.86$&
$25.9\pm1.2$&
no&
2&
X\tabularnewline
USco 67&
2004-Apr-06&
01.21481&
$-6.01\pm0.74$&
0.55&
&
&
&
&
\tabularnewline
&
2004-May-03&
28.21759&
$-6.83\pm0.59$&
0.31&
$-6.42\pm0.90$&
$18.4\pm0.4$&
no&
&
\tabularnewline
USco 75&
2004-Apr-05&
00.37453&
$-6.75\pm0.67$&
0.44&
&
&
&
&
\tabularnewline
&
2004-May-08&
33.09532&
$-9.88\pm1.94$&
1.88&
$-8.32\pm2.05$&
$55.6\pm3.0$&
yes&
3&
\tabularnewline
USco 100&
2004-Apr-06&
01.30836&
$-6.76\pm2.74$&
2.69&
&
&
&
&
\tabularnewline
&
2004-May-03&
28.26425&
$-10.23\pm1.80$&
1.73&
$-8.47\pm3.28$&
$43.7\pm3.2$&
no&
&
\tabularnewline
USco 101&
2004-Apr-05&
00.30191&
$-4.22\pm0.87$&
0.71&
&
&
&
&
\tabularnewline
&
2004-May-03&
28.14804&
$-6.07\pm0.69$&
0.48&
$-5.15\pm1.11$&
$19.1\pm0.3$&
yes&
2&
\tabularnewline
USco 104&
2004-Apr-05&
00.27545&
$-5.83\pm0.50$&
0.02&
&
&
&
&
\tabularnewline
&
2004-May-03&
28.12391&
$-7.48\pm0.50$&
0.06&
$-6.66\pm0.06$&
$16.7\pm0.4$&
yes&
1&
\tabularnewline
USco 109&
2004-Apr-06&
01.23587&
$-4.15\pm0.52$&
0.16&
&
&
&
&
\tabularnewline
&
2004-May-08&
33.11929&
$-4.41\pm0.50$&
0.03&
$-5.12\pm0.72$&
$8.6\pm1.2$&
possibly&
3&
X\tabularnewline
USco 112&
2004-Apr-05&
00.34486&
$-2.70\pm0.69$&
0.47&
&
&
&
&
\tabularnewline
&
2004-May-08&
33.07139&
$-3.46\pm0.51$&
0.11&
$-3.08\pm0.86$&
$5.8\pm1.2$&
possibly&
3&
\tabularnewline
USco 121&
2004-Apr-25&
20.19466&
$-39.47\pm0.51$&
0.11&
&
&
&
&
\tabularnewline
&
2004-May-03&
28.18537&
$-42.43\pm0.50$&
0.02&
$-40.95\pm0.71$&
$17.6\pm1.3$&
yes&
1&
\tabularnewline
USco 128&
2004-May-14&
39.28595&
$-7.41\pm0.85$&
0.69&
&
&
&
&
\tabularnewline
&
2004-May-18&
43.09885&
$-6.94\pm1.16$&
1.05&
$-7.18\pm1.44$&
$3.6\pm1.1$&
no&
&
\tabularnewline
USco 130&
2004-May-10&
35.26060&
$-4.83\pm0.54$&
0.21&
&
&
&
&
\tabularnewline
&
2004-May-14&
39.34201&
$-4.95\pm0.74$&
0.55&
$-4.89\pm0.92$&
$15.2\pm1.1$&
no&
&
\tabularnewline
USco 132&
2004-May-12&
37.28394&
$-7.18\pm0.58$&
0.30&
&
&
&
&
\tabularnewline
&
2004-Jun-18&
43.12719&
$-7.37\pm1.02$&
0.89&
$-7.28\pm1.17$&
$9.1\pm0.7$&
no&
&
\tabularnewline
\hline
\end{tabular}

USCO-121==> Only 7 days apart!

1USco-128 and USco-130 ==>only 4 days apart!

\end{center}

\end{table*}

%==========================================

%==========================================

\begin{figure*}

\begin{center}

\includegraphics[%
  clip,
  scale=1.3]{figures/radvel_all01.eps}

\end{center}

\caption{Relative radial velocities (RVs) of objects measured in two different epochs. The vertical axes indicate are the amount of diviation from the the average radial velocity ($\overline{\mathrm{RV}}$) in Table~\ref{tab:summary_results}), and the horizontal axes indicate the time of the observation in modified Julian date (MJD). The objects are considered to have a non-constant  radial velocity when the error bars of two data points do not overlap each other.}

\label{fig:radvel01}

\end{figure*}

%==========================================

\addtocounter{figure}{-1}

%==========================================

\begin{figure*}

\begin{center}

\includegraphics[%
  clip,
  scale=1.3]{figures/radvel_all02.eps}

\end{center}

\caption{continued}

\label{fig:radvel02}

\end{figure*}

%==========================================

\bibliographystyle{mn}

\bibliography{local}

%

\label{lastpage}
\end{document}
