%% LyX 1.3 created this file.  For more info, see http://www.lyx.org/.
%% Do not edit unless you really know what you are doing.
\documentclass[oneside,english,useAMS, usenatbib]{mn2e}
\usepackage[T1]{fontenc}
\usepackage[latin1]{inputenc}
\setcounter{tocdepth}{3}
\usepackage{graphicx}

\makeatletter

%%%%%%%%%%%%%%%%%%%%%%%%%%%%%% LyX specific LaTeX commands.
%% Bold symbol macro for standard LaTeX users
\newcommand{\boldsymbol}[1]{\mbox{\boldmath $#1$}}

%% Because html converters don't know tabularnewline
\providecommand{\tabularnewline}{\\}

%%%%%%%%%%%%%%%%%%%%%%%%%%%%%% User specified LaTeX commands.
\usepackage{times}
\usepackage{amssymb}

\newcommand{\aap}{A\&A}
\newcommand{\aaps}{A\&AS}
\newcommand{\apj}{ApJ}
\newcommand{\apjl}{\apj}
\newcommand{\pasp}{PASP}
\newcommand{\aj}{AJ}
\newcommand{\mnras}{MNRAS}
\newcommand{\apjs}{ApJS}
\newcommand{\aapr}{A\&AR}
\newcommand{\pasj}{PASJ}

\newcommand{\kmps}{\mathrm{km~s^{-1}}}
\newcommand\ion[2]{#1$\,${\sc {#2}}}   % ion, i.e., CII = \ion{C}{ii}
\newcommand{\Kelvin}{\mathrm{K}}

\usepackage{babel}
\makeatother
\begin{document}
\title[Radial and rotational velocities of young BDs and VLM stars]{Radial and rotational velocities of young brown dwarfs and very low-mass stars in the Upper Scorpius OB association and the $\rho$~Ophiuchi cloud core}

\author[R. Kurosawa et\,al.]{Ryuichi Kurosawa\thanks{E-mail: rk@astro.ex.ac.uk} and  Tim J.~Harries\\ School of Physics, University of Exeter, Stocker Road, Exeter EX4 4QL}

\date{Dates to be inserted}

\pagerange{\pageref{firstpage}--\pageref{lastpage}} \pubyear{2006}

\maketitle

\label{firstpage}

\begin{abstract}

We present the results of a radial velocity (RV) survey of 14 brown
dwarfs (BDs) and very low-mass (VLM) stars in the Upper Scorpius OB
association (UScoOB) and 3 BD candidates in the $\rho$~Ophiuchi dark
cloud core. We obtained high-resolution echelle spectra at VLT using
Ultraviolet and Visual Echelle Spectrograph (UVES) at two different
epochs for each object, and measured the shifts in their RVs to
identify candidates for binary/multiple systems in the sample. The
average time separation of the RV measurements is 21.6~d, and our
survey is sensitive to the binaries with separation $<0.12$~au.  We
found 4 out of 17 objects (or $24^{+16}_{-14}$~per~cent by fraction)
show a significant RV change in 4--33~d time scale, and are 
considered as binary/multiple `candidates.'  We found no double-lined
spectroscopic binaries in our sample, based on the shape of
cross-correlation curves. The RV dispersion of the objects in UScoOB
is found to be very similar to that of the BD and VLM stars in
Chamaeleon~I (Cha~) previously studies by others. We also found the
distribution of the mean rotational velocities ($v\,\sin i$) of the
UScoOB objects is similar to that of the Cha~I, but the dispersion of
$v\,\sin i$ is significantly larger than that of the Cha~I objects. 

\end{abstract}

\begin{keywords}
stars: binaries:spectroscopic -- stars: low-mass, brown dwarfs -- stars: formation -- stars:planetary system: formation
\end{keywords}


\section{Introduction }

\label{sec:Introduction}

Most stars are member of binary systems and it is therefore important that
a complete star formation theory be able to predict the binary fraction,
period distribution, and mass-ratio distribution of newly born stellar
objects across a wide range of masses. Furthermore, the study of the individual
binary system is the only direct means to determine the fundamental
stellar properties such as stellar masses and radii. 

Recent high-resolution imaging studies of young brown dwarfs (BDs)
and very low-mass (VLM) stars have placed strong constraints on binaries
with separations of $\sim1-100\,\mathrm{au}$. For example, Hubble
Space Telescope (\emph{HST}) observations of $\alpha$~Per and the
Pleiades indicates a binary fraction ($f$) of $>10$~per~cent with
a bias towards separations ($a$) of less than $15$~au, and a mass-ratio
($q$) of $>0.7$ \citep{martin:2003} for objects around and below
the hydrogen burning limit (see also \citealt{bouy:2006}). A similar
lack of wide binaries was found in the field T-dwarf study \citep{burgasser:2003},
while $f=15$~per~cent was determined by \citet{close:2003} using
the adaptive optics at Gemini North. They also found an upper limit
to the semimajor axis distribution of $\sim20\,\mathrm{au}$. An
\emph{HST} study of more than 80 field late M and L dwarfs \citep{gizis:2003}
indicated $f=15$~per~cent with separations in the range of 1.6--16~au.
For s small sample (12) of BDs and VLM stars (0.04--0.1~$\mathrm{M_{\odot}}$)
in Upper Scorpius OB association (UScoOB), \citet{kraus:2005} found
$f=25_{-8}^{+16}$~per~cent for $5\,\mathrm{au}<a<18\,\mathrm{au}$
by using a similar imaging technique. More recently,
\citet{basri:2006}, combined with the results of earlier works, found
the upper limit of the overall binary fraction for VLM stars of
$26\pm10$~per~cent.   

Using a Monte Carlo simulation, the data from radial velocity
surveys available in the literature, and by carefully considering the sensitivity
and sampling biases, \citet{maxted:2005}  found an overall BD/VLM
binary frequency of 32--45 per~cent assuming $f=15$~per~cent for
$a>2.6$~au. A recent photometric study \citep{pinfield:2003} of
low-mass objects in Pleiades and Praesepe suggested, albeit indirectly,
$f$ as large as 50~per~cent. which would only be compatible with
direct imaging studies if 70--80~per~cent of those binaries have
$a<1\,\mathrm{au}$. For a more comprehensive review of the current
status of BD/VLM binary fraction and the separation distribution,
readers are refer to a recent review of multiplicity studies
by \cite{burgasser:2006}.

The extensive imaging surveys provide excellent observational constraints
on wider BD+BD binaries, but it is now necessary to search for shorter
period BD+BD binaries systematically. Binaries with the separation
of less than 1~au are not resolved by current imaging techniques,
but will be detectable as spectroscopic binaries, providing the mass-ratio
is not too extreme, and velocity separation is large enough. The first
BD+BD spectroscopic binary, PPl~15  \citep{basri:1999}, showed a
double-peaked cross-correlation function with a maximum velocity separation
of $>70\,\kmps$. The binary was found to have an eccentric orbit
($e=0.4$) with a period of $\sim5.8\,\mathrm{d}$. \citet{basri:1999}
suggested that the formation process of substellar objects is biased
towards smaller separation binaries based on the short period of PPl~15
and the lack of Pleiades BD binaries with separations $>40\,\mathrm{au}$.
Note that the median separation of binaries with solar-type primaries
is 30~au \citet{duquennoy:1991}. Pioneering work on the RVs
of BDs and VLMs are presented by \citet{guenther:2003}, \cite{kenyon:2005}
and \citet{joergens:2006} who found a several binary candidates;
however, the orbital parameters and masses of binaries remains
unknown because the follow-up spectroscopic monitoring is lacking
or still being undertaken. In addition to the follow-up observations,
the number of BDs and VLM star binary candidates needs to be increased
in order to have better statistics on short-period binary parameters. 

The separation distribution of BD/VLM binaries is critical to understanding
their origin. There are two main models for the formation of BDs and
VLM stars: first, they formed in a similar manner to higher mass stars,
but from smaller-mass, denser molecular cloud core (e.g.~\citealt{padoan:2002});
second, BD/VLM objects have low masses because they are ejected from
the dense core in which they form via dynamical interactions in multiple
system, cutting off their accretion before they have reached stellar
masses (\citealt{reipurth:2001}; \citealt{bate:2002}). Alternatively,
there is a third model in which a free-floating BD or planetary-mass
object can be formed in the process of the photo-evaporation \citep{whitworth:2004} with
the outer layers of a pre-stellar core ($\sim0.2\,\mathrm{M_{\odot}}$)
removed by the strong radiation pressure from the nearby massive OB
 stars before the accretion onto the protostar at core centre
occurs. 

Due to the dynamical interaction involved in the second model, BD/VLM
binaries that survive are generally expected to have small separations.
In the first model, wider binaries may be expected to be more common.
\citet{bate:2002} suggested that close binaries ($a<10\,\mathrm{au}$)
do not form directly, but result from hardening of wider systems though
a combination of dynamical interactions, accretion and interactions
with circum-binary discs. If BD/VLM binaries have formed through such
mechanisms, one would not expect to find binaries with 1--10~au separations
without also finding many with separation $<1$~au. If an absence/rarity
of binaries with 1~au were found, it would support the idea that
they are ejected quickly from multiple systems before they have undergone
the interactions that shorten their periods. 

Our immediate aim is to identify spectroscopic and close BD/VLM binaries
using the high-resolution echelle spectroscopy at two epochs. This
experiment is sensitive to VLM binaries with separations of $<0.1$~au
which corresponds to a period of $\sim10\,\mathrm{d}$. A larger sample
of candidates will enable us to measure the binary fraction of these
short-period/close binaries (once confirmed), and address whether
there is a significant population of 'hidden' VLM companions. The
long term goal of this project to follow up the binary candidates
found in this paper by spectroscopically monitoring them over different
time scales, enabling us to obtain the radial velocity curves and
their minimum masses. 

In Section~\ref{sec:observation-reduction}. we describe the observations
and the data reduction. The results of radial velocity and rotational
velocity ($v\,\sin i$) measurements are presented in
Section~\ref{sec:Results}. We discuss the binary/multiplicity 
fraction indicated by our RV survey, and compare our results with
earlier works in Section~\ref{sec:Binary-fraction}, and give
our conclusions in Section~\ref{sec:Conclusions}.

\section{Observations}

\label{sec:observation-reduction}

Our sample consists of 18 young, very low-mass objects: 15 objects in
the Upper Scorpius OB association ($\mathrm{d\approx145\, pc}$,
\citealt{dezeeuw:1999}) from the list of \citet{ardila:2000} and 3
objects in the $\rho$~Ophiuchi cloud core ($\mathrm{d\approx150\,
pc}$, \citealt{dezeeuw:1997}) from \citet{luhman:1999}. The spectral
type of the objects range between M5 and M8.5, and the age
$<\sim10\,\mathrm{Myr}$ (\citealt{luhman:1999}; \citealt{ardila:2000};
\citealt{muzerolle:2003}; \citealt{kraus:2005}).  The sample is not
complete, and the selection was solely based on brightness and the
observability. The basic properties of the targets based on the
literature is summarised in Table~\ref{tab:literatures}.

We obtained high-resolution spectra with the Kueyen telescope of VLT
(Cerro Parnal, Chile) using the UVES echelle spectrograph. The
observations were carried out between 2004 April 5 and 2004 May 17 in
the service mode. For each object, spectra were obtained at two different
epochs separated by 4--33~d. For each object at a
given night, two separate spectra are obtained consecutively. This
allows us to derive more reliable uncertainty estimates in the RV
values of our targets (c.f.~\citealt{joergens:2006}).  The data were
obtained using the red arm of UVES spectrograph with two mosaic CCDs
(EEV + MIT/LL with 2k$\times$4k pixels). The wavelength coverage of
6708 -- 10,250~\AA\, and the spectral resolution $R\approx40,000$ were
used. The slit width and length of 1'' and 12'' were used respectively
with a typical seeing of 0.8''.

The data were reduced via the standard ESO pipeline procedures for
UVES echelle spectra. In summary, the data were corrected for bias,
interorder background, sky background, sky emission lines and cosmic
ray hits. They were then flattened, optimally extracted, and finally
the different orders were merged. No binning was performed to achieve
high resolution required for the RV measurements. The wavelength was
calibrated using the Thorium-Argon arc spectra with a typical value
of the standard deviation of the dispersion solution of 5~m\AA\,
which corresponds to 0.2~$\kmps$ at the central wavelength 8600~\AA.
However, the autoguiding of the telescope keeps the star at the centre
of the slit with about a tenth of the FWHM ($1\,\kmps$) which sets
the upper limit for the systematic error in the RV measurements \citep{bailer-jones:2004}.
The accuracy of the wavelength calibration will be demonstrated in
the RV measurements of a RV standard in the following section. A typical
signal-to-noise ratio (S/N) per wavelength bin of the spectra is about
15, and the heliocentric velocity correction was applied to the final
spectra.

%==========================================

\begin{table*}


\caption{Summary of known properties of the targets from literatures:
a.~\citet{luhman:1999} (original list for $\rho$~Oph),
b.~\citet{ardila:2000} (original list for UScoOB),
c.~\citet{wilking:1999}, d.~\citet{muzerolle:2003},
e.~\citet{kraus:2005}, and f.~\citet{mohanty:2005}. }

\label{tab:literatures}

\begin{center}

\begin{tabular}{llcccc}
\hline 
Object&
Sp.&
mass&
RV&
$v\sin i$&
Known multiple?\tabularnewline
&
&
$\left[\mathrm{M_{\odot}}\right]$&
$\left[\mathrm{km\, s^{-1}}\right]$&
$\left[\mathrm{km\, s^{-1}}\right]$&
\tabularnewline
\hline 
GY 5&
$\mathrm{M7^{c}}$&
$0.07^{\mathrm{d}}$&
$-6.3\pm1.9^{\mathrm{d}}$&
$16.8\pm2.7^{\mathrm{d}}$&
no\tabularnewline
GY 141&
$\mathrm{M8.5^{a}}$&
$0.02^{\mathrm{d}}$&
\ldots{}&
$6.0^{f}$&
no\tabularnewline
GY 310&
$\mathrm{M8.5^{c}}$&
$0.08^{\mathrm{a,d}}$&
\ldots{}&
$10.0^{f}$&
no\tabularnewline
USco 40&
$\mathrm{M5^{b}}$&
$0.1^{\mathrm{b}}$&
\ldots{}&
$37.5^{f}$&
no\tabularnewline
USco 53&
$\mathrm{M5^{b}}$&
$0.1^{\mathrm{b}}$&
\ldots{}&
$45.0^{f}$&
no\tabularnewline
USco 55&
$\mathrm{M5.5^{b}}$&
$0.10+0.07^{\mathrm{e}}$&
\ldots{}&
$12.0^{f}$&
$\mathrm{yes^{e}}$\tabularnewline
USco 66&
$\mathrm{M6^{b}}$&
$0.07+0.07^{\mathrm{e}}$&
$-4.4\pm0.6^{\mathrm{d}}$&
$27.5^{f}$&
$\mathrm{yes^{e}}$\tabularnewline
USco 67&
$\mathrm{M5.5^{b}}$&
$0.10^{\mathrm{e}}$&
\ldots{}&
$18.0^{f}$&
no\tabularnewline
USco 75&
$\mathrm{M6^{b}}$&
$0.07^{\mathrm{e}}$&
$-5.6\pm1.1^{\mathrm{d}}$&
$63.0^{f}$&
no\tabularnewline
USco 100&
$\mathrm{M7^{b}}$&
$0.05^{\mathrm{e}}$&
$-8.9\pm0.6^{\mathrm{d}}$&
$50.0^{f}$&
no\tabularnewline
USco 101&
$\mathrm{M5^{b}}$&
$0.05^{\mathrm{b}}$&
\ldots{}&
\ldots{}&
no\tabularnewline
USco 104&
$\mathrm{M5^{b}}$&
$0.05^{\mathrm{b}}$&
\ldots{}&
$16.0^{f}$&
no\tabularnewline
USco 109&
$\mathrm{M6^{b}}$&
$0.07+0.04^{\mathrm{e}}$&
$-3.8\pm0.7^{\mathrm{d}}$&
$6.0^{f}$&
$\mathrm{yes^{e}}$\tabularnewline
USco 112&
$\mathrm{M5.5^{b}}$&
$0.1^{\mathrm{e}}$&
\ldots{}&
$8.0^{f}$&
no\tabularnewline
USco 121&
$\mathrm{M6^{b}}$&
$0.02^{\mathrm{b}}$&
$-38.9\pm1.0^{\mathrm{d}}$&
\ldots{}&
no\tabularnewline
USco 128&
$\mathrm{M7^{b}}$&
$0.05^{\mathrm{e}}$&
$-3.0\pm1.6^{\mathrm{d}}$&
$0.0^{f}$&
no\tabularnewline
USco 130&
$\mathrm{M7.5^{e}}$&
$0.04^{\mathrm{e}}$&
\ldots{}&
$14.0^{f}$&
no\tabularnewline
USco 132&
$\mathrm{M7^{b}}$&
$0.05^{\mathrm{e}}$&
$-8.2\pm1.1^{\mathrm{d}}$&
\ldots{}&
no\tabularnewline
\hline
\end{tabular}

\end{center}

\end{table*}

%==========================================


\section{Results}

\label{sec:Results}


\subsection{Radial velocities}

\label{sub:Radial-velocities}

%==========================================

\begin{table*}


\caption{Summary of the observations, the heliocentric radial
velocities ($\mathrm{RV}$) from two-epoch and the average rotational
velocities ($v\sin i$). The uncertainties of relative radial velocities
($\sigma_{\mathrm{RRV}}$) with respect to the template star LHS~049
and the average radial velocities ($\overline{\mathrm{RV}}$) are also
given. }

\label{tab:summary_results}

\begin{center}

\begin{tabular}{lcrrrrr}
\hline 
Object&
Date&
HJD-2453100&
RV&
$\sigma_{\mathrm{RRV}}$&
$\overline{\mathrm{RV}}$&
$v\sin i$\tabularnewline
&
&
&
$\left[\mathrm{km\, s^{-1}}\right]$&
$\left[\mathrm{km\, s^{-1}}\right]$&
$\left[\mathrm{km\, s^{-1}}\right]$&
$\left[\mathrm{km\, s^{-1}}\right]$\tabularnewline
\hline 
GY 5&
2004-Apr-24&
20.7198440&
$-6.14\pm0.84$&
0.68&
&
\tabularnewline
&
2004-May-07&
33.7721938&
$-5.96\pm0.60$&
0.34&
$-6.05\pm1.03$&
$16.5\pm0.6$\tabularnewline
GY 141&
2004-May-10&
36.6712108&
$-4.39\pm0.60$&
0.34&
&
\tabularnewline
&
2004-May-17&
43.6683007&
$-2.95\pm0.51$&
0.11&
$-3.67\pm0.79$&
$4.4\pm1.4$\tabularnewline
GY 310&
2004-Apr-24&
20.8373504&
$-4.83\pm0.74$&
0.54&
&
\tabularnewline
&
2004-May-09&
35.8224052&
$-8.43\pm0.51$&
0.11&
$-6.63\pm0.90$&
$11.1\pm6.0$\tabularnewline
USco 40&
2004-Apr-05&
1.7661754&
$-7.15\pm0.74$&
0.54&
&
\tabularnewline
&
2004-May-07&
33.7620554&
$-6.80\pm0.51$&
0.11&
$-6.98\pm0.90$&
$34.2\pm0.5$\tabularnewline
USco 53&
2004-Apr-04&
0.9036653&
$-7.27\pm0.93$&
1.21&
&
\tabularnewline
&
2004-May-02&
28.7394676&
$-5.43\pm0.74$&
0.55&
$-6.35\pm1.19$&
$40.0\pm0.6$\tabularnewline
USco 55&
2004-Apr-05&
1.8422807&
$-5.39\pm0.50$&
0.02&
&
\tabularnewline
&
2004-May-02&
28.8141198&
$-6.38\pm0.53$&
0.27&
$-6.38\pm0.73$&
$22.9\pm0.8$\tabularnewline
USco 66&
2004-Apr-05&
1.7972634&
$-5.32\pm0.57$&
0.29&
&
\tabularnewline
&
2004-May-02&
28.7956003&
$-6.41\pm0.65$&
0.42&
$-5.87\pm0.86$&
$25.9\pm1.2$\tabularnewline
USco 67&
2004-Apr-05&
1.7188620&
$-6.01\pm0.74$&
0.55&
&
\tabularnewline
&
2004-May-02&
28.7113799&
$-6.83\pm0.59$&
0.31&
$-6.42\pm0.90$&
$18.4\pm0.4$\tabularnewline
USco 75&
2004-Apr-04&
0.8840376&
$-6.75\pm0.67$&
0.44&
&
\tabularnewline
&
2004-May-07&
33.6065432&
$-9.88\pm1.94$&
1.88&
$-8.32\pm2.05$&
$55.6\pm3.0$\tabularnewline
USco 100&
2004-Apr-06&
1.8179138&
$-6.76\pm2.74$&
2.69&
&
\tabularnewline
&
2004-May-02&
28.7752928&
$-10.23\pm1.80$&
1.73&
$-8.47\pm3.28$&
$43.7\pm3.2$\tabularnewline
USco 101&
2004-Apr-04&
0.8120734&
$-4.22\pm0.87$&
0.71&
&
\tabularnewline
&
2004-May-02&
28.6591660&
$-6.07\pm0.69$&
0.48&
$-5.15\pm1.11$&
$19.1\pm0.3$\tabularnewline
USco 104&
2004-Apr-04&
0.7850385&
$-5.83\pm0.50$&
0.02&
&
\tabularnewline
&
2004-May-02&
28.6349480&
$-7.48\pm0.50$&
0.06&
$-6.66\pm0.06$&
$16.7\pm0.4$\tabularnewline
USco 109&
2004-Apr-05&
1.7453989&
$-4.15\pm0.52$&
0.16&
&
\tabularnewline
&
2004-May-07&
33.6304878&
$-4.41\pm0.50$&
0.03&
$-5.12\pm0.72$&
$8.6\pm1.2$\tabularnewline
USco 112&
2004-Apr-04&
0.8552168&
$-2.70\pm0.69$&
0.47&
&
\tabularnewline
&
2004-May-07&
33.5826368&
$-3.46\pm0.51$&
0.11&
$-3.08\pm0.86$&
$5.8\pm1.2$\tabularnewline
USco 121&
2004-Apr-24&
20.7059408&
$-39.47\pm0.51$&
0.11&
&
\tabularnewline
&
2004-May-02&
28.6969641&
$-42.43\pm0.50$&
0.02&
$-40.95\pm0.71$&
$17.6\pm1.3$\tabularnewline
USco 128&
2004-May-13&
39.7978276&
$-7.41\pm0.85$&
0.69&
&
\tabularnewline
&
2004-May-17&
43.6108024&
$-6.94\pm1.16$&
1.05&
$-7.18\pm1.44$&
$3.6\pm1.1$\tabularnewline
USco 130&
2004-May-09&
35.7724090&
$-4.83\pm0.54$&
0.21&
&
\tabularnewline
&
2004-May-13&
39.8538830&
$-4.95\pm0.74$&
0.55&
$-4.89\pm0.92$&
$15.2\pm1.1$\tabularnewline
USco 132&
2004-May-13&
39.8268683&
$-7.18\pm0.58$&
0.30&
&
\tabularnewline
&
2004-May-17&
43.6391138&
$-7.37\pm1.02$&
0.89&
$-7.28\pm1.17$&
$9.1\pm0.7$\tabularnewline
\hline
\end{tabular}

\end{center}


\end{table*}

%==========================================

%==========================================

\begin{figure*}

\begin{center}

\includegraphics[%
  clip,
  scale=1.3]{figures/radvel_all01.eps}

\end{center}

\caption{Relative radial velocities (RVs) of objects measured in two
different epochs. The vertical axes indicate the amount of deviation
($\Delta \mathrm{RV}$) from the `average' radial velocity
($\overline{\mathrm{RV}}$) in Table~\ref{tab:summary_results}), and
the horizontal axes indicate the time of the observation in
heliocentric Julian date (HJD). The objects are considered to have a
non-constant RV when the error bars of two data points do not overlap
each other. The non-constant RV objects are considered as
binary/multiple candidates.}

\label{fig:radvel01}

\end{figure*}

%==========================================

\addtocounter{figure}{-1}

%==========================================

\begin{figure*}

\begin{center}

\includegraphics[%
  clip,
  scale=1.3]{figures/radvel_all02.eps}

\end{center}

\caption{continued}

\label{fig:radvel02}

\end{figure*}

%==========================================

The radial velocities of each object were determined by using the cross-correlation
function of the object spectrum with that of a template star which has
a similar spectra type. By visual inspection,
the wavelength ranges used for the cross-correlation calculations
are chosen by avoiding the regions of spectra affected by the telluric
lines, and defects and fringes (in near infrared) of the CCDs. The
radial velocities of objects with respect to the template are obtained
by measuring the location of the peak in the cross-correlation
function. The location of the peak is determined by fitting the
cross-correlation function by a function which consists of two gaussians 
(with a common centre) plus a constant term. LHS~49 (Proxima Cen, M5.5) was
chosen as the template for this purpose. The radial velocity of the
template object LHS~49 was obtained by measuring the wavelength shifts
of the prominent photospheric absorption features \ion{K}{i}~$\lambda\lambda$7664.911,~7698.974.
This gives us $\mathrm{RV_{LHS49}=}-22.6\pm0.5\,\kmps$, which is
in good agreement with the earlier measurement of \citet{garcia-sanchez:2001}
who found $\mathrm{RV_{LHS49}=}-21.7\pm1.8\,\kmps$. The heliocentric
RV of each object can be then calculated by adding $\mathrm{RV_{LHS49}}$
with the RV of each object with respect to LHS~49. In the following
measurements of the heliocentric radial velocities, our measurement
($\mathrm{RV_{LHS49}=}-22.6\pm0.5\,\kmps$) will be used for consistency. 

Before applying the cross-correlation technique to our main targets,
we have applied the technique to the radial velocity standard HD~140538
for which an high-accuracy RV measurement via the fixed-configuration,
cross-dispersed �chelle spectrograph Elodie \citep{baranne:1996}
is available. This was done so to ensure not only the validity of
the cross-correlation technique, but also the validity of the wavelength
calibration. In this test, we found the heliocentric $\mathrm{RV_{HD\,140538}=}18.8\pm0.6\,\kmps$
which is in good agreement with the Elodie radial velocity measurement
of $19.00\pm0.05\,\mathrm{km\, s^{-1}}$ \citep{udry:1999}.

The result of the heliocentric RV measurements (from two epochs for
each object) is summarised in Table~\ref{tab:summary_results} along
with the uncertainties. The table also lists the uncertainties in
the `relative' radial velocities ($\sigma_{\mathrm{RRV}}$) which
is the uncertainty of RV with respect to the template star. The
two consecutive measurements of RVs (from same nights) are used to
find the (absolute/heliocentric) RV of the night, and their uncertainties are obtained from
the standard deviations of the mean ($\sigma_{\mathrm{RRV}}$) and
the uncertainty of the template RV value (c.f.~\citealt{joergens:2006}).
Note that for finding a shift in the radial velocities from two
different epochs, only $\sigma_{\mathrm{RRV}}$ is important since one
only requires relative (with respect to a template) RV values.
The `average' radial velocities $\left (\overline{\mathrm{RV}} \right )$ of the
two epochs are also given in the same table. The $\overline{\mathrm{RV}}$
values of the objects in UScoOB are consistent with the earlier measurements
of \citet{muzerolle:2003} in Table~\ref{tab:literatures}.

For each object and for each RV measurement, the deviations ($\Delta\mathrm{RV}$)
from the average RV are computed and summarized in Fig.~\ref{fig:radvel01}
along with their uncertainties ($\sigma_{\mathrm{RRV}}$) in order
to aid the identification of multiplicity candidates. Note that in
computing $\Delta\mathrm{RV}$ we do not require the knowledge of
absolute radial velocities, but only the relative
velocities (with respect to a template). 

To identify an object with a RV variation with a statistical
significance from our sample, we apply the method described by
\citet{maxted:2005}, which we briefly summarize next, to our data.
There are three steps in this method: 
(1)~compute the $\chi^{2}$ by fitting the two-epoch RV data for each 
object with a constant function (a zero-th order polynomial),
(2)~compute the corresponding $\chi^{2}$ probability ($p$),
(3)~designate the object as a non-constant RV object or a binary
candidate if $p<10^{-3}$ (1~per~cent).  When computing $\chi^{2}$, we
use the uncertainties in the relative RV ($\sigma_{\mathrm{RRV}}$). 
The number of degree of freedom in the fitting procedure is obviously
1.  A similar method was also used in a recent RV survey of VLM stars
by \citet{basri:2006}. 

We have computed $p$ for all the objects, and have plotted the results 
as a histogram of the $-\log{p}$, shown in Fig.~\ref{fig:chisq-test}
(excluding the non-member USco~121; see explanation later). 
The figure clearly shows two distinctive populations: one on the left
(with small $-\log{p}$ values) occupied by the RV constant objects,
and one on in the right-most bin (with the largest $-\log{p}$)
occupied by the RV variable objects which have $p<10^{-3}$.  The
expected distribution of $p$ computed consistently with our
uncertainty measurements is also shown in the same figure. The RV
constant population on the left side reasonably matches the expected
curve. This reassures that our uncertainties in RV values are
reasonable, and an additional systematic error may not be necessary in
this analysis (but see Section~\ref{sec:Binary-fraction}). According
to the figure, there are four objects which show significant RV
variations out of 17 samples. They are GY~141, GY~310, USco~55, and
USco~104, and considered as our preliminary binary/multiple
candidates.  We will discuss the effect of the systematics error in detail, and
list our final binary/multiple candidates in Section~\ref{sec:Binary-fraction}. 


%==========================================

\begin{figure}

\begin{center}
\includegraphics[%
  clip,
  scale=0.63]{figures/chisq_hist.eps}
\end{center}

\caption{Histogram of the $\chi^2$ probability ($p$) for fitting the
  observed (relative) RV values with a constant (a horizontal
  line). The constant used in the fit is determined from the weighted mean of
  the two RV measurements for each object.  The objects with
  $p<10^{-3}$ are identified as \emph{non-constant} or \emph{multiple},
  which appear in the right-most bin in the histogram. There
  are four objects which satisfy this condition. The
  expected distribution (dashed) for RV constant objects based on the RV
  uncertainties in the observations (Table~\ref{tab:summary_results})
  and based on the $\chi^{2}$ probability function is also shown for a
  comparison.  The match between the expected distribution and the
  histogram is reasonable, indicating that our uncertainty estimates
  in RV values are also reasonable.  }


\label{fig:chisq-test}

\end{figure}

%==========================================


Finally, the histogram of $\overline{\mathrm{RV}}$ for the objects in UScoOB
is given in Fig.~\ref{fig:vrad_histrogram}. The total number
of the objects is 14. Note that USco~121 is excluded from the graph
since it is identified as a non-member of the UScoOB association based
on the RV value (see Table~\ref{tab:summary_results}).  \citet{muzerolle:2003}
also found it to be a likely non-member based on the radial velocity
and the low lithium abundance. The distribution of the RVs in the
figure was fitted by a gaussian function. We found that the standard
deviation and the peak position of the radial velocity distribution
are $1.2\,\kmps$ and $-5.9\,\kmps$ respectively. The former is very
similar to the standard deviation (0.9~$\kmps$) of the radial velocity
distribution of 9 BDs and VLM objects in Cha~I found by \citet{joergens:2006b}.
They also studied the radial velocity distribution of more massive
25 T~Tauri stars in Cha~I, and found the standard deviations ($1.3\,\kmps$)
is not significantly different from that of the brown dwarfs and the
very low-mass objects. Unfortunately, we do not have the radial velocity
measurements of the higher mass counter parts (T~Tauri stars) in
Upper Sco OB association. This is planned for a near future investigation
as this is important for the study of the mass dependency of the kinematics
in a young stellar cluster. 

According to the hydrodynamical simulations of a low-mass star-forming
cluster of \cite{bate:2003} which yields a stellar density of
$1.8\times10^{3}\,\mathrm{stars\, pc^{-3}}$, the rms dispersion (1-D)
of the stars and the BDs is $1.2\,\kmps$.  Similarly for the model
with a higher stellar density ($2.6\times10^{3}\,\mathrm{stars\,
pc^{-3}}$), the rms dispersion is 2.5~$\kmps$
(\citealt{bate:2005}). The standard deviation of
$\overline{\mathrm{RV}}$ ($1.2\,\kmps$) found in our analysis is more
comparable the lower stellar density model.

%==========================================

\begin{figure}

\begin{center}

\includegraphics[%
  clip,
  scale=0.63]{figures/radvel_hist.eps}

\end{center}

\caption{Histogram of the average heliocentric radial velocities of 14 UScoOB BD and VLM
objects listed in Table~\ref{tab:summary_results} (excluding USco~121,
a non-member). The gaussian fit (dashed) of the radial velociity
distribution gives a standard deviation of $1.2\,\kmps$ and the peak
position of $-5.9\,\kmps$. }
\label{fig:vrad_histrogram}

\end{figure}

%==========================================




\subsection{Rotational velocities}

\label{sub:Rotational-velocities}

The rotational velocities of the objects were determined by measuring
the widths of the cross correlation functions of the target spectra against
a template spectrum from an object which is known to have a very small
rotational velocity. The line broadening of the targets is assumed
to be dominated by rotational broadening. As in the cases for the
radial velocity measurements, LHS~49 is chosen as the template.
Using its rotational period ($P\approx83\,\mathrm{d}$, \citealt{benedict:1998})
and radius ($R_{*}\approx0.145\,\mathrm{R_{\odot}}$ from the VLTI
measurement by \citealt{segransan:2003}), the rotational velocity
of LHS~49 is estimated as $v\,\sin i=2\pi R_{*}/P\approx0.1\,\kmps$;
negligibly small. 

The width of the cross-correlation curves ($\sigma_{\mathrm{CCF}}$)
are calibrated with the rotational velocities ($v\,\sin i$) by cross
correlating the template spectra against the same template spectra
with added rotation (convolved with a given $v\,\sin i$), as
done by e.g.~\citet{tinney:1998}, \citet{mohanty:2003} and \citet{white:2003}.
A linear limb-darkening law with a solar-like parameter ($\epsilon=0.6$)
was assumed in the formulation of the rotational profile described
by \citet{gray:1992}, his Eq.~17.12. For each object, two measurements
of rotational velocities are computed from two independent spectra
obtained at different epochs. As for the RV measurements,
the mean and the standard deviation of the mean are used as the final
rotational velocity and its uncertainty. The final results are recorded
in Table~\ref{tab:summary_results}. In general, our measurements
are in good agreement with the earlier measurements of \citet{muzerolle:2003}
and \citet{mohanty:2005}, given in Table~\ref{tab:literatures}.
For example, \citet{muzerolle:2003} found $v\,\sin i=16.8\pm2.7\,\mathrm{km\, s^{-1}}$
for GY~5 while we found $v\,\sin i=16.5\pm0.6\,\mathrm{km\, s^{-1}}$. 

The range of $v\,\sin i$ found among our objects is $3.6$--$55.6\,\kmps$,
and a similar range is also found by \citet{mohanty:2005}. 
Fig.~\ref{fig:vrot_histrogram} shows the histogram
of $v\,\sin i$ distribution for the UScoOB objects (14 objects excluding
USco 121, non member). The log-normal fit of this distribution gives
the peak position at $16.9\,\kmps$ with a standard deviation $\sigma=27.8\,\kmps$.
Using the $v\,\sin i$ data in \citet{joergens:2001}, the same histogram
bin size used for UScoOB obects and the log-normal fit, we find the
$v\,\sin i$ distribution of the BD and VLM stars (8 objects) in Cha~I
peaks at $15.4\,\kmps$, and has the standard deviation of $8.0\,\kmps$.
The peak of the distribution is similar to that of Upper Sco objects,
but the standard deviation of the distribution is significantly smaller
than that of our Upper Sco objects. The difference maybe due to the
very small sample. A similar fit was applied to the $v\,\sin i$
distribution of 14 T~Tauri stars in Cha~I using the data of \citet{joergens:2001},
and we found a peak at $17.0\,\kmps$ with a  standard deviation
$25.9\,\kmps$ which are very similar to those of the Upper Sco brown
dwarf candidates and VLM stars. 

%==========================================
\begin{figure}

\begin{center}

\includegraphics[%
  clip,
  scale=0.63]{figures/vrot_hist.eps}

\end{center}

\caption{Histogram of the rotational velocities ($v\,\sin{i}$) of 14 UScoOB BD and VLM
objects listed in Table~\ref{tab:summary_results} (excluding USco~121,
a non-member).  The log-normal fit (dashed) of the
rotational velocity distribution gives a standard deviation
$27.8\,\kmps$ and the peak poistion $16.9\,\kmps$.}
\label{fig:vrot_histrogram}

\end{figure}
%==========================================


\section{Binary fraction}
\label{sec:Binary-fraction}

In Section~\ref{sub:Radial-velocities}, we found 4 out of 17
(excluding USco~121; non-member) objects show a statistically
significant RV variation, indicating that they are binary/multiple
candidates. In order to estimate the uncertainty in the
binary/multiple fraction from this relatively small sample, we will
follow the method used by \citet{basri:2006} who considered the binomial
distribution, $P_{B}\left ( x, n, p \right )$ of $x$ positive event
out of $n$ trials with the probability $p$ for a positive event in each
trial. In our case, $n=17$ (the number sample)  and $p=4/17=0.24$
(the fraction of binary/multiple).  The uncertainties was estimated by
plotting $P_{B}\left ( x, 17, 0.24 \right )$, and finding the values
of $x$ at which $P_{B}$ reduced to $e^{-1}$ of the peak value.  In this
analysis, we find the binary/multiple fraction along with the
uncertainties of our sample to be $f=24^{+16}_{-14}$~per~cent.

Next, we investigate the range of semimajor axes of binaries
(equivalently the range of binary periods) to which our RV survey is
sensitive. For this purpose, we will consider the detection
probability for binaries or the RV variables, given the time
separations of two-epoch observations and the ranges of estimated
primary masses (c.f.~Tables~\ref{tab:literatures} and
\ref{tab:summary_results}).  The probability is calculated based
on the simulated RV observations of binaries whose orbit are randomly
selected from a model. A similar method was used by
\citet{maxted:2005} and \citet{basri:2006}.  The most important
factors in determining the detection probability are the size of
uncertainties in RV measurements (which we use the average
$\sigma_{\mathrm{RRV}}$ from our observation), and the time
separations of observations.  The smaller the errors in RV
measurements, the larger the probability for a given binary orbit and
a time separation of orbit. The larger the time separation of
observations, the larger the upper limit of the semimajor axis to
which an observation is sensitive.  The average time separation of the
two-epoch RV observations of our targets is $21.6$~d.
In the following, we will briefly discuss our model assumptions and
parameters which are essentially the same as those of
\citet{maxted:2005} but with some simplifications.  

There are six basic parameters our Monte Carlo simulation: 
primary mass ($M_{1}$), mass ratio ($q$), eccentricity
($e$), orbital phase ($\phi$), orbital inclination ($i$),
longitude of periastron ($\lambda$). The primary mass $M_{1}$ is
assigned from the adopted mass of 
the targets in Table~\ref{tab:literatures}, with a uniform random
deviation of $\pm0.002\,M_{\sun}$. The mass ratio is assumed to be
uniformly distributed between $q=0$ and $0.2$. The eccentricity $e$ is
assumed to be zero (circular orbits).  Both \citet{maxted:2005}
and \citet{basri:2006} found the detection probability is insensitive
to the assumed distribution of $q$ and $e$. The orbital phase is
randomly chosen between 0 and 1.  The inclination $i$ is randomly
chosen from the cumulative distribution of $\cos{i}$. The longitude of
periastron $\lambda$ is not necessary since we assumed $e=0$. 

In order to compute the detection probability as a function of
semimajor axis $a$, we take the following procedure: (1)~for each
object in our targets, we randomly select $10^5$ binaries using the
assumption stated above for a given value of $a$, and compute the RV
of the primary ($V_{1}$), (2)~evolve the orbit by the time separation
of the RV measuremets use in the observations for this object, and
take another simulated measurement of RV ($V_{2}$), (3)~from $V_{1}$,
$V_{2}$ and the average uncertainty in RV ($\sigma_{\mathrm{RRV}}$)
from the observations, we compute the $\chi^{2}$ probability $p$,  and
flag the trial as a detection if $p<10^{-3}$ as done for the real
data, (4)~find the fraction of detections out of all random trials,
(5) repeat 1--4 for the range of $a$ between $10^{-3}$~and 10~au, and
(6) repeat 5 for the all targets, and find the detection probability
averaged over all targets as a function of $a$.

The result of the simulation is shown in
Fig.~\ref{fig:detect_prob}. The detection probability curve shown here
is very similar that of \citet{basri:2006} for a constant time
separation (20~d) case (see their Fig.~4). The probability remains
fairly constant up to  $a \approx 0.1\,\mathrm{au}$, and it rapidly
decreases beyond $a\approx 0.3\,\mathrm{au}$.  This turning point will
increase if we had used larger time separations in our observation. 
From this figure, we find that the 80~per~cent detection probability
up to for binaries with $a<0.12\,\mathrm{au}$, given our time
separations and the average uncertainty ($\sim0.5\,\kmps$) in the RV
measurements. 

%==========================================
\begin{figure}

\begin{center}
\includegraphics[%
  clip,
  scale=0.63]{figures/detect_prob.eps}
\end{center}

\caption{The binary detection probability (solid) as a function of the semimajor
  axis ($a$), based on Monte~Carlo simulations of RV measurements, is
  shown.  For each object in our target list, the simulation was performed
  using the time separation ($\Delta t$) actually used in our
  two-epoch observations (c.f.~Table~\ref{tab:summary_results}) and
  the estimated primary mass  (c.f.~Table~\ref{tab:literatures}).  The
  final detection probability is obtained by averaging
  over the simulated observations of all the objects.  The average
  time separation in the two-epoch observations  is $\sim
  21.6\,\mathrm{d}$, and the average uncertainty in RV is
  $~0.5\,\kmps$. The probability remains fairly
  constant up to $a \approx 0.1\,\mathrm{au}$, and it rapidly decreases
  beyond $a\approx 0.3\,\mathrm{au}$.  The 80~per~cent detection
  probability is achieved for binaries with $a<0.12\,\mathrm{au}$. }  

\label{fig:detect_prob}

\end{figure}

%==========================================


%% The expected binary fraction for our survey can be obtained
%% by integrating the product of the detection probability found above
%% with an assumed binary separation distribution function over a range
%% of $a$. Here, we simply adopt one of the separation distribution
%% function used by \citet{maxted:2005}. The binary separation ($a$)
%% distribution is assumed to be  gaussian in  log($a$) space with its
%% centre $\log{a_{o}}=0.6$ and width $\sigma_{\log{a}}=1.0$, and
%% truncated at $\log{a}=1.176$ where $a$ is in au. This form of the
%% distribution function is motivated by the distribution of solar-type
%% binaries found in \citet{duquennoy:1991}, and the truncation is
%% suggested from the lack of wide binaries for $a>15$~au 
%% (e.g.~\citealt{burgasser:2003}). The distribution is normalized such
%% that the total binary fraction of 26~per~cent as
%% suggested by \citet{basri:2006}.  The resulting separation
%% distribution function is shown also in Fig.~\ref{fig:detect_prob}. 
%% Using this and the detection probability in the same figure, we find
%% the expected binary fraction to be $f_{\mathrm{exp}}=3.2$~per~cent
%% (for $a<0.12$~au), which is
%% much smaller than our observed binary fraction of
%% $f_{\mathrm{obs}}=24^{+16}_{-14}$. The two values are not consistent
%% with each other. There are two possible reasons for the discrepancy:
%% (1)~the assumed binary separation distribution mentioned above is
%% unrealistic, and (2)~the uncertainties in our RV measurements are
%% underestimated.  The first reason is less likely since both
%% \citet{maxted:2005} and \citet{basri:2006} found this form of the $a$
%% distribution function is consistent with the observed binary
%% fractions estimated from many earlier surveys.  We will investigate
%% the possibility for the second reason in the following.

In Section~\ref{sub:Radial-velocities}
(see Fig.~\ref{fig:chisq-test}), we found the histogram of the
$\chi^{2}$ probability $p$, for the RV data fit assuming constant RV,
reasonably matches the expected distribution.  This leads us to
conclude that the uncertainty estimates in RV values were
reasonable. Here, we will inspect the effect of the systematic error
on the observed binary fraction more closely. There are two possible
source of errors: (1)~the RV shifts caused by surface/atmospheric
activity, and (2)~the RV shifts due to the limit of the stability of
autoguiding of the telescope.  \citet{joergens:2006} found that (1) is
negligibly small for a young VLM objects with $M_{*}<0.12\,M_{\sun}$.
The size of RV errors caused by (2) is expected to be $<1\,\kmps$
\citep{bailer-jones:2004}. Note that \citet{maxted:2005} found an
additional systematic error was not needed to be added to the RV
uncertainties of \citet{joergens:2006} who used the same instruments
as ours, and used the same method of estimating the
uncertainties. Hence, we expect the systematic error to be much
smaller than $<1\,\kmps$ if one is required.

The observed binary fraction is expected to be sensitive to the
assumed size of the systematic error ($\sigma_{\mathrm{sys}}$) since
the RV shifts and the systematic errors are in the same order of
magnitude. The number of binary detections would decrease as
$\sigma_{\mathrm{sys}}$ increases.  We have computed the $\chi^2$
probabilities ($p$) for each object being a RV constant object as done
earlier (Section\ref{sub:Radial-velocities}), but with the assumed 
systematics error of $\sigma_{\mathrm{sys}}=0.2\,\kmps$, which was added to
the uncertainties in the RVs of each object.  The results are
summarised in Table~\ref{tab:prob_compare} along with the probabilities
computed without the systematic error (as in
Section\ref{sub:Radial-velocities}).  The number of objects with
$p<10^{-3}$ decreases from 4 to 2 as the systematic error is
introduced. Unfortunately, the probabilities computed with
$\sigma_{\mathrm{sys}}=0.0\,\kmps$ for GY~141 and USco~55 are just
below the $p=10^{-3}$ boundary value; hence, they become RV constant
objects as the systematic error is introduced. This
clearly demonstrates the importance of the uncertainty estimates in
the measured RV values.  We found that the histogram of the $\chi^2$
probabilities with $\sigma_{\mathrm{sys}}=0.2\,\kmps$ (as in
Fig~\ref{fig:chisq-test}), is also in reasonable agreement with the
expected distribution of $-\log{p}$ (not shown here).  
The binary fraction estimated with the assumed systematic error is
$2/17$ or $f=12^{+13}_{-10}$~per~cent for $a<0.12$~au.



%% %==========================================
%% \begin{figure}

%% \begin{center}

%% \includegraphics[%
%%   clip,
%%   scale=0.63]{figures/detect_err_add.eps}

%% \end{center}

%% \caption{The number of binaries/multiples (non-constant RV objects)
%%   detected as a function of the size of additional systematic error ($\sigma_{\mathrm{sys}}$)
%%   in RV measurements. The number of detections decreases rapidly from 4
%%   to 2 as $\sigma_{\mathrm{sys}}=0.1\,\kmps$, which may be caused by
%%   the instability of the autoguiding of the telescope, is added to
%%   the uncertainties estimated in
%%   Table~\ref{tab:summary_results}. The number of  detections
%%   remains constant at 2 for a relatively wide range of the additional
%%   errors. i.e.~$0.05\,\kmps < \sigma_{\mathrm{sys}} < 0.275\,
%%   \kmps$. No binary is found for  $\sigma_{\mathrm{sys}} > 0.375\,\kmps$.} 
%% \label{fig:adding_sys_err}

%% \end{figure}
%% %==========================================


%==========================================
\begin{table}


\caption{Comparison of the $\chi^{2}$ probability ($p$), for each object
  being a RV constant, computed with an addition systematic error
  $\sigma_{\mathrm{sys}}$ (right), and without (left). The objects are
  listed in ascending order of $p$. The ellipses represent the
  boundary ($p=10^{-3}$) between RV constant objects and RV 
  variable objects. }

\label{tab:prob_compare}

\begin{center}
\begin{tabular}{lclc}
\hline 
&
$\sigma_{\mathrm{sys}}=0.0\,\kmps$&
&
$\sigma_{\mathrm{sys}}=0.2\,\kmps$\tabularnewline
obj.~ID&
$p$&
obj.~ID&
$p$\tabularnewline
\hline
USco~104&
$4.3\times10^{-12}$&
USco~104&
$3.5\times10^{-6}$\tabularnewline
GY~310&
$1.8\times10^{-10}$&
GY~310&
$2.0\times10^{-5}$\tabularnewline
GY~141&
$1.6\times10^{-4}$&
$\cdots$&
$\cdots$\tabularnewline
USco~55&
$7.2\times10^{-4}$&
GY~141&
$5.8\times10^{-2}$\tabularnewline
$\cdots$&
$\cdots$&
USco~55&
$1.6\times10^{-1}$\tabularnewline
USco~101&
$8.9\times10^{-2}$&
USco~130&
$2.3\times10^{-1}$\tabularnewline
USco~66&
$9.2\times10^{-2}$&
USco~132&
$2.6\times10^{-1}$\tabularnewline
USco~75&
$2.8\times10^{-1}$&
USco~101 &
$2.8\times10^{-1}$\tabularnewline
USco~109&
$3.0\times10^{-1}$&
GY~5&
$2.8\times10^{-1}$\tabularnewline
USco~112&
$3.1\times10^{-1}$&
USco~75&
$3.9\times10^{-1}$\tabularnewline
USco~132&
$3.2\times10^{-1}$&
USco~66&
$4.3\times10^{-1}$\tabularnewline
USco-~30&
$3.2\times10^{-1}$&
USco~128&
$4.6\times10^{-1}$\tabularnewline
GY~5&
$3.7\times10^{-1}$&
USco~53&
$5.8\times10^{-1}$\tabularnewline
USco~53 &
$4.2\times10^{-1}$&
USco40&
$6.0\times10^{-1}$\tabularnewline
USco~67&
$4.8\times10^{-1}$&
USco~112&
$6.5\times10^{-1}$\tabularnewline
USco~128&
$5.4\times10^{-1}$&
USco~100&
$6.6\times10^{-1}$\tabularnewline
USco~100&
$6.2\times10^{-1}$&
USco~67&
$7.0\times10^{-1}$\tabularnewline
USco~40&
$7.2\times10^{-1}$&
USco~109&
$7.1\times10^{-1}$\tabularnewline
\hline
\end{tabular} 


\end{center}

\end{table}
%==========================================


%% %==========================================
%% \begin{table}


%% \caption{Summary of the observed binary fractions ($f_{\mathrm{obs}}$) compared with the
%%   expected binary fractions ($f_{\mathrm{exp}}$) with and without a
%%   systematic error ($\sigma_{\mathrm{sys}}$). }

%% \label{tab:binary_fraction}

%% \begin{center}

%% \begin{tabular}{ccccl}
%% \hline 
%% $\sigma_{\mathrm{sys}}$&  
%% N. detected&  
%% $f_{\mathrm{obs}}$&
%% $f_{\mathrm{exp}}$&  
%% obj.~ID  \tabularnewline
%% ($\kmps$)&
%% &
%% (per~cent)&
%% (per~cent)&
%% \tabularnewline
%% \hline
%% $0.0$&
%% $4$&
%% $24^{+16}_{-14}$&
%% $3.2$&
%% GY~141\tabularnewline
%% &
%% &
%% &
%% &
%% GY~310\tabularnewline
%% &
%% &
%% &
%% &
%% USco~55\tabularnewline
%% &
%% &
%% &
%% &
%% USco~104\tabularnewline
%% $0.1$&
%% $2$&
%% $12^{+13}_{-10}$&
%% $2.9$&
%% GY~310\tabularnewline
%% &
%% &
%% &
%% &
%% USco~104\tabularnewline
%% \hline
%% \end{tabular}

%% \end{center}

%% \end{table}
%==========================================

\section{Conclusions}

\label{sec:Conclusions}

We have presented two-epoch RV survey of 18 young BDs and VLM stars
($0.02~M_{\sun} < M_{*} < 0.1~M_{\sun}$) in
UScoOB and $\rho$~Oph dark cloud core using the high resolution UVES
echelle spectroscopy at VLT.  The average time separation of RV
measurements are 21.6~d, and our RV survey is sensitive to binaries
with separation smaller than 0.12~au.  One of our targets, USco~121, is
most likely a non-member of the UScoOB association based on the
deviation of the RV from the result of the population in the
association. A similar conclusion was found by \citet{muzerolle:2003}
from their RV study and the low lithium abundance.  

We found 4 (GY~141, GY~310, USco~55 and USco~104) out of 17 objects as our 
binary/multiple candidates. This corresponds to the binary fraction of
$24^{+16}_{-14}$~per~cent for the binary separation $a<0.12$~au.  We found
that the estimated binary fraction is very sensitive to
the assumed size of systematic error. If an additional systematic
error of $0.2\,\kmps$  is added to the uncertainties in the RV
measurements, we find only 2 (GY~310 and USco~104) out of 17 objects
as our binary/multiple candidates, which corresponds to the binary
fraction of $12^{+13}_{-10}$~per~cent.

The recent high-resolution imaging survey of brown dwarfs and very
low-mass objects (M5.5--M7.5 ) in the UScoOB by \citet{kraus:2005} (which was
not known to authors at the time of our observation: April--May, 2004)
confirms that USco~55 and USco~66 are multiple systems, and USco~109
is most likely a multiple system.  Their projected separations are
$a>4.0$~au which is well above our sensitivity limit of
0.12~au; therefore, it is not surprising that we did not identify USco~66 and
USco~109 as multiples. Interestingly, we also found USco~55 as a
candidate for a multiple system. This may indicate that USco~55 is  
a triple system with one of object located within 0.12~au from the
primary.  In addition, we identify USco~104 as a binary candidate, but this
was not in their target list.  Further they found USco~67, 75, 112, 128,
130, 132 as non-multiple, and similarly we found them as
'non-candidates' for multiplicity. 

We found the RV dispersion ($1.2\,\kmps$) of the objects in UScoOB is
very similar to that of the BDs and VLM stars in Chamaeleon~I (Cha~I)
previous study by \citet{joergens:2006b}. The rotational velocities
($v\,\sin i$) of the samples were also measured. The distribution of
$v\,\sin i$ for the UScoOB objects peaks around $16.9\,\kmps$ which is
also similar to that of the Cha~I population found by
\citet{joergens:2006b}; however, the dispersion of $v\,\sin i$ for the
UScoOB objects ($27.8\,\kmps$) is found to be much larger than that of
the Cha~I objects ($8.0\,\kmps$)

Follow-up spectroscopic observations of the binary candidates
presented here are planned in near future. There are only a few RV
variable binary candidates identified in earlier surveys
(\citealt{guenther:2003}; \citealt{kenyon:2005}; \citealt{joergens:2006}).
As \citet{burgasser:2006} points out most of the current RV and
imaging surveys use samples from magnitude-limited survey, but one
should attempt to use the samples from volume-limited survey in order
to find a correct statistics on binary parameters more straightforwardly,
i.e. without correcting for bias.

\section*{Acknowledgements}

We thank the staff of VLT of the ESO for carrying out the observations
in service mode. RK is grateful for Rob Jeffries for helpful suggestions on the
data analysis presented in this paper. This work was supported by
PPARC rolling grant PP/C501609/1.
 

%

%

%

%----------------------------------------------------

% Bibliography follows here.

%-----------------------------------------------------

%

%

%





\bibliographystyle{mn}

\bibliography{local}

%

\label{lastpage}
\end{document}

% LocalWords:  BDs VLM OB Ophiuchi al Ryuichi Harries QL RV UScoOB BD echelle
% LocalWords:  VLT UVES RVs Cha bouy burgasser gizis kraus basri maxted PPl EEV
% LocalWords:  pinfield substellar duquennoy VLMs guenther joergens padoan CCDs
% LocalWords:  reipurth whitworth dezeeuw ardila luhman muzerolle Kueyen Parnal
% LocalWords:  LL ESO interorder autoguiding FWHM jones Oph wilking mohanty Sp
% LocalWords:  GY USco LHS HJD gaussian Cen garcia sanchez HD RRV pc CCF exp
% LocalWords:  obs
