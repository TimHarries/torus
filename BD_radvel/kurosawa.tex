%% LyX 1.3 created this file.  For more info, see http://www.lyx.org/.
%% Do not edit unless you really know what you are doing.
\documentclass[oneside,english,useAMS, usenatbib]{mn2e}
\usepackage[T1]{fontenc}
\usepackage[latin1]{inputenc}
\setcounter{tocdepth}{3}
\usepackage{graphicx}

\makeatletter

%%%%%%%%%%%%%%%%%%%%%%%%%%%%%% LyX specific LaTeX commands.
%% Bold symbol macro for standard LaTeX users
\newcommand{\boldsymbol}[1]{\mbox{\boldmath $#1$}}

%% Because html converters don't know tabularnewline
\providecommand{\tabularnewline}{\\}

%%%%%%%%%%%%%%%%%%%%%%%%%%%%%% User specified LaTeX commands.
\usepackage{times}
\usepackage{amssymb}

\newcommand{\aap}{A\&A}
\newcommand{\aaps}{A\&AS}
\newcommand{\apj}{ApJ}
\newcommand{\apjl}{\apj}
\newcommand{\pasp}{PASP}
\newcommand{\aj}{AJ}
\newcommand{\mnras}{MNRAS}
\newcommand{\apjs}{ApJS}
\newcommand{\aapr}{A\&AR}
\newcommand{\pasj}{PASJ}

\newcommand{\kmps}{\mathrm{km~s^{-1}}}
\newcommand\ion[2]{#1$\,${\sc {#2}}}   % ion, i.e., CII = \ion{C}{ii}
\newcommand{\Kelvin}{\mathrm{K}}

\usepackage{babel}
\makeatother
\begin{document}
\title[Radial and rotational velocities of young BDs and VLM stars]{Radial and rotational velocities of young brown dwarfs and very low-mass stars in the Upper Scorpius OB association and the $\rho$~Ophiuchi cloud core}

\author[R. Kurosawa et\,al.]{Ryuichi Kurosawa\thanks{E-mail: rk@astro.ex.ac.uk} and  Tim J.~Harries\\ School of Physics, University of Exeter, Stocker Road, Exeter EX4 4QL}

\date{Dates to be inserted}

\pagerange{\pageref{firstpage}--\pageref{lastpage}} \pubyear{2006}

\maketitle

\label{firstpage}

\begin{abstract}

We present the results of a radial velocity (RV) survey of 15 brown
dwarfs (BDs) and very low-mass (VLM) stars in the Upper Scorpius OB
association (UScoOB) and 3 BD candidates in the $\rho$~Ophiuchi
dark cloud core. We obtained high-resolution echelle spectra at VLT
using Ultraviolet and Visual Echelle Spectrograph (UVES) at two different
epochs for each object, and measured the shifts in their RVs to identify
candidates for binary/multiple systems in the sample.
We found 9 out of 18 (50 per cent) in our sample show a significant
RV change in 4--33~d time scale, and are considered as binary/multiple
`candidates.' The fraction slightly decreases to 43 per cent (6 out
of 14) if we consider the objects only in the UScoOB. These fractions
do not necessary represent that of a real binary population among
young BDs and VLM stars since their true multiplicities are still
to be confirmed by follow-up spectroscopic monitoring. We found no
double-lined spectroscopic binaries in our sample, based on the shape of cross-corelation
curves. Our RV-shift tests successfully detected the recently confirmed 3 binaries in UScoOB
via high-resolution imaging technique. We found the RV dispersion
of the objects in UScoOB is very similar to that of the BDs and VLM
stars in Chamaeleon~I (Cha~I) previously studied by others. We also
measured the rotational velocities ($v\,\sin i$) of the sample,
and found good agreements with earlier studies. The distribution of
$v\,\sin i$ for the UScoOB objects peaks around $16.9\,\kmps$ which
is similar to that of the Cha~I population. 

\end{abstract}

\begin{keywords}
stars: binaries:spectroscopic -- stars: low-mass, brown dwarfs -- stars: formation -- stars:planetary system: formation
\end{keywords}


\section{Introduction }

\label{sec:Introduction}

Most stars are member of binary systems and it is therefore important that
a complete star formation theory be able to predict the binary fraction,
period distribution, and mass-ratio distribution of newly born stellar
objects across a wide range of masses. Furthermore, the study of the individual
binary system is the only direct means to determine the fundamental
stellar properties such as stellar masses and radii. 

Recent high-resolution imaging studies of young brown dwarfs (BDs)
and very low-mass (VLM) stars have placed strong constraints on binaries
with separations of $\sim1-100\,\mathrm{au}$. For example, Hubble
Space Telescope (\emph{HST}) observations of $\alpha$~Per and the
Pleiades indicates a binary fraction ($f$) of $>10$~per~cent with
a bias towards separations ($a$) of less than $15$~au, and a mass-ratio
($q$) of $>0.7$ \citep{martin:2003} for objects around and below
the hydrogen burning limit (see also \citealt{bouy:2006}). A similar
lack of wide binaries was found in the field T-dwarf study \citep{burgasser:2003},
while $f=15$~per~cent was determined by \citet{close:2003} using
the adaptive optics at Gemini North. They also found an upper limit
to the semi-major axis distribution of $\sim20\,\mathrm{au}$. An
\emph{HST} study of more than 80 field last M and L dwarfs \citep{gizis:2003}
indicated $f=15$~per~cent with separations in the range of 1.6--16~au.
For s small sample (12) of BDs and VLM stars (0.04--0.1~$\mathrm{M_{\odot}}$)
in Upper Scorpius OB association (UScoOB), \citet{kraus:2005} found
$f=25_{-8}^{+16}$~per~cent for $5\,\mathrm{au}<a<18\,\mathrm{au}$
by using a similar imaging technique.. 

Using a Monte Carlo simulation, the data from radial velocity
surveys available in the literature, and by carefully considering the sensitivity
and sampling biases, \citet{maxted:2005}  found an overall BD/VLM
binary frequency of 32--45 per~cent assuming $f=15$~per~cent for
$a>2.6$~au. A recent photometric study \citep{pinfield:2003} of
low-mass objects in Pleiades and Praesepe suggested, albeit indirectly,
$f$ as large as 50~per~cent. which would only be compatible with
direct imaging studies if 70--80~per~cent of those binaries have
$a<1\,\mathrm{au}$. For a more comprehensive review of the current
status of BD/VLM binary fraction and the separation distribution,
readers are refer to a recent review of multiplicity studies
by \cite{burgasser:2006}.

The extensive imaging surveys provide excellent observational constraints
on wider BD+BD binaries, but it is now necessary to search for shorter
period BD+BD binaries systematically. Binaries with the separation
of less than 1~au are not resolved by current imaging techniques,
but will be detectable as spectroscopic binaries, providing the mass-ratio
is not too extreme, and velocity separation is large enough. The first
BD+BD spectroscopic binary, PPl~15  \citep{basri:1999} showed a
double-peaked cross-correlation function with a maximum velocity separation
of $>70\,\kmps$. The binary was found to have an eccentric orbit
($e=0.4$) with a period of $\sim5.8\,\mathrm{d}$. \citet{basri:1999}
suggested that the formation process of substellar objects is biased
towards smaller separation binaries based on the short period of PPl~15
and the lack of Pleiades BD binaries with separations $>40\,\mathrm{au}$.
Note that the median separation of binaries with solar-type primaries
is 30~au \citet{duquennoy:1991}. Pioneering work on the RVs
of BDs and VLMs are presented by \citet{guenther:2003}, \cite{kenyon:2005}
and \citet{joergens:2006} who found a several binary candidates;
however, the orbital parameters and masses of binaries remains
unknown because the follow-up spectroscopic monitoring is lacking
or still being undertaken. In addition to the follow-up observations,
the number of BDs and VLM star binary candidates needs to be increased
in order to have better statistics on short-period binary parameters. 

The separation distribution of BD/VLM binaries is critical to understanding
their origin. There are two main models for the formation of BDs and
VLM stars: first, they formed in a similar manner to higher mass stars,
but from smaller-mass, denser molecular cloud core (e.g.~\citealt{padoan:2002});
second, BD/VLM objects have low masses because they are ejected from
the dense core in which they form via dynamical interactions in multiple
system, cutting off their accretion before they have reached stellar
masses (\citealt{reipurth:2001}; \citealt{bate:2002}). Alternatively,
there is a third model in which a free-floating BD or planetary-mass
object can be formed in the process of the photo-evaporation \citep{whitworth:2004} with
the outer layers of a pre-stellar core ($\sim0.2\,\mathrm{M_{\odot}}$)
removed by the strong radiation pressure from the nearby massive OB
 stars before the accretion onto the protostar at core centre
occurs. 

Due to the dynamical interaction involved in the second model, BD/VLM
binaries that survive are generally expected to have small separations.
In the first model, wider binaries may be expected to be more common.
\citet{bate:2002} suggested that close binaries ($a<10\,\mathrm{au}$)
do not form directly, but result from hardening of wider systems though
a combination of dynamical interactions, accretion and interactions
with circum-binary discs. If BD/VLM binaries have formed through such
mechanisms, one would not expect to find binaries with 1--10~au separations
without also finding many with separation $<1$~au. If an absence/rarity
of binaries with 1~au were found, it would support the idea that
they are ejected quickly from multiple systems before they have undergone
the interactions that shorten their periods. 

Our immediate aim is to identify spectroscopic and close BD/VLM binaries
using the high-resolution echelle spectroscopy at two epochs. This
experiment is sensitive to VLM binaries with separations of $<0.1$~au
which corresponds to a period of $\sim10\,\mathrm{d}$. A larger sample
of candidates will enable us to measure the binary fraction of these
short-period/close binaries (once confirmed), and address whether
there is a significant population of 'hidden' VLM companions. The
long term goal of this project to follow up the binary candidates
found in this paper by spectroscopically monitoring them over different
time scales, enabling us to obtain the radial velocity curves and
their minimum masses. 

In Section~\ref{sec:observation-reduction}. we describe the observations
and the data reduction. The results of radial velocity and rotational
velocity ($v\,\sin i$) measurements are presented in Sections~\ref{sub:Radial-velocities}
and \ref{sub:Radial-velocities} respectively. We discuss the binary/multiplicity
fraction indicated by our RV survey in Section~\ref{sec:Binary-frequency}, and give
our conclusions in Section~\ref{sec:Conclusions}.

\section{Observations}

\label{sec:observation-reduction}

Our sample consists of 18 young, very low-mass objects: 15 objects in the Upper Scorpius OB association ($\mathrm{d\approx145\, pc}$, \citealt{dezeeuw:1999}) from the list
of \citet{ardila:2000} and 3 objects in the $\rho$~Ophiuchi cloud
core ($\mathrm{d\approx150\, pc}$, \citealt{dezeeuw:1997}) from
 \citet{luhman:1999}. The spectral type of the objects range between
M5 and M8.5, and the age $<\sim10\,\mathrm{Myr}$ (\citealt{luhman:1999};
\citealt{ardila:2000}; \citealt{muzerolle:2003}; \citealt{kraus:2005}).
The sample is not complete, and the selection was solely based on 
brightness and the observability. The basic properties of the
targets based on the literature is summarised in Table~\ref{tab:literatures}.

We obtained high-resolution spectra with the Kueyen telescope of VLT
(Cerro Parnal, Chile) using the UVES echelle spectrograph. The observations
were carried out between 2004 April 5 and 2004 May 17 in  service
mode. For each object, spectra were obtained at two different
epochs separated by several (between 4 and 33~d). For each object at a given 
night, two separate spectra are obtained consecutively. This allows us to derive more
reliable uncertainty estimates in the RV values of our targets (c.f.~\citealt{joergens:2006}).
The data were obtained using the red arm of UVES spectrograph with
two mosaic CCDs (EEV + MIT/LL with 2k$\times$4k pixels). The wavelength
coverage of 6708 -- 10,250~\AA\, and the spectral resolution $R\approx40,000$
were used. The slit width and length of 1'' and 12'' were used respectively with a typical seeing
of 0.8''. 

The data were reduced via the standard ESO pipeline procedures for
UVES echelle spectra. In summary, the data were corrected for bias,
interorder background, sky background, sky emission lines and cosmic
ray hits. They were then flattened, optimally extracted, and finally
the different orders were merged. No binning was performed to achieve
high resolution required for the RV measurements. The wavelength was
calibrated using the Thorium-Argon arc spectra with a typical value
of the standard deviation of the dispersion solution of 5~m\AA\,
which corresponds to 0.2~$\kmps$ at the central wavelength 8600~\AA.
However, the autoguiding of the telescope keeps the star at the centre
of the slit with about a tenth of the FWHM ($1\,\kmps$) which sets
the upper limit for the systematic error in the RV measurements \citep{bailer-jones:2004}.
The accuracy of the wavelength calibration will be demonstrated in
the RV measurements of a RV standard in the following section. A typical
signal-to-noise ratio (S/N) per wavelength bin of the spectra is about
15, and the heliocentric velocity correction was applied to the final
spectra.

%==========================================

\begin{table*}

\label{tab:literatures}

\caption{Summary of known properties of the targets from literature: a.~\citet{luhman:1999} (original list for $\rho$~Oph), b.~\citet{ardila:2000} (original list for UScoOB),  c.~\citet{wilking:1999}, d.~\citet{muzerolle:2003}, e.~\citet{kraus:2005}, and f.~\citet{mohanty:2005}. }

\begin{center}

\begin{tabular}{llcccc}
\hline 
Object&
Sp.&
mass&
RV&
$v\sin i$&
Known multiple?\tabularnewline
&
&
$\left[\mathrm{M_{\odot}}\right]$&
$\left[\mathrm{km\, s^{-1}}\right]$&
$\left[\mathrm{km\, s^{-1}}\right]$&
\tabularnewline
\hline 
GY 5&
$\mathrm{M7^{c}}$&
$0.07^{\mathrm{d}}$&
$-6.3\pm1.9^{\mathrm{d}}$&
$16.8\pm2.7^{\mathrm{d}}$&
no\tabularnewline
GY 141&
$\mathrm{M8.5^{a}}$&
$0.02^{\mathrm{d}}$&
\ldots{}&
$6.0^{f}$&
no\tabularnewline
GY 310&
$\mathrm{M8.5^{c}}$&
$0.08^{\mathrm{a,d}}$&
\ldots{}&
$10.0^{f}$&
no\tabularnewline
USco 40&
$\mathrm{M5^{b}}$&
$0.1^{\mathrm{b}}$&
\ldots{}&
$37.5^{f}$&
no\tabularnewline
USco 53&
$\mathrm{M5^{b}}$&
$0.1^{\mathrm{b}}$&
\ldots{}&
$45.0^{f}$&
no\tabularnewline
USco 55&
$\mathrm{M5.5^{b}}$&
$0.10+0.07^{\mathrm{e}}$&
\ldots{}&
$12.0^{f}$&
$\mathrm{yes^{e}}$\tabularnewline
USco 66&
$\mathrm{M6^{b}}$&
$0.07+0.07^{\mathrm{e}}$&
$-4.4\pm0.6^{\mathrm{d}}$&
$27.5^{f}$&
$\mathrm{yes^{e}}$\tabularnewline
USco 67&
$\mathrm{M5.5^{b}}$&
$0.10^{\mathrm{e}}$&
\ldots{}&
$18.0^{f}$&
no\tabularnewline
USco 75&
$\mathrm{M6^{b}}$&
$0.07^{\mathrm{e}}$&
$-5.6\pm1.1^{\mathrm{d}}$&
$63.0^{f}$&
no\tabularnewline
USco 100&
$\mathrm{M7^{b}}$&
$0.05^{\mathrm{e}}$&
$-8.9\pm0.6^{\mathrm{d}}$&
$50.0^{f}$&
no\tabularnewline
USco 101&
$\mathrm{M5^{b}}$&
$0.05^{\mathrm{b}}$&
\ldots{}&
\ldots{}&
no\tabularnewline
USco 104&
$\mathrm{M5^{b}}$&
$0.05^{\mathrm{b}}$&
\ldots{}&
$16.0^{f}$&
no\tabularnewline
USco 109&
$\mathrm{M6^{b}}$&
$0.07+0.04^{\mathrm{e}}$&
$-3.8\pm0.7^{\mathrm{d}}$&
$6.0^{f}$&
$\mathrm{yes^{e}}$\tabularnewline
USco 112&
$\mathrm{M5.5^{b}}$&
$0.1^{\mathrm{e}}$&
\ldots{}&
$8.0^{f}$&
no\tabularnewline
USco 121&
$\mathrm{M6^{b}}$&
$0.02^{\mathrm{b}}$&
$-38.9\pm1.0^{\mathrm{d}}$&
\ldots{}&
no\tabularnewline
USco 128&
$\mathrm{M7^{b}}$&
$0.05^{\mathrm{e}}$&
$-3.0\pm1.6^{\mathrm{d}}$&
$0.0^{f}$&
no\tabularnewline
USco 130&
$\mathrm{M7.5^{e}}$&
$0.04^{\mathrm{e}}$&
\ldots{}&
$14.0^{f}$&
no\tabularnewline
USco 132&
$\mathrm{M7^{b}}$&
$0.05^{\mathrm{e}}$&
$-8.2\pm1.1^{\mathrm{d}}$&
\ldots{}&
no\tabularnewline
\hline
\end{tabular}

\end{center}

\end{table*}

%==========================================


\section{Results}

\label{sec:Results}


\subsection{Radial velocities}

\label{sub:Radial-velocities}

%==========================================

\begin{table*}

\caption{Summary of the observations, the heliocentric radial velocities ($\mathrm{RV}$) from two-epoch and the average rotational velocities ($v\sin i$). The uncetainties of relative radial velocties ($\sigma_{\mathrm{RRV}}$) with respect to the template star LHS~049 and the average radial velocties ($\overline{\mathrm{RV}}$) are also given. The last column indicates whether a target is a candidate for multiplicity i.e.~the measured radial velocity changes from two different epoch is larger than $1\,\sigma_{\mathrm{RRV}}$ of each others (c.f.~Fig.~\ref{fig:radvel01}).}


\begin{center}

\begin{tabular}{lcrrrrrc}
\hline 
Object&
Date&
HJD-2453100&
RV&
$\sigma_{\mathrm{RRV}}$&
$\overline{\mathrm{RV}}$&
$v\sin i$&
candidate?\tabularnewline
&
&
&
$\left[\mathrm{km\, s^{-1}}\right]$&
$\left[\mathrm{km\, s^{-1}}\right]$&
$\left[\mathrm{km\, s^{-1}}\right]$&
$\left[\mathrm{km\, s^{-1}}\right]$&
\tabularnewline
\hline 
GY 5&
2004-Apr-24&
20.7198440&
$-6.14\pm0.84$&
0.68&
&
&
\tabularnewline
&
2004-May-07&
33.7721938&
$-5.96\pm0.60$&
0.34&
$-6.05\pm1.03$&
$16.5\pm0.6$&
no\tabularnewline
GY 141&
2004-May-10&
36.6712108&
$-4.39\pm0.60$&
0.34&
&
&
\tabularnewline
&
2004-May-17&
43.6683007&
$-2.95\pm0.51$&
0.11&
$-3.67\pm0.79$&
$4.4\pm1.4$&
yes\tabularnewline
GY 310&
2004-Apr-24&
20.8373504&
$-4.83\pm0.74$&
0.54&
&
&
\tabularnewline
&
2004-May-09&
35.8224052&
$-8.43\pm0.51$&
0.11&
$-6.63\pm0.90$&
$11.1\pm6.0$&
yes\tabularnewline
USco 40&
2004-Apr-05&
1.7661754&
$-7.15\pm0.74$&
0.54&
&
&
\tabularnewline
&
2004-May-07&
33.7620554&
$-6.80\pm0.51$&
0.11&
$-6.98\pm0.90$&
$34.2\pm0.5$&
no\tabularnewline
USco 53&
2004-Apr-04&
0.9036653&
$-7.27\pm0.93$&
1.21&
&
&
\tabularnewline
&
2004-May-02&
28.7394676&
$-5.43\pm0.74$&
0.55&
$-6.35\pm1.19$&
$40.0\pm0.6$&
no\tabularnewline
USco 55&
2004-Apr-05&
1.8422807&
$-5.39\pm0.50$&
0.02&
&
&
\tabularnewline
&
2004-May-02&
28.8141198&
$-6.38\pm0.53$&
0.27&
$-6.38\pm0.73$&
$22.9\pm0.8$&
yes\tabularnewline
USco 66&
2004-Apr-05&
1.7972634&
$-5.32\pm0.57$&
0.29&
&
&
\tabularnewline
&
2004-May-02&
28.7956003&
$-6.41\pm0.65$&
0.42&
$-5.87\pm0.86$&
$25.9\pm1.2$&
no\tabularnewline
USco 67&
2004-Apr-05&
1.7188620&
$-6.01\pm0.74$&
0.55&
&
&
\tabularnewline
&
2004-May-02&
28.7113799&
$-6.83\pm0.59$&
0.31&
$-6.42\pm0.90$&
$18.4\pm0.4$&
no\tabularnewline
USco 75&
2004-Apr-04&
0.8840376&
$-6.75\pm0.67$&
0.44&
&
&
\tabularnewline
&
2004-May-07&
33.6065432&
$-9.88\pm1.94$&
1.88&
$-8.32\pm2.05$&
$55.6\pm3.0$&
yes\tabularnewline
USco 100&
2004-Apr-06&
1.8179138&
$-6.76\pm2.74$&
2.69&
&
&
\tabularnewline
&
2004-May-02&
28.7752928&
$-10.23\pm1.80$&
1.73&
$-8.47\pm3.28$&
$43.7\pm3.2$&
no\tabularnewline
USco 101&
2004-Apr-04&
0.8120734&
$-4.22\pm0.87$&
0.71&
&
&
\tabularnewline
&
2004-May-02&
28.6591660&
$-6.07\pm0.69$&
0.48&
$-5.15\pm1.11$&
$19.1\pm0.3$&
yes\tabularnewline
USco 104&
2004-Apr-04&
0.7850385&
$-5.83\pm0.50$&
0.02&
&
&
\tabularnewline
&
2004-May-02&
28.6349480&
$-7.48\pm0.50$&
0.06&
$-6.66\pm0.06$&
$16.7\pm0.4$&
yes\tabularnewline
USco 109&
2004-Apr-05&
1.7453989&
$-4.15\pm0.52$&
0.16&
&
&
\tabularnewline
&
2004-May-07&
33.6304878&
$-4.41\pm0.50$&
0.03&
$-5.12\pm0.72$&
$8.6\pm1.2$&
yes\tabularnewline
USco 112&
2004-Apr-04&
0.8552168&
$-2.70\pm0.69$&
0.47&
&
&
\tabularnewline
&
2004-May-07&
33.5826368&
$-3.46\pm0.51$&
0.11&
$-3.08\pm0.86$&
$5.8\pm1.2$&
yes\tabularnewline
USco 121&
2004-Apr-24&
20.7059408&
$-39.47\pm0.51$&
0.11&
&
&
\tabularnewline
&
2004-May-02&
28.6969641&
$-42.43\pm0.50$&
0.02&
$-40.95\pm0.71$&
$17.6\pm1.3$&
yes\tabularnewline
USco 128&
2004-May-13&
39.7978276&
$-7.41\pm0.85$&
0.69&
&
&
\tabularnewline
&
2004-May-17&
43.6108024&
$-6.94\pm1.16$&
1.05&
$-7.18\pm1.44$&
$3.6\pm1.1$&
no\tabularnewline
USco 130&
2004-May-09&
35.7724090&
$-4.83\pm0.54$&
0.21&
&
&
\tabularnewline
&
2004-May-13&
39.8538830&
$-4.95\pm0.74$&
0.55&
$-4.89\pm0.92$&
$15.2\pm1.1$&
no\tabularnewline
USco 132&
2004-May-13&
39.8268683&
$-7.18\pm0.58$&
0.30&
&
&
\tabularnewline
&
2004-May-17&
43.6391138&
$-7.37\pm1.02$&
0.89&
$-7.28\pm1.17$&
$9.1\pm0.7$&
no\tabularnewline
\hline
\end{tabular}

\end{center}

\label{tab:summary_results}

\end{table*}

%==========================================

%==========================================

\begin{figure*}

\begin{center}

\includegraphics[%
  clip,
  scale=1.3]{figures/radvel_all01.eps}

\end{center}

\caption{Relative radial velocities (RVs) of objects measured in two
different epochs. The vertical axes indicate the amount of deviation
($\Delta \mathrm{RV}$) from the `average' radial velocitiy
($\overline{\mathrm{RV}}$) in Table~\ref{tab:summary_results}), and
the horizontal axes indicate the time of the observation in
heliocentric Julian date (HJD). The objects are considered to have a
non-constant RV when the error bars of two data points do not overlap
each other. The non-constant RV objects are considered as
binary/multiple candidates.}

\label{fig:radvel01}

\end{figure*}

%==========================================

\addtocounter{figure}{-1}

%==========================================

\begin{figure*}

\begin{center}

\includegraphics[%
  clip,
  scale=1.3]{figures/radvel_all02.eps}

\end{center}

\caption{continued}

\label{fig:radvel02}

\end{figure*}

%==========================================

The radial velocities of each object were determined by using the cross-correlation
function of the object spectrum with that of a template star which has
a spectra type. By visual inspection,
the wavelength ranges used for the cross-correlation calculations
are chosen by avoiding the regions of spectra affected by the telluric
lines, and defects and fringes (in near infrared) of the CCDs. The
radial velocities of objects with respect to the template are obtained
by measuring the location of the peak in the cross-correlation function
which is fitted by a function which consists of two gaussian functions
(with a common centre) plus a constant term . LHS~49 (Proxima Cen, M5.5) was
chosen as the template for this purpose. The radial velocity of the
template object LHS~49 was obtained by measuring the wavelength shifts
of the prominent photospheric absorption features \ion{K}{i}~$\lambda\lambda$7664.911,~7698.974.
This gives us $\mathrm{RV_{LHS49}=}-22.6\pm0.5\,\kmps$, which is
in good agreement with the earlier measurement of \citet{garcia-sanchez:2001}
who found $\mathrm{RV_{LHS49}=}-21.7\pm1.8\,\kmps$. The heliocentric
RV of each object can be then calculated by adding $\mathrm{RV_{LHS49}}$
with the RV of each object with respect to LHS~49. In the following
measurements of the heliocentric radial velocities, our measurement
($\mathrm{RV_{LHS49}=}-22.6\pm0.5\,\kmps$) will be used for consistency. 

Before applying the cross-correlation technique to out main targets,
we have applied the technique to the radial velocity standard HD~140538
for which an high-accuracy RV measurement via the fixed-configuration,
cross-dispersed �chelle spectrograph Elodie \citep{baranne:1996}
is available. This was done so to ensure not only the validity of
the cross-correlation technique, but also the validity of the wavelength
calibration. In this test, we found the heliocentric $\mathrm{RV_{HD\,140538}=}18.8\pm0.6\,\kmps$
which is in good agreement with the Elodie radial velocity measurement
of $19.00\pm0.05\,\mathrm{km\, s^{-1}}$ \citep{udry:1999}.

The result of the heliocentric RV measurements (from two epochs for
each object) is summarised in Table~\ref{tab:summary_results} along
with the uncertainties. The table also lists the uncertainties in
the `relative' radial velocities ($\sigma_{\mathrm{RRV}}$) which
is the uncertainty of RV with respect to the template star. The
two consecutive measurements of RVs (from same nights) are used to
find the (absolute/heliocentric) RV of the night, and their uncertainties are obtained from
the standard deviations of the mean ($\sigma_{\mathrm{RRV}}$) and
the uncertainty of the template RV value (c.f.~\citealt{joergens:2006}).
Note that for finding a shift in the radial velocities from two
different epochs, only $\sigma_{\mathrm{RRV}}$ is important since one
only requires relative (with respect to a template) RV values.
The 'average' radial velocities ($\overline{\mathrm{RV}}$) of the
two epochs are also given in the same table. The $\overline{\mathrm{RV}}$
values of the objects in UScoOB are consistent with the earlier measurements
of \citet{muzerolle:2003} in Table~\ref{tab:literatures}, except
for the values for USco~66 and USco~75. This is unsurprising
since the former is a known binary \citep{kraus:2005} and the latter
is a non-constant RV object (a candidate for a multiple) found in
the RV-shift test which will be discussed next. 

For each object and for each RV measurement, the deviations ($\Delta\mathrm{RV}$)
from the average RV are computed and summaries in Fig.~\ref{fig:radvel01}
along with their uncertainties ($\sigma_{\mathrm{RRV}}$) in order
to aid the identification of multiplicity candidates. Note that in
computing $\Delta\mathrm{RV}$ we do not require the knowledge of
absolute or the heliocentric radial velocities, but only the relative
velocities (with respect to a template). We identify an object as a candidate
when the 1$\sigma_{\mathrm{RRV}}$ error bar from each RV measurement
overlaps one another. Using this simple criteria, the candidate were
selected and indicated in the same table (Table~\ref{tab:summary_results}).
We found 9 out of 18 (50 per cent of) our sample are considered as
candidates. The recent observation of \citet{kraus:2005} (which was
not known to authors at the time of our observation: April--May, 2004)
confirms that USco~65 and USco~66 are multiple systems, and USco~109
is most likely a multiple system. In our RV-shift tests, we found
all the multiples confirmed by \citet{kraus:2005} in our multiple
candidates. Further they found USco~67, 128, 130, 132 as non-multiple,
and similarly we found them as 'non-candidates' for multiplicity.
The discrepancy between their results and ours are only in USco~75
and USco~112. While our results suggests that these are candidates
for multiple systems, \citet{kraus:2005} did not find them as multiple
systems. This difference may be caused by the difference in sensitivity
(on a binary separation) of the two different methods. Another possible
caused of the difference is that the uncertainties in RVs may be underestimated, since
surface activity of the objects may also contribute noise in
the RV measuements; however, the levels of th noise is expeted be
small for BD and VLM stars (c.f.~\citealt{joergens:2006}).

Finally, the histogram of the RVs using only the objects in UScoOB
is given in Figure~\ref{fig:vrad_vrot_histrogram}. The total number
of the objects is 14. Note that USco~121 is excluded from the graph
since it is identified as a non-member of the UScoOB association based
on the RV value (see Table~\ref{tab:summary_results}).  \citet{muzerolle:2003}
also found it to be a likely non-member based on the radial velocity
and the low lithium abundance. The distribution of the RVs in the
figure was fitted by a gaussian function. We found that the standard
deviation and the peak position of the radial velocity distribution
are $1.2\,\kmps$ and $-5.9\,\kmps$ respectively. The former is very
similar to the standard deviation (0.9~$\kmps$) of the radial velocity
distribution of 9 BDs and VLM objects in Cha~I found by \citet{joergens:2006b}.
They also studied the radial velocity distribution of more massive
25 T~Tauri stars in Cha~I, and found the standard deviations ($1.3\,\kmps$)
is not significantly different from that of the brown dwarfs and the
very low-mass objects. Unfortunately, we do not have the radial velocity
measurements of the higher mass counter parts (T~Tauri stars) in
Upper Sco OB association. This is also planned for a near future investigation
as this is important for the study of the mass dependency of the kinematics
in a young stellar cluster. 

According to the hydrodynamical simulations of a low-mass star-forming
cluster of \cite{bate:2003} which yields a stellar density of $1.8\times10^{3}\,\mathrm{stars\, pc^{-3}}$,
the rms dispersion (1-D) of the stars and the BDs is $1.2\,\kmps$.
Similarly for the model with a higher stellar density ($2.6\times10^{3}\,\mathrm{stars\, pc^{-3}}$),
the rms dispersion is 2.5~$\kmps$ (\citealt{bate:2005}). The standard
deviation of $\overline{\mathrm{RV}}$ ($1.2\,\kmps$) found in our
analysis is more comparable the lower stellar density model. 

%==========================================

\begin{figure*}

\begin{center}

\begin{tabular}{cc}
\includegraphics[%
  clip,
  scale=0.45]{figures/radvel_hist.eps}&
\includegraphics[%
  clip,
  scale=0.45]{figures/vrot_hist.eps}\tabularnewline
\end{tabular}

\end{center}

\caption{Histogram of the average heliocentric radial velocities (left) and the rotational velocities (right) of 14 UScoOB BD and VLM objects listed in Table~\ref{tab:summary_results} (excluding USco~121, a non-member). The gaussian fit (dashed line) of the radial velociity distribution gives a standard deviation of  $1.2\,\kmps$ and the peak position of $-5.9\,\kmps$.  The log-normal fit (dash-dot) of  the rotational velocity distribution gives  a standard deviation $27.8\,\kmps$ and the peak poistion $16.9\,\kmps$.}
\label{fig:vrad_vrot_histrogram}

\end{figure*}

%==========================================


\subsection{Rotational velocities}

\label{sub:Rotational-velocities}

The rotational velocities of the objects were determined by measuring
the widths of the cross correlation functions of the target spectra against
a template spectrum from an object which is known to have a very small
rotational velocity. The line broadening of the targets is assumed
to be dominated by rotational broadening. As in the cases for the
radial velocity measurements, LHS~49 is chosen as the template.
Using its rotational period ($P\approx83\,\mathrm{d}$, \citealt{benedict:1998})
and radius ($R_{*}\approx0.145\,\mathrm{R_{\odot}}$ from the VLTI
measurement by \citealt{segransan:2003}), the rotational velocity
of LHS~49 is estimated as $v\,\sin i=2\pi R_{*}/P\approx0.1\,\kmps$;
negligibly small. 

The width of the cross-correlation curves ($\sigma_{\mathrm{CCF}}$)
are calibrated with the rotational velocities ($v\,\sin i$) by cross
correlating the template spectra against the same template spectra
with added rotation (convolved with a given $v\,\sin i$), as
done by e.g.~\citet{tinney:1998}, \citet{mohanty:2003} and \citet{white:2003}.
A linear limb-darkening law with a solar-like parameter ($\epsilon=0.6$)
was assumed in the formulation of the rotational profile described
by \citet{gray:1992}, his Eq.~17.12. For each object, two measurements
of rotational velocities are computed from two independent spectra
obtained at different epochs. As for the RV measurements,
the mean and the standard deviation of the mean are used as the final
rotational velocity and its uncertainty. The final results are recorded
in Table~\ref{tab:summary_results}. In general, our measurements
are in good agreement with the earlier measurements of \citet{muzerolle:2003}
and \citet{mohanty:2005}, given in Table~\ref{tab:literatures}.
For example, \citet{muzerolle:2003} found $v\,\sin i=16.8\pm2.7\,\mathrm{km\, s^{-1}}$
for GY~5 while we found $v\,\sin i=16.5\pm0.6\,\mathrm{km\, s^{-1}}$. 

The range of $v\,\sin i$ found among our objects is $3.6$--$55.6\,\kmps$,
and a similar range is also found by \citet{mohanty:2005}. The left
panel of Figure~\ref{fig:vrad_vrot_histrogram} shows the histogram
of $v\,\sin i$ distribution for the UScoOB objects (14 objects excluding
USco 121, non member). The log-normal fit of this distribution gives
the peak position at $16.9\,\kmps$ with a standard deviation $\sigma=27.8\,\kmps$.
Using the $v\,\sin i$ data in \citet{joergens:2001}, the same histogram
bin size used for UScoOB obects and the log-normal fit, we find the
$v\,\sin i$ distribution of the BD and VLM stars (8 objects) in Cha~I
peaks at $15.4\,\kmps$, and has the standard deviation of $8.0\,\kmps$.
The peak of the distribution is similar to that of Upper Sco objects,
but the standard deviation of the distribution is significantly smaller
than that of the Upper Sco objects. The difference maybe due to the
very small sample. A similar fit was applied to the $v\,\sin i$
distribution of 14 T~Tauri stars in Cha~I using the data of \citet{joergens:2001},
and we found a peak at $17.0\,\kmps$ with a  standard deviation
$25.9\,\kmps$ which are very similar to those of the Upper Sco brown
dwarf candidate objects. 


\section{Binary frequency}

\label{sec:Binary-frequency}

According to Table~\ref{tab:summary_results}, 9 out of 18 (50 per
cent of) the sample are considered as binary/multiple candidates. The
fraction slightly decreases to 43 per cent (6 out of 14) if we consider
the objects only in the UScoOB. These fractions are significantly
larger than the binary fraction (25~per~cent, 3 out of 12) found
by the recent high-resolution imaging survey of brown dwarfs and very
low-mass objects (M5.5--M7.5 ) in the UScoOB \citep{kraus:2005}.
The imaging surveys are sensitive to wider binaries (separation$\gtrsim4\,\mathrm{au}$)
while our technique is more sensitive to narrower ($\lesssim1\,\mathrm{au}$)
or shorter period ($\sim10\,\mathrm{d}$) binaries. However, our RV-shift
test also detected all the binaries found in the imaging survey of
\citet{kraus:2005}, demonstrating that the method is also effective
in the systems with a separation 5--18~au. In Section~\ref{sub:Radial-velocities},
we have found all the binaries confirmed by \citet{kraus:2005} also
as our binary/multiple candidates. In addition to those samples commonly
found, we have two more candidates (USco~75 and USco~112) which
are in the sample of \citet{kraus:2005}, but not detected as binaries
by them. If we assume these objects are real binary/multiple systems
but not detected by the imaging survey because of possibly very small
separations, their multiplicity fraction would increase to 5 out of
9 (56~per~cent) which is more consistent with the fraction suggested
by our RV survey.

Another possible reason for the different binary fraction suggested
by our data is that the uncertainties in the RV values may be underestimated.
Since the our suggested binary fraction is merely that for the binary
`candidates', the follow-up RV monitoring of the candidates must be
done to confirm the true multiplicity, and to determined firmer constraint
on the binary/multiple faction in the UScoOB. This is planned for
a near future. The binary fraction of slightly more massive pre-main-sequence
stars was studied by \citet{kohler:2000} using the K-band speckle
interferometry of X-ray selected samples. For the UScoOB members with
spectral type M0--M5 (0.7--0.13~$\mathrm{M_{\odot}}$), they found
the binary fraction of $52\left(\pm10\right)$~per~cent (26 out
of 50) with the separation 21--1000~au. On the other hand, \citet{kohler:2000}
found no wide ($>20\,\mathrm{au}$) binary in their survey, and suggested
that there may be a discontinuity in separation distribution around
$M_{*}\sim0.1\,\mathrm{M_{\odot}}$. 

A similar imaging survey of the brown dwarf candidates in Cha~I cloud
by \citet{neuhaeuser:2002} and \citet{neuhaeuser:2003} also found
the multiplicity rate $\lesssim10\,\mathrm{per\, cent}$ using 12
objects (Cha~H$\alpha$~1--12 with M6--M8). Using a subset of the
same sample and the same radial velocity technique used in this paper,
\citet{joergens:2006} found only 11~per~cent (1 out of 9) of the
sample showed non-constant RVs in her multi-epoch data, indicating
a much smaller multiplicity fraction. It is unclear at this moment
whether the difference is caused by the physical nature of the two
different star forming regions.


\section{Conclusions}

\label{sec:Conclusions}

We have presented two-epoch RV survey of 18 young BDs and VLM stars
in UScoOB and $\rho$~Oph dark cloud core using the high resolution
UVES echelle spectroscopy at VLT.  We found 9 out of 18 objects (50~per~cent) are possible candidates
in this analysis. For the objects in UScoOB, we found 6 out of 14
objects (43~per~cent) as candidates. Compared to the average binary
fraction ($\sim20\,\mathrm{per\, cent}$) of magnitude-limited imaging
surveys for BDs and VLM stars (M6 and later) in the field (c.f.~\citealt{burgasser:2006}),
the fraction of the binary `candidates' found by this RV survey is
significantly higher. The possible reasons to for the discrepancy
are: (1) caused by the different sensitivity on the binary separation
in the imaging and RV technique, (2) caused by possibly under estimated
uncertainties in the measured RV values. The true multiplicity of
the candidates must be confirmed by follow-up spectroscopy monitoring
of candidates. 

We found a good agreement with the high resolution imaging survey
of the BD and VLM objects in UScoOB of \citet{kraus:2005}. All three
confirmed binaries in UScoOB by \citet{kraus:2005} are found as the
multiplicity candidate this RV survey; however, we found two objects
(USco~75 and USco~112) also as binary candidates while they found
them as singles.

From the distribution of RV values in UScoOB, we confirmed that USco~121
is most likely a non-member of the association, as similarly found
so by  \citet{muzerolle:2003}. We found the RV dispersion of the
objects in UScoOB is very similar to that of the BDs and VLM stars
in Chamaeleon~I (Cha~I) previously studied by others. The rotational
velocities ($v\,\sin i$) of the samples are also measured. The distribution
of $v\,\sin i$ for the UScoOB objects peaks around $16.9\,\kmps$
which is similar to that of the Cha~I population. 

The follow-up spectroscopic observations of the binary candidates
presented here are planned in near future. There are only a few RV
variable binary candidates are identified from the earlier surveys
(\citealt{guenther:2003}; \citealt{kenyon:2005}; \citealt{joergens:2006}).
As \citet{burgasser:2006} pointed out most of the current RV and
imaging surveys use samples from magnitude-limited survey, but one
should attempt to use the samples from volume-limited survey in order
to find a correct statistics on binary parameters more straight forwardly,
i.e. without correcting for bias.

\section*{Acknowledgements}

We thank the staff of VLT of the ESO for the observations carried out in
service mode.  This work is supported by PPARC rolling grant PP/C501609/1. 

%

%

%

%----------------------------------------------------

% Bibliography follows here.

%-----------------------------------------------------

%

%

%





\bibliographystyle{mn}

\bibliography{local}

%

\label{lastpage}
\end{document}
