% mn2esample.tex
%
% v2.1 released 22nd May 2002 (G. Hutton)
%
% The mnsample.tex file has been amended to highlight
% the proper use of LaTeX2e code with the class file
% and using natbib cross-referencing. These changes
% do not reflect the original paper by A. V. Raveendran.
%
% Previous versions of this sample document were
% compatible with the LaTeX 2.09 style file mn.sty'
% v1.2 released 5th September 1994 (M. Reed)
% v1.1 released 18th July 1994
% v1.0 released 28th January 1994

\documentclass[useAMS,usenatbib]{mn2e}

% If your system does not have the AMS fonts version 2.0 installed, then
% remove the useAMS option.  
% useAMS allows you to obtain upright Greek characters.
% e.g. \umu, \upi etc.  See the section on "Upright Greek characters" in
% this guide for further information.
%
% If you are using AMS 2.0 fonts, bold math letters/symbols are available
% at a larger range of sizes for NFSS release 1 and 2 (using \boldmath or
% preferably \bmath).
%
% The usenatbib command allows the use of Patrick Daly's natbib.sty for
% cross-referencing.
%
% If you wish to typeset the paper in Times font (if you do not have the
% PostScript Type 1 Computer Modern fonts you will need to do this to get
% smoother fonts in a PDF file) then uncomment the next line
% \usepackage{Times}
\usepackage{lscape}
\usepackage{graphicx}
\usepackage{subfigure}

%%%%% AUTHORS - PLACE YOUR OWN MACROS HERE %%%%%
\newcommand\mnras{MNRAS}
\newcommand\pasp {PASP}
\newcommand\aj   {AJ}
\newcommand\apj  {ApJ}
\newcommand\apjl  {ApJL}
\newcommand\apjs  {ApJS}
\newcommand\aap  {A\&A}
\newcommand\aaps  {A\&AS}
\newcommand\apss  {AP\&SS}
\newcommand\gca  {Geochim. Cosmochim. Acta}
\newcommand\araa {ARA\&A}
\newcommand\logmdot {$\log \dot{M}$}
\usepackage{multirow}

%%%%%%%%%%%%%%%%%%%%%%%%%%%%%%%%%%%%%%%%%%%%%%%%

\title[**]{**}  \author[N.J.  Mayne and Tim
Harries]{N.J. Mayne$^{1}$\thanks{E-mail: nathan@astro.ex.ac.uk
    (NJM)} and Tim Harries$^{1}$\\
  $^{1}$ School of Physics,
  University of Exeter, Stocker Road, Exeter, EX4 4QL.\\
}
 \begin{document}


\date{Accepted ?. Received ?; in
  original form ?}

\pagerange{\pageref{firstpage}--\pageref{lastpage}} \pubyear{2009}

\maketitle

\label{firstpage}

\begin{abstract}
To Do
\begin{enumerate}
\item{Read new papers and put in any relevant stuff.}
\item{Sort structure paragraph at end of intro.}
\item{Sort out headings of sections.}
\item{Sort clearpage problem, perhaps change to includegraphics.}
\item{Sort out the webpage stuff.}
\end{enumerate}
\end{abstract}

\begin{keywords}
  stars:evolution -- stars:formation -- stars: pre-main-sequence --
  techniques: photometric -- catalogues -- (stars) Hertzsprung-Russell
  H-R diagram
\end{keywords}


\section{Introduction}
\label{intro}

Theoretical study of Brown Dwarfs (BDs) and the subsequent development
of testable hypotheses for observational studies is a critical, and
somewhat uncertain, area within the field of star and planet
formation. Several of the key physical parameters and mechanisms
remain unclear within this mass regime \citep{2007ApJ...664.1154S}.
Uncertainty remains over fundamental issues such as the initial
formation mechanisms, the subsequent evolution and dominance of a
star-disc interaction through to the possible dependence of planet
formation mechanisms on stellar mass \citep[for a review of BD
formation mechanisms see][]{2006astro.ph..2367W}.

Perhaps the two most fundamental challenges are formulating the
formation mechanism of BDs and deriving the resulting
initial-mass-function (IMF). Whether the BD IMF or formation
mechanisms are different to their higher mass, solar-type, equivalents
(T Tauri Stars, hereafter termed TTS) is an open question. Derivations
of reliable IMFs depend, obviously, on robust derivations of stellar
mass, and more subtly, on derivations of stellar ages (as IMFs are
usually defined for assumed coeval populations). The study of
classical T Tauri (CTTS) and disc systems of
\cite{1997AJ....114..288M} found that given an expected range of
parameters the scatter in infrared photometry matched the observed
CTTS locus well. However the work of \cite{1997AJ....114..288M},
discussed in more detail later, only included parametrised `flat'
discs. Once disc flaring is considered it is clear from previous work
that misidentification of high mass stars with circumstellar discs is
likely. \cite{2004ApJ...617.1177W} show that high-mass (HM) stars with
circumstellar discs are likely to be mis-identified as younger Class I
sources. This problem also exists at lower masses where the presence
of circumstellar discs can dramatically shift the spectral energy
distributions (SEDs) and therefore colours and magnitudes of BDs
\citep{2004MNRAS.351..607W}. Therefore, deriving robust masses (and
ages), and subsequently IMFs for BDs requires an understanding of
these effects of disc presence.

The presence, or fraction (for populations) of associated discs is
itself also crucial in isolating the formation mechanisms of
BDs. Precise determination of disc parameters for BDs and comparison
with the TTS results may allow the differentiation of currently
competing theories of the BD formation mechanism. Essentially, a
mechanism similar to that seen in higher mass stars i.e. cloud
collapse and fragmentation will result in a similar fraction of discs
across the BD mass boundary, whereas a formation mechanism dominated
by dynamical interactions \citep[ejections, accretion disc stripping
from near collisions or tidal interactions etc.]{2003IAUS..211...27B}
will result in a markedly different disc parameters, such as fractions
of stars with discs or radial sizes of the discs themselves, across
the BD mass boundary. Fundamentally a different disc fraction for BD
stars to TTS would provide persuasive evidence of a different
formation mechanism across the two mass regimes, as would an increase
in the average radial size of discs, indicative of truncation for BD
objects. Therefore, to progress our understanding of the fundamental
issues of both the formation mechanism and emerging IMFs for BDs,
consideration of the effects of an associated circumstellar disc and
their subsequent ubiquity are essential.

Another fundamental, and important, unresolved issue is the mechanism
of planet formation. A critical, and currently ill defined constraint
for the planet formation models is the timescale available for planet
formation within the circumstellar disc. Current favoured formation
mechanisms such as core accretion or gravitational instability are
separated by these timescales. Disc fractions have again been used, in
conjunction with stellar ages, to derive the disc dissipation
timescale, which obviously constrains the planet formation
timescale. Studies of TTS have derived a disc dissipation timescale of
$\approx$ 5 Myrs \citep{2001ApJ...553L.153H}, with very few discs
shown to survive out to $\approx$ 20 Myrs
\citep{2004ApJ...612..496M}. However, the derivation of a disc
dissipation timescale from disc fractions suffers from several
intrinsic problems, for instance the inconsistent (and often
contradictory) methods of deriving ages \citep[see][for a full
discussion]{2008MNRAS.386..261M}. Notwithstanding this, disc fraction
surveys have also been published for lower mass and BD systems, such
as \cite{2004A&A...427..245S} and \cite{2004ApJ...609L..33M}, deriving
a similar order-of-magnitude disc dissipation timescale. The
probability of forming planets around BDs is clearly sensitive to any
changes in disc dissipation timescale across the TTS to BD mass
boundary.

Some studies of disc lifetimes find no significant difference between
BD stars and their higher mass counterparts
\citep{2003AJ....126.1515J}, however a possible correlation of disc
lifetime with stellar mass is proposed in \cite{2007ApJ...660.1517S}.
Additionally, \cite{2009arXiv0901.2603K} find that, for TTS older than
3 Myrs, the disc dispersal time is mass dependent for intermediate
mass (IM) to solar-type stars. This result only holds for the mass
transition stated (IM to solar-type) and is based on stars with
combined dust signatures, from infrared (IR) excesses and accretion
indicators (using a EW[H$\alpha$], criterion).
\cite{2009arXiv0901.4120D} also find, for TTS, a mass dependent
general SED shape for disc candidates, with higher mass stars showing
a featureless continuum at $\lambda>8.0$\,$\mu$m and lower mass stars
generally showing strong silicate and PAH features at 10 and 2\,$\mu$m,
indicative of mass dependent disc evolution. If the disc lifetimes and
therefore rate or type of involved physical processes are a function
of stellar mass. This could provide another piece of evidence
separating the previously assumed continuous (in mass) evolutionary
principles of stars. Resulting in a dichotomy separated by the
hydrogen-burning mass limit. This would also lead to a change in the
probability of forming planets in the BD regime, compared to CTTS.

To test the hypothesis that disc dissipation timescales are
independent of stellar mass one again requires accurate masses for BD
with associated discs (hereafter, BDD) systems, alongside reliable
disc fractions. The derivation of masses for BDD systems is
complicated by disc presence, but additionally, the detection of disc
presence at a given age is not simply a function of the disc
dissipation timescale.

Several mechanisms may act to completely dissipate a circumsteller
disc. These include accretion from the disc ($\delta M_{\rm
  disc}=\dot{M}$), photoevaporation or photoionisation due to flux
from the central and neighbouring stars, and the formation of giant
planets or interaction with binary partners. However, the entire disc
does not have to be dissipated to be missed by disc fraction surveys.
More subtly, disc fractions can be altered due to wavelength dependent
detection. This is due to the peak flux of the thermal blackbody
emission of the disc being proportional to the temperature,
$\lambda^{\rm thermal}_{\rm peak}\propto T_{\rm disc}$, and the
temperature in the disc being approximately a function of distance
from the star. Thus for a given system, as one increases the
wavelength of observation, the corresponding radial position of the
effective source will increase. Moreover, as the disc flux must be
separated from the photospheric flux the important quantity is the
separation in wavelength between the emission peaks for the stellar
and thermal disc components. Therefore the presence of an inner disc
hole can change derived disc fractions, as a function of
wavelength. Interestingly, \cite{2009arXiv0901.2603K} find that
results for disc dissipation timescales based only on their accretion
indicators or inferred disc presence do not agree, perhaps due to
inclusion of transitional objects, where the inner disc is being
cleared, and is therefore not detected at their target
wavelength. \cite{2009arXiv0901.2603K} suggest that this is caused by
photoevaporation of the inner disc, leading to a mass dependent disc
lifetime, at their wavelength, for their sample. However, as discussed
in \cite{2009arXiv0901.4120D}, the inner hole can be created via dust
grain growth and mid plane settling, giant planet formation or
photoevaporation by ultraviolet (UV) and far-UV (FUV) flux
\citep{2007MNRAS.378..369N}.  \cite{2009arXiv0901.4120D} argue that if
binarity or giant planet formation or photoevaporation cause inner
disc clearing the resulting system would not show active accretion
from radii below the inner disc boundary. For the case of clearing by
dust grain growth and settling accretion from a hot gas disc within
the inner radius is still possible, and indeed evidence for such a hot
gas inner disc has been found through SED fitting
\citep{2005ApJ...623..952E,2008ApJ...689..513T}. Additionally,
\cite{2009arXiv0901.4445E} argue that for M stars, with a pure
reprocessing disc where the inner disc radius ($R_{\rm inner}$) is
equal to the dust sublimation radius ($R_{\rm dust}$, discussed in
Section \ref{physics}), the stellar and thermal components are not
separable at wavelengths shorter than 6\,$\mu$m. In the case of
\cite{2009arXiv0901.4445E}, the inner disc hole was created by
photoionising radiation from the central star destroying the dust
component of the disc. Recent direct evidence for a photoevaporating
disc has been found for the circumstellar disc around a low mass TTS
in the Sigma Orionis cluster \citep{2009arXiv0902.0457R}. The
detection was characterised by a tenuous photoevaporation wind and
little or no accretion, and a disc detection at 24\,$\mu$m indicative of
a large inner hole. Additionally, for BDs, \cite{2009arXiv0902.0638M}
find that the disc mass and radius distributions in the Orion
Trapezium cluster are consistent with photoevaporation over 1 Myr (the
adopted age of the cluster), from UV and FUV flux generated by the
massive $\theta^1$ Ori C.

Whichever mechanism is responsible for clearing the inner disc hole it
is clear that its presence leads to confusion in the validity of disc
fractions at a given wavelength. The important quantity here is the
separation in wavelength between the emission peaks for the stellar
and thermal disc components, as it is these components which must be
separated. For BDD systems it is likely that difficulties detecting
the disc will be exacerbated by the lower temperatures of the
photosphere and therefore the smaller separation in wavelength from
the thermal disc emission component, compared to TTS
systems. Therefore, to derive disc fractions for BD populations we
require detailed comparison models with which to guide disc candidate
selection. These models must necessarily include a treatment of
accretion and dust destruction, given that these factors deplete the
inner (and total) disc thereby affecting disc detection.

Although the creation of the inner disc hole could have several
causes, at least part of the effect is undoubtedly due to
accretion. Accretion acts to directly deplete the inner disc (where
the material is accreted from). Accretion also depletes the inner disc
indirectly, as accretion generates hot continuum emission
\citep{1998ASPC..154.1709G}, increasing the photoionising UV and FUV
flux available \citep{1998ASPC..154.1709G}, and therefore affecting
the derived disc fraction \citep{2009arXiv0901.3684H}. This effect may
be amplified for BDD systems where the combined photospheric emission
is dominated by accretion flux. However, accretion columns modeled by
\cite{2004MNRAS.348..879A} were found to produce negligible levels of
photoionising flux due in part to the high optical depth of the column
itself to Lyman continuum photons. Additionally, due to the high
concentrations of hydrogen in the accretion column and hotspot the
emission is more likely to be characterised by a stellar atmosphere
than a black body. However, for our work as emission from a stellar
atmosphere or a black body of equal temperature are approximately
equal for wavelengths longer than the Lyman edge we have neglected
this effect. Whilst the accretion column and hotspot may not
contribute to the photoinising flux within the system they will of
course alter the disc evolution by an increase of the emitted
flux. This increase in flux will lead to an increase in temperature of
the disc inner and outer regions, and more particularly a change in the
dust sublimation effects at the inner edge.

The current proposed mechanism of accretion is a magnetically
channelled accretion `column', where material from a truncated (at the
co-rotation radius $R_{\rm co}$) inner disc boundary falls onto the
surface of the star along magnetic field lines
\citep{1990RvMA....3..234C,1991ApJ...370L..39K}. This mechanism is
widely accepted, with strong evidence found from the excessive widths
of H$\alpha$ ($>$10\,\AA) emission lines, over that caused by
chromospheric activity and stellar rotation
\citep{1998ASPC..154.1793W}, caused by the high velocities of
infalling matter, $~$100-200 kms$^{-1}$ \citep{1998ApJ...492..743M}.

Accretion rates can be derived by modelling this magnetospheric
accretion with a characteristic `column' temperature and assuming
statistical equilibrium of the state populations. The resulting line
ratios can then be fitted to the observed spectra. Accretion rates can
also be calculated from model comparison of the Equivalent Width (EW)
of the H$\alpha$ line itself, or perhaps more reliably, as they are
less affected by extinction, that of the hydrogen recombination lines
of Pa$\beta$ and Br$\gamma$. Accretion rates have been derived using
these methods for many TTS and more recently for BDs
\citep{2003ApJ...592..266M,2004A&A...424..603N,2006A&A...452..245N},
with accretion rates in the range $\dot{M}_{\rm
    acc}=$10$^{-6}$--10$^{-12} M_{\odot}yr^{-1}$, and typical
accretion rates in the range \logmdot = log$(\frac{\dot{M}}{M_{\odot}}
yr^{-1})=-10$ to $-8$ with only FU Orionis type objects accreting at
rates higher than this. The mechanism producing accretion rates as
high as those observed episodically in FU Orionis type objects is
likely to be produced by some accretion mechanism instability, or
perhaps by encounters with other cluster members \citep[see][for
discussion]{2008arXiv0810.2854P}. Additionally, much lower accretion
rates of $\dot{M}\approx$ 10$^{-13}M_{\odot}yr^{-1}$ have been derived
for BDs using flux blueward of 4000\,\AA where the photospheric flux
becomes negligible \citep{2009arXiv0901.3684H}.

Analysis of the derived accretion rate data has led several authors to
propose a strong correlation between accretion rate and stellar mass
\citep{2003ApJ...592..266M,2004A&A...424..603N,2006A&A...452..245N},
of an approximate form: $\dot{M_{\rm acc}}\propto M_{*}^{~2}$. This
correlation has no clear physical basis and it has been suggested that
it could be caused solely by selection biases and detection thresholds
\citep{2006MNRAS.370L..10C}. \cite{2006ApJ...648..484H} points out
that Bondi-Hoyle accretion naturally results in a clear
$\dot{M}\propto M^2_*$ relationship, but is not a desirable formalism
to adopt as it completely ignores angular momentum of the accreting
material. This proposed correlation could have important implications
with regards to disc lifetimes and planet formation probabilities as a
function of mass.

As discussed, discs associated with BD systems lead to difficulties in
accurate classification of the underlying BD
\citep{2004MNRAS.351..607W}. Furthermore, accretion, both direct and
indirect, can act to alter the disc signature, and clearly the
photospheric signature. For cooler BD systems, where the stellar flux
is generally redward of any accretion flux, increased accretion fluxes
could dominate the resulting SED. Given that the derivation of a
robust relationship of the form $\dot{M}\propto M^{\alpha}_*$ requires
accurate mass and accretion derivations, then clearly the effects of
circumstellar disc and increased accretion rates must be considered,
before any such relationship can be accepted. As there is currently no
clear theory linking accretion rate and mass (ignoring Bondi-Hoyle
accretion), this relationship is driven by observations.  However, the
observational constraints must be interpreted using a self-consistent
set of accreting BDD and BD systems to better understand their
implications. In other words, what do the observations really tell us?
Is there a real relationship between mass and accretion rate or would
simulated systems outside this relationship (and attributed scatter)
not be detected or included in the current observational surveys, due
to misclassification?

A further use of disc fractions, in conjunction with stellar ages, is
to aid the understanding of the angular momentum evolution of star and
disc systems. Recent evidence from TTS, from correlations of stars
with IR excesses and slower rotational rates
\citep{2007ApJ...671..605C} has seemed to strongly support a mechanism
of disc-locking, where the central star is rotationally braked through
magnetic locking to the accretion disc
\citep{1993AJ....106..372E,1990RvMA....3..234C,1991ApJ...370L..39K}. A
similar mechanism has been put forward, but has yet to be
unambiguously confirmed, for BD stars. However, current studies have
found a much weaker correlation between the slower rotators and those
where a disc presence is inferred for BDs. Some authors have therefore
proposed an `imperfect' disc-locking mechanism for stars in the lower
mass regime ($M_*<0.4M_{\odot}$, inclusive of the BD mass regime,
$M_*<0.08M_{\odot}$) \citep{2005A&A...430.1005L}. This weaker
correlation could also be due to `imperfect' disc-locking or
observational effects and mass dependence of the disc dissipation
timescale \citep{2004A&A...419..249S,2005A&A...429.1007S}.

If disc-locking is present for BDD systems, as for TTS stars, one
would expect accretion of material from a magnetic truncation radius,
$R_{\rm T}$ \citep{2007prpl.conf..479B}. \cite{2007prpl.conf..479B}
show that the magnetic truncation radius, $R_{\rm T}$, must be less
than the co-rotation radius, $R_{\rm co}$ (where the Keplerian angular
rotation rate of the disc is the same as the angular rotation rate of
the star), as material outside this radius will have too much angular
momentum to fall onto the star. This is implicitly adopted in the
proposed accretion mechanism, where the disc is magnetically
truncated, and therefore locked to the star. This magnetic truncation
would lead to an inner wall (and inner hole) of the accretion disc.
Evidence for an inner disc wall has been found from near-IR excesses
from the inner rims of Herbig Ae stars, with emission peaking at
2--3\,$\mu$m \citep{2001ApJ...560..957D}. The actual shape and radial
position of this wall is affected by several factors, discussed
further in Sections \ref{dust_edge} and \ref{inner_edge}. The previous
work of \cite{1997AJ....114..288M} explored the effects of simple mass
accretion and the radius to the inner disc on the infrared colours of
CTTS, using a prescribed range of inner disc radii.
\cite{1997AJ....114..288M} find that the scatter in infrared colours
for there parameter range, $R_{\rm inner}=$1$-$12$R_*$ and
$\dot{M_{\rm acc}}=$10$^{-9}$to 10$^{-5} M_{\odot}yr^{-1}$, matches
the scatter of the CTTS locus found in the data. The models of
\cite{1997AJ....114..288M} used physically flat discs however meaning
their results will not stand for discs with significant flaring. More
importantly for our work \cite{1997AJ....114..288M} find a correlation
of the infrared locus of models with the accretion and inner disc
location. Essentially, changes in the stellar flux levels and position
of the inner wall lead to temperature changes of this inner edge and
therefore changes in the balance of flux across the infrared
wavelength regime. As this study does not include the effects of
vertical hydrostatic equilibrium or even the study of flared discs one
would expect, perhaps, a more complex relationship between the inner
edge position and infrared excess in `real' systems. This means that
studies aiming to provide evidence of disc locking by comparing disc
indicators with rotation rates may include unforeseen biases. As the
rotation rate will effect the co-rotation radius and therefore
truncate the disc at different radii, thereby enhancing or suppressing
any relationship of infrared excess (or disc presence) with rotation
rate.

In summary, several key areas in the theoretical studies of BDs hinge
critically on observational constraints, where the derived values are
heavily affected by disc and/or accretion presence. Formation
mechanisms, planet formation, IMFs, disc dissipation timescales,
accretion rate--stellar mass relationships and studies of disc-locking
mechanisms using correlations of rotation rates with infrared
excesses, all require derivations of either masses, ages or disc
fractions (or a combination), supplemented by other parameters. The
consequences of the disc structure for BD accretion discs,
particularly vertical structure, given the assumption of vertical
hydrostatic equilibrium, have already been discussed for a limited
range of BD parameters in \cite{2004MNRAS.351..607W}. Indeed
\cite{2004MNRAS.351..607W} have shown that BDD systems seen at higher
inclinations can be confused with CTTS systems, having similar colours
and magnitudes. A benchmark model grid of spectra for star and disc
systems in the mass range 0.1$M_{\odot}<M_*<$50$M_{\odot}$ has
previously been published in \cite{2006ApJS..167..256R}.

In this work we explore the effects of accretion and disc presence on
the derivation of the stellar parameters and explore the intrinsic
biases, or relationships, which will effect or inhibit conclusions on
mass to accretion rate or rotation to infrared excess correlations. We
have extended the similar work of \cite{2004MNRAS.351..607W} by
inclusion of a simple treatment of accretion and a sophisticated
implementation of dust sublimation (amongst other improvements). The
complete model grid, with derived photometry and isochrones, is
available online through our browsing tool
\footnote{http://www.astro.ex.ac.uk/research/bd\textunderscore disc}
which is described in Appendix \ref{website}. The aim of this online
database is to complement that of \cite{2006ApJS..167..256R}. Finally,
subsequent papers will introduce expansion to the model grid and an
online fitting tool.

**THIS PARAGRAPH IS CURRENTLY WRONG, WILL REQRITE AT THE END**

The structure of this paper is as follows. In Section \ref{model} we
outline the model used, including an explanation of our physical
assumptions (Section \ref{physics}), a brief description of the
radiative transfer code (Section \ref{torus}), and an explanation of
the derivation of emergent quantities such as photometric magnitudes
(Section \ref{derived}). The ranges of the input variables are then
explained and, where possible, justified using observational
constraints in Section \ref{par_space}, with the total parameter space
covered shown in Table \ref{par_space_table}. Section
\ref{disc_struct} then contains analysis and description of the
resulting temperature and dust density structures of the discs. The
consequences of these changes in disc structure on the SEDs of these
objects is then discussed in Section \ref{sed_analysis}. Section
\ref{photometry} then details problems caused for the derivation of
the primary parameters age, mass and disc fractions from photometry
and the resulting relations built on these parameters; accretion rate
to stellar mass and rotational period to disc presence. Finally,
conclusions are drawn, with a summary, in Section \ref{conclusions}..

\section{Model}
\label{model}

In this Section we detail the physical model adopted and assumptions
made (Section \ref{physics}), then explain key elements of the
radiative transfer code (Section \ref{torus}). Then we discuss the
derived values, such as broadband photometric magnitudes and colours
(in Section \ref{derived}). Some internal consistency checks are
detail in Appendix \ref{consistency}.

\subsection{Physical model and assumptions}
\label{physics}

\subsubsection{Photospheric flux}
\label{phot_flux}

To model an accreting BDD system we must first model the underlying
photospheric flux. To provide a self-consistent set of model spectra
we have adopted a BD stellar interior and atmospheric model grid. We
have then constructed the total photospheric flux for any input value
of stellar age and mass by interpolating for surface gravity (log(g)),
effective temperature ($T_{\rm eff}$), radius ($R_*/R_{\odot}$) and
luminosity ($L_*/L_{\odot}$). These values were then used to
interpolate atmospheric spectra for flux (ergs s$^{-1}$cm$^{-2}
{\rm \AA}^{-1}$) from 1200 to 2$\times$ 10$^7$ \AA (or 0.12 to 2000\,
$\mu$m). These spectra were then smoothed using 200 logarithmically
spaced points. Careful inspection ensured that no spectral features
were removed during smoothing. The stellar interior models used for
this study are the `DUSTY00' models of \cite{2000ApJ...542..464C}
combined with the `AMES-Dusty', atmospheric models of
\cite{2000ApJ...542..464C}, which are all available online
\footnote{http://perso.ens-lyon.fr/france.allard/}. For our
approximate $T_{\rm eff}$ (effective temperature) range of $\approx
$3000\,K$<T_{\rm eff}<$1600\,K the AMES-Dusty atmospheres are most
applicable (2700\,K$>T_{\rm eff}>$1700\,K). We did try including dynamic
application of atmospheres based on the derived $T_{\rm eff}$,
i.e. using AMES-Cond for $T_{\rm eff}<$1700\,K, however this resulted in
large discontinuities between the model atmospheres and resulting
spectra. As this only affects stars at the very edge of our
temperature range, i.e. for the oldest and lowest mass objects (for
the AMES-Cond case), we have adopted the same model throughout.

\subsubsection{Accretion flux}
\label{acc_flux}

Secondly, we require a model of the accretion flux for the system. The
flux due to accretion was added onto the intrinsic photospheric flux
using the simple approximation of blackbody emission. The selected
accretion rate was used to derive an accretion luminosity ($L_{\rm acc}$),
where the material was modelled as falling from the disc inner edge
onto the surface of the star. $L_{\rm acc}$ is calculated according to
\ref{Lacc}, 
\begin{equation}
{L_{\rm acc}=\frac{GM_*\dot{M}}{R_*} \left( 1-\frac{R_*}{R_{\rm inner}}\right)},
\label{Lacc}
\end{equation}
where $M_*$ is the stellar mass, $\dot{M}$ the mass accretion rate,
$R_*$ the stellar radius and $R_{\rm inner}$ the radius of the disc inner
boundary.

The initial inner disc radius was set to be the co-rotation radius
(this is discussed in more detail in Section
\ref{disc_parameters}). During the radiative transfer simulations of
the disc the final inner dust disc radius may be extended past the
co-rotation radius. This is due to dust sublimation effects (see
Sections \ref{dust_edge} and \ref{inner_edge} for an
explanation). However, as the gas within the disc is not sublimated it
will always extend down to the co-rotation radius and we assume
accretion will still occur from this radius. A full discussion of the
differences between the initial and resulting final radii of the dust
disc can be found in Section \ref{inner_edge}. Once the accretion
luminosity was derived, an adopted areal coverage ($A$), over the
stellar surface, was used to calculate an effective temperature
($T_{\rm acc}$), for the accretion `hot' spot, where
\begin{equation}
{T_{\rm acc}=\left(\frac{L_{\rm acc}}{4\pi R^2_{*} \sigma A}\right)^{\frac{1}{4}}}.
\label{Tacc}
\end{equation}

Finally, a blackbody flux distribution is generated at $T_{\rm acc}$
and added onto the intrinsic stellar photospheric flux. In general one
would expect this to be an overestimate of the accretion flux, as for
pre-MS BDs large convective zones are expected on the stellar surface,
and some of the accretion energy may act to further drive these
convective currents, meaning flux is lost. It is worth noting however
that observationally UV excesses are often used to recreate and then
subtract an assumed accretion flux using a blackbody flux curve,
essentially the reverse of this method.

\subsubsection{Disc parameters}
\label{disc_parameters}

In this study we assume that accretion from the central star occurs
along magnetically channelled columns from the inner disc boundary.
For TTS stars, \cite{2007prpl.conf..479B} show that the magnetic
truncation radius ($R_{\rm mag}$) is less than the co-rotation radius
($R_{\rm co}$), where the angular Keplerian velocity of the
disc is equal to the surface angular velocity of the central star.
Calculations of the magnetic truncation radius depend on derivations
of the surface magnetic field \citep{1991ApJ...370L..39K}. This is
currently unavailable for BD stars, due to increased molecular species
obscuring the Zeeman splitting signatures, that are normally used to
derive stellar surface magnetic fields. Therefore, for our model grid
we have adopted an initial inner disc radius as the co-rotation
radius,
\begin{equation}
{R_{\rm inner}=\left({GM_*\tau^2\over{4\pi ^2}}\right)^{1\over{3}}},
\label{inner_eq}
\end{equation}
where $\tau$ is the stellar rotation period and $R_{\rm inner}$ is the
inner radius. This is effectively adopting a disc-locking mechanism
(without associated angular momentum loss), as for disc-locked stars,
$R_{\rm mag}\approx R_{\rm co}$,
\citep{1991ApJ...370L..39K,1994ApJ...429..781S}. Therefore, for our
model simulations, this inner edge radius is dependent on, and derived
from, the adopted value of the rotational period for the central star,
as well as being weakly dependent on the stellar mass. As discussed in
Section \ref{intro}, the inner disc can be cleared through a number of
mechanisms, including binarity or Giant planet formation,
photoevaporation or photoionisation of the disc and dust grain growth
or settling. For BDD systems where a disc is modeled a treatment of
dust sublimation is included (discussed in Section \ref{dust_edge}).
However, the effects of binarity or Giant planet formation are
neglected. Further to the stellar mass and period required prior to
calculation of the inner disc radius, we require a disc mass (in
stellar masses).

For this work the disc outer edge was set at 300 AU, this was chosen
as a maximum size of the circumstellar
disc. \cite{2008A&A...486..877B} have shown that the disc outer radius
has little effect on the resulting SED. However, in our subsequent
paper we will include models for outer radii of 100 AU and plan to
extend this parameter range further to smaller values of the outer
radius in the future.

\subsubsection{naked and disc systems}
\label{naked_BDD}

The combined (accretion plus photosphere) SED is then used as a
boundary condition for the {\sc torus} radiative transfer code and
as a benchmark set of SEDs to model `naked' BD systems. The set of
`naked' photospheres (plus accretion) are diluted by the factor
$(R_*/{\rm distance})^2$ to a distance of 10pc. The `negligibly' accreting,
`naked' stars can be used to produce absolute magnitude (and intrinsic
colour) isochrones for comparison. Whilst, the remainder are used to
model systems showing active accretion where no disc is detected. For
instance, \cite{2009arXiv0901.2603K} find 43 stars within their sample
are actively accreting whilst no disc is detected (out of a total
sample of 1253).

In summary the key required input variables to setup the model grid
are as follows. Stellar age and mass, which are used to derive the
stellar flux. Further supplemented by an accretion rate, areal
coverage and rotation period, used to generate the the accretion flux.
The inner disc radius is then derived using the rotation rate, but a
disc mass fraction must be adopted.


\subsection{Radiative transfer code: {\sc torus}}
\label{torus}

In this section we briefly explain the key elements of the radiative
transfer code used to model the BDD systems.

We have used the {\sc torus} radiative transfer code which is
described in \cite{2000MNRAS.315..722H}, including the subsequent
refinements i.e.  addition of Adaptive-Mesh-Refinement (AMR)
introduced in \cite{2004MNRAS.350..565H}. {\sc torus} uses the
criterion of \cite{1999A&A...344..282L} in a radiative equilibrium
algorithm. The simulation also calculates \textit{vertical hydrostatic
  equilibrium} and dust sublimation for the disc, similar to that
described in \cite{2007ApJ...661..374T}.

\subsubsection{Dust Size distribution}
\label{dust_size}
 
We have adopted a similar treatment to \cite{2002ApJ...564..887W}.
With the size distribution of dust particle given by,
\begin{equation}
\label{particle_dist}
n(a)da=C_ia^{-q}\times exp^{[-(a/a_c)^p]}da
\end{equation}.
Where $n(a)da$ is the number of particles of size $a$ (within the
increment $da$), $a_{\rm c}$ is the characteristic particle size, with $p$
and $q$ simply used to control the shape of the distribution. $C_i$
controls the relative abundance of each constituent species ($i$) in
the dust. \cite{2002ApJ...564..887W} found that simulated SEDs fit
observed SEDs better using this adjusted size distribution for the
dust particles, as opposed to a simple power law. The best fitting
values found in \cite{2002ApJ...564..887W} were $q=$3.0, $p=$0.6 with
$a_{\rm c}=$50\,$\mu$m, also with an associated maximum and minimum grain size
of $a_{\rm min}=$5 nm and $a_{\rm max}=$1mm. Subsequent works have adopted a
slightly steeper power law dependence, with $q=$3.5
\citep{2008A&A...486..877B,2008ApJ...676L.143M,2008arXiv0810.2552P}.
Therefore we have adopted the values of \cite{2002ApJ...564..887W}
except for the parameter $q$ where we have used $q=$3.5.
\cite{2002ApJ...564..887W} calculated the $C_i$ values for amorphous
Carbon and silicon by requiring the dust to deplete a solar abundance
of either component completely \citep[using abundances
from][]{1989GeCoA..53..197A,1993ASPC...40..410N}. We have set $C_i=$1
and adjusted the species using a grain fractional abundance \cite[an
equivalent process to that of][]{2002ApJ...564..887W}, however we have
adjusted these grain fractions using the updated solar abundances of
\cite{2006CoAst.147...76A}. The resulting difference in opacity
between grain fractions matching the work of
\cite{2002ApJ...564..887W} and the new grain fractions is only a
slight enhancement of the silicate feature (due to the relative
abundance of Silicon increasing) which has little effect on the
resulting SEDs. Figure \ref{albedo} shows the
resulting albedo, and scattering and absorption opacities, for our
dust population, with a the vertical dashed line showing 10\,$\mu$m.

\begin{figure}
  \vspace*{174pt}
  \special{psfile="Fig/albedo.ps"
   hoffset=235 voffset=0 hscale=30 vscale=30 angle=90}
 \caption{Figures of the albedo (\textit{top panel}), and scattering
   (dashed line) and absorption (solid line) opacities (\textit{bottom
     panel}) against log($\lambda$) (in $\mu$m) for our adopted dust
   population. For \textit{both panels} the vertical dashed line is
   plotted at 10\,$\mu$m to highlight the silicate features.
   \label{albedo}}
\end{figure}

\subsubsection{Dust sublimation and the Inner Disc Edge}
\label{dust_edge}

We have also applied dust sublimation during our simulations. This
treatment is similar to that detailed in \cite{2007ApJ...661..374T},
but improved and generalised with respect to the applicable
geometries.

The practical implementation is as follows: An initial coarse
approximate temperature structure is found for the grid without dust,
only a fixed gas distribution. Then, first three full \textit{Lucy}
algorithm iterations are run for a grid with a maximum cell opacity of
0.1 (with dust added to this level). Three more iterations are then
performed with dust opacity limited to $\tau=$1 per cell. A further
three \textit{Lucy} algorithm iterations are then performed with the
full dust opacity, where dust is added until the gas-to-dust ratio is
achieved($\epsilon =$100). For each iteration (from the first to the
ninth) dust in cells which reach the sublimation temperature and
density is removed from the disc by setting the opacity to zero for
the cell. Once these nine iterations are complete, the grid is refined
around the disc photosphere, and the radiative transfer algorithm is
iterated through to convergence. After this first convergence, a
vertical hydrostatic equilibrium algorithm is run, after which the
disc grid is unrefined and then re-refined around the disc
photosphere. This process continues for five hydrostatic equilibrium
calculations. Finally, once the hydrostatic equilibrium iterations are
complete we perform a final dust sublimation run to remove any cells
where the dust temperature has moved above the sublimation temperature
and further refine the photosphere and inner edge of the disc (whilst
also unrefining cells where the opacity has reduced significantly and
high resolution is no longer required). In this model the dust
sublimation is effectively digital i.e. when the density and
temperature for sublimation is reached for a given cell the dust is
destroyed. We did attempt a more realistic model where the dust
sublimation causes the dust to be destroyed at an evanescently
decaying rate into the particular region. This was expected to lead to
a more realistic situation where the dust destruction propagates into
the disc material, and does not instantly destroy an entire
cell. However, this implementation lead to incremental sections of
dust being removed after each iteration, resulting in little change in
the overall density structure of the disc. This is perhaps interesting
for real systems, in that the dust sublimation should not be expected
to act as a digital `switch'.

As our models are magnetically truncated at the co-rotation radius, we
have implicitly generated an inner hole. This also means that the disc
will have an inner wall at this radius. Evidence for inner walls in
circumstellar discs is apparent from the SEDs of disc systems, where a
peak in emission is found between 2 and 3\,$\mu$m. The temperatures
reached by such inner walls are expected to generate thermal flux
contributions within this wavelength range
\citep{2001ApJ...560..957D}. In fact the structure, radius and
generation mechanism of the inner wall will lead to markedly different
contributions to the SED. As discussed in Section \ref{intro} several
mechanisms can act to clear the inner disc.

Binarity and Giant planet formation would be expected to clear an
inner disc and shut down accretion from within the inner wall due to
gravitational perturbations. In addition, photoevaporation from UV and
FUV flux is expected to halt accretion from radii within the inner
disc due to the removal of this material via a stellar/disc wind
\citep{2006MNRAS.369..229A}. These two clearing mechanisms could be
differentiated by the resulting masses of the discs
\citep{2007MNRAS.378..369N}. However, for our model systems we have
neglected both of these effects.

For our models we have initially cleared the inner disc by invoking a
magnetic truncation radius, however, further disc clearing will be
caused by the dust destruction/sublimation mechanism we have included.
In some cases where the radius in the disc at which the dust
sublimation temperature and densities are reached, ($R_{\rm sub}$)
will be less than the co-rotation radius ($R_{\rm co}$). For these
cases, $R_{\rm sub}<R_{\rm co}$, the inner wall radius and shape will
be controlled by sublimation. Effectively meaning the final inner disc
radius, after the radiative transfer simulation, will not equal the
initial inner disc radius. The differences in the starting and
resulting inner wall radii and the shapes of the inner edges are
discussed in Section \ref{inner_edge}. If the inner disc is cleared
via dust sublimation one would expect a residual hot gas disc interior
to the inner wall. This is as the gas component will not be destroyed
and is free to extend to the co-rotation radius.  Evidence for a hot
gas inner disc component has been found through SED fitting by
\cite{2005ApJ...623..952E} and \cite{2008ApJ...689..513T}.  In
particular, \cite{2008ApJ...689..513T}, found that emission apparent
from within the inner wall radius could not be well fit by a modeled
spherical halo, or optically thin disc, dust component.

Furthermore, as the dust sublimation limit is temperature and density
dependent \citep{1994ApJ...421..615P}, dust sublimation can lead to
changes in shape of the inner wall. As the density of the
circumstellar disc will vary with scale height one would expect the
dust sublimation radius to vary concurrently with scale height. This
results in a curved inner wall \citep{2001A&A...371..186N}. As
explained by \cite{2007ApJ...661..374T}, the fact that near-IR
excesses of disc systems do not appear strongly dependent on
inclination indicates that the inner walls are not
vertical. \cite{2007ApJ...661..374T} also showed that grain growth and
midplane settling will lead to a curved inner wall (and a hot inner
gas disc). The curvature is caused by changes in the vertical dust
density distributions as a function of dust size, or midplane
settling. Essentially, larger grains will sink more efficiently
towards the midplane of the disc, and due to their larger surface
areas cool more efficiently. Therefore, as the scaleheight increases
the relative proportion of larger to smaller grains will reduce and
the cooling become less efficient. This will result in higher
temperatures being reached for larger scale heights at a fixed radius,
thereby curving the inner wall as the upper layers of the disc reach
the sublimation temperature at closer radii.
\cite{2007ApJ...661..374T}, modeled this effect by using two dust
distributions, a small and a large, with different vertical
distributions indicative of increased midplane settling for the larger
grains. For this study we have adopted a size distribution for one
population of dust grains, with one characteristic vertical
distribution. As such curvature of the inner wall in our simulations
will be dominated simply by the dust sublimation temperature, and
density. The resulting shapes and radial positions of the inner walls
after the radiative transfer simulations are complete are discussed in
detail in Section \ref{inner_edge}.

For our model grid we have adopted a gas-to-dust ratio, $\epsilon$, of
100, and the gas population is assumed essentially static with a zero
optical depth. Therefore, it does not contribute to the resulting SED,
and we have not included a hot inner gas disc as found by
\cite{2008ApJ...689..513T}.

\subsubsection{SEDs}
\label{seds}

Once the radiative transfer code was completed simulated SEDs were
generated. These SEDs can be generated for any distance and for any
system-observer inclination. For our models we have set the distance,
to 10pc to create absolute flux SEDs and selected ten inclinations (0,
27, 39, 48, 56, 64, 71, 77, 84 and 90$^{\circ}$). A further useful
feature of the {\sc torus} code is that the emitted photons which make
up the SED are tagged on their way to the observer. These tags
separate the photons into four groups. Firstly, photons are separated
by source into thermal (disc) or stellar groups. These groups are then
subdivided into photons which reach the observer either directly or
after scattering (for example see Figure \ref{disc_eg_173}). The
resulting SEDs are discussed in Section \ref{seds}.

\subsection{Photometric systems and derived quantities}
\label{derived}

To allow us to investigate the effects of disc presence and accretion
luminosity on the derivation of parameters for BD stars we can first
study the produced SEDs. However, many observational studies use
non-spectroscopic data to derive the pertinent parameters. Therefore,
to examine the practical effects in the `observational plane' we have
used the SEDs to produce broadband photometric magnitudes, and
subsequently colours. Broadband magnitudes were also derived in a
large range of other filter sets not used explicitly in the analysis
within this paper. These magnitudes are available online
\footnote{http://www.astro.ex.ac.uk/research/bd\textunderscore disc}
and are briefly discussed in Appendix \ref{website}. In addition,
monochromatic fluxes have been derived for all filters, and again are
available online and discussed in Appendix \ref{website}. The
additional magnitudes and monochromatic fluxes are also discussed in
detail in the subsequent paper introducing the online fitting tool.

To derive broadband photometric magnitudes and colours the SEDs of
either the disc or naked systems were folded through the filter
responses of the required photometric system. As the fluxes in all
cases are absolute, derived for an observer to object distance of
10pc, no conversion is required to derive absolute magnitudes and
therefore intrinsic colours.

For the naked systems the flux is spherically symmetric however,
clearly, for the BDD systems different inclinations will result in
very different SEDs and therefore magnitudes and colours. Therefore,
for the BDD systems, photometric magnitudes and colours must be
derived for each inclination selected.

We have integrated using photon counting and calibrated using a Vega
spectrum for magnitudes in the optical and near-IR regimes. The filter
bandpasses selected are, the optical system of
\cite{1998A&A...333..231B} for \textit{UBVRI} and the \textit{CIT}
system of \cite{1982AJ.....87.1029E,2004PASP..116....9S} for
\textit{JHK}, with the required shift of $-0.015$\,$\mu$m as prescribed by
\cite{2004PASP..116....9S}.

We have also derived magnitudes in the mid-IR range using the IRAC and
MIPS systems of the \textit{Spitzer} space telescope. These magnitudes
were derived using a conversion of flux to Data Number (DN) and
calibrated using zero points derived from the zero magnitude fluxes
published in the IRAC handbook
\footnote{http://ssc.spitzer.caltech.edu/documents/som/som8.0.irac.pdf}
and the MIPS instrument calibration website
\footnote{http://ssc.spitzer.caltech.edu/mips/calib/}.

These specific filter sets have been chosen due to their ubiquitous
use and suitability for the derivation of the key stellar parameters
of age, mass and, for populations, disc fractions. Therefore, by
studying the changes of magnitudes (and colours) in these photometric
systems we can test the predicted effects on these derived parameters
caused by changes in the input parameters of our model grid. The
optical magnitudes \textit{VI} are used to explore age related effects
of varying the parameter space. Flux in the \textit{VI} bands is only
minimally affected by accretion flux \citep{1998ASPC..154.1709G} and
disc thermal flux \citep{1998apsf.book.....H} and, additionally, in a
\textit{V, V-I} CMD, the reddening vector lies parallel to the pre-MS,
minimising any age effect of extinction uncertainty, \citep[for a full
discussion see][]{2008MNRAS.386..261M,2007MNRAS.375.1220M}. The
near-IR passbands of \textit{JHK} are most often used to derive
stellar masses as for pre-MS objects the reddening vector, in for
instance, a \textit{J, J-K} CMD, is almost perpendicular to the
isochrones, minimising the mass effect of extinction uncertainties
\citep[the \textit{CIT} systems was chosen to
match][]{2000ApJ...542..464C}. Finally, as is now well documented the
\textit{Spitzer} IRAC passbands provide the best data with which to
unambiguously separate naked and star-disc systems
\citep{2005ApJ...620L..51L}. In addition, at longer wavelengths, the
MIPS instrument provides sensitivity to disc systems at much larger
radii (or debris discs).

Once the model grid was completed several checks were performed to
verify the consistency of our results. For each individual model these
checks were passed to our satisfaction before publication. Some
problems remain, and these are explained in Appendix
\ref{consistency}.

\section{Parameter space}
\label{par_space}

This section details the range of each of the input parameters we have
varied and, where possible, gives justification for the selected
ranges from published observations. The simulations in this paper
cover variations in the stellar mass, stellar age, stellar rotation
rate, accretion rate, the areal coverage of the accretion stream, disc
mass fraction and the system inclination. In a subsequent paper we
introduce variation of the outer radius and using an analytical
vertical disc structure as oppose to vertical hydrostatic equilibrium.
A summary of the values adopted for each input variable is shown in
Table \ref{par_space_table}.

\subsection{Mass} 
\label{par_mass}

Representative masses within the BD regime were chosen as follows:
0.01, 0.02, 0.03, 0.04, 0.05, 0.06, 0.07 \& 0.08 $M_{\odot}$.

\subsection{Age}
\label{par_age}

Typical disc lifetimes for solar type stars are of order 10 Myrs
\citep{2001ApJ...553L.153H}. Therefore, we have adopted input ages of
1 and 10 Myrs for our model grid, to span the approximate range of
ages over which the discs influence will be important.

\subsection{Rotation rate}
\label{par_rotation}

Data for rotation rates, from periodic variability surveys, are widely
available for a range of different age clusters of TTS. However, fewer
studies exist on the rotation rates of BD mass objects. To select a
feasible range of periods for our model grid we have collated the
results from some recent studies of BD variability. Rotation period
data for $\sigma$ Ori, at an age of $\approx$ 3 Myrs
\citep{2008MNRAS.386..261M}, is studied in \cite{2004A&A...419..249S},
where periods are found over the range 5.78$-$74.4 hours ($\approx$
0.24$-$3.1 days) for BD mass objects. In some cases the periodic
variability is irregular and assumed to come from active accretion hot
spots on the BD surface \citep[see][for discussion of variability
causes]{1995A&A...299...89B,2007prpl.conf..297H}. Infrared excesses
and evidence of accretion are also found to correlate with slower
rotating TTS, which the authors construe as evidence for a disc
locking scenario. \cite{2004A&A...419..249S} also find a $mass\propto
period$ relationship extending into the BD mass regime, this faster
rotation in younger mass objects is taken as evidence for a decreasing
effectiveness of the angular momentum removal of the disc locking
mechanism, i.e. imperfect disc locking. In addition,
\cite{2004A&A...419..249S} argue that when compared with data from
younger clusters disc lifetimes can be seen to decrease with central
object mass.

\cite{2005A&A...429.1007S} study rotation period data for stars in the
vicinity of $\epsilon$ Ori, with an assumed age of $\approx$3 Myrs
\citep{2002A&A...384..937Z} and the ONC at an age of $\approx$2 Myrs
\citep{2008MNRAS.386..261M}. Rotation periods, from photometric
variability, in the range 4.7$-$87.6 hours ($\approx$ 0.2$-$3.65 days)
for BD mass stars are found. Again, \cite{2005A&A...429.1007S} find
some BD objects with irregular variability indicative of accretion hot
spots on the stellar surface. \cite{2005A&A...429.1007S} then analyse
the mass-period relationship of these data in comparison with data
from $\sigma$ Ori \citep[3 Myr,][]{2004A&A...419..249S}, NGC2264
\citep[2 Myr,][]{2004A&A...417..557L} and then ONC \citep[1
Myr][]{2002A&A...396..513H}. They conclude that either the $\epsilon$
Ori sample has an age of around 2$-$3 Myrs or there is still
significant disc locking (concluded using the slope of the mass-period
relation). More specifically for the BD mass regime, they find
\citep[in agreement with][]{2004A&A...419..249S} an age independent BD
lower period limit of 2$-$4 hours, therefore at any age a fraction of
BD stars rotate at short and constant period. Given that these objects
should spin up as they hydrostatically contract they invoke rotational
braking in the BD mass regime as a mechanism of removing significant
angular momentum.

\cite{2003ApJ...594..971J} study the rotational periods of BD (and
very low mass stars) in the Chameleon I region. This region is $\le$ 1
Myrs old, and the authors find rotation periods of 2.19, 3.376 and
3.21 days for their BD mass counterparts. Using supplementary data
from the literature \cite{2003ApJ...594..971J} propose that a shorter
lifetime of $\approx$5 Myrs for accretions discs around BDs is more
probable and is inferred from the derivation of a shorter locking
timescale. However, as discussed in the previous studies, an imperfect
disc locking mechanisms is also hypothesised as responsible for the
less significant loss in angular momentum out to ages of 10 Myrs. The
data mentioned above and other datasets are summarised and discussed
in \cite{2007prpl.conf..297H}, where a strong argument for the
mass-period relationship can be found, shown in their Fig 6, alongside
further support for either a shorter disc lifetime or imperfect disc
locking mechanism for BD stars.

Therefore, in summary, it is currently unclear if any disc locking
mechanisms efficiency or disc lifetimes are a function of mass, with
an associated phase change over the TTS to BD mass limit. Rotation
rate and disc presence data are being used to explore issue.
Therefore, to create a set of useful models to help contextualise the
observational constraints we must adopt a realistic, and bounding,
range of rotation rates. For our model grid, and associated age range
($<$10 Myrs), we have selected 0.5 and 5 days.  The lower limit is
placed at around the approximate median for the faster rotating
samples, providing a useful benchmark for exploring possible detection
limits of faster rotators with discs. The longer period is set at the
approximate edge of the slow rotating tail, providing a limiting case
for the slower rotating, disc-locked, candidates.

\subsection{Areal Coverage}
\label{par_cov}

As discussed in Section \ref{par_rotation} evidence for irregular
periodic variability has been found in BD stars with detections of
associated stellar discs. This is construed as evidence for accretion
hot spots formed as magnetically channeled material hits the stellar
surface \citep[see discussion
in][]{1995A&A...299...89B,2007prpl.conf..297H}. The irregularity is
thought to be caused by changes in the magnetospheric structure and
accretion rate \citep{1995A&A...299...89B}. For our model we have
assumed that disc material is disrupted at the co-rotation radius and
channeled onto the star in the form of accretion hot spots with a
characteristic temperature. Therefore, to calculate the characteristic
temperature and the resulting blackbody accretion flux we must adopt
an accretion rate and areal coverage of the accretion stream.

Little observational evidence can be found for approximate sizes of
accretion hot spots due to their more transient nature and often
smaller coverages, when compared to cooler or `plage' spots
\citep{2007prpl.conf..297H}. \cite{1995A&A...299...89B} modeled the
size of the cool spots on solar-type stars for a selection of
periodically variable candidates. They found projections of cooler
spots, onto the stellar disc, of a few to $~$60\%.
\cite{1995A&A...299...89B} also found projected sizes, onto the
stellar disc, of typically a few \% to around 10\% for hot spots.
\cite{1996AJ....112.2159B} used observations of YY Orionis monitoring
flux amplitude variations as a function of wavelength to derive a
probable hot spot area of around 10\%. The spot temperature was also
modeled for YY Orionis in \cite{1996AJ....112.2159B}, resulting in a
best fitting areal coverage of 11\%. Therefore, to bound the probable
areal coverage range of the accretion hot spots we have adopted areal
coverages of 1 and 10\%.

\subsection{Accretion Rate}
\label{par_accn}

Accretion rates derived for pre-MS stars are of order \logmdot = $-6$
to $-11$ \citep{2006A&A...452..245N}.  With the largest accretion
rates found in so-called FU Orionis type objects (after the object
where they were first observed). For the more typical accretion rates
\citep[$\dot{M}=$10$^{-11}$ to 10$^{-8.9} M_{\odot}yr^{-1}$, for
  TTS,][]{2009arXiv0901.4120D}, several studies have now suggested
that the accretion rate is strongly correlated with the mass of the
central star. This relationship was perhaps first suggested by
\cite{2003ApJ...592..266M} using various accretion diagnostics. Later,
\citep{2005ApJ...625..906M} derived a relationship of approximately
$\dot{M}\propto M_*^{~2}$. Further evidence was put forward by
\cite{2004A&A...424..603N}, where accretion rates as low as 5$\times$
10$^{-12}M_{\odot} yr^{-1}$ were found for BD stars. More recently,
even lower accretion rates of $\approx$10$^{-13}M_{\odot} yr^{-1}$,
have been derived for BDs by \cite{2009arXiv0901.3684H}.  Further
support for a dependence of accretion rate on stellar mass was
apparent in the significantly more homogeneous dataset of
\cite{2006A&A...452..245N}. \cite{2006A&A...452..245N} analysed a set
of accretion rates and masses derived for pre-MS stars in $\rho$
Ophiuchi and compared these results to stars in Taurus. They found
that the accretion rate scales with central object mass, into the BD
regime. However, work by \cite{2006MNRAS.370L..10C} suggests that much
of this result is likely to be caused by biases and sensitivity
limits.

For any $\dot{M}\propto M_*$ relationship to be ratified, clearly, the
associated mass derivations must be shown to be accurate. For the BD
mass regime it has been demonstrated that classification, either
spectroscopic or photometric, and therefore accurate mass derivation
is exceedingly difficult for BDD systems
\citep{2004MNRAS.351..607W}. The inclusion of significant levels of
accretion flux will act to exacerbate this problem. Essentially, as BD
mass objects have relatively cool photospheres, compared to the
temperature of the accretion hotspots, meaning that for lower
accretion rates, than for their CTTS counterparsts, the accretion flux
dominates the small wavelength regime of the SED.

Therefore, to strengthen, or reject, any stellar mass to accretion
rate relationship, the limits of observational classification (and
mass derivation) of accreting BDD systems must be understood. To
provide a benchmark model grid of such systems, with which to aid
interpretation of the observational data, we must select a range of
accretion rates including and exceeding those detected for systems
previously classified as BDs. 

As the relationship $\dot{M}\propto M_*^{~2}$ predicts lower accretion
rates for BD mass objects it is essential that we model systems at
higher accretion rates, which may have been missed in current
observational studies. Therefore, we have adopted accretion rates of
\logmdot = $-6$, $-7$, $-8$, $-9$, $-10$, $-11$ \& $-12$.

\subsection{Disc Mass}
\label{par_mdisc}

A further required input parameter is the mass of the circumstellar
disc. Previously studies modeling BD discs have adopted a range of
disc mass fractions, for instance \citep{2004MNRAS.351..607W} use 0.1,
0.01 and 0.001$M_*$. \cite{2002ApJ...564..887W} fit observed spectra
with modeled SEDs to derive a disc mass of 0.003$M_*$, for HH 30
IRS. Subsequent derivations of disc masses have converged to within an
order of magnitude, with the following specific results: 0.03$M_*$
\citep[$\rho$ Ophiuchi,][]{2002A&A...393..597N}, 0.055$M_*$ \citep[GM
Aurigae,][]{2003MNRAS.342...79R}, 0.03$M_*$ \citep[GY 5, GY 11, and GY
310,][]{2004ApJ...609L..33M} and 0.022$M_*$\citep[2MASS
J04442713+2512164,][]{2008A&A...486..877B}. As the derived disc masses
all have a similar order of magnitude we have adopted $M_{\rm
  disc}\approx $0.01$M_*$. As changes in disc masses are expected to
change the resulting SED less than perhaps, accretion rate for
example, we have not varied the disc mass for this study. The results
of simulations varying this parameter will be published in a future
paper.

\subsection{Inclination}
\label{par_inc}

Discs around BD stars exhibit increased flaring, due to the reduced
surface gravity in the disc \citep{2004MNRAS.351..607W}. This
increased flaring, and therefore larger scaleheight of the disc
results in a smaller opening angle, when compared to higher mass stars
and their circumstellar discs. As has been shown in
\cite{2004MNRAS.351..607W} effects caused by variations in the system
inclination angle are much more significant for BDD systems, again
compared to their higher mass analogues. Therefore, we have simulated
ten observer to system inclination angles spaced evenly in
$cos(inclination)$ space, namely, 0, 27, 39, 48, 56, 64, 71, 77, 84
and 90$^{\circ}$.

A final list of all varied parameters and their values can be seen
in Table \ref{par_space_table}.

\begin{table*}
\begin{tabular}{|l|l|}
Input parameter&Values (\# values)\\
\hline
Mass ($M_{\odot}$)&0.01, 0.02, 0.03, 0.04, 0.05, 0.06, 0.07 \& 0.08 (8)\\
Age (Myr)&1 \& 10 (2)\\
Rotation period (days)&0.5 \& 5 (2)\\
Areal coverage (of $\dot{M}$, \%)&1 \& 10 (2)\\
Accretion rate (log$(\frac{\dot{M}}{M_{\odot}} yr^{-1})$)&$-6$, $-7$, $-8$, $-9$, $-10$,
$-11$ \& $-12$ (7)\\
Disc mass ($M_*$)&0.01 (1)\\
Inclination ($^{\circ}$)&0, 27, 39, 48, 56, 64, 71, 77, 84 \& 90
(10)\\
\end{tabular}
\caption{List of all varied input parameters. Resulting in a total
  number of models of 448 (plus 40 models without radiative transfer
  simulations for the naked BDs) and 4480 SEDs (plus 40 for naked
  BDs). \label{par_space_table}}
\end{table*}

\section{Result and Analysis}
\label{results}

Uncertainty remains over the formation mechanisms, resulting IMFs and
subsequent interaction mechanisms with associated circumstellar disc,
for BDs. Furthermore, no current theoretical explanation can be
demonstrated for the suspected stellar mass-accretion rate
relationship \citep[$\dot{M}\propto
M_*^2$,][]{2003ApJ...592..266M,2004A&A...424..603N,2006A&A...452..245N}
derived from observations. In addition doubt remains over the
existence of a disc-locking mechanism in BDD systems, with predictions
of a correlation of rotation rate with infrared excess possibly
providing an intrinsic bias \citep{1997AJ....114..288M}. Theoretical
models are built upon constraints, such as ages, masses and disc
fractions, and correlations, such as the accretion rate to stellar
mass relation and rotation rate to period relation (disc-locking),
derived from the data. Critically, these resulting constraints and
relationships involve both the application of existing models, such as
isochrones to derive ages, and the selection of bias-minimised
samples, for instance removal of foreground or background contaminant
stars. Clearly, if one is to adopt a new theoretical model or use an
observed correlation as a constraint one must explore the suspected
intrinsic biases and the reliability of the selection criteria.

For our model grid we have selected a range of input parameters
covering expected, from observations, ranges. We have input the
fundamental parameters of age and mass by adoption of a set of
theoretical isochrones. We have intrinsically included a correlation
of rotation rate with initial inner edge position by setting this at
the co-rotation radius, which is expected to result in a correlation
of rotation rate with IR excess \citep{1997AJ....114..288M} (the
reverse of that expected due to angular momentum loss through
disc-locking). We have not assumed a relationship between accretion
rate and stellar mass for our model grid. Therefore, analysis of our
model grid in an attempt to recover the fundamental parameters of age
and mass, and show accretion rate to stellar mass or rotation rate to
infrared excess relationships, will allow us to assess the reliability
of these current theoretical models or relationships.

In this Section we first discuss the physical structure, both density
and temperature, of the BDD disc systems (Section \ref{disc_struct})
across our parameter space. Then we discuss the resulting simulated
SEDs for these BDD systems (Section \ref{sed_analysis}). Finally, we
produce simulated photometric observations (Section \ref{photometry}).
These simulated observations allows us to show that recovery of the
input masses and ages for our accreting BDD systems is unreliable. We
show that despite an intrinsic relationship between rotation rate and
initial inner edge position we do not recover a rotation rate
correlation with IR excess. In addition, despite not intrinsically
including a dependence of accretion rate on stellar mass in our grid.
We show that current observational techniques and theoretical models
applied to the grid would result in a relationship of this type being
derived.

\subsection{Disc Structure}
\label{disc_struct}

The initial density structure in our BDD systems is setup as described
in \cite{2004MNRAS.351..607W} with the the disc surface density
conserved, $\Sigma ^{\beta - \alpha}$. The initial scaleheight,
$h\propto r^{\beta}$, where r is the radial distance from the star,
and the density, $\rho \propto r^{-\alpha}$. The initial values of
$\alpha$ and $\beta$ were 2.1 and 1.1 respectively \citep[see][for
full discussion of disc structure]{2004MNRAS.351..607W}. The values
for $\alpha$ and $\beta$ were chosen to optimise resolution of the
vertically evolving disc, but minor variations are largely
inconsequential as the systems evolves from this state. In our models
we have, however, placed at the inner disc edge at the co-rotation
radius, as opposed to the dust destruction radius used in
\cite{2004MNRAS.351..607W}. The initial disc scaleheight at 100 AU,
$h(100)$, was set to 25 AU. As the simulation used vertical
hydrostatic equilibrium and dust sublimation, both the disc
scaleheight and inner edge location then evolved in the systems
dependent on the input parameters. In this section we discuss the
structure of the discs in terms of these two generated
characteristics, i.e the disc scaleheight and inner edge location.

\subsubsection{Disc Flaring}
\label{disc_flaring}

As discussed in \cite{2004MNRAS.351..607W} discs around CTTS stars
have scaleheights at 100 AU of between $h$(100)$=$7 to 20 AU, whereas
for BDD systems, $h$(100)=20 to 60 AU (for 0.08 and 0.01 $M_{\odot}$
respectively). Our models adopt the same formalism for solving
hydrostatic equilibrium and therefore, at negligible accretion rates,
our disc scaleheights are comparable. As the accretion rate increases
the flux levels of the central star increase and lead to heating of
the disc. This in turn leads to vertical expansion in the disc. For
our models we found that levels of vertical flaring increased
marginally with accretion rate. Significant differences, more than
$>$5 AU increase in $h(50)$, in the vertical structure were not
apparent until the high accretion rates of \logmdot = $-7$ and
$-6$. Figures \ref{flare_-12_183} and \ref{flare_-7_193} show the
density structure ($log({\rho})$) in the disc from radial distances of
0 to 50 AU for example systems ($M_*=0.04M_{\odot}$, Age=1 Myrs,
$\tau$=5 and areal coverage=10\%), with accretion rates of \logmdot =
$-12$ and $-7$, respectively.

\begin{figure}
  \vspace*{174pt}
  \special{psfile="Fig/flare_-12_183.ps"
   hoffset=10 voffset=-40 hscale=33 vscale=33 angle=0}
 \caption{Figure showing the density structure ($log({\rho}$)) of the
   BDD system with $M_*=0.04M_{\odot}$, Age=1 Myrs, $\tau$=5, areal
   coverage=10\% and accretion rate of \logmdot=$-12$.
   \label{flare_-12_183}}
\end{figure}

\begin{figure}
  \vspace*{174pt}
  \special{psfile="Fig/flare_-7_193.ps"
  hoffset=10 voffset=-40 hscale=33 vscale=33 angle=0}
\caption{Figure showing the density structure ($log({\rho}$)) of the
  BDD system with $M_*=0.04M_{\odot}$, Age=1 Myrs, $\tau$=5, areal
  coverage=10\% and accretion rate of \logmdot=$-7$.
   \label{flare_-7_193}}
\end{figure}

Figure \ref{flare_-12_183} shows that for a typical BDD system the
resulting disc is highly flared and therefore, the opening angle for
stellar radiation is small. As discussed the flaring for BDD systems
was found to be larger than CTTS systems by
\cite{2004MNRAS.351..607W}. \cite{2004MNRAS.351..607W}, state that the
degree of disc flaring depends on the disc temperature structure and
the mass of the central star, with the disc scaleheight $h\propto
\frac{T_{\rm disc}}{M_*}^{1/2}$ \citep{1973A&A....24..337S}. Recently,
\cite{2009arXiv0901.4445E} suggest the inverse relation of flaring
with stellar mass, i.e. $h\propto M_*$. This suggestion was based on
evidence from \cite{2006ApJ...644..364A}, where SEDs for 17 systems in
the mass range 6$M_{\rm Jup}<M_*<$350$M_{\rm Jup}$ were fit with
flared or flat disc models. In general, \cite{2006ApJ...644..364A}
find that lower mass objects achieve better fits with the flat disc
models and higher mass objects with the flared discs.

The results of \cite{2006ApJ...644..364A} show that above a mass of
50$M_{\rm Jup}$ all objects (6/17) are better fit with flared
discs. Whilst at masses below 50$M_{\rm Jup}$ only one object is better
fit by the flared disc model, with the remaining objects (10/17)
better fit with flat models. Whether, this result is statistically
significant enough to assert a $h\propto M_*$ is doubtful as the
fitting process contains, presumably two fixed scaleheight
distributions. Therefore, for our study we continue to assume that our
flared BDD systems will have larger characteristic scaleheights than
typical CTTS systems.

A quick comparison of Figures \ref{flare_-12_183} and
\ref{flare_-7_193} show an increase in the scaleheight at 50 AU of
$>$5 AU, as the accretion rate moves from \logmdot =$-12$ to
$-7$. However, despite this small change with high levels of accretion
our grid shows scaleheights comparable to the work of
\cite{2004MNRAS.351..607W} and as such result in similar consequences
for the SEDs and photometric magnitudes. The effects of this flaring
and the increase in flaring for very high accretion levels on SEDs and
photometric magnitudes are discussed in Sections \ref{sed_analysis}
and \ref{photometry}, respectively.

\subsubsection{Inner Edge}
\label{inner_edge}

Our models include a sophisticated treatment of dust sublimation
(described in Section \ref{dust_edge}). In contrast to the models of
\cite{2004MNRAS.351..607W}, who apply a dust destruction radius, we
set the initial inner radius of the disc at the co-rotation radius of
the system. However, as our models then sublimate dust during the
temperature convergence routine this inner edge location may change.
As the flux levels of the central star increase with increasing
accretion rates the flux incident on the inner edge increases and
leads to increasing erosion of the inner edge of the dust disc.

As the inner edge moves its temperature is expected to change, this
has been predicted to lead to a correlation of inner edge position and
IR excess \citep{1997AJ....114..288M}. This may act to bias surveys
correlating rotation rates with IR excess to search for evidence of
disc-locking. However, the flux from the inner edge will usually peak
between 2 and 3\,$\mu$m \citep{2001ApJ...560..957D}, given that the dust
sublimation temperature peaks at $\approx$1400\,K for canonical
densities. This means that for models where dust is sublimated the
inner edge will have maximum temperatures of $\approx$1400\,K and this
correlation of disc position and temperature will be lost.

Equation \ref{inner_eq} shows that as the rotation period, of the
star, increases the co-rotation radius decreases. This will result in,
initially, shorter period systems having closer and hotter inner edges
than their longer period counterparts. In addition, if the accretion
rate is increased in these systems the incident flux on the inner wall
will increase leading to a rise in the temperature of the inner edge.
At some point, in both cases the dust sublimation temperature may be
reached, leading to a change in radial position of the wall. In
addition, the temperature of the inner edge will then tend to a
maximum temperature, of the dust sublimation temperature. Figure
\ref{lum_sub} shows the total stellar luminosity (accretion luminosity
plus the intrinsic stellar luminosity), in log($L_*/L_{\odot}$)
against the change in inner disc location, $R_{\rm co}-R_{\rm inner}$
in units of AU (\textit{top panel}) and $R_*$ \textit{bottom panel}.
This change in inner edge location is from the initially set
co-rotation radius to the final radius after dust sublimation, if
applicable.

\begin{figure}
  \vspace*{174pt}
  \special{psfile="Fig/lum_sub.ps"
   hoffset=235 voffset=0 hscale=30 vscale=30 angle=90}
 \caption{Figure showing the combined stellar luminosity (accretion
   plus intrinsic stellar luminosity) in units of
   log($L_*/L_{\odot}$) against the change in inner edge location
   shown in units of AU (\textit{top panel}) and $R_*$ (\textit{bottom
     panel}).
   \label{lum_sub}}
\end{figure}

The \textit{top panel} of Figure \ref{lum_sub} shows that as the
luminosity increases the total change in the inner radius increases as
more of the disc is sublimated. The total change for lower
luminosities at lower than $~$0.2 log($L_*/L_{\odot}$) is small (less
than $~$0.02 AU). As the luminosities increases to larger, and extreme
values, due to increasing accretion, the dust sublimation
increases. In the \textit{bottom panel} the change in inner radius
tends towards around $~$120 $R_*$. Increases in accretion rate, for
extreme accretors, dominates the increase in stellar luminosity. As
pre-MS stars age they contract. Therefore, for lower accretion rates,
younger, and therefore larger stars will have greater luminosities
than the equivalent older stars. Therefore, the major contributors to
the X-axis of Figure \ref{lum_sub} are controlled by the input
parameters of accretion rate and age. The initial inner wall radius
was set as the magnetic co-rotation radius (see Section
\ref{physics}). This distance is a strong function of rotational
period, with $R_{\rm inner}\propto \tau^{2/3}$ (see Equation
\ref{inner_eq}) which is in turn a function of stellar radius. The
stellar radius itself is then a function of the stellar
age. Therefore, the initial inner wall edge position will change as a
function of the input variables of rotational period and age. The
initial position of the inner edge will control the amount of the
stellar flux intercepted by the inner edge wall. Therefore the amount
that the disc will be sublimated will also be a function of the
initial position of the inner edge.

Figure \ref{delta_radius} shows the initial inner edge position, or
the co-rotation radius ($R_*$) against the change in inner edge
location ( $R_{\rm inner}-R_{\rm co}$, $R_*$). The \textit{panels} of
Figure \ref{delta_radius} then shows these data for two accretion
levels and at the two ages and rotation periods. These input variables
for our model grid are the underlying variables controlling, as
discussed, the dust sublimation process. The \textit{top panels} are
for models with accretion rates of -9, -10, -11 and -12
log$(\frac{\dot{M}}{M_{\odot}}yr{-1})$, classed as typical accretors
here and throughout the rest of this work. The typical accretors are
then separated further into three groups in the \textit{top left
  panel} specifically, -9, -10 and -11 \& -12
log$(\frac{\dot{M}}{M_{\odot}}yr^{-1})$ (blue, red and black dots
respectively). The \textit{bottom panels} show the data for models
with accretion rates of \logmdot = $-6$, $-7$ and $-8$ (blue, red and
black dots respectively), which we have classed as extreme
accretors. The \textit{right panels} show the data separated by age,
with 1 and 10 Myrs (as crosses and triangles respectively), and
rotation period, with 0.5 and 5 days (as red and blue symbols
respectively).

\begin{figure}
  \vspace*{174pt}
  \special{psfile="Fig/delta_radius.ps"
   hoffset=235 voffset=0 hscale=30 vscale=30 angle=90}
 \caption{Figure showing data for the co-rotation radius ($R_{\rm
     co}$) against the change in inner edge location ( $R_{\rm
     inner}-R_{\rm co}$) both in units of $R_*$. The typical accretors
   are shown in the \textit{top panels} and the extreme accretors in
   the \textit{bottom panels} (see text for explanation). The
   accretion rates are then further subdivided. The \textit{top left
     panel} shows accretion rates of \logmdot = $-9$, $-10$, $-11$
   \& $-12$ as blue, red and black dots
   respectively. In the \textit{bottom left panel} accretion rates of
   \logmdot = $-6$, $-7$ and $-8$ are shown as
   blue, red, and black dots. In addition the \textit{left panels}
   separates the systems by age and period. The systems with ages of 1
   and 10 Myrs are shown as crosses and triangles respectively. The
   systems with rotation rates of 0.5 and 5 days are then shown as red
   or blue symbols respectively. \label{delta_radius}}
\end{figure}

The classification of models into typical and extreme accretors is
based on observations (see discussion in \ref{par_accn}) and also
evidence from the simulations themselves. Figure \ref{delta_radius},
\textit{left panels}, shows that dust sublimation does not become
significant until accretion rates of around \logmdot=$-9$ and
increases dramatically with increasing accretion rate thereafter.
This accretion rate is also, unsurprisingly the accretion rate at
which the underlying photospheric emission is overwhelmed by the
accretion flux, and the photospheric features are lost (see Section
\ref{sed_analysis}). The \textit{right panels} of Figure
\ref{delta_radius} show that dust sublimation occurs only for rotation
periods of 0.5 days for typical accretors. For the extreme accretors
the systems with both rotation periods undergo sublimation but the
effect is more significant in the systems with shorter rotation
periods.  The \textit{right panels} also show that for extreme
accretors the older systems undergo more dust sublimation, whereas for
the typical accretors the older systems undergo little or no dust
sublimation.

The density of material in the disc falls with increasing radius from
the star ($\rho\propto R_*^{\alpha}$). The dust sublimation
temperature is also density dependent \citep{1994ApJ...421..615P}.
Therefore, for systems where the inner edge has been eroded
significantly from the co-rotation radius the temperature of the inner
edge will fall as the final radius increases. In this case a
correlation of inner edge temperature and inner edge radius is
expected with a resulting slope determined, at least in part, by the
radial density distribution of the disc. In this case the the final
inner edge radius and therefore temperature is no longer strongly
correlated with the rotation rate. Figure \ref{inner_temp} shows the
final inner edge location ($R_{\rm inner}$, $R_*$) against the
temperature of the inner wall ($T_{\rm inner}$, K). The \textit{left
  panel} shows the systems designated as typical accretors and the
\textit{right panel} those with extreme accretion rates. For
\textit{both panels} the systems with rotation periods of 0.5 and 5
days are plotted in red and blue respectively. Those systems where the
change in inner radius was greater than 1$R_*$, and therefore classes
as significantly sublimated, are plotted as crosses. These systems are
only present for the extreme accretors, as shown in Figure
\ref{delta_radius}. Furthermore, for the typical accretors Figure
\ref{delta_radius} shows that only the shorter period objects, in
general, experience any disc erosion at the inner edge.

\begin{figure}
  \vspace*{174pt}
  \special{psfile="Fig/inner_temp.ps"
   hoffset=235 voffset=0 hscale=30 vscale=30 angle=90}
 \caption{Figure showing data for the inner edge location ($R_{\rm
     inner}$, $R_*$). The typical accretors are shown in the
   \textit{left panel} and the extreme accretors in the \textit{right
     panel} (see text for explanation). In \textit{both panels}
   separate the systems by period, with those systems with rotation
   periods of 0.5 and 5 days shown as red and blue symbols
   respectively. In addition systems where the total change in the
   inner radius is greater than 1$R_*$ are marked as crosses (this is
   only achieved by some systems classed as extreme
   accretors).\label{inner_temp}}
\end{figure}

Figure \ref{inner_temp} shows that for those systems where significant
disc erosion ($\Delta R_{\rm inner} > 1.0 R_*$) has occurred the
resulting temperature of the inner edge is weakly correlated with the
inner edge radius, but, critically, not correlated with rotation rate.
The inner edge temperatures of the remaining systems for the extreme
accretors are slightly anti-correlated with the radius to the inner
edge. For the systems with typical acccretion rates, and longer
periods, Figure \ref{inner_temp} shows there is again a weak
correlation between the radius to, and the temperature of the inner
edge. For the shorter period models with typical accretion rates there
is no clear correlation between the temperature at the inner edge and
radius to this edge. The \textit{left panel} of Figure
\ref{inner_temp} shows that, taken as a whole, our models with typical
accretion rates show a clear correlation between the temperature at
the inner edge and the radius to this boundary. This agrees with the
work of \cite{1997AJ....114..288M} where this correlation is found for
there $R_{\rm inner}=$1$-$12$R_*$ and $\dot{M_{\rm acc}}=$10$^{-9}$to
10$^{-5} M_{\odot}yr^{-1}$, for flat disc models.
\cite{1997AJ....114..288M} use this correlation, and the derivation of
IR magnitudes, to predict a relationship between IR excess and radius
to the inner wall. Our data indicate that this correlation, for
typically accreting systems will translate into a correlation between
rotation rate and IR excess. This could have important implications
for studies of disc-locking where disc presence is examined as a
function of rotation rates, provide an intrinsic bias. In practice
however, this correlation is weak, an unobservable, in our data due to
inner disc wall shape, inclination and flaring effects. This is
discussed in more detail in Section \ref{photometry}.

For simulations without dust sublimation, and analytic density
distributions, the inner disc edge is characterised solely by the
initial disc density prescription. If one includes vertical
hydrostatic equilibrium the scaleheight of the disc inner edge can
change but the radial distribution of dust will not. Dust sublimation
physics will alter the vertical structure, but crucially will also
alter the radial distribution of dust. This means that dust
sublimation effects will control the shape of the inner disc wall and
possibly change this from a vertical wall. Evidence that real inner
disc boundaries (for CTTS) are not vertical walls comes from a lack of
dependence of the derived IR excess on system inclination
\citep{2007ApJ...661..374T}. As discussed in Section \ref{dust_edge}
the disc inner hole can be created by several mechanisms, however, for
this study we have only included a treatment of magnetic truncation
(initially) and subsequent dust sublimation. The effects of dust
sublimation have been shown to produce curved inner walls due to the
temperature and density dependence of the dust sublimation temperature
\citep{2005A&A...438..899I}. The curved boundary in this case is
convex, as density increases towards the midplane of the disc, the
dust sublimation temperature increases, meaning the destruction radius
moves closer to the star. A curved inner boundary is also found to
result from dust sublimation in \cite{2007ApJ...661..374T}. In
\cite{2007ApJ...661..374T} two populations of dust grains where
included with different scaleheight distributions. This is essentially
a simple model of grain growth and subsequent midplane settling.
\cite{2007ApJ...661..374T}, found a concave inner disc boundary, this
is due to the larger grains cooling more efficiently. These larger
grains dominate the dust towards smaller vertical distances from the
midplane, and therefore the increased cooling means that the dust
sublimation temperature is reached at closer radii. The implications
of the inner edge shape for the simulated SEDs and photometry are
discussed in Section \ref{sed_analysis} and \ref{photometry}.

For systems where dust sublimation does not occur, or is minimal the
inner edge of the system will remain as a vertical wall, as prescribed
by our initial density distribution. However, as one increases the
flux levels of a given system, by increasing the accretion rates (or
moving to younger ages) the inner edge will increase in
temperature. As the flux levels increase, in a given model, the dust
at the inner edge will sublimate. The dust in the regions of lowest
density or maximum flux interception will be preferentially
destroyed. This will lead to a reduction in the scaleheight and a
slight concave cavity at the inner wall. The inner edge will then move
towards a convex boundary as the inner edge is forced radially outward
towards regions of lower density with increasing flux levels. This
outward movement will result in the outer, vertical, regions of the
dust disc eroding. Increasing the flux still further will lead to
significant dust sublimation and an outward migration. This subsequent
migration will move the inner edge into regions of lower density and
greater scaleheight.

The transition in shape and position of the inner edge for a given
model as the accretion rate, and therefore flux levels, are increased
is shown in Figures \ref{no_sub_183} to \ref{sub_181}. The model
chosen for illustrative purposes has the following parameters,
$M_*$=0.04$M_{\odot}$, Age=1 Myrs and areal coverage=10\%. The model
shown in Figure \ref{no_sub_183} has $\tau$=5 days and accretion
rate \logmdot$=-12$ (model shown in Figure
\ref{flare_-12_183}), with Figures \ref{sub_169} to \ref{sub_181}
showing models with $\tau$=0.5 days and accretion rates \logmdot=$-12$, $-9$, $-7$
and $-6$. The \textit{right
  panels} of Figures \ref{no_sub_183} to \ref{sub_181} show the
temperature structure of the final converged disc structure. The
\textit{left panels} of Figures \ref{no_sub_183} to \ref{sub_181} show
the final log($\rho$) distribution for regions where dust is present,
scaled to a different minimum density (log($\rho_{\rm
  min}$)$=1^{-13}$) compared to Figures \ref{flare_-12_183} and
\ref{flare_-7_193}. The lower (log) density threshold is used in this
case to better trace the inner edge and denser regions of the disc,
whereas Figure \ref{flare_-12_183} and \ref{flare_-7_193} show the
outer diffuse regions of the disc, or disc limits. Additionally, for
Figures \ref{sub_169} to \ref{sub_181} (the sublimated systems), the
\textit{left panels} also show the initial density, in the same units
as the final density, this is shown as a background grey scale (with
the same maximum and minimum as for the final density) to illustrate
the effects of sublimation.

\begin{figure*}
  \centering
  \subfigure[]{\includegraphics[scale=0.4,angle=0]{./Fig/no_sub_183_density.ps}\label{no_sub_183_a}}
  \subfigure[]{\includegraphics[scale=0.4,angle=0]{./Fig/no_sub_183_temp.ps}\label{no_sub_183_b}}
  \caption{Figure showing final density (log($\rho$)) where dust is
    present (colour scale) as the \textit{left panel} and temperature,
    \textit{right panel}, for a system with no dust
    sublimation. System parameters are: $M_*$=0.04$M_{\odot}$, Age=1
    Myrs and areal coverage=10\%, $\tau$=5 days and accretion rate \logmdot$=-12$.}
  \label{no_sub_183}
\end{figure*}

\begin{figure*}
  \centering
  \subfigure[]{\includegraphics[scale=0.4,angle=0]{./Fig/sub_169_density.ps}\label{sub_169_a}}
  \subfigure[]{\includegraphics[scale=0.4,angle=0]{./Fig/sub_169_temp.ps}\label{sub_169_b}}
  \caption{Figure showing both the initial (grey scale) and final
    density (log($\rho$)) where dust is present (colour scale) as the
    \textit{left panel} and temperature, \textit{right panel}, for a
    system with no dust sublimation. System parameters are:
    $M_*$=0.04$M_{\odot}$, Age=1 Myrs and areal coverage=10\%,
    $\tau$=0.5 days and accretion rate \logmdot=$-12$.}
  \label{sub_169}
\end{figure*}

\begin{figure*}
  \centering
  \subfigure[]{\includegraphics[scale=0.4,angle=0]{./Fig/sub_175_density.ps}\label{sub_175_a}}
  \subfigure[]{\includegraphics[scale=0.4,angle=0]{./Fig/sub_175_temp.ps}\label{sub_175_b}}
  \caption{Figure showing both the initial (grey scale) and final
    density (log($\rho$)) where dust is present (colour scale) as the
    \textit{left panel} and temperature, \textit{right panel}, for a
    system with no dust sublimation. System parameters are:
    $M_*$=0.04$M_{\odot}$, Age=1 Myrs and areal coverage=10\%,
    $\tau$=0.5 days and accretion rate \logmdot$=-9$.}
  \label{sub_175}
\end{figure*}

\begin{figure*}
  \centering
  \subfigure[]{\includegraphics[scale=0.4,angle=0]{./Fig/sub_179_density.ps}\label{sub_179_a}}
  \subfigure[]{\includegraphics[scale=0.4,angle=0]{./Fig/sub_179_temp.ps}\label{sub_179_b}}
  \caption{Figure showing both the initial (grey scale) and final
    density (log($\rho$)) where dust is present (colour scale) as the
    \textit{left panel} and temperature, \textit{right panel}, for a
    system with no dust sublimation. System parameters are:
    $M_*$=0.04$M_{\odot}$, Age=1 Myrs and areal coverage=10\%,
    $\tau$=0.5 days and accretion rate \logmdot$=-7$.}
  \label{sub_179}
\end{figure*}

\begin{figure*}
  \centering
  \subfigure[]{\includegraphics[scale=0.4,angle=0]{./Fig/sub_181_density.ps}\label{sub_181_a}}
  \subfigure[]{\includegraphics[scale=0.4,angle=0]{./Fig/sub_181_temp.ps}\label{sub_181_b}}
  \caption{Figure showing both the initial (grey scale) and final
    density (log($\rho$)) where dust is present (colour scale) as the
    \textit{left panel} and temperature, \textit{right panel}, for a
    system with no dust sublimation. System parameters are:
    $M_*$=0.04$M_{\odot}$, Age=1 Myrs and areal coverage=10\%,
    $\tau$=0.5 days and accretion rate \logmdot$=-6$.}
  \label{sub_181}
\end{figure*}

Figure \ref{no_sub_183} shows the inner disc edge as a vertical wall
with no dust sublimation, the inner wall has not changed location from
the initial disc description. The \textit{right panel} of Figure
\ref{no_sub_183} shows that the corresponding temperature structure is
smooth and interpolation at the inner edge yields a temperature of
$~$900\,K. For the shorter period model the inner edge has undergone
some dust sublimation. The \textit{left panel} of Figure \ref{sub_169}
shows that the inner edge has lost dust slightly, preferentially,
along the midplane. Again for Figure \ref{sub_169} the change in inner
edge location is small, $~0.45R_*$ ($~1.0\times 10^{-3}$AU), at the
sublimation (and inner wall) temperature is $~$1450\,K, for a density
of $~2.5\times 10^{-8}$ ($g/cm^3$) (\textit{right panel} of Figure
\ref{sub_169}).The \textit{left panel} of Figure \ref{sub_169} shows
that for even for a negligible accretion rate of \logmdot=$-12$,
sublimation has begun to erode the inner edge, for the younger and
faster rotating systems.  Here the final density distribution has
moved radially outward as can be seen by comparing the grey and colour
scales in the \textit{left panel} of Figure
\ref{sub_169}. Additionally, the inner edge has become slightly
concave. As the flux levels of the central star increase the
temperature of the inner edge surpasses the sublimation temperature at
an increasing radius, destroying more of the disc and pushing the
inner edge radially outwards. The \textit{left panel} of Figure
\ref{sub_175} shows that for an accretion rate of \logmdot$=-9$,
increased sublimation has pushed the inner wall slightly farther from
the star (when compared to Figure \ref{sub_169}), here, $\Delta
R~0.52R_*$ ($~1.1\times 10^{-3}$AU), and $T_{\rm inner}~1450$. Again
the final density distribution has moved radially outward as can be
seen by comparing the grey and colour scales in the \textit{left
  panel} of Figure \ref{sub_175}. The \textit{right panel} of Figure
\ref{sub_175} shows that the gas interior to the inner edge extending
down to the co-rotation radius, which is not sublimated, will reach
temperatures in excess, or equal to, the inner edge, which in this
model is $~$1400\,K. Such hot gas inner discs interior to the dust
disc have been postulated through SED fitting
\citep{2005ApJ...623..952E,2008ApJ...689..513T}. Again, the shape of
the inner wall has changed in Figure \ref{sub_175}, the top and bottom
sections of the disc have eroded and the inner wall is convex. The
model in Figure \ref{sub_175} has an accretion rate of \logmdot=$-9$,
which is the transition point between our two groups of accretors,
typical and extreme. Figure \ref{sub_175} shows that for typical
accretion rates, and faster rotation periods (and younger ages)
significant dust sublimation is expected to occur at the inner edge,
resulting in a curved inner wall at around $~$1400\,K. Increasing the
accretion rate, and therefore flux levels still further leads to
significant erosion of the inner disc.  For the extreme accretors the
dust sublimation at the inner edge now changes the inner wall position
by many multiples of the stellar radius. Figures \ref{sub_179} and
\ref{sub_181} are for systems with accretion rates typical in outburst
or episodic accretion systems, \logmdot$=-7$ and $-6$
respectively. The inner edges for these systems are eroded radially by
7.1 and 36.1 ($R_*$) respectively (0.016 and 0.080 AU, respectively),
visible by comparing the grey and colour scales of Figure
\ref{sub_179} and \ref{sub_181}.  As this occurs the disc inner edge
moves far enough that the density drop (with radius) occurring in the
disc causing the sublimation temperature to fall. Consequently the
disc inner edge becomes cooler as it can not sustain temperatures
above the sublimation temperature.  This is shown in the \textit{right
  panels} of Figures \ref{sub_179} and \ref{sub_181}, with approximate
temperatures at the inner edge of 1370 and 1260\,K. In addition, the
inner edge has become concave and, again due to changes in the
underlying disc density distribution, much larger in scaleheight.

A similar sequence is apparent across the grid for set
masses. However, the balance of the rotation period, and therefore
inner edge location, and age and areal coverage, therefore flux
levels, leads to changes in the accretion rate at which the dust
sublimation starts. However, in almost all cases the dust sublimation
does not become significant until at least \logmdot=$-9$ (as discussed
above).

\subsection{SEDs}
\label{sed_analysis}

The resulting converged disc structures, as discussed in Section
\ref{seds} are then used to create simulated SEDs. In this section we
explore features in these derived SEDs and link them back to the
physical structures within the disc discussed in Section
\ref{disc_struct}. For analysis purposes our SEDs will comprise
contributions from four main luminosity sources. Firstly, the
accretion luminosity ($L_{\rm acc}$) and the intrinsic stellar
luminosity ($L_{*}$), will effectively be combined and provide flux at
the shorter wavelength section of the SED. For our parameter space the
temperature of the photospheres has a range of $~$3000\,K$<T_{\rm
  eff}<$1600\,K, this will comprise flux contributions peaking in the
range, 0.97 to 1.8\,$\mu$m ($\lambda_{\rm peak}\propto \frac{1}{T_{\rm
    acc}}$). The accretion spot temperatures, for typical accretors,
will range from $~$340$>T_{\rm acc}>$9400\,K, emitting in the range 0.3
to 8.5\,$\mu$m. For the extreme accretors temperatures which can reach
as high as 53\,000\,K, provide flux to $~$0.05\,$\mu$m. The second pair of
luminosity sources are from constituent parts of the circumstellar
disc. The inner edge, ($L_{\rm wall}$), at temperatures of $~$370 to
$~$1500\,K, or flux contributions from $~$1.85 to 7.7\,$\mu$m
\citep[nominally 2 to 3\,$\mu$m][]{2001ApJ...560..957D}, and the outer
disc ($L_{\rm disc}$), at temperatures of $~$300\,K or less contributing
flux longward of $~$10\,$\mu$m. As discussed in Section \ref{seds} the
{\sc torus} radiative transfer code tags the photons before they reach
the simulated observer in one of four ways, stellar and thermal,
either direct or scattered. These components allow us to analyse the
contributions from the main flux contributors discussed, i.e. $L_{\rm
  acc}$ and $L_*$ equals stellar scattered or direct, with thermal
emission comprising $L_{\rm wall}$ and $L_{\rm disc}$. Figure
\ref{disc_eg_173} shows a typically accreting BDD system with the
component photon contributions highlighted, with $M_*=0.04M_{\odot}$
at 1 Myr with an areal coverage of 10\%, accretion rate of
\logmdot=$-10$ and rotation period of 0.5
days. The solid black line is the total simulated SED, with the
stellar photons shown as blue lines and the thermal as red. The
scattered and direct contributions are then shown as dashed and solid
lines respectively. The horizontal and vertical dotted lines then show
the approximate ranges of the flux contributions from the (from left
to right), $L_{\rm acc}$ (blue), $L_*$ (black), $L_{\rm wall}$ (red)
and $L_{\rm disc}$ (black). The vertical dashed lines at the bottom of
Figure \ref{disc_eg_173} show the approximate ranges of the $V$, $I$,
$J$, and $K$ photometric bands we have adopted.

\begin{figure}
  \vspace*{174pt}
  \special{psfile="Fig/disc_eg_173.ps"
   hoffset=250 voffset=-20 hscale=36 vscale=36 angle=90}
 \caption{Figure showing the SED log($\lambda \mu$m) against
   log(flux), flux in ergs$/s/cm^2/{\rm \AA}$) for $M_*=0.04M_{\odot}$
   at 1 Myr with an areal coverage of 10\%, accretion rate of \logmdot
   = $-10$ and rotation period of 0.5 days. The solid black line is
   the total simulated SED. The stellar components of the flux are
   shown in blue and the thermal in red, with direct emission
   appearing as solid lines and scattered photons as dashed lines. The
   vertical dotted bars linked by horizontal bars show the approximate
   (log) wavelength ranges of the luminosity components. With the blue
   lines denoting the accretion luminosity ($L_{\rm acc}$) and the
   first (left to right) black lines showing the photospheric
   contribution ($L_*$). The final two sets of lines show the inner
   edge contribution ($L_{\rm wall}$) in red and the outer disc
   ($L_{\rm disc}$) in black. The vertical and horizontal dashed lines
   (in the \textit{lower panels}) denote the approximate sensitivity
   ranges of our chosen $V$, $I$, $J$ and $K$
   filters. \label{disc_eg_173}}
\end{figure}

Figure \ref{disc_eg_173} shows that the photosphere and accretion
luminosities dominate the SED at wavelengths shortward of
$~$1$\,\mu$m. After this, for this model, the SED is dominated by
contributions from the disc, at shorter wavelengths, the inner edge
and then at larger wavelengths the outer disc. As shown in Figure
\ref{disc_eg_173} the intrinsic stellar photosphere and accretion
luminosities overlap, in terms of peak wavelength of emission. This
overlap essentially means that the combined stellar flux ($L_*+L_{\rm
  acc}$) will eventually become dominated by the accretion flux, as
one increases the accretion rate. With the features of the intrinsic
stellar photosphere or atmosphere eventually lost or swamped by the
accretion flux. This will lead to difficulties in using spectral
features to identify BDD systems for wavelengths shorter than $~$1
$\mu$m, and, of course, for wavelengths longer than this the SED
becomes dominated by disc emission. Therefore, once the SED of a BDD
system, at shorter wavelengths, becomes dominated by accretion
luminosity spectral classification of these systems (or at least their
central stars) will become unreliable. As discussed in Section
\ref{inner_edge} observations of the inner edge location separate the
BDD systems into two groups, the typical accretors ($\dot{M}\leq$-9
log$(\frac{\dot{M}}{M_{\odot}} yr^{-1})$) and extreme accretors
($\dot{M}>-9$). This division
between typical and extreme accretors is based on observational
results (see Section \ref{par_accn}) as well as consequence of
sublimation. The typical accretors show limited sublimation at the
disc inner edge leading to a weak correlation in inner edge
temperature with edge radius. Whereas, for the extreme accretors the
inner edge is significantly eroded and the dust sublimation
temperature reached, which can not be exceeded. This effectively means
that the inner edge temperature is weakly correlated with temperature
(with a different characteristic slope to the typical accretors), as
the dust sublimation temperature is density dependent, which is in
turn dependent on radial position (see Figures \ref{delta_radius} and
\ref{inner_temp}).

\subsubsection{Accretion Dominance}
\label{acc_dominance}

Figure \ref{acc_dom} shows the affect of increasing the accretion
blackbody flux (for increasing accretion rates) for a BD star,
$M=$0.04$M_{\odot}$, at 1 Myr. Whilst the photons originating from the
star (both from $L_*$ and $L_{\rm acc}$) will be tagged as stellar by
{\sc torus} we can separate these flux contributions simply by
observing the naked star system. The panels in Figure \ref{acc_dom}
show the flux from a naked system, with no treatment of the disc. This
enables us to view the effect of increasing accretion rate on the
photospheric flux in isolation. The accretion rates included in all
panels are \logmdot = $-8$, $-9$ and $-12$ (blue, black and red lines
respectively). The \textit{bottom panels} show the systems with a
rotation period of 5 days and \textit{top panels} for those with a
rotation period of 0.5 days. Given our assumption that accretion
occurs from the co-rotation radius, decreasing the rotational period
moves this accretion radius closer to the star, $R_{\rm inner}\propto
\tau^{2/3}$ (see Equation \ref{inner_eq}). As the accretion radius
moves father from the star the potential energy released by the
accreted material is reduced. This effect can be seen when comparing
the \textit{top} and \textit{bottom panels}, although the effect is
marginal for all but the highest displayed accretion rates. The
\textit{left panels} show accretion streams with an areal coverage of
10\% and the \textit{left panels} 1\%. As the areal coverage reduces
the effective temperature of the accretion hot spot increases,
resulting in an increase in accretion flux, and resulting shift to
bluer wavelengths of the peak flux. This can be seen clearly by
comparing the \textit{left} and \textit{right panels} of Figure
\ref{acc_dom}. Perhaps the most important, albeit qualitative, result
shown is Figure \ref{acc_dom} is an insight into the accretion rate at
which the accretion blackbody flux dominates over the photospheric
flux. Figure \ref{acc_dom} shows that as the accretion rate raises
above \logmdot = $-9$ for systems with 1 or 10\% areal coverage, the
accretion flux dominates the emergent SED at both periods. Therefore
for reasonable coverages (1-10\%) and rotational periods (0.5-5 days)
the photospheric flux is effectively swamped by accretion flux for
accretion rates \logmdot$>-9$. This suggests that accreting BD systems
with accretion rates as low as \logmdot$=-9$ could be difficult to
identify from SEDs as BD stars, and may well be classified as higher
mass stars.

\begin{figure*}
  \vspace*{348pt}
  \special{psfile="Fig/acc_dom.ps"
   hoffset=470 voffset=0 hscale=60 vscale=60 angle=90}
 \caption{Figure showing the photospheric flux (log(ergs$/s/cm^2/{\rm
     \AA}$)) against $\lambda$ (log($\mu$m)) of a Brown Dwarf with
   $M=$0.04$M_{\odot}$, at 1 Myr. No disc is included, but blackbody
   fluxes from an accretion stream at the rates of \logmdot $=-8$,
   $-9$ and $-12$ are shown as blue, black and red lines
   respectively. The \textit{bottom panels} show accretion for a star
   rotating at 5 days, with the \textit{top panels} showing that of
   0.5 days. The \textit{left panels} systems with an areal coverage
   of 10\% and the \textit{right panels} has 1\%. The vertical and
   horizontal dashed lines (in the \textit{lower panels}) denote the
   approximate sensitivity ranges of our chosen $V$, $I$, $J$ and $K$
   filters.\label{acc_dom}}
\end{figure*}

As is well known the shorter wavelength component of the SED for
extreme accretors is dominated by the accretion luminosity. For the
typical accretors the intrinsic photospheric emission is important and
the flux levels of this component are dependent on the stellar
parameters of mass and age, through evolution of the stellar radius
and temperature, as is well understood. In general, the bluer (short
wavelength) components of the SEDs are therefore controlled by the
stellar input parameters (mass and age) and the accretion input
parameters (accretion rate and areal coverage, and to a lesser extent
rotation rate).

\subsubsection{Flaring and Obscuration}
\label{flare_obscure}

As shown in Figures \ref{flare_-12_183} and \ref{flare_-7_193} and
discussed in Section \ref{disc_flaring} BDD systems under vertical
hydrostatic equilibrium have highly flared discs. As discussed in
\cite{2004MNRAS.351..607W} this increased flaring (when compared to
CTTS stars) leads to occulltation at the star at lower inclination
angles and, therefore, significant changes to the SED. Increases in
the inclination angle, for these systems, quickly lead a significant
proportion of the stellar flux being intercepted, and reprocessed, by
the highly flared disc. This reprocessing will lead to a change in the
flux levels at the shorter, bluer, wavelengths as more stellar flux is
reprocessed and intercepted by the disc. It will also lead to
significant changes in the flux reaching the observer from the inner
and outer regions of the disc. 

Figure \ref{flare_sed_fast} shows the SEDs for a range of inclinations
for BDD systems with two accretion rates. The BDD system has a central
stellar mass of $0.04M_{\odot}$, an age of 1Myr, a rotation period of
0.5 days and an areal coverage of the accretion stream of 10\%. The
\textit{top panels} show the SEDs for a star with an accretion rate of
\logmdot$=-12$ (the lowest in our typical
accretion range). The \textit{bottom panels} then show SEDs for
systems with an accretion rate of \logmdot$=-8$ (classed as an extreme
accretor). From left to right the \textit{panels} then show the naked
stellar flux then the fluxes from BDD systems seen at inclination
angles of 0$^{\circ}$, 64$^{\circ}$ and 90$^{\circ}$, or face-on, the
approximate expectation value of inclination and edge-on. For each of
the \textit{second} to the \textit{fourth panels} the flux components
are then shown as in Figure \ref{disc_eg_173}, i.e. the total SED
shown as a black line, then stellar light shown in blue and thermal in
red, with direct shown as bold and scattered as dashed lines. The
inset graph in the \textit{bottom right} panel of Figure
\ref{flare_sed_fast} simply shows a lower flux scale, as the flux
levels have fallen to the lowest shown division in the main Figure.

\begin{figure*}
  \vspace*{348pt}
  \special{psfile="Fig/flare_sed_fast.ps"
   hoffset=470 voffset=0 hscale=60 vscale=60 angle=90}
 \caption{Figure showing SEDs of a system with a central stellar mass
   of $0.04M_{\odot}$, an age of 1Myr, a rotation period of 0.5 days
   and an areal coverage of the accretion stream of 10\%. The
   \textit{panels}, both \textit{top} and \textit{bottom}, (from left
   to right) show the naked stellar flux (\textit{first panel}), then
   constituent fluxes for SEDs at inclination angles of 0$^{\circ}$,
   64$^{\circ}$ and 90$^{\circ}$. The \textit{second} to
   \textit{fourth panels} on each line then show the total flux as a
   black line, with the component fluxes presented as explained Figure
   \ref{disc_eg_173}. The \textit{top panels} are for systems with an
   accretion rate of -12 log$(\frac{\dot{M}}{M_{\odot}}yr^{-1})$ and
   the \textit{lower panels} are for an accretion rate of -8
   log$(\frac{\dot{M}}{M_{\odot}}yr^{-1})$. The \textit{bottom right
     panel} inset Figure shows an enlargement of the lower flux levels
   of this SED. The vertical and horizontal dashed lines (in the
   \textit{third panels} across) denote the approximate sensitivity
   ranges of our chosen
   $V$, $I$, $J$ and $K$ filters. \label{flare_sed_fast}}
\end{figure*}

Figure \ref{flare_sed_fast} shows, by comparing the \textit{top
  panels} that as a disc is added to the system and the inclination of
the BDD system is increased more stellar (and accretion flux) is
reprocessed by the disc, for a negligibly accreting system. As one
moves from inclinations of 0 to 64$^{\circ}$ the stellar flux has
reduced as has the thermal component. At edge-on inclinations the
system is only visible through scattered light, and some (slight)
thermal emission from the outer disc at wavelengths $\lambda>30\,\mu$m
(indicative of temperatures of $~$100\,K. Increasing the accretion rate
from -12 to -8 log$(\frac{\dot{M}}{M_{\odot}}yr^{-1})$ leads to a
significant increase in the naked stellar flux (compare \textit{top
  left} and \textit{bottom left panels}). As we add a disc and
increase the viewing inclination for the extreme accreting system
(\textit{bottom panels}), we also lose stellar flux. In the extreme
accretor case however the fall in flux contribution to the total SED
from the photosphere (and accretion) is much more significant. Indeed,
at edge-on inclinations the system becomes only visible in the thermal
($~$100\,K) disc regime, with the scattered light significantly reduced.

As shown in Section \ref{disc_struct}, and more specifically, Figures
\ref{flare_-12_183} and \ref{flare_-7_193} as one increases the
accretion rate for BDD systems, in general, the outer disc structure
increases in vertical height. Effectively, the greater flux levels
increase the temperature of the outer disc and the condition of
vertical hydrostatic equilibrium leads to a vertical expansion. This
vertical expansion leads to a significant change in the angle at which
photons emitted from the stellar photosphere are intercepted by the
outer disc. This is an extension of the effected found in
\cite{2004MNRAS.351..607W}, where it was found that this effect alone
can shift the SED, and therefore colours and magnitudes, of inclined
BDD systems to those indicative of higher mass CTTS. The actual
changes in flux are complicated by dust sublimation and shape changes
of the inner edge (discussed later). As the inclination of the BDD
system is increased we see the total flux drop significantly once we
exceed the opening angle of the BDD system. This opening angle will
decrease with vertical scaleheight of the outer disc. Figure
\ref{flare_inc} shows the total SEDs for the BDD system of Figures
\ref{flare_-12_183} and \ref{flare_-7_193}, with the corresponding
accretion rates as \textit{left} and \textit{right panels}
respectively. The solid lines of \textit{both panels} of Figure
\ref{flare_inc} show the total SED for each of our ten simulated
inclinations, namely, 0, 27, 39, 48, 56, 64, 71, 77, 84 and
90$^{\circ}$ (with the flux decreasing with increasing
inclination). The dashed line then highlights the inclination at which
the flux observed drops significantly, at 71 and 56$^{\circ}$ for the
accretion rates of -12 and -7 (or \textit{left} and \textit{right
  panels}), respectively. Whilst this effect is probably artificially
enhanced for such extreme accretion rates it is illustrative of the
general trend, notwithstanding complications from inner edge shape.

\begin{figure*}
  \vspace*{348pt}
  \special{psfile="Fig/flare_inc.ps"
   hoffset=470 voffset=0 hscale=60 vscale=60 angle=90}
 \caption{Figure of total SED flux reaching the observer for viewing
   angles of 0, 27, 39, 48, 56, 64, 71, 77, 84 \& 90$^{\circ}$. SEDs
   shown are for systems shown in Figures \ref{flare_-12_183} and
   \ref{flare_-7_193}, as the \textit{left} and \textit{right panels}
   respectively. In both panels the dashed line highlights the angle
   at which a significant fall in flux reaching the observer occurs.
   The values of inclination at which the fall in flux occurs are
   71$^{\circ}$ and 56$^{\circ}$ for the accretion rates of -12 and -7
   log$(\frac{\dot{M}}{M_{\odot}}yr^{-1})$, \textit{left} and
   \textit{right panels} respectively. The vertical and horizontal
   dashed lines denote the approximate sensitivity ranges of our chosen
   $V$, $I$, $J$ and $K$ filters. \label{flare_inc}}
\end{figure*}

\subsubsection{Inner Edge Shape}
\label{inner_shape}

As discussed in Section \ref{disc_struct} the addition of dust
sublimation leads to a change in the shape of the inner edge, where
the temperature is sufficient. Figures \ref{no_sub_183} to
\ref{sub_181} show a range in inner edge shapes, from flat walls
through concave to convex curves, caused by the radial density profile
of the disc and the dependence of the sublimation temperature on
density. This change in shape, as noted by \cite{2007ApJ...661..374T},
will lead to changes in the characteristics of the SED, or derived IR
excess, with inclination angle \citep{2007ApJ...661..374T}. For
concave and vertical walls the inner edge is visible over a smaller
range of viewing angles than for a convex inner edge.

Figures \ref{typ_curve} and \ref{ext_curve} show the thermal direct
radiation, originating from the disc, as a function of inclination for
the two groups of accretors, typical and extreme respectively. The
models included in Figure \ref{typ_curve} as the \textit{left} and
\textit{right panels} are those shown in Figures \ref{sub_169} and
\ref{sub_175} respectively. Whereas, the models included in Figure
\ref{ext_curve} as the \textit{left} and \textit{right panels} are
those shown in Figures \ref{sub_179} and \ref{sub_181}
respectively. For each \textit{panel} of Figures \ref{typ_curve} and
\ref{ext_curve} the thermal direct flux is plotted as a solid black
line at each modeled inclination, namely, 0, 27, 39, 48, 56, 64, 71,
77, 84 \& 90$^{\circ}$. The dashed line in each case shows where a
significant fall in the flux reaching the observer from direct thermal
emission from the disc is apparent.

\begin{figure*}
  \vspace*{348pt}
  \special{psfile="Fig/typ_curve.ps"
   hoffset=470 voffset=0 hscale=60 vscale=60 angle=90}
 \caption{Figure showing thermal direct flux reaching the observer at
   inclination angles of 0, 27, 39, 48, 56, 64, 71, 77, 84 \&
   90$^{\circ}$, for typical accretors. The \textit{left panel} shows
   the flux from the model shown in Section \ref{disc_struct}, Figure
   \ref{sub_169}. The \textit{right panel} shows the flux from the
   model shown as Figure \ref{sub_175} in Section
   \ref{disc_struct}. In both panels the dashed line highlights the
   angle at which a significant fall in flux from direct thermal
   emission is observed.  The values of inclination at which the fall
   in flux occurs are 77$^{\circ}$ for \textit{both panels} and
   therefore accretion rates. The vertical and horizontal dashed lines
   denote the approximate sensitivity ranges of our chosen $V$, $I$,
   $J$ and $K$ filters. \label{typ_curve}}
\end{figure*}

\begin{figure*}
  \vspace*{348pt}
  \special{psfile="Fig/ext_curve.ps"
   hoffset=470 voffset=0 hscale=60 vscale=60 angle=90}
 \caption{Figure showing thermal direct flux reaching the observer at
   inclination angles of 0, 27, 39, 48, 56, 64, 71, 77, 84 \&
   90$^{\circ}$, for extreme accretors. The \textit{left panel} shows
   the flux from the model shown in Section \ref{disc_struct}, Figure
   \ref{sub_179}. The \textit{right panel} shows the flux from
   the model shown as Figure \ref{sub_181} in Section
   \ref{disc_struct}. In both panels the dashed line highlights the
   angle at which a significant fall in flux from direct thermal
   emission is observed.  The values of inclination at which the fall
   in flux occurs are 71$^{\circ}$ and 64$^{\circ}$ for the
   \textit{left} and \textit{right panels}
   respectively.  The vertical and horizontal dashed lines (in the
   \textit{lower panels}) denote the approximate sensitivity ranges of
   our chosen $V$, $I$, $J$ and $K$ filters.\label{ext_curve}}
\end{figure*}

For typically accreting systems, where little or no dust sublimation
has occurred the inner edge remains either a vertical or slightly
concave wall, as shown in Figures \ref{sub_169} and \ref{sub_175}. For
these systems flux reaching an observer from direct thermal emission,
i.e. from, in the main, the disc inner edge, falls rapidly with
increasing viewing angle, after the maximum opening angle is achieved.
This is shown in Figure \ref{typ_curve} where at the viewing angle of
77$^{\circ}$ the flux shown falls significantly, for both accretion
rates. This opening angle is larger than found for the slower
rotating, counter parts, as shown by Figure \ref{flare_inc}, where a
transition in occultation occurs at 71$^{\circ}$. For the case shown
in Figure \ref{typ_curve} the system has the shorter rotation period,
leading to a larger opening angle, as the flux from the central star
does not reach the outer disc, and subsequently flare it. Figure
\ref{typ_curve} shows that after an angle of 77$^{\circ}$ the flux
from the inner edge falls to zero. The systems displayed in Figure
\ref{typ_curve} have inner edge temperatures of $~$1400\,K, meaning
their flux emission peaks at a wavelength of $~$2\,$\mu$m. The flux
observed at the expected peak wavelength falls to zero at inclinations
of 77$^{\circ}$, indicative of a vertical wall being obscured by the
outer disc, before the limit of the opening angle is reached.

Figure \ref{inner_inc_sed} shows the full SEDs and the components (as
tagged by {\sc torus}), for the models displayed in Figures
\ref{sub_169} and \ref{sub_179} and the \textit{left panels} of
Figures \ref{typ_curve} and \ref{ext_curve}. The \textit{top panels}
show the SEDs for an accretion rate of -12
log$(\frac{\dot{M}}{M_{\odot}}yr^{-1})$ (model from Figure
\ref{sub_169}) for inclinations of 64, 71 and 77$^{\circ}$, moving
from the \textit{leftmost} to \textit{far right panel}. The same
inclinations are shown in the \textit{bottom panels}, which show the
SEDs, and components, for the model with an accretion rate of -7
log$(\frac{\dot{M}}{M_{\odot}}yr^{-1})$, displayed in Figure
\ref{sub_179}. As discussed in Section \ref{disc_struct}, the model
with the higher accretion rate has a a much larger, in scaleheight,
and convex inner boundary. The lower accretion rate model has a
vertical inner wall, close to the star. The analysis of Figures
\ref{typ_curve} and \ref{ext_curve} show that the flux from the inner
wall should fall dramatically at different limiting inclinations for
either accretion rate.

\begin{figure*}
  \vspace*{348pt}
  \special{psfile="Fig/inner_inc_sed.ps"
   hoffset=470 voffset=0 hscale=60 vscale=60 angle=90}
 \caption{Figure showing SEDs of a system with a central stellar mass
   of $0.04M_{\odot}$, an age of 1Myr, a rotation period of 0.5 days
   and an areal coverage of the accretion stream of 10\%. The
   \textit{panels}, both \textit{top} and \textit{bottom}, (from left
   to right) show the constituent fluxes for SEDs at inclination
   angles of 64$^{\circ}$, 71$^{\circ}$ and 77$^{\circ}$. The total
   flux is shown as a black line, with the component fluxes presented
   as explained Figure \ref{disc_eg_173}. The \textit{top panels} are
   for systems with an accretion rate of -12
   log$(\frac{\dot{M}}{M_{\odot}}yr^{-1})$ and the \textit{lower
     panels} are for an accretion rate of -7
   log$(\frac{\dot{M}}{M_{\odot}}yr^{-1})$. The SEDs from the
   \textit{top} and \textit{bottom panels} are derived from the models
   presented in Figures \ref{sub_169} and \ref{sub_179}, respectively.
   The vertical and horizontal dashed lines (in the \textit{second
     panels} across) denote the approximate sensitivity ranges of our
   chosen $V$, $I$, $J$ and $K$ filters.\label{inner_inc_sed}}
\end{figure*}

Figure \ref{inner_inc_sed} shows that for negligible accretion rates
(\textit{top panels}), where the inner edge is a vertical wall, the
total flux falls rapidly after exceeding a viewing angle of
71$^{\circ}$. After an inclination of 71$^{\circ}$, the negligibly
accreting system is dominated by scattered light from the central
star. The system accreting at -7
log$(\frac{\dot{M}}{M_{\odot}}yr^{-1})$, however, maintains high flux
levels, and thermal contributions to the SED, for inclination angles
above 71$^{\circ}$, particularly between 1 and 3\,$\mu$m. Essentially,
as shown by comparing Figures \ref{sub_169} and \ref{sub_179}, with
increasing flux and subsequent dust sublimation the inner edge has
grown vertically and become convex. As shown previously by
\cite{2007ApJ...661..374T}, a vertical (or concave) inner edge will
have strong dependence on inclination angel as the inner edge is
obscured at lower inclinations. For the larger convex wall, the SEDs
shown in \ref{inner_inc_sed} show that the resulting flux is less
dependent on inclination angle.

Once accretion rates reach extreme levels the inner edge is sublimated
far from the central star. For these systems the resulting inner edge
is both larger, in terms of scaleheight, and convex. Figure
\ref{ext_curve} shows that the flux falls dramatically at an
occultation angle of 71 and 64$^{\circ}$ for accretion rates of -7 and
-6 log$(\frac{\dot{M}}{M_{\odot}}yr^{-1})$, respectively
(\textit{left} and \textit{right panels}, respectively). For the
accretion rate of -7 log$(\frac{\dot{M}}{M_{\odot}}yr^{-1})$ this is
indicative of a larger opening angle (71$^{\circ}$) for the shorter
period object, when compared to the more slowly rotating equivalent
(56$^{\circ}$, from Figure \ref{flare_inc}, \textit{right
  panel}). Additionally, Figure \ref{inner_inc_sed} shows that the
total SED, has a strong thermal component and is only weakly dependent
on inclination angle, between 1 and 3\,$\mu$m . This increase in
opening angle with increase in rotation rate is again due to more flux
being intercepted by a larger and closer inner wall. This interception
of flux by the inner edge leads to a cooler and less vertically
extended outer disc in the shorter period objects when compared to the
slower rotating systems. Figures \ref{ext_curve} and
\ref{inner_inc_sed} also show that, contrary to the typical accretors,
the flux from the inner edge does not fall off as quickly with
inclination angle after occultation. The inner edge temperatures are
$~$1300 and $~$1200\,K for the accretion rates of -7 (\textit{left
  panel}) and -6 (\textit{right panel})
log$(\frac{\dot{M}}{M_{\odot}}yr^{-1})$ respectively. This leads to
peak emission at $~$2.2 to 2.4\,$\mu$m. Figures \ref{ext_curve} and
\ref{inner_inc_sed} shows that even for inclinations above the
occulation or opening angle of the disc flux from the inner edge
persists. This is indicative of a large and convex inner edge, as is
observed in Figures \ref{sub_179} and \ref{sub_181}. Once again these
extreme accretion rates probably over accentuate the result but help
illuminate the consequences of changes in the disc structure.

Therefore, the longer (in wavelength) component of the final SED is
dominated by the disc inner edge ($L_{\rm wall}$), temperature and
shape and the outer disc scaleheight ($L_{\rm disc}$). The shape and
location of the inner disc walls are controlled, primarily, by the
stellar rotation rate (and stellar radius) and accretion rate (and
stellar flux). Given that the stellar flux and radius decrease as the
pre-MS star ages (or decreases in mass) the flux levels and the
position of the inner edge are also weakly affected by the age of the
central star. 

As shown in Figure \ref{inner_temp} the location of the inner edge and
the corresponding temperature are correlated (differently) for typical
and extreme accretors. This correlation is dominated by the dust
sublimation temperature for the extreme accretors, and therefore the
radial density profile. However, for the typical accretors the
correlation between temperature and inner radius retains some
dependence on the co-rotation radius (temperature is correlated with
rotation period). This is as the sublimation temperature is not
reached for the longer period models. Therefore, one might expect the
resulting SEDs for typical accretors to show a correlation in shape,
between 2 and 3\,$\mu$m, caused by the inner wall temperature changes,
in turn caused by changes in the rotation period. Figures
\ref{rot_-12_1Myr} and \ref{rot_-9_1Myr} show the SEDs (and
constituent parts as explained in Figures \ref{disc_eg_173}) for a BDD
model at the two rotation periods. The model has a mass of
0.04$M_{\odot}$ and an age of 1 Myr, with accretion rates of -12 and
-9 log$(\frac{\dot{M}}{M_{\odot}}yr^{-1})$ in Figures \ref{rot_-12_1Myr}
and \ref{rot_-9_1Myr} respectively, both at an areal coverage of
10\%. The \textit{top panels} show the SEDs for a rotation period of
0.5 days, and the \textit{bottom panels} the SEDs for a rotation
period of 5 days. The \textit{first}, \textit{second}, \textit{third}
and \textit{fourth panels} then show inclination angles of 0, 56, 64
and 90$^{\circ}$ respectively.

\begin{figure*}
  \vspace*{348pt}
  \special{psfile="Fig/rot_-12_1Myr.ps"
   hoffset=470 voffset=0 hscale=60 vscale=60 angle=90}
 \caption{Figure showing SEDs of a
   system with a central stellar mass of $0.04M_{\odot}$, an age of
   1Myr an accretion rate of -12
   log$(\frac{\dot{M}}{M_{\odot}}yr^{-1})$ with an an areal coverage
   of 10\%. The \textit{panels}, both \textit{top} and
   \textit{bottom}, (from left to right) show the SEDs and constituent
   fluxes at inclination angles of 0$^{\circ}$, 56$^{\circ}$,
   64$^{\circ}$ and 90$^{\circ}$. The component fluxes are presented
   as explained Figure \ref{disc_eg_173}. The \textit{top panels} are
   for systems with a rotation period of 0.5 days and the
   \textit{lower panels} are for a rotation period of 5 days. The
   vertical and horizontal dashed lines (in the \textit{third panels}
     across) denote the approximate sensitivity ranges of our chosen
   $V$, $I$, $J$ and $K$ filters. \label{rot_-12_1Myr}}
\end{figure*}

\begin{figure*}
  \vspace*{348pt}
  \special{psfile="Fig/rot_-9_1Myr.ps"
   hoffset=470 voffset=0 hscale=60 vscale=60 angle=90}
 \caption{Figure showing SEDs of a system with a central stellar mass
   of $0.04M_{\odot}$, an age of 1Myr an accretion rate of -9
   log$(\frac{\dot{M}}{M_{\odot}}yr^{-1})$ with an an areal coverage
   of 10\%. The \textit{panels}, both \textit{top} and
   \textit{bottom}, (from left to right) show the SEDs and constituent
   fluxes at inclination angles of 0$^{\circ}$, 56$^{\circ}$,
   64$^{\circ}$ and 90$^{\circ}$. The component fluxes are presented
   as explained Figure \ref{disc_eg_173}. The \textit{top panels} are
   for systems with a rotation period of 0.5 days and the
   \textit{lower panels} are for a rotation period of 5
   days.  The vertical and horizontal dashed lines (in the
   \textit{third panels across}) denote the approximate sensitivity
   ranges of our chosen $V$, $I$, $J$ and $K$ filters.
   \label{rot_-9_1Myr}}
\end{figure*}

Figures \ref{rot_-12_1Myr} and \ref{rot_-9_1Myr} show that for these
BDD systems with typical accretion rates, the changes in SED with
rotation period are small, apart from perhaps the face-on
configuration. The systems in the \textit{top} and \textit{bottom
  panels} of Figures \ref{rot_-12_1Myr} and \ref{rot_-9_1Myr} have
inner edge temperatures of $~$1470 \& $~$900\,K, and $~$1460 \& $~$940\,K
respectively. This leads to different peak emissions for the slower
and faster rotating stars, of approximately 2 and 3\,$\mu$m. However,
as is shown by Figures \ref{rot_-12_1Myr} and \ref{rot_-9_1Myr} very
little resulting change is apparent in any of the SEDs apart from the
face-on configuration. Essentially, any change in the SED shape caused
by changing emission at the inner disc edge is smoothed out by the
resulting interactions with the rest of the disc structure. As pre-MS
stars age they contract and reduce in luminosity, this will lead to
changes in the co-rotation radius and inner edge temperature. The
models in Figures \ref{rot_-12_1Myr} and \ref{rot_-9_1Myr} are for an
age of 1 Myr. Therefore, changes in the SED could also be more
apparent for older systems, with lower stellar luminosities. Figures
\ref{comp_rot_-12} and \ref{comp_rot_-9} show the same system as
Figures \ref{rot_-12_1Myr} and \ref{rot_-9_1Myr}, i.e. with accretion
rates of -12 and -9 log$(\frac{\dot{M}}{M_{\odot}}yr^{-1})$
respectively, but for ages of 1 and 10 Myrs. In Figures
\ref{comp_rot_-12} and \ref{comp_rot_-9} only the total SED are shown
for the shorter period, a solid black line, and the longer period, as
a dashed blue line. The \textit{top panels} show the SEDs for 1 Myr
systems and the \textit{bottom panels} the 10 Myr systems. The
\textit{first}, \textit{second}, \textit{third} and \textit{fourth
  panels} again show the viewing angles of 0, 56, 64 and 90$^{\circ}$
respectively.

\begin{figure*}
  \vspace*{348pt}
  \special{psfile="Fig/comp_rot_-12.ps"
   hoffset=470 voffset=0 hscale=60 vscale=60 angle=90}
 \caption{Figure showing SEDs of a system with a central stellar mass
   of $0.04M_{\odot}$ and accretion rate of -12
   log$(\frac{\dot{M}}{M_{\odot}}yr^{-1})$ with an an areal coverage
   of 10\%. The SEDs are shown in \textit{each panel} for a rotation
   period of 0.5 (solid black line) and 5 (dashed blue line) days. The
   \textit{panels}, both \textit{top} and \textit{bottom}, (from left
   to right) show the total SEDs inclination angles of 0$^{\circ}$,
   56$^{\circ}$, 64$^{\circ}$ and 90$^{\circ}$. The \textit{top
     panels} are for systems with an age of 1 Myr and the
   \textit{lower panels} are for an age of 10
   Myrs. The vertical and horizontal dashed lines (in the
   \textit{third panels across}) denote the approximate sensitivity
   ranges of our chosen $V$, $I$, $J$ and $K$
   filters\label{comp_rot_-12}}
 \end{figure*}

\begin{figure*}
  \vspace*{348pt}
  \special{psfile="Fig/comp_rot_-9.ps"
   hoffset=470 voffset=0 hscale=60 vscale=60 angle=90}
 \caption{Figure showing SEDs of a system with a central stellar mass
   of $0.04M_{\odot}$ and accretion rate of -9
   log$(\frac{\dot{M}}{M_{\odot}}yr^{-1})$ with an an areal coverage
   of 10\%. The SEDs are shown in \textit{each panel} for a rotation
   period of 0.5 (solid black line) and 5 (dashed blue line) days. The
   \textit{panels}, both \textit{top} and \textit{bottom}, (from left
   to right) show the total SEDs inclination angles of 0$^{\circ}$,
   56$^{\circ}$, 64$^{\circ}$ and 90$^{\circ}$. The \textit{top
     panels} are for systems with an age of 1 Myr and the
   \textit{lower panels} are for an age of 10
   Myrs. The vertical and horizontal dashed lines (in the
   \textit{third panels across}) denote the approximate sensitivity
   ranges of our chosen $V$, $I$, $J$ and $K$ filters
   \label{comp_rot_-9}}
\end{figure*}

Figures \ref{comp_rot_-12} and \ref{comp_rot_-9} show that for typical
accretors the resulting SEDs do not, in general, change significantly
with changing rotation rate. For the face-on configuration
(\textit{far left panels}) the SED slope over 2 to 3\,$\mu$m is
marginally different only. Essentially, flux redistribution from the
highly flared outer regions of the disc and inner edge shape changes
lead to a loss of correlations of SED shape (from 2 to 3\,$\mu$m)with
inner edge temperature. This leads to a loss of any correlation
between IR excess and inner edge location as predicted for the, flat,
discs of \cite{1997AJ....114..288M}.

\subsubsection{SED summary}
\label{sed_summary}

The shape of the combined SED, $L_*+L_{\rm acc}+L_{\rm wall}+L_{\rm
  disc}$, is controlled by the combined stellar flux (intrinsic
stellar temperature and luminosity, and accretion stream temperature
and luminosity), and the disc characteristics (inner edge location and
temperature, and outer disc flaring). The stellar flux and temperature
both increase with increasing accretion rates and reducing areal
coverages. The stellar flux (at constant temperature) also reduces as
a pre-MS star contracts with age (or reduces in mass). Furthermore,
the inner disc flux increases as the inner wall becomes hotter, larger
and more visible with increasing incident flux. The outer disc
component also increases as the outer disc flares. The flaring and
temperature of the disc are controlled by the stellar flux and the
rotation rate. Finally, and most importantly, the shape of the SED
will be dominated by the inclination of the system, especially for
extremely flared systems. This results in the observable SED being a
major function of the input variables, age (and mass), accretion rate
(and areal coverage), rotation period and system inclination.

As we have discussed in this Section and Section \ref{disc_struct} the
underlying disc structures are caused by complicated interplays
between the sublimation physics, vertical hydrostatic equilibrium and
emission of the disc components. However, several underlying trends
have been observed in the physical characteristics of the systems.
Most notably the correlation, to different strengths, of the disc
inner wall location and its resulting temperature, as found in
\cite{1997AJ....114..288M}. However, \cite{1997AJ....114..288M} go on
to predict a correlation of IR excess with inner edge location. Given
that the initial inner edge location is dependent (chiefly) on the
rotation period one might expect a correlation of rotation rate with
IR excess. This has been shown to be unlikely due to complications in
the inner edge location due to dust sublimation, and obscuration of
the inner disc edge at higher inclination angles. In addition, for
large groups of stars there is evidence for a correlation in stellar
mass with accretion rates. Although we have shown that the observable
SEDs of BDD systems at higher accretion rates are difficult to
classify as such, most surveys supporting this relationship use
photometric data. Therefore, we must examine whether these problems
translate into the broadband photometric magnitudes.

\subsection{Photometry}
\label{photometry}

Practically, most parameters for young pre-MS stars are derived from
surveys of populations, usually open clusters, using broadband
photometry and subsequently constructed colour-magnitude and
colour-colour diagrams (CMDs and CoCoDs respectively, hereafter).
Therefore, in this Section, we demonstrate the prohibitive effects on
the broadband photometry of varying our input parameters. In this
Section we, in particular, relate the discussed features of disc
structures (\ref{disc_struct}) and the resulting simulated SEDs
(\ref{sed_analysis}) in terms of the derived photometry, in Section
\ref{indv_phot}. Furthermore, we explore the consequences of our model
grid on the derivation of the primary parameters of age, mass and disc
fractions, from populations, in Section \ref{populations}. This in
turn leads to highlighting selection effects with, for instance, the
mass to accretion rate relation.

\subsection{Effects of Disc Presence and Accretion on Photometry}
\label{indv_phot}

For the derivation of ages optical CMDs, in particular in \textit{V,
  V-I}, are most often used, and indeed most suitable. Whereas, IR
CMDs, such as a \textit{J, J-K} CMD, are most suitable for mass
derivation \cite[see references and discussions in][ and Section
\ref{derived}]{2007MNRAS.375.1220M,2008MNRAS.386..261M}.

As discussed in Section \ref{sed_analysis} the presence of a
circumstellar disc and accretion result in significant changes to the
simulated SEDs for our models. Firstly, the photophere of the central
star is dominated by the accretion luminosities for accretion rates of
$\dot{M}>$10$^{-9}M_{\odot}yr^{-1}$ as shown in Section
\ref{acc_dominance}. Secondly, for increasing accretion rates the
outer disc flaring increases, resulting in occultation of the star at
smaller inclination angles, as discussed in Section
\ref{flare_obscure}. Finally, the location of the inner edge and flux
intercepted by this edge lead to changes in its shape and subsequent
dependence of the observed SED with inclination, as discussed in
Section \ref{inner_shape}. In this Section we explore the effects of
these changes in SED shape on the derived photometry in CMDs used for
age and mass derivation for stars. The use of isochrones for the
derivation of single star parameters is completely unreliable
\citep[see discussion in][]{2008MNRAS.386..261M}. However, exploring
the changes in the CMDs of single objects will elucidate the causes of
scatters from the expected locii in CMDs of populations of BDs.

\subsubsection{Accretion Dominance-Photometry}
\label{acc_dom_phot}

Accretion flux was shown to dominate the underlying, intrinsic,
photospheric SED for accretion rates of
$\dot{M}>$10$^{-9}M_{\odot}yr^{-1}$ (see Figure \ref{acc_dom} in
Section \ref{acc_dominance}). As one would expect this leads to
significant changes in the derived photometric magnitudes for bands
blueward of a few microns, where the accretion and photospheric flux
dominate. This is shown in Figure \ref{naked_acc}. Figure
\ref{naked_acc} shows CMDs in $M_V$, $(V-I)_0$ (\textit{left panels})
and $M_J$, $(J-K)_0$ (\textit{right panels}) of naked stars with
accretion rates of -12, -9 and -8
log$(\frac{\dot{M}}{M_{\odot}}yr^{-1})$, shown as black, red and blue
lines respectively (in \textit{all panels}). The \textit{top panels},
then show the two rotational periods of 5 and 0.5 days, as dashed and
solid lines respectively, for a coverage of 10\%. The \textit{bottom
  panels}, then show the two areal spot coverages of 1 and 10\%, as
dashed and solid lines respectively, for a rotational period of 5
days.

\begin{figure*}
  \vspace*{348pt}
  \special{psfile="Fig/naked_acc.ps"
   hoffset=470 voffset=0 hscale=60 vscale=60 angle=90}
 \caption{Figure showing the naked systems for accretion rates of -8,
   -9 and -12 log$(\frac{\dot{M}}{M_{\odot}}yr^{-1})$, shown in
   \textit{all panels} as blue, red and black lines respectively. The
   model shown for \textit{all panel} has a mass of 0.04$M_{\odot}$
   and an age of 1Myr. The \textit{top panels} are models with areal
   spot coverages of 10\%, and the \textit{lower panels} are for
   models with rotational periods of 5 days. The dashed and solid
   lines then delineate the rotation periods for the \textit{top
     panels} and coverages for the \textit{bottom panels}. The dashed
   lines show the rotational periods of 5 days (\textit{top panels})
   and coverages of 1\% (\textit{bottom panels}). The solid lines
   denotes periods of 0.5 days (\textit{top panels}) and coverages of
   10\% (\textit{bottom panels}). For accretion rates of -12
   log$(\frac{\dot{M}}{M_{\odot}}yr^{-1})$ (black line, \textit{all
     panels}), are representative of all models at this rate, as the
   models are coincident within the CMD.\label{naked_acc}}
\end{figure*}


As the rotation rate increases the co-rotation radius also increases,
as shown in Equation \ref{inner_eq}, $R_{\rm inner}\propto
\tau^{\frac{2}{3}}$. Therefore, as Equations \ref{Lacc} and
\ref{Tacc}, show, $L_{\rm acc}\propto 1- \frac{R_*}{R_{\rm inner}}$
and $T_{\rm acc}\propto (\frac{L_{\rm
    acc}}{A})^{\frac{1}{4}}$. Therefore, the temperature of the
accretion hot spot increases as the coverage lowers and the rotational
period increases (the latter is due to the increase in potential
energy lost by the mass accreted). This can be seen in Figure
\ref{acc_dom} with lower areal coverages and longer rotational periods
leading to accretion fluxes at peaking at shorter wavelengths. The
\textit{left panels} of Figure \ref{naked_acc} show that as the
accretion rate increases the stars move blueward and to brighter
magnitudes in the $M_V$, $(V-I)_0$ CMD. Furthermore, as the accretion
spot temperature is maximised i.e. moving to longer periods of
rotation or smaller areal coverages (solid to dashed lines), the stars
also move blueward. For the \textit{right panels}, or $M_J$, $(J-K)_0$
CMDs the behaviour is less intuitive. Increases in accretion rate lead
to a slight brightening and an initial move redward, followed by a
move blueward. This is as the lower accretion rates have lower hot
spot temperatures, with peak wavelengths between 1 and 3\,$\mu$m
(corresponding to temperatures of $~$2800\,K to $~$900\,K, the range of
the BD photosphere). This leads to increases in the red flux and
changes in the slope from $J$ to $K$. At the higher accretion rates
the accretion flux dominates and cause the blueward shift as the slope
becomes larger. For the highest accretion rate, in the \textit{bottom
  right panel}, a change to a hotter spot (reduction in coverage, from
solid to dashed lines) results in a move redward in the CMD. This can
be understood by observing the solid blue lines in the \textit{top
  panels} of Figure \ref{acc_dom}. Essentially, the reduction in spot
temperature moves the accretion flux to much shorter wavelengths,
$~$0.5 to $~$0.3\,$\mu$m, meaning that the flux in the $J$ and $K$
bands becomes more like the photosphere, and less dominated by
accretion. The \textit{top right panel} of Figure \ref{naked_acc} then
shows that increases in the rotation period (from solid to dashed
lines) shows a move blueward in the CMD, resulting in changes of the
peak accretion flux emission from $~$0.58 to $~$0.5\,$\mu$m, meaning
significant flux from the accretion hot spot will still fall in the
$J$ and $K$ bands, and the slope will be increased.

\subsubsection{Effect of Flaring and Obscuration on Photometry}
\label{flare_obscure_phot}

As described by \cite{2004MNRAS.351..607W} the increased flaring, in
BDD systems when compared to CTTS, leads to more significant changes
in the photometric characteristics of the BDD stars with inclination
changes. The increased flaring shown in Figures \ref{flare_-12_183}
and \ref{flare_-7_193} results in increased obscuration of the central
star (and inner edge) at earlier inclinations. This obscuration is
shown to occur at lower inclination angles for objects accreting at
higher levels, as shown in Figure \ref{flare_inc}. Figure
\ref{flare_inc} shows that significant obscuration occurs at
inclinations of 71 and 56$^{\circ}$ for accretion rates of -12 and -7
log$(\frac{\dot{M}}{M_{\odot}}yr^{-1})$, respectively. The changes in
flux with inclination will clearly lead to a change in the observed
magnitude with inclination. Figure \ref{flare_mag} shows the $M_V$ and
$M_J$ magnitude, in the \textit{top} and \textit{bottom panels}
respectively, as a function of inclination with the individual points
marked as crosses. The \textit{top} and \textit{bottom panels} show
accretion rates of -12, -9 and -7
log$(\frac{\dot{M}}{M_{\odot}}yr^{-1})$ as black, red and blue lines.
The solid lines are for objects with rotational periods of 0.5 days
and the dashed lines are for periods of 5 days. The \textit{inset
  panel} within the \textit{bottom panels} shows an enlargement of the
section of inclinations from 40 to 80$^{\circ}$. For \textit{all
  panels} the vertical dotted lines show the inclinations for which
significant drops in flux are observed (see Figure \ref{flare_inc})
for -12 and -7 log$(\frac{\dot{M}}{M_{\odot}}yr^{-1})$ in black and
blue respectively.

\begin{figure*}
  \vspace*{348pt}
  \special{psfile="Fig/flare_mag.ps"
   hoffset=470 voffset=0 hscale=60 vscale=60 angle=90}
 \caption{Figure showing magnitude change with inclination for $M_V$
   (\textit{top panel}) and $M_J$ (\textit{bottom panel}) for the
   models shown in Section \ref{disc_struct}, Figures \ref{no_sub_183}
   to \ref{sub_181}. The accretion rates of -7, -9 and -12
   log$(\frac{\dot{M}}{M_{\odot}}yr^{-1})$ are shown as blue, red and
   black lines respectively. The individual points are shown as
   crosses for all data. The dashed lines show the systems with
   rotational periods of 5 days and the solid lines 0.5 days, in
   \textit{all panels}. A section of the \textit{bottom panel} has
   been enlarged and is shown as the \textit{inset panel}. Note the
   models for -12 log$(\frac{\dot{M}}{M_{\odot}}yr^{-1})$ are coincident
   in this Figure. \textit{All panels} show the inclinations are which
   obscuration occurs for -12 and -7
   log$(\frac{\dot{M}}{M_{\odot}}yr^{-1})$ as dashed black and blue
   lines respectively. \label{flare_mag}}
\end{figure*}

Figure \ref{flare_mag} shows that the occultation starts earlier for
the higher accretion rates, shown by the increase in both magnitudes
(\textit{top} and \textit{bottom panels}) with inclination angle at
around 56 and 71$^{\circ}$, as also shown in Figure
\ref{flare_inc}. The slower rotation rates, for the highest accretion
rates, show a generally fainter magnitude, than the faster rotating
systems, caused by a larger disc scaleheight. For $M_J$ the fall in
magnitude is again earlier and more significant for the larger
accretion rates. However, at higher inclinations the systems become
brighter again, show in the \textit{inset panel}. This brightening is
due to the curved inner wall, as shown in Figures \ref{inner_inc_sed}
(discussed in more detail in Section \ref{inner_edge_phot}.

The change in magnitude, will also lead to a change in colour. As
described $V$, $V-I$ and $J$, $J-K$ CMDs are ubiquitously used for
derivation of ages and masses. The use of isochrones to derive ages
and masses of individual stars is not reliable. However, it is
informative to explore the changes in the CMD position of our BDD
systems. Figure \ref{flare_VVI} shows $M_V$, $(V-I)_o$ CMDs for the
accretion rates of -12 and -7 log$(\frac{\dot{M}}{M_{\odot}}yr^{-1})$,
as the \textit{left} and \textit{right panels} respectively. The
systems are those shown in Figure \ref{flare_-12_183} and
\ref{flare_-7_193}, at all modeled inclinations. For \textit{both
  panels} the red dashed line, with the data points marked as crosses,
shows the changes in CMD position with inclination of the BDD system
at different inclinations for a 1 Myr system (with an areal coverage
of 10\% and a $\tau$ of 5 days). The solid blue line is the matching
isochrone for naked systems, with the naked $M_*=0.04M_{\odot}$ marked
as a triangle. Again, for \textit{both panels} the inclinations at
which significant drops in flux were found in Figure \ref{flare_inc}
are noted in red text.

\begin{figure*}
  \vspace*{348pt}
  \special{psfile="Fig/flare_VVI.ps"
   hoffset=470 voffset=0 hscale=60 vscale=60 angle=90}
 \caption{Figure showing the positions within a $M_V$, $(V-I)_0$ of
   the models in Figures \ref{flare_-12_183} and \ref{flare_-7_193}
   (Section \ref{disc_struct}) as the \textit{left} and \textit{right
     panels} respectively. The solid blue line are the 1Myr naked
   isochrones at the accretion rates of -12 and -7
   log$(\frac{\dot{M}}{M_{\odot}}yr^{-1})$ for the \textit{left} and
   \textit{right panels} respectively (with an areal coverage of 10\%
   and a $\tau$ of 5 days). The dashed red line is the BDD system with
   $M_*=0.04M_{\odot}$ at all simulated inclinations. The triangles,
   in both panels are the 0.04$M_{\odot}$ stars on the naked
   isochrones and the crosses show this mass over the range of
   inclinations with an associated disc. Figure \ref{flare_inc}
   predicts an occultation angle of 71 and 56$^{\circ}$ for the -12
   and -7 log$(\frac{\dot{M}}{M_{\odot}}yr^{-1})$ models respectively,
   and these are denoted by the text within the
   figure.\label{flare_VVI}}
\end{figure*}

Figure \ref{flare_VVI} shows that as the system changes in inclination
it leads to a significant dimming in both cases. The inclinations from
Figure \ref{flare_inc} where significant flux is lost are shown and it
can be seen that a significant drop in magnitude is apparent just
before this inclination (as also shown in Figure \ref{flare_mag}). The
models in \textit{both panels} get slightly redder as we increase to
the obscuration angle, caused by scattered light from the disc this
can be observed for the accretion rate of -12
log$(\frac{\dot{M}}{M_{\odot}}yr^{-1})$ in Figures \ref{rot_-12_1Myr}
and \ref{comp_rot_-12}.

Figure \ref{flare_JJK} shows the same data as Figure \ref{flare_VVI}
with the same notation and symbol meanings.

\begin{figure*}
  \vspace*{348pt}
  \special{psfile="Fig/flare_JJK.ps"
   hoffset=470 voffset=0 hscale=60 vscale=60 angle=90}
 \caption{Figure showing the positions within a $M_J$, $(J-K)_0$ of
   the models in Figures \ref{flare_-12_183} and \ref{flare_-7_193}
   (Section \ref{disc_struct}) as the \textit{left} and \textit{right
     panels} respectively. The solid blue line are the 1Myr naked
   isochrones at the accretion rates of -12 and -7
   log$(\frac{\dot{M}}{M_{\odot}}yr^{-1})$ for the \textit{left} and
   \textit{right panels} respectively (with an areal coverage of 10\%
   and a $\tau$ of 5 days). The dashed red line is the BDD system with
   $M_*=0.04M_{\odot}$ at all simulated inclinations. The triangles,
   in both panels are the 0.04$M_{\odot}$ stars on the naked
   isochrones and the crosses show this mass over the range of
   inclinations with an associated disc. Figure \ref{flare_inc}
   predicts an occultation angle of 71 and 56$^{\circ}$ for the -12
   and -7 log$(\frac{\dot{M}}{M_{\odot}}yr^{-1})$ models respectively,
   and these are denoted by the text within the
   figure.\label{flare_JJK}}
\end{figure*}

Figure \ref{flare_JJK} again, as in Figure \ref{flare_VVI}, shows the
BDD systems dimming with increasing inclination, with some move
towards redder colours. There is an offset between the naked and BDD
systems at face on inclinations, caused by the inner disc wall. This
is particularly large for the higher accretion rate (\textit{right
  panel}). This increase in reddening for the higher accretion rates
is due to the much larger and curved inner disc edge as shown in
Figure \ref{sub_179}. This curved inner edge also leads to less
dependence of $(J-K)_0$ on inclination angle, as compared to the lower
accretion rate.

\subsubsection{Inner Edge and Photometry}
\label{inner_edge_phot}

It is clear that the scatter within a CMD, in particular for the
$M_J$, $(J-K)_0$ CMD, is strongly affected by the emission from the
inner edge (as shown in Figure \ref{flare_JJK}).

Figure \ref{inner_VVI} shows the position in a CMD of the models from
Figures \ref{sub_169}, \ref{sub_175}, \ref{sub_179} and
\ref{sub_181}. The accretion rates shown are -12, -7, -9 and -6
log$(\frac{\dot{M}}{M_{\odot}}yr^{-1})$, as the \textit{top left},
\textit{bottom left}, \textit{top right} and \textit{bottom right
  panels}. The solid blue line in \textit{all panels} of Figure
\ref{inner_VVI} show the naked isochrone at 1 Myr (with an areal
coverage of 10\% and $\tau$ of 0.5 days), at the shown accretion rate.
The dashed red line shows the BDD system matching the naked isochrone,
except with $M_*=0.04M_{\odot}$, at all inclination rates, with the
datapoints marked as crosses. The blue triangles are then the naked
stars at $M_*=0.04M_{\odot}$. The points at which the thermal direct
flux reaching the observer drops significantly are marked in red text
for each system, as shown in Figures \ref{typ_curve} and
\ref{ext_curve}.

\begin{figure*}
  \vspace*{348pt}
  \special{psfile="Fig/inner_VVI.ps"
   hoffset=470 voffset=0 hscale=60 vscale=60 angle=90}
 \caption{Figure showing the positions within a $M_V$, $(V-I)_0$ of
   the models in Figures \ref{sub_169}, \ref{sub_175}, \ref{sub_179}
   and \ref{sub_181} as the \textit{top left}, \textit{bottom left},
   \textit{top right} and \textit{bottom right panels}
   respectively. As for Figure \ref{flare_VVI}, the solid blue line is
   the naked isochrone at 1 Myr for the corresponding accretion rate
   (-12, -9, -7 or -6 log$(\frac{\dot{M}}{M_{\odot}}yr^{-1})$). The
   dashed red line is the BDD isochrone at the same age and accretion
   rate. The inclination of each model at which obscuration is
   expected to occur, from Figures \ref{typ_curve} and
   \ref{ext_curve}, is denoted in text.\label{inner_VVI}}
\end{figure*}

Figure \ref{inner_VVI} shows for the $M_V$, $(V-I)_0$, the inclination
increase again dims the star for these models and moves it initially
redward, until it exceeds the expected obscuration angle, found from
Figures \ref{typ_curve} and \ref{ext_curve}. However, as we move
through the accretion rates from -12
log$(\frac{\dot{M}}{M_{\odot}}yr^{-1})$ (\textit{top left panel}) to -6
log$(\frac{\dot{M}}{M_{\odot}}yr^{-1})$ (\textit{bottom right panel})
the transition from edge-on to face-on system tends to a sharper
feature, with the reddest point at the obscuration angle. As would be
expected from the SEDs shown in Figure \ref{inner_inc_sed}. This is
caused by maximising the amount of the inner edge which is visible,
whilst the stellar photosphere is still visible (which also
contributes in these passbands).

Figure \ref{inner_JJK} shows the same information as Figure
\ref{inner_VVI} but for a $M_J$, $(J-K)_0$ CMD. Here, the BDD
isochrones are initially much redder than their naked counterparts, as
expected due to IR excesses caused by the disc inner edges. As the
inclination increase for the lower accretion rates (\textit{top left}
and \textit{bottom left panels}) the system moves first redward and
then sharply blueward as the SED becomes dominated by scattered light
only. As can be seen in Figure \ref{inner_inc_sed}. For the extreme
accretion rates the wall is no longer vertical and the change in
infrared colours is smoother up until the obscuration angle.

\begin{figure*}
  \vspace*{348pt}
  \special{psfile="Fig/inner_JJK.ps"
   hoffset=470 voffset=0 hscale=60 vscale=60 angle=90}
 \caption{Figure showing the positions within a $M_J$, $(J-K)_0$ of
   the models in Figures \ref{sub_169}, \ref{sub_175}, \ref{sub_179}
   and \ref{sub_181} as the \textit{top left}, \textit{bottom left},
   \textit{top right} and \textit{bottom right panels}
   respectively. As for Figure \ref{flare_VVI}, the solid blue line is
   the naked isochrone at 1 Myr for the corresponding accretion rate
   (-12, -9, -7 or -6 log$(\frac{\dot{M}}{M_{\odot}}yr^{-1})$). The
   dashed red line is the BDD isochrone at the same age and accretion
   rate. The inclination of each model at which obscuration is
   expected to occur, from Figures \ref{typ_curve} and
   \ref{ext_curve}, is denoted in text.\label{inner_JJK}}
\end{figure*}

\subsection{Populations}
\label{populations}

Practically, the derivation of parameters for pre-MS populations uses
CMDs and isochrones applied to clusters or populations of stars. In
this section we explore the scatters of our BDD systems across our
grid of input values within typical or commonly used CMDs and CoCoDs
for deriving ages, masses and disc fractions.

To delineate the effects of the accretion rate and circumstellar discs
we have subdivided the grid into two groups (as discussed in Sections
\ref{disc_struct} and \ref{sed_analysis}), those with accretion rates
typical for higher mass CTTS objects, defined as
$\dot{M}=$10$^{-12}M_{\odot}yr^{-1}$ (negligible) to
$\dot{M}=$10$^{-9}M_{\odot}yr^{-1}$ and those with elevated accretion
rates, where $\dot{M}>$10$^{-9}M_{\odot}yr^{-1}$. For several of the
plots in this section the magnitude and colours for the
$M_*=0.01M_{\odot}$ systems at high inclinations become extremely
faint and red. In some cases these objects appear slightly without the
scale of the diagram the axes were limited in this way to be able to
show the changes in colour and magnitude for the majority of stars
better. The colours and magnitudes for these lowest mass stars are not
necessarily unreliable but simply hinder the aesthetics of the plots,
and as they are at the limits of our grid we have decided to omit them
from some Figures by trimming the axes. Additionally, stars at the
highest inclinations, i.e edge on disc systems, are often omitted from
the Figures due to their extremely faint magnitudes, meaning they
would not be practically observable.

\subsubsection{Age and Mass Derivations}
\label{age_mass_pop}

As discussed in Section \ref{indv_phot} for the derivation of the
primary parameters of age and mass the CMDs most suitable and commonly
used are $M_V$, $(V-I)_0$ and $M_J$, $(J-K)_0$. Furthermore, these
CMDs are indicative of the associated CMDs, for instance $M_R$,
$(R-I)_0$ or $M_J$, $(J-H)_0$ etc. Also as discussed in
\cite{2008MNRAS.386..261M} the use of isochrones for the derivation of
masses and ages for individual pre-MS stars is unreliable at
best. Practically, therefore, median ages are derived from
populations. Subsequently, derived masses are still unreliable but at
least based on a consistent age. This problem is being addressed by
Bell et al (in prep), where $K$ band photometry and known eclipsing
binaries are being used to refine pre-MS isochrones. In this section
we plot the data for our 1 Myr systems only and explore the resulting
scatters caused by the disc presence and accretion luminosity.

Figures \ref{spread_opt_VVI_typ} and \ref{spread_opt_VVI_ext} shows
four CMDs in $M_V$, $(V-I)_0$ for the typical and extreme accretors
respectively. For the typical accretors, Figure
\ref{spread_opt_VVI_typ}, shows the isochrones constructed for naked
systems at 1 Myr, for accretion rates of -12 and -9
log$(\frac{\dot{M}}{M_{\odot}}yr^{-1})$, as the solid and dashed black
lines. Whereas in Figure \ref{spread_opt_VVI_ext} the naked isochrones
of -9 and -6 log$(\frac{\dot{M}}{M_{\odot}}yr^{-1})$, as dashed and
solid black lines respectively. The \textit{top left panel}, in both
Figures (\ref{spread_opt_VVI_typ} and \ref{spread_opt_VVI_ext}) shows
a pre-MS isochrone at 1 Myr from \cite{2000A&A...358..593S} adjusted
to a distance of 250pc and an extinction of $A_V=2$ mag, simulating a
background population of CTTS stars. The dots in \textit{each panel},
for both Figures (\ref{spread_opt_VVI_typ} and
\ref{spread_opt_VVI_ext}) are the 1 Myr BDD systems for all
inclinations, rotational periods, areal spot coverages and the
individual accretion rates within the accretion type classes. The
\textit{panels} of the Figures \ref{spread_opt_VVI_typ} and
\ref{spread_opt_VVI_ext} then separate the systems by the input
variables. The \textit{top left panel} shows either the accretion
rates of -9, -10 and -11 \& -12 or -6, -7 and -8
log$(\frac{\dot{M}}{M_{\odot}}yr^{-1})$ in Figures
\ref{spread_opt_VVI_typ} and \ref{spread_opt_VVI_ext}
respectively. These three classes are then shown as blue, black and
red dots respectively. The \textit{bottom left panels} then show those
systems with areal coverages of 1 and 10\% as blue and red dots
respectively. The \textit{top right panels} then shows systems with
rotational periods of 0.5 and 5 days as blue and red dots
respectively. Finally, the \textit{bottom right panels} show the
systems separated by inclination. These groups are $\theta \leq
48^{\circ}$ as blue dots (classed as face-on systems), $\theta> 56$ \&
$64^{\circ}$ (classed as the expected systems, as the expectation
value of $cos(\theta)=60^{\circ}$) as black dots and $\theta \geq
71^{\circ}$ (classed as edge-on systems) as red dots. Figure
\ref{spread_opt_RRI_typ} also shows the same data as for Figure
\ref{spread_opt_VVI_typ}, but for $M_R$, $(R-I)_0$ CMDs.

\begin{figure*}
  \vspace*{348pt}
  \special{psfile="Fig/spread_opt_VVI_typ.ps"
   hoffset=470 voffset=0 hscale=60 vscale=60 angle=90}
 \caption{Figure showing CMDs in $M_V$, $(V-I)_0$ for typical
   accretors ($\dot{M} \leq$ -9
   log$(\frac{\dot{M}}{M_{\odot}}yr^{-1})$). The black solid and dashed
   lines in \textit{all panels} are the naked star isochrones for
   accretion rates of -12 and -9 log$(\frac{\dot{M}}{M_{\odot}}yr^{-1})$
   respectively. The \textit{top left} only shows the 1 Myr pre-MS
   isochrone of \citet{2000A&A...358..593S} adjusted to a distance
   modulus of 7 and an extinction of $A_V=2$, simulating a reddened
   background population. The dots are the data for typical accretors
   for models at 1 Myr. The data are then separated by input
   variables, in the \textit{panels}. The \textit{top left panel}
   shows three accretion classes, -9, -10 and -11 \& -12
   log$(\frac{\dot{M}}{M_{\odot}}yr^{-1})$, as blue, black and red dots
   respectively (the lowest two accretion rates are grouped as they
   are coincident). The \textit{top right panel} shows the rotational
   periods of 0.5 and 5 days as blue and red dots respectively. The
   \textit{bottom left panel} shows the areal spot coverages of 1 and
   10\% as blue and red dots resepctively. The \textit{bottom right
     panel} shows three groups of inclinations, $\theta \leq$
   48$^{\circ}$, $\theta \geq$ 77$^{\circ}$ and $\theta=$ 56, 64 \&
   71$^{\circ}$ , as blue, red and black dots
   respectively.\label{spread_opt_VVI_typ}}
\end{figure*}

As can be seen in Figure \ref{spread_opt_VVI_typ} current accretion in
a star and disc systems, for rates typical in the BD mass regime,
creates a scatter in our simulated photometry indicative of a much
larger isochronal age spread than 10 Myr. Indeed, for many BDD systems
even at nominal accretion rates of -11 or -12
log$(\frac{\dot{M}}{M_{\odot}}yr^{-1})$, for our simulations, the
colours of these stars move significantly blueward of the expected BD
locus in a $M_V$, $(V-I)_0$ CMD. As the dots in Figure
\ref{spread_opt_VVI_typ} are the simulated photometry of BDD systems
over a range of typical input parameters (see Section
\ref{par_space}), an observed coeval 1 Myr population could reasonably
be expected to show a similar scatter. The naked isochrones show that
this spread in simulated photometry, would lead to a spread in
isochronal age greater than the $\approx$10 Myr spread claimed for
higher mass stars in some star forming regions (for instance the
ONC). Furthermore, for the higher accretion rates of -9 or -10
log$(\frac{\dot{M}}{M_{\odot}}yr^{-1})$ the movement of the star
within the CMD will effectively move the star into the contamination
region expected for background CTTS or MS stars at a $(V-I)_0$ of
$\leq$1.5 and $M_V$ 12-10, and as such the star would not be included
in a photometrically selected BD sample. The solid green line, in the
\textit{top left panel}, showing the 1 Myr isochrone of
\cite{2000A&A...358..593S} at a distance of 250 pc and extinction of
$A_V=2$ mags shows that the BDD systems with higher accretion rates
could easily be confused for a background CTTS or MS population (as MS
stars are simply less luminous at a roughly constant \textit{V-I} than
the CTTS counterparts). Indeed miss classification of a BDD system as
a CTTS system has already been revealed in
\cite{2003ApJ...582.1109W}. Furthermore, scatter, although somewhat
reduced, can be observed in the simulated photometry of the negligibly
accreting systems, for example in the \textit{top left panel} some red
dots, accreting at -12 or -11 log$(\frac{\dot{M}}{M_{\odot}}yr^{-1})$
(which will be negligible compared to the photospheric flux, see
Figure \ref{acc_dom}), still scatter significantly. These objects may
still be included in a `wide' photometric selection. This scatter is a
strong function of inclination, as shown in the \textit{bottom right
  panel}, where, as the inclination is increased the objects are
pushed lower in the CMD. Indeed, for the edge on cases some objects
have magnitudes fainter than show in Figure \ref{spread_opt_VVI_typ}
($M_V\approx 20$). This is expected as the star becomes obscured by
the flared disc, interestingly for these systems the \textit{bottom
  right panel} shows that this occurs for inclinations above around
71$^{\circ}$ (as found in Figure \ref{flare_inc}) in most cases.
However, even for the lower inclination angles some objects have very
faint magnitudes, this is due to the disc flaring leading to a smaller
opening angle and is discussed in Section \ref{disc_struct}. The
\textit{bottom left panel} shows that for naked systems accretion
rates of -9 log$(\frac{\dot{M}}{M_{\odot}}yr^{-1})$ lead to the stars
being scattered into the region occupied by background CTTS and
contamination (the lower accretion rates are all coincident with the
-12 log$(\frac{\dot{M}}{M_{\odot}}yr^{-1})$ for 1 Myr). Crucially, the
\textit{top left panel} shows that the scatter from the isochrone is,
generally, correlated with accretion rate. Effectively, as the
accretion rate increases the BDD system moves farther away from the
isochrone and is therefore less likely to be classified as a BDD
system and included in any target samples of such objects. The
\textit{top right panel} shows that the scatter, in $M_V$, in our
simulated photometry from the isochrones is significant for both areal
coverages. The \textit{bottom left panel}, shows significant scatter
in the simulated photometry for both adopted rotation periods. This
panel also shows that scatter in this diagram is not obviously
correlated with the rotation period and therefore inner edge position.
Overall, as found in \cite{2004MNRAS.351..607W}, the dominant
scattering effect for BD disc systems in an optical CMD appears to be
caused by accretion rate and inclination, and therefore obscuration
effects of the disc on the star. This suggests that for a given
photometric survey of BDD systems accreting at typical accretion rates
and with an expected range of inclinations (centred on around
60$^{\circ}$) one would expect to exclude a significant fraction of
these objects from any isochrone based selection.

\begin{figure*}
  \vspace*{348pt}
  \special{psfile="Fig/spread_opt_VVI_ext.ps"
   hoffset=470 voffset=0 hscale=60 vscale=60 angle=90}
 \caption{As Figure \ref{spread_opt_VVI_typ} except for extreme
   accretors, $\dot{M}\geq$ -8
   log$(\frac{\dot{M}}{M_{\odot}}yr^{-1})$. In this Figure the solid
   black line now shows naked systems with an accretion rate of -6
   log$(\frac{\dot{M}}{M_{\odot}}yr^{-1})$ (as opposed to -12
   log$(\frac{\dot{M}}{M_{\odot}}yr^{-1})$, as in Figure
   \ref{spread_opt_VVI_typ}, the dashed black line is still for an
   accretion rate of -9
   log$(\frac{\dot{M}}{M_{\odot}}yr^{-1})$. \label{spread_opt_VVI_ext}}
\end{figure*}

Figure \ref{spread_opt_VVI_ext} shows that for extreme accretion rates
the BDD objects move significantly blueward and brighter than the
naked isochrone (the naked isochrone at -9
log$(\frac{\dot{M}}{M_{\odot}}yr^{-1})$ is shown for
comparison). Comparison of the \textit{four panels} again show that,
in general, the scatter is chiefly correlated with accretion rate and
inclination. Essentially, Figure \ref{spread_opt_VVI_ext} shows that
for, admittedly extreme, accretion rates BDD systems will not be
included in a photometrically (or indeed spectroscopically, from
Section \ref{sed_analysis}) selected sample.

\begin{figure*}
  \vspace*{348pt}
  \special{psfile="Fig/spread_opt_RRI_typ.ps"
   hoffset=470 voffset=0 hscale=60 vscale=60 angle=90}
 \caption{As Figure \ref{spread_opt_RRI_typ}, but in $M_R$, $(R-I)_0$,
   with the pre-MS isochrone omitted.\label{spread_opt_RRI_typ}}
\end{figure*}

Similar scattering, as observed in Figure \ref{spread_opt_VVI_typ},
occurs when using CMDs constructed using $M_R$, $(R-I)_0$ \citep[using
the photometric system of][]{1998A&A...333..231B}, except that the 1
Myr pre-MS isochrone has been omitted. Figure \ref{spread_opt_RRI_typ}
shows the $M_R$, $(R-I)_0$ CMD.

Similar consequences are apparent in CMDs best suited, and most used,
for derivation of masses using isochrones. Figure
\ref{spread_ir_JJK_typ} shows the same data as Figure
\ref{spread_opt_VVI_typ} in the same format except in this case the
pre-MS isochrone of \cite{2000A&A...358..593S} has been smoothed,
purely for aesthetics and to remove the sharper features. Again Figure
\ref{spread_ir_JJK_ext} contains the same data in the same format as
Figure \ref{spread_opt_VVI_ext} except that the naked star isochrone
at an accretion rate of -6 log$(\frac{\dot{M}}{M_{\odot}}yr^{-1})$ is
not shown as it lies significantly blueward of the CMD.

\begin{figure*}
  \vspace*{348pt}
  \special{psfile="Fig/spread_ir_JJK_typ.ps"
   hoffset=470 voffset=0 hscale=60 vscale=60 angle=90}
 \caption{As Figure \ref{spread_opt_VVI_typ}, in $M_J$, $(J-K)_0$, in
   this Figure the pre-MS isochrone of \citet{2000A&A...358..593S},
   has been smoothed.\label{spread_ir_JJK_typ}}
\end{figure*}

\begin{figure*}
  \vspace*{348pt}
  \special{psfile="Fig/spread_ir_JJK_ext.ps"
   hoffset=470 voffset=0 hscale=60 vscale=60 angle=90}
 \caption{As for Figure \ref{spread_opt_VVI_ext}, except that the -6
   log$(\frac{\dot{M}}{M_{\odot}}yr^{-1})$ isochrone is not shown (lies
   significantly blueward of locus of data), and the pre-MS isochrone
   has been smoothed.\label{spread_ir_JJK_ext}}
\end{figure*}

Figure \ref{spread_ir_JJK_typ} shows that for typical accretion rates
and expected ranges of the other input variables the spread in CMD
positions is large. Indeed. the \textit{top left panel} again shows
that, as in the $M_V$ $(V-I)_0$ case, some of the BDD objects will
appear as redenned background CTTS objects. Scrutiny of the
\textit{individual panels} shows no clear correlation of CMD position
with input variable except for the inclination. As shown in the
\textit{bottom right panel} as the inclination increases the systems
move to fainter magnitudes and redder colours, in general. The
\textit{top right panel} of Figure \ref{spread_ir_JJK_typ} shows that
for typical accretion rates there does not appear to be a correlation
of $(J-K)_0$ colour with rotation rate as expected from the modeling
of flat disc by \cite{1997AJ....114..288M}. It is clear that assigning
masses to accreting BDD systems using isochrones, for our model grid,
would result in the incorrect masses. In addition, photometric sample
selection for BD systems will also exclude the majority of our higher
accretion rate objects. Figure \ref{spread_ir_JJK_ext} shows that for
extreme accretors the scatter is much larger, and again correlated
only with inclination, and perhaps accretion rate to a lesser degree
(\textit{bottom right} and \textit{top left panels} of Figure
\ref{spread_ir_JJK_ext}). 

\subsubsection{Summary}

Therefore, if one adopts the range of input parameters we have used
(see Section \ref{par_space} for justification), our simulated
photometry shows, qualitatively, that a coeval 1 Myr population of
accreting BD stars and BDD systems, with typical accretion rates and
range of inclinations, will exhibit a significant scatter in apparent
isochronal age of ($>$10 Myr). Furthermore, objects with typical (and
extreme) accretion rates are scattered sufficiently in CMD space to
prohibit their identification as pre-MS BDs. Indeed, these objects
would not be included in a photometrically selected sample of BDs, and
as such are unlikely to be assigned the correct masses or ages. The
scatter from the naked 1 Myr systems, generally, increases with
increasing accretion rate. It is important to note at this point that
these conclusions are qualitative, and obviously based on our
assumptions. However, the repercussion for isochronal age derivation
and sample selection in the BD regime could be profound. Indeed this
study has only included the effects of current or ongoing accretion
from a disc, it has neglected any effects of accretion, both past and
present, on the evolution of the central star. This past accretion
could also act to reduce the stars radius, accelerating contraction
\citep{1999MNRAS.310..360T,1999A&A...342..480S} and introducing
additional scatter in a coeval population proportional to the range in
accretion rates \citep[see][for full discussion]{2008MNRAS.386..261M}.
Our findings support those of \cite{2004MNRAS.351..607W}, showing that
a significant scatter is caused by obscuration of the central source
by the central disc. In addition we have shown that the typical
accretion rates may scatter BDD systems into the region of a CMD
occupied by the CTTS or background MS locus.

The fact that scatter in the CMD increases with increasing accretion
rate casts doubt on the veracity of the mass to accretion rate
relationship. For our model grid we have not assumed any such
relation, therefore, as our data would also show a similar relation it
suggests that the observed result may be caused by intrinsic
scattering. The relation, $\dot{M}\propto M_*^{2}$ suggests that their
is a dearth of lower mass stars accreting at higher rates. We have
shown that for accretion rates in the range -12 to -9
log$(\frac{\dot{M}}{M_{\odot}}yr^{-1})$ the BDD systems with higher
accretion rates would be preferentially missed using photometric or
isochronal selection.

Additionally, we have adopted the co-rotation radius as the initial
location of our inner wall. This leads to an initial proportionality
between the rotation rate and inner edge temperature. As discussed in
Section \ref{disc_struct} this correlation will be modified due to
dust sublimation effects, weakening the correlation. The derived
photometry however, at varying inclinations, does not show a strong
correlation in IR, $(J-K)_0$, colour with rotational rate, as proposed
by \cite{1997AJ....114..288M}.

\subsubsection{Disc fractions}
\label{disc_fractions_phot}

Disc fractions have been derived using infrared excesses previously in
\textit{JHK}, however recent works pre-dominantly use \textit{Spitzer}
IRAC magnitudes. Furthermore, MIPS magnitudes are used to identify
so-called debris discs, where IR excesses are not apparent at shorter
wavelengths. Finally, disc fractions have also been derived using the
$\alpha$ criteria, where $\alpha=\frac{dlog\lambda
  F\lambda}{dlogF\lambda}$ between two limiting wavelengths,
originally used to distinguish amongst Class I, II or II sources, but
now used to detect disc presence
\citep{2006AJ....131.1574L,2009arXiv0901.2603K}. An $\alpha>$-2 is
used as a selection criterion for disc presence for TTS stars. We have
constructed the $\alpha$ values for our model grid by adopting the
limiting wavelengths of \cite{2009arXiv0901.2603K}, namely 3.6 to
8.0\,$\mu$m.

As shown in Figure \ref{inner_temp}, Section \ref{disc_struct}, the
inner edge temperature for typical accretors retains some correlation
with rotation rate. In general the more slowly rotating objects having
cooler inner edges. For the more extreme rotators the inner edge
location is still weakly correlated with the inner edge
temperature. In the extreme accretors case the correlation is caused
by the radial fall in density, and therefore dust sublimation
temperature as the inner edge is ablated. Therefore, for the extreme
accretors the correlation between rotational period and inner edge
temperature is lost. The inner edge temperature is in fact correlated
with the flux and therefore accretion rate of the system. As shown in
Figures \ref{spread_ir_JJK_typ} and \ref{spread_ir_JJK_ext} (in
Section \ref{age_mass_pop}), there is no clear correlation in CMD
position, and $(J-K)_0$ colour with rotation rate for either typical
or extreme accretors. In fact any scatter in these Figures
(\ref{spread_ir_JJK_typ} and \ref{spread_ir_JJK_ext}) appears to be
most strongly correlated with accretion rate and inclination.

Figures \ref{disc_frac_JHJK_typ} and \ref{disc_frac_JHJK_ext} show the
$(J-H)_0$, $(J-K)_0$ CoCoDs for typical and extreme accretors
respectively. The dots are the simulate BDD systems for both ages (as
position in these CoCoDs is not a strong function of age) and masses,
and the black crosses the naked systems, again at both ages (and all
masses), for the accretion rates matching the shown BDD data. The
\textit{panels} of these Figures then separate the BDD systems by
accretion rate (\textit{top left}), areal spot coverage
(\textit{bottom left}), rotational period (\textit{top right}) and
inclination (\textit{bottom right}). The accretion rates of -9, -10
and -11 \& -12 log$(\frac{\dot{M}}{M_{\odot}}yr^{-1})$ are shown as
blue, black and red dots respectively in the \textit{top left panel}
of Figure \ref{disc_frac_JHJK_typ}. For Figure
\ref{disc_frac_JHJK_ext}, the \textit{top left panel} shows accretion
rates of -6, -7 and -8 log$(\frac{\dot{M}}{M_{\odot}}yr^{-1})$ are shown
as blue, black and red dots respectively. The \textit{bottom left
  panels} of both Figures \ref{disc_frac_JHJK_typ} and
\ref{disc_frac_JHJK_ext} show the systems with areal spot coverages of
1 and 10\% as blue and red dots respectively. The \textit{top right
  panels} of both Figures \ref{disc_frac_JHJK_typ} and
\ref{disc_frac_JHJK_ext} show the systems with rotational periods of
0.5 and 5 days as blue and red dots respectively. Finally, The
\textit{bottom right panels} of both Figures \ref{disc_frac_JHJK_typ}
and \ref{disc_frac_JHJK_ext} show the systems with inclinations of
$\theta \leq 48^{\circ}$ as blue dots, $\theta> 56$ \& $64^{\circ}$ as
black dots and $\theta \geq 71^{\circ}$ as red dots.

\clearpage

\begin{figure*}
  \vspace*{348pt}
  \special{psfile="Fig/disc_frac_JHJK_typ.ps"
   hoffset=470 voffset=0 hscale=60 vscale=60 angle=90}
 \caption{Figure showing $(J-H)_0$, $(J-K)_0$ colour-colour diagram
   for typical accretors ($\dot{M}<leq$ -9
   log$(\frac{\dot{M}}{M_{\odot}}yr^{-1})$). The dots are then all
   models (at 1 and 10 Myrs) with discs and the black crosses are the
   naked systems, in \textit{all panels}. The data are then separated
   by input variable in \textit{each panel}. The \textit{top left
     panel} shows accretion rates of -9, -10 and -11 \& -12
   log$(\frac{\dot{M}}{M_{\odot}}yr^{-1})$ as blue, black and red dots
   respectively. The \textit{top right panel} shows the systems with
   rotational periods of 0.5 and 5 days as blue and red dots
   respectively. The \textit{bottom left panel} shows systems with
   areal spot coverages of 1 and 10\% as blue and red dots
   respectively. Finally, the \textit{bottom right panel} separates
   the systems into three groups inclination, $\theta \leq$
   48$^{\circ}$, $\theta \geq$ 77$^{\circ}$ and $\theta=$ 56, 64 \&
   71$^{\circ}$ , as blue, red and black dots
   respectively.\label{disc_frac_JHJK_typ}}
\end{figure*}

\begin{figure*}
  \vspace*{348pt}
  \special{psfile="Fig/disc_frac_JHJK_ext.ps"
   hoffset=470 voffset=0 hscale=60 vscale=60 angle=90}
 \caption{As for Figure \ref{disc_frac_JHJK_typ} but for extreme
   accretors, $\dot{M} \geq$ -8 log$(\frac{\dot{M}}{M_{\odot}}yr^{-1})$
   .\label{disc_frac_JHJK_ext}}
\end{figure*}

Figures \ref{disc_frac_JHJK_typ} and \ref{disc_frac_JHJK_ext} show
that for typical and extreme accretion rates there is no clear
correlation of rotation rate, areal coverage or inclination with the
position in the CoCoD. Significantly no clear correlation between IR
colours and rotation rate can be observed for the typical accretors.
This suggests that the weak correlation found between inner edge
temperature and rotational period does not produce a matching
correlation with IR colours. For the extreme accretors, however, a
correlation between IR colours and accretion rate is clear. As the
accretion rate increases the systems move to redder CoCoDs. This is
caused, as discussed by the dust inner edge being sublimated and the
resulting sublimation decreasing, with density, radially. Therefore,
one can not extend the predictions of \cite{1997AJ....114..288M},
after adoption of a co-rotation radius to suggest that IR excesses may
be correlated with rotation rates, for typical or extreme accretors.
As can be seen for all the panels of Figures \ref{disc_frac_JHJK_typ}
and \ref{disc_frac_JHJK_ext} there is a slight overlap between the
naked or disc less objects (crosses) and the BDD systems (dots). This
overlap is small, only some of the lowest mass naked objects appear
redward of the main population. Generally, a well placed cut should
identify a large proportion of the disc candidates as is well known,
if there is little complication from variable extinction and a well
defined photometric system. However, the fact that an overlap exist at
all will lead to some confusion as to the position of any disc excess
cut. For real data a cut will be placed at a user identified paucity
or gap in the CoCoD. Therefore, in practice, if any other region of
the population appears equally, or more, sparse the cut is likely to
be miss-placed. In practice these disc fractions are known to be lower
limits of the true values, so ostensibly a miss placed cut, missing
some disc candidates is not problematic.

The analysis in Section \ref{age_mass_pop} and Figures
\ref{disc_frac_JHJK_typ} and \ref{disc_frac_JHJK_ext}, shows that
accretion rate and inclination are the dominant variables controlling
the photometric scatter, of the BDD systems. Therefore, for our
subsequent analysis we will separate the systems using only these
input variables.

Figures \ref{disc_frac_irac_typ} and \ref{disc_frac_irac_ext} show
CoCoDs for colours constructed using simulated IRAC photometry, for
the typical and extreme accreting systems respectively. These two
Figures contain the same data as Figures \ref{disc_frac_JHJK_typ} and
\ref{disc_frac_JHJK_ext}, respectively. The naked systems are again
shown as crosses and the BDD systems as dots. The \textit{left panels}
of these Figures show $([3.6]-[4.5])_0$, $([4.5]-[5.8])_0$ CoCoDs and
the \textit{right panels} $([3.6]-[4.5])_0$, $([5.8]-[8.0])_0$
CoCoDs. The \textit{top panels} then show the accretion rates of -9,
-10, -11 \& -12 log$(\frac{\dot{M}}{M_{\odot}}yr^{-1})$), for Figure
\ref{disc_frac_irac_typ}, and -6, -7 and -8
log$(\frac{\dot{M}}{M_{\odot}}yr^{-1})$), for Figure
\ref{disc_frac_irac_ext}, as blue, black and red dots
respectively. The \textit{bottom panels} of both Figures show the
systems with inclinations of $\theta \leq 48^{\circ}$ as blue dots,
$\theta> 56$ \& $64^{\circ}$ as black dots and $\theta \geq
71^{\circ}$ as red dots.

\begin{figure*}
  \vspace*{348pt}
  \special{psfile="Fig/disc_frac_irac_typ.ps"
   hoffset=470 voffset=0 hscale=60 vscale=60 angle=90}
 \caption{Figure showing typically accreting systems ($\dot{M}\leq$-9
   log$(\frac{\dot{M}}{M_{\odot}}yr^{-1})$), in irac photometric
   bands. The \textit{left panels} show the $([3.6]-[4.5])_0$ against
   $([4.5]-[5.8])_0$ CoCoD, and the \textit{right panels} show the
   $([3.6]-[4.5])_0$ against $([5.8]-[8.0])_0$ CoCoD. In \textit{all
     panels} the dots are disc systems and crosses naked systems. The
   \textit{top panels} then separate the systems by accretion rate
   with, -9, -10 and -11 \& -12 log$(\frac{\dot{M}}{M_{\odot}}yr^{-1})$
   as blue, black and red dots respectively. The \textit{bottom
     panels} separate the systems by inclination with $\theta \leq$
   48$^{\circ}$, $\theta \geq$ 77$^{\circ}$ and $\theta=$ 56, 64 \&
   71$^{\circ}$ , as blue, red and black dots
   respectively. \label{disc_frac_irac_typ}}
\end{figure*}

\begin{figure*}
  \vspace*{348pt}
  \special{psfile="Fig/disc_frac_irac_ext.ps"
   hoffset=470 voffset=0 hscale=60 vscale=60 angle=90}
 \caption{As for Figure \ref{disc_frac_irac_typ} but for extreme
   accretors, $\dot{M} \geq$ -8
   log$(\frac{\dot{M}}{M_{\odot}}yr^{-1})$.\label{disc_frac_irac_ext}}
\end{figure*}

Figure \ref{disc_frac_irac_typ} and \ref{disc_frac_irac_ext} show that
the naked and BDD systems are well separated. Supporting the view that
IRAC magnitudes are well suited for disc candidate identification. For
the typical accretors there is a possible correlation between
inclination and position in the $([3.6]-[4.5])_0$, $([5.8]-[8.0])_0$
CoCoD, (\textit{bottom right panel} of Figure
\ref{disc_frac_irac_typ}). The systems which are edge-on
(i.e. $\theta\geq$77$^{\circ}$) appear closer in colour to the naked
systems. In the case of extreme the accretors the separation between
the naked and BDD systems is correlated with accretion rate for both
CoCoDs (\textit{top panels} of Figure
\ref{disc_frac_irac_ext}). Effectively, as the accretion rate
increases, in general the system moves to redder colours.

Figure \ref{disc_frac_irac_mips_typ} and
\ref{disc_frac_irac_mips_ext}, show the same data as Figure
\ref{disc_frac_irac_typ} and \ref{disc_frac_irac_ext}, in the same
format and with the same notations. Figures
\ref{disc_frac_irac_mips_typ} and \ref{disc_frac_irac_mips_ext} show
CoCoDs of simulated photometry from IRAC and MIPS magnitudes. For both
Figures the \textit{left panels} show $([3.6]-[4.5])_0$,
$([8.0]-24)_0$ and the \textit{right panels} $([3.6]-[5.8])_0$,
$([8.0]-24)_0$ CoCoD, where 24, is the 24\,$\mu$m MIPS channel.

\begin{figure*}
  \vspace*{348pt}
  \special{psfile="Fig/disc_frac_irac_mips_typ.ps"
   hoffset=470 voffset=0 hscale=60 vscale=60 angle=90}
 \caption{Figure showing the $([3.6]-[4.5])$, $([8.0]-24)$ CMDs as the
   \textit{left panels} and $([3.6]-[5.8])$, $([8.0]-24)$ CMDs as the
   \textit{right panels}, combining IRAC and MIPS photometric
   channels. The systems are again separated by accretion rate and
   inclination angle in the \textit{top} and \textit{bottom} panels as
   in figure \ref{disc_frac_irac_typ}.\label{disc_frac_irac_mips_typ}}
\end{figure*}

\begin{figure*}
  \vspace*{348pt}
  \special{psfile="Fig/disc_frac_irac_mips_ext.ps"
   hoffset=470 voffset=0 hscale=60 vscale=60 angle=90}
 \caption{As for Figure \ref{disc_frac_irac_mips_typ} but for extreme
   accretors ($\dot{M} \geq$ -8
   log$(\frac{\dot{M}}{M_{\odot}}yr^{-1})$).\label{disc_frac_irac_mips_ext}}
\end{figure*}

Figures \ref{disc_frac_irac_mips_typ} and
\ref{disc_frac_irac_mips_ext} show that the BDD and naked systems are
very well separated in CoCoDs utilising IRAC and longer wavelength
MIPS data. For the systems with typical accretion rates, Figure
\ref{disc_frac_irac_mips_typ}, no correlation between CoCoD position
and accretion rate is obvious (\textit{top panels}). The \textit{lower
  panels} of this Figure show that, in general, the locus of BDD stars
is spread similarly for all but edge-on systems, which sometimes
appear a little redder in $([8.0]-24)_0$. For the extreme accretors,
Figure \ref{disc_frac_irac_mips_ext}, however, correlations in CoCoD
position with both inclination and accretion rate are clear. The
\textit{top panels} of Figure \ref{disc_frac_irac_mips_ext} show that
as the accretion rate increases, in general, the systems move to
redders colours in $([3.6]-[4.5])_0$ or $([3.6]-[5.8])_0$
(\textit{left} and \textit{right panels} respectively). This is to be
expected from our previous analysis of IRAC CoCoDs using this
photometry. The \textit{lower panels} of Figure
\ref{disc_frac_irac_mips_ext} show a correlation of inclination angle
with the colours in both axes. These two correlations essentially mean
that we can roughly identify systems, with regard to inclination and
accretion rate, by their position within this CoCoD. For instance,
extreme accretors with lower accretion rates seen face-on will occupy
the bottom left of the BDD locus, with edge-on systems with the
maximum accretion rate appearing to the top right. Comparing the
scales on Figures \ref{disc_frac_irac_mips_typ} and
\ref{disc_frac_irac_mips_ext} shows that these systems slightly
overlap, but in general the extreme accretors will lie redward of the
typical accretor locus in both colours.

Figure \ref{disc_frac_mips} shows CoCoDs constructed solely from the
longer wavelength MIPs photometry. Figure \ref{disc_frac_mips} shows
the BDD data shown in previous Figures (such as Figure
\ref{disc_frac_irac_typ} and \ref{disc_frac_irac_ext}) in $(24-70)_0$,
$(70-160)_0$ CoCoDs. The \textit{left} and \textit{right panels} show
the typically and extreme accreting systems, respectively. The
\textit{inset panels}, in \textit{both top panels}, show larger areas
to include the naked systems, as they are well removed from the BDD
locus.

\begin{figure*}
  \vspace*{348pt}
  \special{psfile="Fig/disc_frac_mips.ps"
   hoffset=470 voffset=0 hscale=60 vscale=60 angle=90}
 \caption{Figure showing CMDs of mips photometry for the typical
   ($\dot{M} \leq$ -9 log$(\frac{\dot{M}}{M_{\odot}}yr^{-1})$) and
   extreme ($\dot{M} \geq$ -8 log$(\frac{\dot{M}}{M_{\odot}}yr^{-1})$)
   accretors, as the \textit{left panels} and \textit{right panels}
   respectively. The \textit{top panels} separate the systems by
   accretion rate in the way described in Figures
   \ref{disc_frac_irac_typ} and \ref{disc_frac_irac_ext} in the
   \textit{left} and \textit{right panels} respectively. The
   \textit{bottom panels} then separate the systems by inclination as
   described in Figure \ref{disc_frac_irac_typ}. The \textit{inset
     panels} in the \textit{top panels} then show a larger region with
   the BDD systems as red dots and the naked systems as blue
   crosses.\label{disc_frac_mips}}
\end{figure*}

Figure \ref{disc_frac_mips} shows, in the \textit{inset panels}, that
BDD and naked systems are very well separated in CoCoDs constructed
using MIPs photometry. For the typical accretors, \textit{left panel},
the position within the CoCoD appears strongly correlated with
inclination only, with the edge on systems moving away from the main
BDD locus, as expected. For the extreme accretors, \textit{right
  panel}, significantly more structure is apparent. In this case as
the accretion rate increases the systems move redward in both colours.
Additionally, as the inclination increases the systems move to redder
$(24-70)_0$ colours. This suggests, as with Figure
\ref{disc_frac_irac_mips_ext}, for objects with extreme accretion
rates, CoCoDs of this sort may help distinguish the characteristic
inclination and accretion rate of the system. In practice, due to
saturation and reddening effects and small sample sizes this may prove
difficult. For our study this is interesting in that the fluxes for
wavelengths longward of 24\,$\mu$m come from regions at temperatures of
$~$120 K and below. This suggests that the outer disc flaring, as
discussed in Section \ref{disc_struct}, leads to significantly
structural changes in the outer, cooler disc, as a function of
accretion rate. In addition, observing the separation of the naked and
BDD systems within the CoCoDs it is clear that for systems within our
grid, the best disc indicators are at longer wavelengths, as is well
known from observations.

Disc fractions are also derived using the $\alpha$ value
\citep{2006AJ....131.1574L}, essentially a slope of the SED between
two wavelengths (at wavelengths longer than the stellar flux peak).
Figure \ref{disc_frac_alpha} shows the derived $\alpha$ values between
3.6 and 8.0\,$\mu$m for our entire model grid (against an arbitrary
model number), with dots showing BDD systems and crosses showing naked
systems. The \textit{left} and \textit{right panels} of Figure
\ref{disc_frac_alpha} show the typical and extreme accreting systems
respectively. For TTS $\alpha>-$2 distinguishes between systems with
and without discs \citep{2009arXiv0901.2603K}. As can be seen in
Figure \ref{disc_frac_alpha} almost all the typical, and all of the
extreme accreting BDD systems would be successfully identified using
this criterion. Suggesting that the $\alpha$ value is a reliable disc
indicator for BD systems. The \textit{top panels} of Figure
\ref{disc_frac_alpha} show the alpha values separated by accretion
rates, with -9, -10 and -11 \& -12, or -6, -7 and -8
log$(\frac{\dot{M}}{M_{\odot}}yr^{-1})$ shown as blue, black and red
dots respectively. The \textit{bottom panels} differentiate systems
seen at different inclinations with $\theta \leq 48^{\circ}$ as blue
dots, $\theta> 56$ \& $64^{\circ}$ as black dots and $\theta \geq
71^{\circ}$ as red dots.

\begin{figure*}
  \vspace*{348pt}
  \special{psfile="Fig/disc_frac_alpha.ps"
   hoffset=470 voffset=0 hscale=60 vscale=60 angle=90}
 \caption{Figure showing $\alpha$ value \citep{2006AJ....131.1574L}
   (against an arbitrary model number) used
   \citet{2009arXiv0901.2603K}, $\alpha=\frac{dlog(\lambda
     F_{\lambda})}{dlog(\lambda)}|^{3.6}_{8.0}$ ($\lambda$ in
   $\mu$m). Crosses show the simulated objects without discs and dots
   are those with circumstellar disc, the horizontal line is the
   $\alpha>-$2 cut used to identify disc candidates for solar type
   stars in \citet{2009arXiv0901.2603K}. Note, that stars of
   $M_*<$0.01$M_{\odot}$ are not included due to unreliable flux
   estimates, in the atmosphere models, for longer wavelengths
   ($\lambda>$4\,$\mu$m). The \textit{left panels} show typical
   ($\dot{M} \leq$ -9 log$(\frac{\dot{M}}{M_{\odot}}yr^{-1})$) accretors
   and the \textit{right panels} the extreme ($\dot{M} \geq$ -8
   log$(\frac{\dot{M}}{M_{\odot}}yr^{-1})$) accretors. The \textit{top
     panels} separate the systems by accretion rate in the way
   described in Figures \ref{disc_frac_irac_typ} and
   \ref{disc_frac_irac_ext} in the \textit{left} and \textit{right
     panels} respectively. The \textit{bottom panels} then separate
   the systems by inclination as described in Figure
   \ref{disc_frac_irac_typ}.\label{disc_frac_alpha}}
\end{figure*}

Figure \ref{disc_frac_alpha} shows that except for a few systems
($~\frac{1}{50}$), all of which are typical accretors at edge-on
inclinations, a cut of $\alpha>-$2 would identify all of our BDD
systems. Furthermore, the \textit{top left} panel shows that for
typically accreting systems there would be no significant bias in
selection with accretion rate. The \textit{bottom left panel} shows
that the lowest $\alpha$ values are only apparent for the edge-on
systems. The \textit{right panels} of Figure \ref{disc_frac_alpha},
show that for extreme accretors the systems are well above the disc
discriminator line, and move to greater $\alpha$ values with
increasing accretion rate.

In summary, for our model grid, the best disc discriminators are MIPs
CoCoDs and the $\alpha$ values. With, in general, correlations in
position within the CoCoDs or the $\alpha$ values seen with accretion
rate only for the extreme accretors.

\subsubsection{Practical Observations}
\label{object}

Some recent studies of disc fractions for BD populations have used
data from the \textit{Spitzer} IRAC camera. Figures \ref{irac_cut},
\ref{irac_cut_2} and \ref{irac_cut_3} show the simulated photometry
for our complete model grid. Naked BDs are shown as crosses and BDD
systems as dots, with 1 and 10 Myrs data shown as blue and red dots
respectively. The dashed lines are cuts used in three recent
publications, Figure \ref{irac_cut} is from \cite{2005ApJ...631L..69L}
a study of IC348, Figure \ref{irac_cut_2} from
\cite{2008ApJ...688..362L} a study of $\sigma$ Orionis and Figure
\ref{irac_cut_3} a study of the Taurus region from
\cite{2008ApJ...674..336G}. In these cases the effects of extinction
are either negligible in the plotted colours, with values of
$E([3.6]-[4.5])<0.04$ and $E([4.5]-[5.8])<0.02$ for IC348 and $A_V
\leq$4 mag \citep[which will be negligible in the IRAC
CoCoD,][]{2004ApJS..154..363A}, or the cuts have been placed in
intrinsic colour space as for $\sigma$ Orionis.

\begin{figure*}
  \vspace*{348pt}
  \special{psfile="Fig/irac_cut.ps"
   hoffset=470 voffset=0 hscale=60 vscale=60 angle=90}
 \caption{Figure showing all simulated photometry for the entire
   studied parameter range at ages 1Myr and 10Myr as dots (blue and
   red respectively), and the naked photometry as crosses. The dashed
   lines are a recent BD disc excess cuts used in
   \citet{2005ApJ...631L..69L} for IC348 at a nominal age of 3-4Myr
   \citet{2008MNRAS.386..261M}. It must be noted that the naked stars
   of $M=$0.01$M_{\odot}$ are not included in this figure as their
   SEDs do not extend far enough into the IRAC passbands to derive
   reliable colours. The \textit{left panel} shows the typical
   ($\dot{M} \leq$ -9 log$(\frac{\dot{M}}{M_{\odot}}yr^{-1})$) accretors
   and the \textit{right panel} the extreme ($\dot{M} \geq$ -8
   log$(\frac{\dot{M}}{M_{\odot}}yr^{-1})$) accretors.\label{irac_cut}}
\end{figure*}

\begin{figure*}
  \vspace*{348pt}
  \special{psfile="Fig/irac_cut_2.ps"
   hoffset=470 voffset=0 hscale=60 vscale=60 angle=90}
 \caption{Figure showing an observational cut used in
   \citet{2008ApJ...688..362L} for data of the $\sigma$ Orionis
   cluster. It must be noted that the naked stars of
   $M=$0.01$M_{\odot}$ are not included in this figure as their SEDs
   do not extend far enough into the IRAC passbands to derive reliable
   colours. The systems are separated by age as described in Figure
   \ref{irac_cut}. The \textit{left} and \textit{right panels} again
   show the typical ($\dot{M} \leq$ -9
   log$(\frac{\dot{M}}{M_{\odot}}yr^{-1})$) and extreme ($\dot{M} \geq$
   -8 log$(\frac{\dot{M}}{M_{\odot}}yr^{-1})$) accretors respectively as
   in Figure \ref{irac_cut}.\label{irac_cut_2}}
\end{figure*}

\begin{figure*}
  \vspace*{348pt}
  \special{psfile="Fig/irac_cut_3.ps"
   hoffset=470 voffset=0 hscale=60 vscale=60 angle=90}
 \caption{Figure showing an observational cut used in
   \citet{2010arXiv1004.2541M} for data of the Taurus cloud. The
   criteria applied for BDD system selection are that of
   \citet{2008ApJ...674..336G}. It must be noted that the naked stars
   of $M=$0.01$M_{\odot}$ are not included in this figure as their
   SEDs do not extend far enough into the IRAC passbands to derive
   reliable colours. The systems are separated by age as described in
   Figure \ref{irac_cut}. The \textit{left} and \textit{right panels}
   again show the typical ($\dot{M} \leq$ -9
   log$(\frac{\dot{M}}{M_{\odot}}yr^{-1})$) and extreme ($\dot{M} \geq$
   -8 log$(\frac{\dot{M}}{M_{\odot}}yr^{-1})$) accretors respectively as
   in Figure \ref{irac_cut}.\label{irac_cut_3}}
\end{figure*}

As can be seen from Figures \ref{irac_cut}, \ref{irac_cut_2} and
\ref{irac_cut_3} almost all of the disc systems from our simulated
photometry would be correctly identified using these cuts. It is
important to note that our conclusions so far have been drawn from
differential photometric arguments, in this case we are using
intrinsic colours and these values are extremely sensitive to changes
in zero point and photometric calibration. Figure \ref{irac_cut} shows
that other than a few typically accreting systems the observational
cut of \cite{2005ApJ...631L..69L} would select all of the BDD systems
in our grid. Figure \ref{irac_cut_2} shows that the cut of
\cite{2008ApJ...688..362L} would miss some typical and a very small
number of extreme accreting BDD systems. Figure \ref{irac_cut_3} shows
only very few BDD systems will be missed. The number of BDD systems
not identified for Figure \ref{irac_cut_2} and \ref{irac_cut_3} is
larger than that of \ref{irac_cut}, but these Figures show that there
is no preferential selection based on age. A comparison between Figure
\ref{irac_cut_2} and the \textit{bottom right panel} of Figure
\ref{disc_frac_irac_typ}, shows that all of the missed BDD systems are
edge-on systems. Given, that extreme reddening may move naked stars
into the BDD region selected, in general, dis fractions are usually
quoted as lower limits. Therefore, a small number of miss-identified
BDD system is a small effect. This allows us to conclude, that
ubiquitously used BDD cuts used in IRAC CoCoDs are reliable when
applied to our simulated dataset.

In a future publication we will release a fitting tool to be used in
conjunction with the online database of naked and BDD systems (for
both photometry and SED fitting). Therefore, we will be performing
fits to observed systems using our grid. In this work we simply pick
an example star and highlight its position in CMD and CoCoD space,
compared to models covering the expected range of parameters.
\cite{2004A&A...424..603N} derive an accretion rate of -9
log$(\frac{\dot{M}}{M_{\odot}}yr^{-1})$ for $\rho$ Ophiucus 102. For
this object \cite{2002A&A...393..597N} find a spectral type of M6,
$T_{\rm eff}$ of 2700\,K, an extinction of $A_V$=3.0, a mass of
$M_*$0.04-0.08$M_{\odot}$ and a distance of 150pc. Photometry for
$\rho$ Ophiucus 102 is also published in \cite{2006A&A...452..245N},
in $J$, $H$ and $K$. Additionally, $R$ and $I$ photometry has been
published for this object in \cite{2005AJ....130.1733W}. We have
adjusted our model spectra in the range $M_*$=0.04-0.08$M_{\odot}$,
with accretion rates of -9 log$(\frac{\dot{M}}{M_{\odot}}yr^{-1})$
(for all remaining input parameters) to the correct distance and
reddening. Figure \ref{rho_oph_102} shows, as dots, our BDD systems
with the parameters in the range quoted, with $\rho$ Ophiucus 102 is
shown as a black star. The \textit{panels} of Figure \ref{rho_oph_102}
then show example CMDs and CoCoDs. The \textit{top left panel} shows a
$M_J$, $(J-H)_0$ CMD, the \textit{bottom left panel} a $M_R$,
$(R-I)_0$ CMD, the \textit{top right panel} a $M_J$, $(J-K)_0$ CMD and
finally, the \textit{bottom right panel} a $(J-K)_0$, $(J-H)_0$ CoCoD.

\begin{figure*}
  \vspace*{348pt}
  \special{psfile="Fig/rho_oph_102.ps"
   hoffset=470 voffset=0 hscale=60 vscale=60 angle=90}
 \caption{Figure showing CMDs and CoCoDs of systems with matching
   parameters to $\rho$ Ophiucus 102 from
   \citet{2002A&A...393..597N}. The object $\rho$ Ophiucus 102 is
   shown in \textit{all panels} as a black star, the BDD systems are
   the separated by age, with the 1 and 10 Myr systems appearing as
   blue and red dots respectively. The \textit{top left panel} shows a
   $M_J$, $(J-H)_0$ CMD, the \textit{top right panel} a $M_J$,
   $(J-K)_0$ CMD, the \textit{bottom left panel} a $M_R$, $(R-I)_0$
   CMD and the \textit{bottom right panel} shows a $(J-H)_0$,
   $(J-K)_0$ CoCoD.\label{rho_oph_102}}
\end{figure*}

We have not performed a formal fitting of the object $\rho$ Ophiucus
102. However, qualitatively, Figure \ref{rho_oph_102} shows that for
$\rho$ Ophiucus 102 the colours are well matched by our grid, yet the
magnitude is somewhat brighter. This could be an indication of
uncertainties in the distance to the object, reddening or magnitudes.
A full investigation is beyond the scope of this paper, but, as stated
in a future publication we will release an online fitting tool to be
used in conjunction with our grid.

\section{Conclusions}
\label{conclusions}

We have constructed a model grid of SEDs, and subsequently photometric
magnitudes and colours, for actively accreting BDs with or without an
associated accretion disc. We have modeled the photospheric flux from
these BDs by adopting (and interpolating) the interior `DUSTY00'
models of \cite{2000ApJ...542..464C} combined with the `AMES-Dusty',
atmospheric models of \cite{2000ApJ...542..464C}. We have then assumed
that accretion occurs from an inner edge of a magnetically truncated
accretion disc (truncated at the co-rotation radius). The accretion
flux is calculated using a simple blackbody emission, given the
derivation of a characteristic spot effective temperature. SEDs were
then produced for both naked BDs and BDD systems. For the BDD systems
we have modeled the disc using the TORUS radiative transfer code using
the Lucy radiative transfer algorithm and incorporating dust
sublimation and including a treatment of vertical hydrostatic
equilibrium (see Section \ref{model} for a discussion of the code). To
produce a `grid' of simulated systems we have varied several input
parameters namely: stellar mass, stellar age, stellar rotation rate,
accretion rate, the areal coverage of the accretion stream and the
system inclination (the disc mass was fixed). The ranges of these
variables were selected to represent and bound typical pre-MS BD
systems, justification is provided using evidence from observational
studies in Section \ref{par_space} and a final list of the values of
these variables can be found in Table \ref{par_space_table}.

Accepting our assumptions, parameter ranges and radiative transfer
code our resulting simulated dataset has allowed us to qualitatively
explore the effects of \emph{active} (current not past accretion)
accretion on disc structure. Furthermore through the simulation of
observations we have explored the effects of accretion, and disc
presence, on both the SEDs, and photometric colours and magnitudes of
these systems. 

As discussed in Section \ref{disc_struct} vertical hydrostatic
equilibrium, when applied to BDs, leads to increased flaring, when
compared to CTTS. This has previously been explored by
\cite{2004MNRAS.351..607W}. However, in our study we have included a
simple treatment of accretion. This leads to increased flaring as more
flux reaches the outer disc, and subsequently lower opening angles for
BDD systems with higher accretion rates. Furthermore, the addition of
dust sublimation has shown that for BDD systems the inner disc
location, temperature and vertical size \& shape also varies with
accretion rate. The inner edge position is correlated with temperature
for the lower accreting models as suggested by
\cite{1997AJ....114..288M}. For the systems with higher accretion
rates the inner edge temperature is weakly correlated with
temperature, mainly due to the radial fall in density and therefore
dust sublimation temperature. The inner disc edge, initially
prescribed as a vertical wall, then becomes concave and finally convex
as dust sublimation is increased (with increasing flux from higher
rates of accretion).

Subsequently, the SEDs of BD systems with typical accretion rates and
associated discs, are changed significantly from the assumed
underlying photospheric model flux, and therefore become difficult to
classify. In Section \ref{sed_analysis} we have shown that the BD
photosphere becomes `swamped' or overwhelmed by the accretion flux for
rates of $\dot{M}>$-9 log$(\frac{\dot{M}}{M_{\odot}}yr^{-1})$. The outer
disc flaring observed in the BDD systems was shown to cause
occultation and a subsequent, sharp, fall in flux at an inclination
which decreases for more systems with higher accretion rates. The
thermal direct flux emanating from the disc inner edge was found to
fall less dramatically with inclination for the systems with a curved
inner boundary, as expected from the work of
\cite{2007ApJ...661..374T}. 

Subsequent derivation of photomertric magnitudes has allowed us to
demonstrate that, as expected, increased accretion without disc
presence, moves our naked systems to bluer and brighter magnitudes.
Once a disc is added the increase in accretion flux interacts with the
disc and does not necessarily lead to a simple motion toward brighter
magnitudes and bluer colours. The increased flaring and obscuration
present in BDD systems, over CTTS, leads to rapid falls in magnitude
with inclination as an accretion (or flaring) dependent inclination.
Furthermore, the disc inner edge leads to a shift redwards with
increasing accretion rate as more flux is intercepted by the inner
edge and the inner edge becomes convex and `puffed up'. For more
typically accreting BDD systems with vertical inner edges, the motion
in a CMD, with inclination, is more disjointed. 

In practice, however, most parameters for BDD systems are derived for
populations. We have shown, in Section \ref{photometry} that
derivation of an \emph{isochronal} (or photometric) age from our
simulated photometry of a coeval BD sample, with typical accretion
rates and associated circumstellar discs, would be inaccurate and
exceedingly difficult. Indeed, the resulting photometric colours and
magnitudes could be indicative of a more distant redenned CTTS
population. For more extreme accretion rates the scatter, in CMD
space, is significantly far from the pre-MS locus and as such these
stars have little chance of being selected as BDs. As discussed in
Section \ref{results} this does not include any effects due to past
accretion on the evolution of the central star, which acts to
accelerate the gravitational contraction and make the star appear
older \citep{1999MNRAS.310..360T,1999A&A...342..480S}, further
scattering the apparent age of a coeval population.

Concordantly, \emph{isochronal} derivations of mass and therefore
IMFs, for our simulated photometry, of a coeval population of
accreting BDs with associated discs, would be inaccurate and
problematic. Again caused by the changes in the SEDs as a result of
the accretion flux and increased occultation by the larger degree of
flaring seen in BD discs \citep[for the latter, as found
by][]{2004MNRAS.351..607W}

We have also qualitatively explored the effects of accretion and disc
presence in our simulated dataset on disc fraction estimates. As is
currently well known, longer wavelength bandpasses are much more
reliable and suitable for disc identification. As shown in Section
\ref{results} the naked and BDD disc loci were much more clearly
separated in the CoCoD constructed using \textit{Spitzer} IRAC
magnitudes than the shorter wavelength CIT \textit{JHK} passbands. In
addition, we that the slope of the SED from 3.6 to 8.0\,$\mu$m, or
$\alpha$ value, is an effective disc indicator. We have also
tentatively shown that current observational cuts, when applied to our
simulated photometry (with its associated photometric system), results
in the reliable detection of disc candidates, for IRAC and MIPS
colours and $\alpha$ values, and therefore a robust lower limit disc
fraction.

A further, derivative area this study impacts on, and perhaps most
significantly, is the recent evidence for a stellar mass to accretion
rate correlation, of the form: $\dot{M_{acc}}\propto M_{*}^{~2}$
\citep{2003ApJ...592..266M,2004A&A...424..603N,2006A&A...452..245N}.
This relationship has been extended into the BD mass regime in
\cite{2006A&A...452..245N}. However, arguments based on selection and
detection thresholds have already cast this relation into doubt
\citep{2006MNRAS.370L..10C}. As we have shown in Section \ref{results}
a relationship of this kind is self-reinforcing as lower mass objects
with higher accretion rates have little chance of being correctly
identified as such due to both the accretion flux and flared
associated disc. Essentially, at present it is unclear how many BD
stars are not included in this relationship due to misidentification.
As explained in \cite{2004MNRAS.351..607W}, BD systems with a disc,
without including accretion effects, can have the characteristics of
higher mass CTTS stars, due to increased disc flaring from a reduced
surface gravity in the disc. The effects of accretion at typical or
larger rates further exacerbate the situation both spectroscopically,
as the photospheric flux essentially becomes swamped or completely
veiled, and photometrically as the resulting colours and magnitudes
are significantly shifted. Therefore, for our simulated dataset a
relationship of this sort may well be derived, if typical methods are
used to identify BD objects with discs and derive masses, ages and
accretion rates, even though it is not present.

Finally, although inner edge locations are correlated with their
temperature we do not find a resulting correlation with IR excess. As
our initial inner edge locations are placed at the co-rotation radius
one might expect a correlation between rotation rate and IR excess.
This in turn might suggest that studies of disc presence correlation
with slower rotation rates, exploring disc-locking, may have intrinsic
biases. However, for our systems with dust sublimation, vertical
flaring, accretion and view over a range of inclinations any
correlation is not apparent. The study of \cite{1997AJ....114..288M}
used analytical prescribed flat discs. Whether, any correlation is
apparent for such systems using our models is studied in an upcoming
paper where we introduce a fitting tool to be used with the model grid
and release a further set of models with analytic disc structures, and
an increased parameter space.

\subsection{Summary}
\label{conc_summary}

In summary, our simulated dataset shows that for typical parameter
ranges for BD stars and BDD systems, disc presence and accretion flux
lead to:

Difficulty deriving the following stellar parameters for a coeval
population:

\begin{itemize}
\item{Isochronal ages}
\item{Isochronal masses}
\item{IMFs}
\end{itemize}

And we have shown:
\begin{itemize}
\item{\textit{Spitzer} IRAC magnitudes are required for reliable disc
    identification}
\item{An expected correlation in inner disc edge with IR excess does
    not occur for systems with dust sublimation and vertical
    hydrostatic equilibrium view over a range of inclinations.}
\item{Low mass, high accretion rate systems are likely to be
    misidentified and therefore not included in any study relating
    $M_*\propto\dot{M}$}
\end{itemize}

\section[]{ACKNOWLEDGMENTS}

\bibliographystyle{mn2e}
\bibliography{references}
\appendix

\section{Consistency Checks and Problems}
\label{consistency}

Firstly, a check was made on the photospheric input flux and the
resulting stellar direct flux (tagged by {\sc torus}) after the radiative
transfer simulation. The resulting flux distributions should match
most closely for face-on configurations, and then match in shape only,
with the stellar direct flux level dropping towards higher
inclinations, as more photons are scattered and absorbed by the disc.

We also directly compared the magnitudes and colours of our naked BD
systems with the lowest accretion rate (-12
log$(\frac{\dot{M}}{M_{\odot}}yr^{-1})$ to those published in
\cite{2000ApJ...542..464C}, in the same photometric system
(\textit{CIT}). We found significant colour differences ($\delta
(J-K)\le$ 0.1), between our derived values and those of
\cite{2000ApJ...542..464C}. As a further check we derived the
magnitudes in the \cite{1998A&A...333..231B} system (by both adopting
the published zero points, and by using a Vega reference spectrum),
and applied conversions of \cite{1992ApJS...82..351L}\footnote{We have
  also included the wavelength shift mention in
  \cite{2004PASP..116....9S}}, to the \textit{CIT} system. For each
method we failed to match the magnitudes and colours published in
\cite{2000ApJ...542..464C}. As a final test we passed the downloaded,
unaltered, atmospheric spectra directly through the filter response
program, without interpolation, for the closest matches in log(g) and
$T_{\rm eff}$ from the interior models published in
\cite{2000ApJ...542..464C}. These magnitudes and colours also failed
to match. Therefore, we must conclude that the most likely cause of
the mismatch is due to improvements in the model atmospheres available
online \footnote{http://perso.ens-lyon.fr/france.allard/} (this is
likely as the models available online, have a later time stamp,
$\approx$ 2005 compared to 2000). For the final published magnitudes,
for the naked systems, we have used a similar wavelength resolution as
in our BDD systems, i.e. 200 logarithmically spaced points. This means
that magnitudes derived from these spectra will differ slightly from
those derived from the full spectra, but this effect is negligible,
and increased resolution for only some of our model grid (i.e naked
stars) will hamper comparison between the models.

The final test of the derived colours and magnitudes was a comparison
of the naked systems with the results for the almost face-on BDD
systems. The results for the optical passbands should be similar and
an appraisal of the component SED, i.e. showing the stellar direct
flux, as can be seen in Figure \ref{disc_eg_173}.

We have adopted zeropoints derived using a Vega reference spectrum for
the optical and near-IR passbands. As a test we have compared our
derived zeropoints using the filter response of
\cite{1998A&A...333..231B} and the Vega reference spectrum against
those published in \cite{1998A&A...333..231B}. For our photometric
systems we integrate the summed number of photons counted by the
simulated telescope systems, however to test the zeropoints and match
the system of \cite{1998A&A...333..231B} we must integrate the summed
energy. The zeropoints we derived \citep[with the values of][in
parenthesis]{1998A&A...333..231B} were: $U=$20.977(20.94),
$B=$20.499(20.498), $V=$21.116(21.10), $R=$21.676(21.655),
$I=$22.376(22.371), $J=$23.735(23.755), $H=$24.989(24.860),
$K=$25.884(26.006) and $L=$27.809(27.875). Our derived zeropoints and
those of \cite{1998A&A...333..231B} match to within 0.05 mags (and
usually much closer) for all bands except the \textit{H,K} and
\textit{L} bands. This is probably due to the previously noted IR
excess (although detected at longer wavelengths) of the observed Vega
spectrum. \cite{1998A&A...333..231B} use a combined model spectrum of
Vega and Sirius as their reference spectrum. As a further test we also
used the synthetic A0V stellar spectrum of \cite{1993ASPC...41...55C}
to derive zeropoints but were still unable to improve the match to the
\cite{1998A&A...333..231B} photometric system for the \textit{JHK}
colours. However, for these colours we have adopted the \textit{CIT}
system, but were unable to find published zeropoints, and therefore
used the Vega reference spectrum. Essentially this may mean there is a
small offset in our \textit{JHK} photometry, however as most of our
results are based on differential photometry this will not affect our
conclusions.

A further complication with our adopted photometric systems is the
range of zeropoints available for the \textit{Spitzer} IRAC
photometry. For this study, as stated, we have adopted zeropoints
calculated using the zero magnitude flux from the IRAC handbook
\footnote{http://ssc.spitzer.caltech.edu/documents/som/som8.0.irac.pdf}.
The resulting zeropoints were: Channel 1[3.6]=19.541, channel
2[4.5]=19.089, channel 3[5.8]=17.395 and channel 4[8.0]=17.966. The
corresponding zeropoints derived for the MIPS passbands where: channel
1[24]=2.139, channel 2[70]=-0.2726 and channel 3[160]=-1.990.

In summary, several careful consistency checks were performed to
confirm that the resulting SEDs and photometric magnitudes behaved as
expected and matched any available published results. A failure to
match the published zeropoints in the near-IR bands of the
\cite{1998A&A...333..231B} using a Vega reference spectrum was
probably due to an IR excess in our observed Vega spectrum. However as
in general most of the conclusions or implications of this study are
based on differential photometry, this should not affect them
adversely.  Furthermore, a failure to match the published magnitudes
(and colours) for the atmospheric models in
\cite{2000ApJ...542..464C}, even using published zeropoints for the
excellently defined system of \cite{1998A&A...333..231B}, and
subsequent conversions to the required \textit{CIT} system
\citep{1992ApJS...82..351L}, was prescribed to an update in the model
atmospheres available online.

\section{Website}
\label{website}

**THIS SECTION NEEDS TO BE SORTED**

As stated throughout this paper the data presented are available from
a web page
\footnote{http://www.astro.ex.ac.uk/research/bd\textunderscore
  disc}. In this Appendix we briefly discuss the data included, and
the different ways of accessing or visualising these data on the
website.

\subsection{Available Data}
\label{web_data}

The magnitudes and colours presented in this paper are avaiable both
as individual magnitudes and as isochrones or mass tracks. Photometric
magnitudes have also been derived for several other systems and are
available online. These are Johnson, Cousins \textit{UBVRI(JHK)}
\citep{1966ARA&A...4..193J,2005ARA&A..43..293B}, \textit{Tycho} $V_t$
and $B_t$ \citep{2000PASP..112..961B}, Bessell \textit{UBVRIJHKL}
\citep{1988PASP..100.1134B,1998A&A...333..231B}, SDSS \textit{UGRIZ}
\citep{1996AJ....111.1748F}, 2MASS $JHK_s$
\citep{2003AJ....126.1090C,2006AJ....131.1163S}, MKO \textit{JHK}
\citep{2002PASP..114..169S,2002PASP..114..180T}, UKIRT \textit{ZYJHK}
\citep{2001MNRAS.325..563H}, IRAS 12, 25, 60 and 100\,$\mu$m
\citep{1984ApJ...278L...1N} and SCUBA \textit{450WB} and
\textit{850WB} \citep{1999MNRAS.303..659H}. For further information on
these magnitudes, such as the filter responses used and the adopted
zeropoints please refer to the website.

\textit{Monochromatic fluxes}

In addition to the magnitudes derived for each of these bands
monochromatic fluxes have also been derived for all bands listed
above. These have been derived following closely the methods of
\cite{2006ApJS..167..256R}, extended to further passbands. For details
of the assumed SED shape, central wavelengths and bandpasses adopted
please refer to the website. The derivations of these monochromatic
fluxes will be detailed in a coming paper, which details a fitting
tool associated with these data.

\subsection{Navigation}
\label{navigation}

The main page contains links to download the entire dataset in several
formats alongside files describing the format.

\begin{enumerate}
\item All SEDs (\AA and ergs s$^{-1}$ cm$^{-2} {\rm \AA} ^{-1}$) 
\item All SEDs ($\mu$m and mJy)
\item All Photometric Magnitudes for all BDD systems.
\item All Photometric Magnitudes for all Naked systems.
\item All Monochromatic Fluxes for all Naked systems ($\mu$m and ergs
  s$^{-1}$ cm$^{-2} {\rm \AA} ^{-1}$)
\item All Monochromatic Fluxes for all Naked systems ($\mu$m and mJy)
\item All Monochromatic Fluxes for all BDD systems ($\mu$m and ergs
  s$^{-1}$ cm$^{-2} {\rm \AA} ^{-1}$)
\item All Monochromatic Fluxes for all BDD systems ($\mu$m and mJy)
\end{enumerate}

Also included is a link to the calibration information listing all the
filter response sources, adopted zero points, central wavelengths,
bandwidths and assumed SED shapes (for monochromatic flux derivation).
A brief scientific overview is also available explaining, in general
terms, the dataset and model.

\subsection{Browsing Tools}
\label{tools}

For users who wish to investigate the dataset in a more specific or
interactive fashion two browsing tools are included.

The isochrones, mass tracks and individual stars magnitudes and
colours generated and presented in the work can be queried using the
``Isochrone and Mass Track Tool''. This allows the user to select the
parameters of the model required and retrieve the specific data.

Secondly, an interactive tool is included allowing a user to select a
given set of parameters and download the ``SEDs, monochromatic fluxes
or magnitudes''. In addition this tool plots the SEDs for the Naked
system or the three inclinations, in the case of BDD systems and
allows users to select filters sets, whose monochromatic flux values
will be overlaid on the displayed SEDs.

\label{lastpage}
\end{document}
