% mn2esample.tex
%
% v2.1 released 22nd May 2002 (G. Hutton)
%
% The mnsample.tex file has been amended to highlight
% the proper use of LaTeX2e code with the class file
% and using natbib cross-referencing. These changes
% do not reflect the original paper by A. V. Raveendran.
%
% Previous versions of this sample document were
% compatible with the LaTeX 2.09 style file mn.sty'
% v1.2 released 5th September 1994 (M. Reed)
% v1.1 released 18th July 
% v1.0 released 28th January 1994

\documentclass[useAMS,usenatbib]{mn2e}

% If your system does not have the AMS fonts version 2.0 installed, then
% remove the useAMS option.  
% useAMS allows you to obtain upright Greek characters.
% e.g. \umu, \upi etc.  See the section on "Upright Greek characters" in
% this guide for further information.
%
% If you are using AMS 2.0 fonts, bold math letters/symbols are available
% at a larger range of sizes for NFSS release 1 and 2 (using \boldmath or
% preferably \bmath).
%
% The usenatbib command allows the use of Patrick Daly's natbib.sty for
% cross-referencing.
%
% If you wish to typeset the paper in Times font (if you do not have the
% PostScript Type 1 Computer Modern fonts you will need to do this to get
% smoother fonts in a PDF file) then uncomment the next line
% \usepackage{Times}
\usepackage{lscape}
\usepackage{graphicx}
\usepackage{subfigure}

%%%%% AUTHORS - PLACE YOUR OWN MACROS HERE %%%%%
\newcommand\mnras{MNRAS}
\newcommand\pasp {PASP}
\newcommand\aj   {AJ}
\newcommand\apj  {ApJ}
\newcommand\apjl  {ApJL}
\newcommand\apjs  {ApJS}
\newcommand\aap  {A\&A}
\newcommand\aaps  {A\&AS}
\newcommand\apss  {AP\&SS}
\newcommand\gca  {Geochim. Cosmochim. Acta}
\newcommand\araa {ARA\&A}
\newcommand\logmdot {$\log \dot{M}$}
\usepackage{multirow}

%%%%%%%%%%%%%%%%%%%%%%%%%%%%%%%%%%%%%%%%%%%%%%%%

\title[Brown dwarf discs]{On the properties of discs around accreting brown dwarfs}  \author[Nathan J.  Mayne and Tim J. Harries]{Nathan J. Mayne\thanks{E-mail: nathan@astro.ex.ac.uk
    (NJM)} and Tim J. Harries\\
  School of Physics,
  University of Exeter, Stocker Road, Exeter, EX4 4QL.\\
}
 \begin{document}


\date{Accepted ?. Received ?; in
  original form ?}

\pagerange{\pageref{firstpage}--\pageref{lastpage}} \pubyear{2010}

\maketitle

\label{firstpage}

\begin{abstract}
To Do
\begin{enumerate}
\item{Read new papers and put in any relevant stuff.}
\item{Sort structure paragraph at end of intro.}
\item{Sort out headings of sections.}
\item{Sort clearpage problem, perhaps change to includegraphics.}
\item{Sort out the webpage stuff.}
\end{enumerate}
\end{abstract}

\begin{keywords}
  stars:evolution -- stars:formation -- stars: pre-main-sequence --
  techniques: photometric -- catalogues -- (stars) Hertzsprung-Russell
  H-R diagram
\end{keywords}


\section{Introduction}
\label{intro}

There is now strong evidence that as brown dwarfs (BDs) form they pass
through a Classical T~Tauri star (CTTS) phase, during which they
possess a flared, dusty circumstellar disc from which they are
actively accreting material
\citep{jayawardhana_2003,mohanty_2004}. The accretion is
thought to proceed via a magnetically-controlled funnel flow mechanism
in which material from a truncated inner disc boundary falls onto the
surface of the star along magnetic field lines
\citep{camenzind_1990,koenigl_1991,mohanty_2008}.

The truncated inner-edge of the circumstellar disc receives direct
photospheric radiation and is strongly heated. This leads to an
increased scaleheight, or `puffing up' of the inner disc rim, and
subsequent shadowing of the region immediately beyond it
\citep{dullemond_2001}. The density-dependent nature of the dust
sublimation temperature \citep{pollack_1994} may also play a
role in shaping the inner rim
\citep{isella_2005,tannirkulam_2007}.

As an aside, although we recognise that the inner hole in protostellar
discs may be the result of giant planet formation or photoevaporation
\citep{dahm_2009,najita_2007}, these two mechanisms
should result in a negligible mass-accretion rate
\citep{dahm_2009} and therefore would represent a distinct
evolutionary state of brown dwarf disc (BDD) sytems to that of the
CTTS-like phase we are considering here.

The interpretation of the spectral energy distributions (SEDs) of
pre-main-sequence (PMS) stars involves trying to distinguish the
various contributions to the continuum from the hot spots at the base
of the accretion flow, the photospheric flux, and the near-IR flux
from the dusty inner disc. The complexity of the interplay between
these contributions is excerbated by other, geometrical, effects such
as the inclination (which changes the projected area of the inner disc
wall visible to the observer) and the outer disc structure (which may
obsure the inner disc for high inclincations)
\citep{walker_2004,tannirkulam_2007}.

In surveys, particularly those attempting to discern the PMS disc
fraction , this disentangling usually takes place using broad-band
photometric measures and cuts in colour-colour or colour-magnitude
space
\citep{luhman_2005,Luhman_2008,gutermuth_2008}.
The important quantity here is the separation in wavelength between
the emission peaks for the stellar and thermal disc components, as it
is these components which must be separated. For BDD systems it is
likely that difficulties detecting the disc will be exacerbated by the
lower temperatures of the photosphere and therefore the smaller
separation in wavelength from the thermal disc emission component,
compared to CTTS systems. Therefore, to derive disc fractions for BD
populations we require detailed comparison models with which to guide
disc candidate selection.

Comparison of accretion rates across the pre-main-sequence (PMS) mass
spectrum has indicated that the accretion rate is strongly correlated
with PMS mass,
\citep{muzerolle_2003,natta_2004,natta_2006},
with an approximate form $\dot{M_{\rm acc}}\propto M_{*}^{~2}$. Since
in the canonical picture the accretion rate is driven by the disc
viscosity, and should be indepedent of the mass of the central object,
this correlation is somewhat surprising. There is some danger that the
correlation is the result of, or at least strengthened by, the
presence of observational biases \citep{clarke_2006}: At the
high-mass (CTTS) end of the mass spectrum the lowest accretion-rates
cannot be measured via continuum methods since the excess is too small
in contrast with the photopsheric emission. The emission lines will
also be weak, and indeed the H$\alpha$ equivalent width (EW) may be
less than the 10\AA\ that traditionally demarcates classical from
weak-lined T Tauri stars. Such objects should show doppler-broadened
profiles, but the line wings will be weak, and the presence of
underlying H$\alpha$ absorption may become dominant. Thus there may be
a population of low-accretion-rate CTTS which are current missing from
the surveys.

There are also risks of observational biases at the low-mass end of
the correlation. Radiative-transfer modelling of H$\alpha$ emission
from accreting BDs requires high temperatures ($>$\, 10\,kK) in the
accretion funnels in order to recover the level of emission that is
observed \citep{muzerolle_2003,natta_2004,natta_2006}. (Cooling rate
arguments can be invoked to explain the presence of such high
temperatures). Recently some continuum measurements of accreting BDs
have been conducted \cite{herczeg_2009}, and these broadly support the H$\alpha$
rates.

Nonetheless it is worth considering the observed effect that mass
accretion at typical CTTS rates onto a BD might have. Parity between
the BD photospheric luminosity and the accretion luminosity occurs at
relatively low accretion rates \citep{clarke_2006}: Such objects may
not be detected as BDs at all in photometric surveys. Even before such
an extreme case is reached, the additional luminosity provided by the
accretion should have measurable effect on the BD colours. Furthermore
the additional flux will be reprocessed by the disc, altering its
scaleheight and possibly the shape of its inner edge.

Here we present a grid of models of BDD systems, including a
self-consistent treatment of the photospheric and accretion luminosity
sources and the interaction of that flux with the circumstellar disc.
We investigate the impact of the luminosity of the central source on
location and shape of the disc inner rim, as well as the large-scale
structure and flaring of the outer disc. We construct synthetic
colour-colour and colour-magnitude diagrams in order to examine the
efficacy of the photometric selections used to isolate brown dwarfs
and measure disc fractions
\citep{luhman_2005,Luhman_2008,gutermuth_2008}.

The complete model grid, with derived photometry and isochrones, is
available online through our browsing tool
\footnote{http://www.astro.ex.ac.uk/research/bd\textunderscore disc}
which is described in Appendix \ref{website}.

\section{Model}
\label{model}

In this Section we detail the physical model adopted and assumptions
made (Section \ref{physics}), then explain key elements of the
radiative transfer code (Section \ref{torus}). Then we discuss the
derived values, such as broadband photometric magnitudes and colours
(in Section \ref{derived}). Some internal consistency checks are
given in Appendix \ref{consistency}.

\subsection{Physical model and assumptions}
\label{physics}

\subsubsection{Photospheric flux}
\label{phot_flux}

In order to model an accreting BDD system we must first model the
underlying photospheric flux. We have adopted a BD stellar interior
and atmospheric model grid and have then constructed the total
photospheric flux for any input value of stellar age and mass by
interpolating for surface gravity ($\log g$), effective temperature
($T_{\rm eff}$), radius ($R_*/R_{\odot}$) and luminosity
($L_*/L_{\odot}$). These values were then used to interpolate
atmospheric spectra for flux (ergs s$^{-1}$cm$^{-2} {\rm \AA}^{-1}$)
from 1200 to 2$\times$ 10$^7$ \AA. The spectra were subsequently
resampled onto 200 logarithmically spaced points. Careful inspection
ensured that no spectral features were removed during resampling. The
stellar interior models used for this study are the `DUSTY00' models
of \cite{chabrier_2000} combined with the `AMES-Dusty',
atmospheric models of \cite{chabrier_2000}, which are all
available online
\footnote{http://perso.ens-lyon.fr/france.allard/}. For our $T_{\rm
  eff}$ range of $\approx $3000\,K$<T_{\rm eff}<$1600\,K the
AMES-Dusty atmospheres are the most applicable (2700\,K$>T_{\rm
  eff}>$1700\,K). We did try including dynamic application of
atmospheres based on the derived $T_{\rm eff}$, i.e. using AMES-Cond
for $T_{\rm eff}<$1700\,K, but this resulted in large discontinuities
between the model atmospheres and resulting spectra. Since this only
affects stars at the very edge of our temperature range, i.e. for the
oldest and lowest mass objects (for the AMES-Cond case), we have
adopted the AMES-Dusty models throughout.

\subsubsection{Accretion flux}
\label{acc_flux}

We assumed blackbody emission for the accretion flux.  The selected
accretion rate was used to derive an accretion luminosity ($L_{\rm
  acc}$), where the material was modelled as free-falling from the
disc inner edge onto the surface of the star. $L_{\rm acc}$ is
calculated according to \ref{Lacc},
\begin{equation}
{L_{\rm acc}=\frac{GM_*\dot{M}}{R_*} \left( 1-\frac{R_*}{R_{\rm inner}}\right)},
\label{Lacc}
\end{equation}
where $M_*$ is the stellar mass, $\dot{M}$ the mass accretion rate,
$R_*$ the stellar radius and $R_{\rm inner}$ the radius of the disc inner
boundary.

The initial inner disc radius was set to be the co-rotation radius
(this is discussed in more detail in Section
\ref{disc_parameters}). During the radiative transfer simulations of
the disc the final inner dust-disc radius may be beyond the
co-rotation radius due to dust sublimation effects (see Sections
\ref{dust_edge} and \ref{inner_edge} for an explanation). Once the
accretion luminosity was derived, an adopted areal coverage ($A$),
over the stellar surface, was used to calculate an effective
temperature ($T_{\rm acc}$), for the accretion `hot' spot, where
\begin{equation}
{T_{\rm acc}=\left(\frac{L_{\rm acc}}{4\pi R^2_{*} \sigma A}\right)^{\frac{1}{4}}}.
\label{Tacc}
\end{equation}

Finally, a blackbody flux distribution is generated at $T_{\rm acc}$
and added onto the intrinsic stellar photospheric flux. In general one
would expect this to be an overestimate of the accretion flux, as for
pre-MS BDs large convective zones are expected on the stellar surface,
and some of the accretion energy may act to further drive these
convective currents, meaning flux is lost. It is worth noting however
that observationally UV excesses are often used to recreate and then
subtract an assumed accretion flux using a blackbody flux curve, which
is essentially the reverse of this method.

\subsubsection{Disc parameters}
\label{disc_parameters}

In this study we assume that accretion from the central star occurs
along magnetically channelled columns from the inner disc boundary.
For CTTS stars, \cite{bouvier_2007} show that the magnetic
truncation radius ($R_{\rm mag}$) is less than the co-rotation radius
($R_{\rm co}$), where the angular Keplerian velocity of the
disc is equal to the surface angular velocity of the central star.
Calculations of the magnetic truncation radius depend on derivations
of the surface magnetic field \citep{koenigl_1991}. This is
currently unavailable for BD stars, due to increased molecular species
obscuring the Zeeman splitting signatures, that are normally used to
derive stellar surface magnetic fields. Therefore, for our model grid
we have adopted an initial inner disc radius as the co-rotation
radius,
\begin{equation}
{R_{\rm inner}=\left({GM_*\tau^2\over{4\pi ^2}}\right)^{1\over{3}}},
\label{inner_eq}
\end{equation}
where $\tau$ is the stellar rotation period and $R_{\rm inner}$ is the
inner radius. This is effectively adopting a disc-locking mechanism
(without associated angular momentum loss), as for disc-locked stars,
$R_{\rm mag}\approx R_{\rm co}$,
\citep{koenigl_1991,shu_1994}. Therefore, for our
model simulations, this inner edge radius is dependent on, and derived
from, the adopted value of the rotational period for the central star,
as well as being weakly dependent on the stellar mass. As discussed in
Section \ref{intro}, the inner disc can be cleared through a number of
mechanisms, including binarity or Giant planet formation,
photoevaporation or photoionisation of the disc and dust grain growth
or settling. For BDD systems where a disc is modeled a treatment of
dust sublimation is included (discussed in Section \ref{dust_edge}).
However, the effects of binarity or Giant planet formation are
neglected. Further to the stellar mass and period required prior to
calculation of the inner disc radius, we require a disc mass (in
stellar masses).

Although we solve for the vertical disc structure, the radial
structure of the disc is a free parameter. We assume that the surface
density varies as $\Sigma(r) \propto r^{-1}$.

For this work the disc outer edge was set at 300\,AU, this was chosen
as a maximum size of the circumstellar
disc. \cite{bouy_2008} have shown that the disc outer radius
has little effect on the resulting SED. However, in our subsequent
paper we will include models for outer radii of 100\,AU and plan to
extend this parameter range further to smaller values of the outer
radius in the future.

\subsubsection{Naked and disc systems}
\label{naked_BDD}

The combined (accretion plus photosphere) SED is then used as a
boundary condition for the {\sc torus} radiative transfer code and
as a benchmark set of SEDs to model `naked' BD systems. The set of
`naked' photospheres (plus accretion) are diluted by the factor
$(R_*/{\rm distance})^2$ to a distance of 10\,pc. The `negligibly' accreting,
`naked' stars can be used to produce absolute magnitude (and intrinsic
colour) isochrones for comparison. The remainder are used to
model systems showing active accretion where no disc is detected. For
instance, \cite{kennedy_2009} find 43 stars within their sample
are actively accreting whilst no disc is detected (out of a total
sample of 1253).

In summary the key required input variables to setup the model grid
are as follows: Stellar age and mass (which are used to derive the
stellar flux) and accretion rate, areal coverage and rotation period
(which are used to determine the the accretion flux and temperature).
The disc parameters are set by a disc mass (expressed as a fraction of
the BD mass), and a power-law distribution for the disc surface
density.

\subsection{Radiative transfer code: {\sc torus}}
\label{torus}

In this section we briefly explain the key elements of the radiative
transfer code used to model the BDD systems.

We have used the {\sc torus} radiative transfer code which is
described in \cite{harries_2000}, including the subsequent
refinements i.e.  addition of Adaptive-Mesh-Refinement (AMR)
introduced in \cite{harries_2004}. {\sc torus} uses the method
of \cite{lucy_1999} solve radiative equilibrium. The
simulation also self-consistently solves the equation of
\textit{vertical hydrostatic equilibrium} and dust sublimation for the
disc (described in \cite{tannirkulam_2007}).

\subsubsection{Dust Size distribution}
\label{dust_size}
 
We have adopted a similar dust model to \cite{wood_2002} 
with the size distribution of dust particle given by,
\begin{equation}
\label{particle_dist}
n(a)da=C_ia^{-q}\times exp^{[-(a/a_c)^p]}da
\end{equation}.
Where $n(a)da$ is the number of particles of size $a$ (within the
increment $da$), $a_{\rm c}$ is the characteristic particle size, with
$p$ and $q$ simply used to control the shape of the
distribution. $C_i$ controls the relative abundance of each
constituent species ($i$) in the dust. \cite{wood_2002}
found that simulated SEDs fit observed SEDs better using this adjusted
size distribution for the dust particles, as opposed to a simple power
law. The best fitting values found in \cite{wood_2002} were
$q=$3.0, $p=$0.6 with $a_{\rm c}=$50\,$\mu$m, also with an associated
maximum and minimum grain size of $a_{\rm min}=$5 nm and $a_{\rm
  max}=$1mm. A slightly steeper power law dependence, with $q=$3.5
(corresponding to the canonical MRN distribution) is more widely
adopted
\citep{bouy_2008,morrow_2008,pascucci_2008}.
Here we have adopted the values of \cite{wood_2002}
except for the parameter $q$ where we have used $q=$3.5.
\cite{wood_2002} calculated the $C_i$ values for amorphous
carbon and silicon by requiring the dust to deplete a solar abundance
of either component completely \citep[using abundances
  from][]{anders_1989,noels_1993}. We have set
$C_i=$1 and adjusted the species using a grain fractional abundance
\cite[an equivalent process to that of][]{wood_2002},
however we have adjusted these grain fractions using the updated solar
abundances of \cite{asplund_2006}. The resulting difference in
opacity between grain fractions matching the work of
\cite{wood_2002} and the new grain fractions is only a
slight enhancement of the silicate feature (due to the relative
abundance of Silicon increasing) which has little effect on the
resulting SEDs. Figure \ref{albedo} shows the resulting albedo, and
scattering and absorption opacities, for our dust population, with a
the vertical dashed line showing 10\,$\mu$m.

\begin{figure}
  \vspace*{174pt}
  \special{psfile="Fig/albedo.ps"
   hoffset=235 voffset=0 hscale=30 vscale=30 angle=90}
 \caption{Figures of the albedo (\textit{top panel}), and scattering
   (dashed line) and absorption (solid line) opacities (\textit{bottom
     panel}) against log($\lambda$) (in $\mu$m) for our adopted dust
   population. For \textit{both panels} the vertical dashed line is
   plotted at 10\,$\mu$m to highlight the silicate features.
   \label{albedo}}
\end{figure}

\subsubsection{Dust sublimation and the inner disc edge}
\label{dust_edge}


As our models assume magnetic-truncation at the co-rotation radius, we
have implicitly generated an inner hole. This also means that the disc
will have an inner wall at this radius. Evidence for inner walls in
circumstellar discs is apparent from the SEDs of disc systems, where a
peak in emission is found between 2 and 3\,$\mu$m. The temperatures
reached by such inner walls are expected to generate thermal flux
contributions within this wavelength range
\citep{dullemond_2001}.

For low-luminosity systems the dust sublimation radius ($R_{\rm sub}$)
will be coincident with the co-rotation radius. However should
the combined photospheric and accretion luminosities be sufficient the
inner disc will be heated sufficiently that the dust close to the
co-rotation radius will be sublimated, and we must account for this
in our models. Furthermore, since the dust sublimation temperature has
a density dependence \citep{pollack_1994}, it is both the location and {\em
  shape} of the inner wall that can change (Isella ref; Tannirkulam ref).

Our treatment of dust sublimation is similar to that detailed in
\cite{tannirkulam_2007}, but with some enhancements to ensure a
swift convergence of the sublimated rim. The dust sublimation proceeds
as follows: An initial temperature distribution is found for the
optically-thin limit by setting the global dust-to-gas ratio to a tiny
value. Subsequent radiative equilibrium iterations are performed using
the adopted dust-to-gas ratio of 0.01, but limiting the maximum
optical depth across a given cell to $\tau_{\rm max}$. Cells whose
temperature exceeds the local dust sublimation temperature have their
dust-to-gas ratio set to zero. Radiative equilibrium iterations and
sublimation sweeps are performed at $\tau_{\max} = $0.1, 1, and 10,
with a final iteration of $\tau_{\max} = \infty$. Adequately solving
the radiative-equilibrium necessiates resolving the disc's effective
photosphere, and adaptive mesh refinement is used to split the grid at
the optically-thin/optically-thick boundary in order that the maximum
cell size at this boundary is $<1$ at a wavelength of 5500\,\AA. Once
a self-consistent sublimated rim has been determined the equation of
hydrostatic equilibrium is solved throughout the disc, and the
sublimation iterations are restarted (the change in the density
structure from the hydrostatical equilibrium step naturally feeds back
into the shape of the inner rim). Five hydrostatic equilibrium steps
are performed to ensure convergence, although a stable disc structure
is normally found after three such iterations.

For our model grid we have adopted a gas-to-dust ratio, $\epsilon$, of
100, and the gas population is assumed essentially static with a zero
optical depth. 

\subsubsection{SEDs}
\label{seds}

Once the radiative transfer code was completed simulated SEDs were
generated. These SEDs can be generated for any distance and for any
system-observer inclination. For our models we have set the distance,
to 10\,pc to create absolute flux SEDs and selected ten inclinations
equi-spaced in $\cos i$ (0, 27, 39, 48, 56, 64, 71, 77, 84 and
90$^{\circ}$). A further useful feature of the {\sc torus} code is
that the emitted photon packets which make up the SED are tagged on their way
to the observer. These tags separate the packets into four
groups. Firstly, packets are separated by source into thermal (disc)
or stellar groups. These groups are then subdivided into those which
reach the observer either directly or after scattering (for example
see Figure \ref{disc_eg_173}). The resulting SEDs are discussed in
Section \ref{seds}.

\subsection{Photometric systems and derived quantities}
\label{derived}

Many observational studies use non-spectroscopic data to derive the
pertinent parameters. Therefore in order to examine the practical
effects in the `observational plane' we have used the SEDs to produce
broadband photometric magnitudes, and subsequently colours. Broadband
magnitudes were also derived in a large range of other filter sets not
used explicitly in the analysis within this paper. These magnitudes
are available online
\footnote{http://www.astro.ex.ac.uk/research/bd\textunderscore disc}
and are briefly discussed in Appendix \ref{website}. In addition,
monochromatic fluxes have been derived for all filters, and again are
available online and discussed in Appendix \ref{website}.

In order to derive broadband photometric magnitudes and colours the
SEDs of either the disc or naked systems were folded through the
filter responses of the required photometric system. As the fluxes in
all cases are absolute, derived for an observer to object distance of
10\,pc, no conversion is required to derive absolute magnitudes and
therefore intrinsic colours.

We have integrated using photon-counting and calibrated using a Vega
spectrum for magnitudes in the optical and near-IR regimes. The filter
bandpasses selected are, the optical system of
\cite{bessell_1998} for \textit{UBVRI} and the \textit{CIT}
system of \cite{elias_1982,stephens_2004} for
\textit{JHK}, with the required shift of $-0.015$\,$\mu$m as prescribed by
\cite{stephens_2004}.

We have also derived magnitudes in the mid-IR range using the IRAC and
MIPS systems of the \textit{Spitzer} space telescope. These magnitudes
were derived using a conversion of flux to Data Number (DN) and
calibrated using zero points derived from the zero magnitude fluxes
published in the IRAC handbook
\footnote{http://ssc.spitzer.caltech.edu/documents/som/som8.0.irac.pdf}
and the MIPS instrument calibration website
\footnote{http://ssc.spitzer.caltech.edu/mips/calib/}.

These specific filter sets have been chosen due to their ubiquitous
use and suitability for the derivation of the key stellar parameters
of age, mass and, for populations, disc fractions. Therefore, by
studying the changes of magnitudes (and colours) in these photometric
systems we can test the predicted effects on these derived parameters
caused by changes in the input parameters of our model grid. The
optical magnitudes \textit{VI} are used to explore age related effects
of varying the parameter space. Flux in the \textit{VI} bands is only
minimally affected by accretion flux \citep{gullbring_1998} and
disc thermal flux \citep{hartmann_1998} and, additionally, in a
\textit{V, V-I} CMD, the reddening vector lies parallel to the pre-MS,
minimising any age effect of extinction uncertainty, \citep[for a full
discussion see][]{mayne_2008,mayne_2007}. The
near-IR passbands of \textit{JHK} are most often used to derive
stellar masses as for pre-MS objects the reddening vector, in for
instance, a \textit{J, J-K} CMD, is almost perpendicular to the
isochrones, minimising the mass effect of extinction uncertainties
\citep[the \textit{CIT} systems was chosen to
match][]{chabrier_2000}. Finally, as is now well documented the
\textit{Spitzer} IRAC passbands provide the best data with which to
unambiguously separate naked and star-disc systems
\citep{luhman_2005}. In addition, at longer wavelengths, the
MIPS instrument provides sensitivity to disc systems at much larger
radii (or debris discs).

Once the model grid was completed several checks were performed to
verify the consistency of our results. For each individual model these
checks were passed to our satisfaction before publication. Some
problems remain, and these are explained in Appendix
\ref{consistency}.

\section{Parameter space}
\label{par_space}

This section details the range of each of the input parameters we have
varied and, where possible, gives justification for the selected
ranges from published observations. The simulations in this paper
cover variations in the stellar mass, stellar age, stellar rotation
rate, accretion rate, the areal coverage of the accretion stream, disc
mass fraction and the system inclination. A summary of the values
adopted for each input variable is shown in Table
\ref{par_space_table}.

\subsection{Mass} 
\label{par_mass}

Representative masses within the BD regime were chosen as follows:
0.01, 0.02, 0.03, 0.04, 0.05, 0.06, 0.07 \& 0.08\,$M_{\odot}$.

\subsection{Age}
\label{par_age}

Typical disc lifetimes for solar type stars are of order 10\,Myrs
\citep{haisch_2001}. Therefore, we have adopted input ages of
1 and 10\,Myrs for our model grid, to span the approximate range of
ages over which the discs influence will be important.

\subsection{Rotation rate}
\label{par_rotation}

Data for rotation rates, from periodic variability surveys, are widely
available for a range of different age clusters of TTS. However, fewer
studies exist on the rotation rates of BD mass objects. Rotation
period data for $\sigma$ Ori, at an age of $\approx$ 3 Myrs
\citep{mayne_2008}, is studied in \cite{scholz_2004}, where periods
are found over the range 5.78$-$74.4 hours ($\approx$ 0.24$-$3.1 days)
for BD mass objects. \cite{scholz_2005} study rotation period data for
stars in the vicinity of $\epsilon$ Ori, with an assumed age of
$\approx$3 Myrs \citep{osorio_2002} and the ONC at an age of
$\approx$2 Myrs \citep{mayne_2008}. Rotation periods, from photometric
variability, in the range 4.7$-$87.6 hours ($\approx$ 0.2$-$3.65 days)
for BD mass stars are found. \cite{joergens_2003} study the rotational
periods of BD (and very low mass stars) in the Chameleon I
region. This region is $\le$ 1 Myrs old, and the authors find rotation
periods of 2.19, 3.376 and 3.21 days for their BD mass counterparts.

In some cases the periodic variability is irregular and assumed to
come from active accretion hot spots on the BD surface \citep[see][for
discussion of variability causes]{bouvier_1995,herbst_2007},
indicative of active accretion. All the studies mentioned infer a disc
locking mechanism. Furthermore, \cite{scholz_2004} and
\cite{scholz_2005} find evidence for a $mass\propto period$
relationship extending into the BD regime. Additionally,
\cite{joergens_2003} propose a shorter lifetime of $\approx$5 Myrs for
BD discs, inferred from a shorter derived disc locking
timescale. However, as discussed in the previous studies, an imperfect
disc locking mechanisms is also hypothesised as responsible for the
less significant loss in angular momentum out to ages of 10 Myrs, for
BD discs. The data on BD rotation rates, disc presence and disc
locking are summarised and discussed in \cite{herbst_2007}.

Therefore, to create a set of useful models to help contextualise the
observational constraints, for study of disc locking mechanisms, we
must adopt a realistic, and bounding range of rotation rates. For our
model grid, and associated age range ($<$10 Myrs), we have selected
0.5 and 5 days. With the limits set at at the approximate
median of faster rotators and the edge of the slower rotators.

\subsection{Areal Coverage}
\label{par_cov}

As discussed in Section \ref{par_rotation} evidence for irregular
periodic variability has been found in BD stars with detections of
associated stellar discs. This is construed as evidence for accretion
hot spots formed as magnetically channeled material hits the stellar
surface \citep[see discussion
in][]{bouvier_1995,herbst_2007}. The irregularity is
thought to be caused by changes in the magnetospheric structure and
accretion rate \citep{bouvier_1995}. For our model we have
assumed that disc material is disrupted at the co-rotation radius and
channeled onto the star in the form of accretion hot spots with a
characteristic temperature. Therefore, to calculate the characteristic
temperature and the resulting blackbody accretion flux we must adopt
an accretion rate and areal coverage of the accretion stream.

Little observational evidence can be found for approximate sizes of
accretion hot spots due to their more transient nature and often
smaller coverages, when compared to cooler or `plage' spots
\citep{herbst_2007}. \cite{bouvier_1995} modeled the
size of the cool spots on solar-type stars for a selection of
periodically variable candidates. They found projections of cooler
spots, onto the stellar disc, of a few to $~$60\%.
\cite{bouvier_1995} also found projected sizes, onto the
stellar disc, of typically a few \% to around 10\% for hot spots.
\cite{bertout_1996} used observations of YY Orionis monitoring
flux amplitude variations as a function of wavelength to derive a
probable hot spot area of around 10\%. The spot temperature was also
modeled for YY Orionis in \cite{bertout_1996}, resulting in a
best fitting areal coverage of 11\%. Therefore, to bound the probable
areal coverage range of the accretion hot spots we have adopted areal
coverages of 1 and 10\%.

\subsection{Accretion Rate}
\label{par_accn}

Accretion rates derived for pre-MS stars are of order \logmdot = $-6$
to $-11$ \citep{natta_2006}, with the largest accretion rates found in
so-called FU Orionis type objects. For the more typical accretion
rates \citep[$\dot{M}=$10$^{-11}$ to 10$^{-8.9} M_{\odot}yr^{-1}$, for
  TTS,][]{dahm_2009}, several studies have now suggested that the
accretion rate is strongly correlated with the mass of the central
star. This relationship was perhaps first suggested by
\cite{muzerolle_2003} using various accretion diagnostics. Later,
\citep{muzerolle_2005} derived a relationship of approximately
$\dot{M}\propto M_*^{~2}$. Further evidence was put forward by
\cite{natta_2004}, where accretion rates as low as 5$\times$
10$^{-12}M_{\odot} yr^{-1}$ were found for BD stars. More recently,
even lower accretion rates of $\approx$10$^{-13}M_{\odot} yr^{-1}$,
have been derived for BDs by \cite{herczeg_2009}.  Further support for
a dependence of accretion rate on stellar mass was apparent in the
significantly more homogeneous dataset of
\cite{natta_2006}. \cite{natta_2006} analysed a set of accretion rates
and masses derived for pre-MS stars in $\rho$ Ophiuchi and compared
these results to stars in Taurus. They found that the accretion rate
scales with central object mass into the BD regime, although with
significant scatter.

As the relationship $\dot{M}\propto M_*^{~2}$ predicts lower accretion
rates for BD mass objects it is essential that we model systems at
higher accretion rates, which may have been missed in current
observational studies. Therefore, we have adopted accretion rates of
\logmdot = $-6$, $-7$, $-8$, $-9$, $-10$, $-11$ \& $-12$.

\subsection{Disc Mass}
\label{par_mdisc}

Previously studies modeling BD discs have adopted a range of
disc mass fractions, for instance \citep{walker_2004} use 0.1,
0.01 and 0.001$M_*$. \cite{wood_2002} fit observed spectra
with modeled SEDs to derive a disc mass of 0.003$M_*$, for HH 30
IRS. Subsequent derivations of disc masses have converged to within an
order of magnitude, with the following specific results: 0.03$M_*$
\citep[$\rho$ Ophiuchi,][]{natta_2002}, 0.055$M_*$ \citep[GM
Aurigae,][]{rice_2003}, 0.03$M_*$ \citep[GY 5, GY 11, and GY
310,][]{mohanty_2004} and 0.022$M_*$\citep[2MASS
J04442713+2512164,][]{bouy_2008}. As the derived disc masses
all have a similar order of magnitude we have adopted $M_{\rm
  disc}\approx $0.01$M_*$. As changes in disc masses are expected to
change the resulting SED less than perhaps, accretion rate for
example, we have not varied the disc mass for this study. The results
of simulations varying this parameter will be published in a future
paper.


\subsection{Inclination}
\label{par_inc}

Discs around BD stars exhibit increased flaring, due to the reduced
surface gravity in the disc \citep{walker_2004}. This
increased flaring, and therefore larger scaleheight of the disc
results in a smaller opening angle, when compared to higher mass stars
and their circumstellar discs. As has been shown in
\cite{walker_2004} effects caused by variations in the system
inclination angle are much more significant for BDD systems, again
compared to their higher mass analogues. Therefore, we have simulated
ten observer to system inclination angles spaced evenly in
$\cos i$ space, namely, 0, 27, 39, 48, 56, 64, 71, 77, 84
and 90$^{\circ}$.

A final list of all varied parameters and their values can be seen
in Table \ref{par_space_table}.

\begin{table*}
\begin{tabular}{|l|l|}
\hline
Input parameter&Values (\# values)\\
\hline
Mass ($M_{\odot}$)&0.01, 0.02, 0.03, 0.04, 0.05, 0.06, 0.07 \& 0.08 (8)\\
Age (Myr)&1 \& 10 (2)\\
Rotation period (days)&0.5 \& 5 (2)\\
Areal coverage (of $\dot{M}$, \%)&1 \& 10 (2)\\
Accretion rate (log$(\frac{\dot{M}}{M_{\odot}} yr^{-1})$)&$-6$, $-7$, $-8$, $-9$, $-10$,
$-11$ \& $-12$ (7)\\
Disc mass ($M_*$)&0.01 (1)\\
Surface density profile & $r^{-1}$ (1) \\
Inclination ($^{\circ}$)&0, 27, 39, 48, 56, 64, 71, 77, 84 \& 90
(10)\\
\hline
\end{tabular}
\caption{List of all varied input parameters. Resulting in a total
  number of models of 448 (plus 40 models without radiative transfer
  simulations for the naked BDs) and 4480 SEDs (plus 40 for naked
  BDs). \label{par_space_table}}
\end{table*}


\section{Result and Analysis}
\label{results}


In this Section we first discuss the physical structure, both density
and temperature, of the BDD disc systems (Section \ref{disc_struct})
across our parameter space. Then we discuss the resulting simulated
SEDs for these BDD systems (Section \ref{sed_analysis}). Finally, we
produce simulated photometric observations (Section \ref{photometry}).
These simulated observations allows us to show that recovery of the
input masses and ages for our accreting BDD systems is unreliable. We
show that despite an intrinsic relationship between rotation rate and
initial inner edge position we do not recover a rotation rate
correlation with IR excess. In addition, despite not intrinsically
including a dependence of accretion rate on stellar mass in our grid.
We show that current observational techniques and theoretical models
applied to the grid would result in a relationship of this type being
derived.

\subsection{Disc Structure}
\label{disc_struct}

XXXX Nathan - add disc structure equation here and remove walker references.

The initial density structure has the the disc surface density
conserved, $\Sigma ^{\beta - \alpha}$. The initial scaleheight,
$h\propto r^{\beta}$, where r is the radial distance from the star,
and the density, $\rho \propto r^{-\alpha}$. The initial values of
$\alpha$ and $\beta$ were 2.1 and 1.1 respectively. The values
for $\alpha$ and $\beta$ were chosen to optimise resolution of the
vertically evolving disc, but minor variations are largely
inconsequential as the systems evolves from this state. In our models
we have, however, placed at the inner disc edge at the co-rotation
radius, as opposed to the dust destruction radius used in
\cite{walker_2004}. The initial disc scaleheight at 100\,AU,
$h(100)$, was set to 25\,AU. As the simulation used vertical
hydrostatic equilibrium and dust sublimation, both the disc
scaleheight and inner edge location then evolved in the systems
dependent on the input parameters. In this section we discuss the
structure of the discs in terms of these two generated
characteristics, i.e the disc scaleheight and inner edge location.

\subsubsection{Disc Flaring}
\label{disc_flaring}

We have previously tested the {\sc torus} code against that used by
Walker and co-workers, using a CTTS disc model and simultaneously
solving for radiative and hydrostatic equilibrium (but not employing
dust sublimation). These tests showed excellent agreement in density
and temperature structure (as well as in the resultant SEDs). The
results of these tests were presented by \cite{walker_2006}. It is
therefore unsurprising that at negligible mass-accretion rates our
disc structures are very similar to those present in
\cite{walker_2004}. Typical discs around CTTS stars have scaleheights
at 100\,AU of between $h$(100)$=$10 to 20\,AU, whereas for BDD
systems, $h$(100)=20 to 60\,AU (for 0.08 and 0.01 $M_{\odot}$
respectively). As the accretion rate increases the flux levels of the
central star increase and lead to heating of the disc which in turn
leads to vertical expansion. We found that levels of vertical flaring
increased only marginally with accretion rate. Significant
differences, more than $>$5\,AU increase in $h(50)$, in the vertical
structure were not apparent until the high accretion rates of \logmdot
= $-7$ and $-6$. Figures \ref{flare_-12_183} and \ref{flare_-7_193}
show the density structure ($\log \rho$) in the disc from radial
distances of 0 to 50\,AU for example systems ($M_*=0.04M_{\odot}$,
Age=1\,Myrs, $\tau$=5\,d and areal coverage=10\%), with accretion
rates of \logmdot = $-12$ and $-7$, respectively.

\begin{figure*}
\begin{center}
  \subfigure[]{\includegraphics[scale=0.4,angle=0]{./Fig/flare_-12_183.ps}\label{flare_-12_183}}
  \subfigure[]{\includegraphics[scale=0.4,angle=0]{./Fig/flare_-7_193.ps}\label{flare_-7_193}}
\end{center}
 \caption{Figure showing the density structure ($\log \rho$) of the
   BDD system with $M_*=0.04M_{\odot}$, Age=1 Myrs, $\tau$=5, areal
   coverage=10\% and accretion rate of (a) \logmdot=$-12$ and (b) \logmdot = $-7$.}
\end{figure*}

\cite{walker_2004} state that the degree of disc flaring depends on
the disc temperature structure and the mass of the central star, with
the disc scaleheight $h\propto \frac{T_{\rm disc}}{M_*}^{1/2}$
\citep{shakura_1973}. Recently however \cite{ercolano_2009} proposed
the inverse relation of flaring with stellar mass, i.e. $h\propto
M_*$. This suggestion was based on evidence from \cite{allers_2006},
where SEDs for 17 systems in the mass range 6$M_{\rm
  Jup}<M_*<$350$M_{\rm Jup}$ were fit with flared or flat disc
models. In general, \cite{allers_2006} find that lower mass objects
achieve better fits with the flat disc models and higher mass objects
with the flared discs.

The results of \cite{allers_2006} show that above a mass of
50$M_{\rm Jup}$ all objects (6/17) are better fit with flared
discs. Whilst at masses below 50$M_{\rm Jup}$ only one object is better
fit by the flared disc model, with the remaining objects (10/17)
better fit with flat models. Whether, this result is statistically
significant enough to assert a $h\propto M_*$ is doubtful as the
fitting process contains, presumably two fixed scaleheight
distributions. Therefore, for our study we continue to assume that our
flared BDD systems will have larger characteristic scaleheights than
typical CTTS systems.

A  comparison of Figures \ref{flare_-12_183} and
\ref{flare_-7_193} shows an increase in the scaleheight at 50\,AU of
$>$5 AU, as the accretion rate moves from \logmdot =$-12$ to
$-7$. However, despite this small change with high levels of accretion
our grid shows scaleheights comparable to the work of
\cite{walker_2004} and as such result in similar consequences for the
SEDs and photometric magnitudes. The effects of this flaring and the
increase in flaring for very high accretion levels on SEDs and
photometric magnitudes are discussed in Sections \ref{sed_analysis}
and \ref{photometry}, respectively.

\subsubsection{Inner edge of the dust disc: location}
\label{inner_edge}

The inner edge of the gas disc is fixed at the co-rotation radius, at
which point the gas threads onto the magnetic field and follows the
field lines in a funnel flow towards the protostellar surface. We make
the assumption that the dust (should it exist at the co-rotation
radius) is destroyed in the funnel flows. This is a reasonable
assumption from both theoretical and observational perspectives:
Radiative-transfer models indicate the temperature in the funnel flows
may be very much greater than the dust sublimation temperature
($>10$\,kK, \citep{muzerolle_2003}), while dusty funnel flows are
likely to be optically thick in the visible, which would lead to
substantial photometric variability as the funnels transit the
photosphere--an effect that is unobserved.

Our models include a sophisticated treatment of dust sublimation
(described in Section \ref{dust_edge}).  As the flux levels of the
central protostar increase with increasing accretion rates the flux
incident on the inner edge increases and leads to increasing erosion
of the inner edge of the dust disc.

As the inner edge moves its temperature is expected to change, this
has been predicted to lead to a correlation of inner edge position and
IR excess \citep{meyer_1997}. This may act to bias surveys correlating
rotation rates with IR excess to search for evidence of
disc-locking. However, the flux from the inner edge will usually peak
between 2 and 3\,$\mu$m \citep{dullemond_2001}, given that the dust
sublimation temperature peaks at $\approx$1400\,K for canonical
densities. This means that for models where dust is significantly
sublimated the inner edge will {\em usually} have a temperature of
$\approx$1400\,K and this correlation of disc position and temperature
will be lost (although see below).

Equation \ref{inner_eq} shows that as the rotation period of the
protostar increases the co-rotation radius decreases. This will result
in, initially, shorter period systems having closer and hotter inner
edges than their longer period counterparts. In addition, if the
accretion rate is increased in these systems the incident flux on the
inner wall will increase leading to a rise in the temperature of the
inner edge.  At some point the dust sublimation temperature may be
reached, leading to a change in radial position of the wall. In
addition, the temperature of the inner edge will then tend to the dust
sublimation temperature.

A further complexity arises when one considers that the density of
material in the disc falls with increasing radius from the star
($\rho(r) \propto r^{\alpha}$), and that the dust sublimation
temperature is density dependent \citep{pollack_1994}.  Therefore, for
systems where the inner edge has been eroded significantly from the
co-rotation radius the temperature of the inner edge will fall
systematically as the inner dust disc radius increases. 

In order to examine the effect of dust sublimation on our models we
have measured the radius of the inner edge of the dust disc by
integrating from the centre along the midplane until an optical depth
of 2/3 (at 5500\AA) is reached. Figure \ref{inner_temp} shows the
radial density distribution of the disc.  In this case the the final
inner edge radius and therefore temperature is no longer strongly
correlated with the rotation rate. Figure \ref{inner_temp} shows the
final inner edge location ($R_{\rm inner}$, $R_*$) against the
temperature of the inner wall ($T_{\rm inner}$, K). The \textit{left
  panel} shows the systems designated as typical accretors and the
\textit{right panel} those with extreme accretion rates.  For
\textit{both panels} the systems with rotation periods of 0.5 and 5
days are plotted in red and blue respectively. Those systems where the
change in inner radius was greater than 1$R_*$, and therefore classes
as significantly sublimated, are plotted as crosses. 

\begin{figure*}
  \vspace*{348pt}
  \special{psfile="Fig/inner_temp.ps"
     hoffset=470 voffset=0 hscale=60 vscale=60 angle=90}
   \caption{Figure showing data for the inner edge location ($R_{\rm
       inner}$, $R_*$). The typical accretors are shown in the
     \textit{left panel} and the extreme accretors in the
     \textit{right panel} (see text for explanation). In \textit{both
       panels} separate the systems by period, with those systems with
     rotation periods of 0.5 and 5 days shown as red and blue symbols
     respectively. In addition, systems where $\Delta R_{\rm inner} <
     1R_*$ are plotted ia crosses and $\Delta R_{\rm inner} > 1R_*$ as
     filled circles (this is only achieved by some systems classed as
     extreme accretors).\label{inner_temp}}
\end{figure*}


The \textit{left panel} of Figure \ref{inner_temp} shows that, taken
as a whole, our models with typical accretion rates show a clear
correlation between the temperature at the inner edge and the radius
to this boundary. This agrees with the work of \cite{meyer_1997} where
this correlation is found for there $R_{\rm inner}=$1$-$12$R_*$ and
\logmdot = $-9$ to $-5$, for flat disc models.  \cite{meyer_1997} use
this correlation, and the derivation of IR magnitudes, to predict a
relationship between IR excess and radius to the inner wall. Our data
indicate that this correlation, for typically accreting systems will
translate into a correlation between rotation rate and IR excess. This
could have important implications for studies of disc-locking where
disc presence is examined as a function of rotation rates, provide an
intrinsic bias. In practice however, this correlation is weak (in fact
unobservable) in our data due to the combined effects of the inner
disc wall shape, inclination and flaring effects. This is discussed in
more detail in Section \ref{photometry}.

Figure \ref{inner_temp} shows that for those systems where significant
disc erosion ($\Delta R_{\rm inner} > 1.0 R_*$) has occurred the
resulting temperature of the inner edge is weakly correlated with the
inner edge radius, but, critically, not correlated with rotation rate.
The inner edge temperatures of the remaining systems for the extreme
accretors are slightly anti-correlated with the radius to the inner
edge. For the systems with typical acccretion rates, and longer
periods, Figure \ref{inner_temp} shows there is again a weak
correlation between the radius to, and the temperature of the inner
edge. For the shorter period models with typical accretion rates there
is no clear correlation between the temperature at the inner edge and
radius to this edge. 

\subsubsection{Inner edge of the dust disc: shape}
\label{inner_edge_shape}

The initial shape of the inner edge of the dust disc is a vertical
wall coincident with the co-rotation radius. In the previous section
we have shown that models with a negligible accretion rate do not
significantly sublimate the dust, and hence the edge remains
vertical. This inner edge is heated by direct radiation from the
protostar, and its scaleheight increases. Disc material behind the
inner rim is shielded from direct radiation and has a smaller
scaleheight, leading to the `puffed up' inner rim predicted by
\cite{dullenmond_2001}. This effect is illustrated in
Figure~\ref{no_sub_183}, a model in which there is negligible dust
sublimation.


\begin{figure*}
  \centering
  \subfigure[]{\includegraphics[scale=0.33,angle=0]{./Fig/no_sub_183_density.ps}\label{no_sub_183_a}}
  \subfigure[]{\includegraphics[scale=0.33,angle=0]{./Fig/no_sub_183_temp.ps}\label{no_sub_183_b}}
  \subfigure[]{\includegraphics[scale=0.33,angle=0]{./Fig/sub_179_density.ps}\label{sub_179_a}}
  \subfigure[]{\includegraphics[scale=0.33,angle=0]{./Fig/sub_179_temp.ps}\label{sub_179_b}}
  \subfigure[]{\includegraphics[scale=0.33,angle=0]{./Fig/sub_181_density.ps}\label{sub_181_a}}
  \subfigure[]{\includegraphics[scale=0.33,angle=0]{./Fig/sub_181_temp.ps}\label{sub_181_b}}

  \caption{Figure showing both the initial (grey scale) final density
    ($\log \rho$) where dust is present (colour scale, left-hand
    panels) and temperature (right-hand panels). System parameters
    are: $M_*$=0.04$\,M_{\odot}$, Age=1\,Myrs and areal coverage=10\%,
    $\tau$=5\,days and accretion rate \logmdot$=-12$ (a, b);
    $M_*$=0.04$M_{\odot}$, Age=1\,Myrs and areal coverage=10\%,
    $\tau$=0.5\,days and accretion rate \logmdot$=-7$ (c, d);
    $M_*$=0.04$M_{\odot}$, Age=1\,Myrs and areal coverage=10\%,
    $\tau$=0.5\,days and accretion rate \logmdot$=-6$ (e, f).}
\end{figure*}

As the flux of the central object increases significant dust
sublimation occurs, leading to a change in the radial position of the
dusty inner edge, but also shaping it: The density drops rapidly away
from the midplane, and since the dust sublimation temperature also
falls the edges of the inner dust disc become curved--this the
mechanism described analytically by \cite{isella_2005}. We illustrate
this effect in Figure~\ref{sub_179_a}, which also shows a scaleheight
decrease behind the curved inner rim, over a significantly larger
radial scale than the previous model.

Finally the most extreme accretor (Figure~\ref{sub_181_a}) shows dust
being destroyed to very large radii ($\sim 0.1$\, AU). A curved rim is
present, without any obvious decrease in scaleheight behind the inner
rim. The distance from the central object is such that the vertical
component of gravity is much diminished, and the disc has a
substantial scaleheight at the inner edge, meaning that it reprocesses
significant protostellar radiation leading to a high near-IR excess.

We note that a similar sequence is apparent across the grid for set
masses. However, the balance of the rotation period, and therefore
inner edge location, and age and areal coverage, therefore flux
levels, leads to changes in the accretion rate at which the dust
sublimation starts. However, in almost all cases the dust sublimation
does not become significant until at least \logmdot=$-9$ (as discussed
above).

\subsection{SEDs}
\label{sed_analysis}


XXX




\subsection{Photometry}
\label{photometry}

Practically, most parameters for young pre-MS stars are derived from
surveys of populations, usually open clusters, using broadband
photometry and subsequently constructed colour-magnitude and
colour-colour diagrams (CMDs and CoCoDs respectively, hereafter).
Therefore, in this Section, we demonstrate the prohibitive effects on
the broadband photometry of varying our input parameters. In this
Section we, in particular, relate the discussed features of disc
structures (\ref{disc_struct}) and the resulting simulated SEDs
(\ref{sed_analysis}) in terms of the derived photometry, in Section
\ref{indv_phot}. Furthermore, we explore the consequences of our model
grid on the derivation of the primary parameters of age, mass and disc
fractions, from populations, in Section \ref{populations}. This in
turn leads to highlighting selection effects with, for instance, the
mass to accretion rate relation.

\subsection{Effects of Disc Presence and Accretion on Photometry}
\label{indv_phot}

For the derivation of ages optical CMDs, in particular in \textit{V,
  V-I}, are most often used, and indeed most suitable. Whereas, IR
CMDs, such as a \textit{J, J-K} CMD, are most suitable for mass
derivation \cite[see references and discussions in][ and Section
\ref{derived}]{mayne_2007,mayne_2008}.

As discussed in Section \ref{sed_analysis} the presence of a
circumstellar disc and accretion result in significant changes to the
simulated SEDs for our models. Firstly, the photophere of the central
star is dominated by the accretion luminosities for accretion rates of
$\dot{M}>$10$^{-9}M_{\odot}yr^{-1}$ as shown in Section
\ref{acc_dominance}. Secondly, for increasing accretion rates the
outer disc flaring increases, resulting in occultation of the star at
smaller inclination angles, as discussed in Section
\ref{flare_obscure}. Finally, the location of the inner edge and flux
intercepted by this edge lead to changes in its shape and subsequent
dependence of the observed SED with inclination, as discussed in
Section \ref{inner_shape}. In this Section we explore the effects of
these changes in SED shape on the derived photometry in CMDs used for
age and mass derivation for stars. The use of isochrones for the
derivation of single star parameters is completely unreliable
\citep[see discussion in][]{mayne_2008}. However, exploring
the changes in the CMDs of single objects will elucidate the causes of
scatters from the expected locii in CMDs of populations of BDs.

\subsubsection{Age and Mass Derivations}
\label{age_mass_pop}

As discussed in Section \ref{indv_phot} for the derivation of the
primary parameters of age and mass the CMDs most suitable and commonly
used are $M_V$, $(V-I)_0$ and $M_J$, $(J-K)_0$. Furthermore, these
CMDs are indicative of the associated CMDs, for instance $M_R$,
$(R-I)_0$ or $M_J$, $(J-H)_0$ etc. Also as discussed in
\cite{mayne_2008} the use of isochrones for the derivation of
masses and ages for individual pre-MS stars is unreliable at
best. Practically, therefore, median ages are derived from
populations. Subsequently, derived masses are still unreliable but at
least based on a consistent age. This problem is being addressed by
Bell et al (in prep), where $K$ band photometry and known eclipsing
binaries are being used to refine pre-MS isochrones. In this section
we plot the data for our 1 Myr systems only and explore the resulting
scatters caused by the disc presence and accretion luminosity.

Figures \ref{spread_opt_VVI_typ} and \ref{spread_opt_VVI_ext} shows
four CMDs in $M_V$, $(V-I)_0$ for the typical and extreme accretors
respectively. For the typical accretors, Figure
\ref{spread_opt_VVI_typ}, shows the isochrones constructed for naked
systems at 1 Myr, for accretion rates of -12 and -9
log$(\frac{\dot{M}}{M_{\odot}}yr^{-1})$, as the solid and dashed black
lines. Whereas in Figure \ref{spread_opt_VVI_ext} the naked isochrones
of -9 and -6 log$(\frac{\dot{M}}{M_{\odot}}yr^{-1})$, as dashed and
solid black lines respectively. The \textit{top left panel}, in both
Figures (\ref{spread_opt_VVI_typ} and \ref{spread_opt_VVI_ext}) shows
a pre-MS isochrone at 1 Myr from \cite{siess_2000} adjusted
to a distance of 250pc and an extinction of $A_V=2$ mag, simulating a
background population of CTTS stars. The dots in \textit{each panel},
for both Figures (\ref{spread_opt_VVI_typ} and
\ref{spread_opt_VVI_ext}) are the 1 Myr BDD systems for all
inclinations, rotational periods, areal spot coverages and the
individual accretion rates within the accretion type classes. The
\textit{panels} of the Figures \ref{spread_opt_VVI_typ} and
\ref{spread_opt_VVI_ext} then separate the systems by the input
variables. The \textit{top left panel} shows either the accretion
rates of -9, -10 and -11 \& -12 or -6, -7 and -8
log$(\frac{\dot{M}}{M_{\odot}}yr^{-1})$ in Figures
\ref{spread_opt_VVI_typ} and \ref{spread_opt_VVI_ext}
respectively. These three classes are then shown as blue, black and
red dots respectively. The \textit{bottom left panels} then show those
systems with areal coverages of 1 and 10\% as blue and red dots
respectively. The \textit{top right panels} then shows systems with
rotational periods of 0.5 and 5 days as blue and red dots
respectively. Finally, the \textit{bottom right panels} show the
systems separated by inclination. These groups are $\theta \leq
48^{\circ}$ as blue dots (classed as face-on systems), $\theta> 56$ \&
$64^{\circ}$ (classed as the expected systems, as the expectation
value of $cos(\theta)=60^{\circ}$) as black dots and $\theta \geq
71^{\circ}$ (classed as edge-on systems) as red dots. Figure
\ref{spread_opt_RRI_typ} also shows the same data as for Figure
\ref{spread_opt_VVI_typ}, but for $M_R$, $(R-I)_0$ CMDs.

\begin{figure*}
  \vspace*{348pt}
  \special{psfile="Fig/spread_opt_VVI_typ.ps"
   hoffset=470 voffset=0 hscale=60 vscale=60 angle=90}
 \caption{Figure showing CMDs in $M_V$, $(V-I)_0$ for typical
   accretors ($\dot{M} \leq$ -9
   log$(\frac{\dot{M}}{M_{\odot}}yr^{-1})$). The black solid and dashed
   lines in \textit{all panels} are the naked star isochrones for
   accretion rates of -12 and -9 log$(\frac{\dot{M}}{M_{\odot}}yr^{-1})$
   respectively. The \textit{top left} only shows the 1 Myr pre-MS
   isochrone of \citet{siess_2000} adjusted to a distance
   modulus of 7 and an extinction of $A_V=2$, simulating a reddened
   background population. The dots are the data for typical accretors
   for models at 1 Myr. The data are then separated by input
   variables, in the \textit{panels}. The \textit{top left panel}
   shows three accretion classes, -9, -10 and -11 \& -12
   log$(\frac{\dot{M}}{M_{\odot}}yr^{-1})$, as blue, black and red dots
   respectively (the lowest two accretion rates are grouped as they
   are coincident). The \textit{top right panel} shows the rotational
   periods of 0.5 and 5 days as blue and red dots respectively. The
   \textit{bottom left panel} shows the areal spot coverages of 1 and
   10\% as blue and red dots resepctively. The \textit{bottom right
     panel} shows three groups of inclinations, $\theta \leq$
   48$^{\circ}$, $\theta \geq$ 77$^{\circ}$ and $\theta=$ 56, 64 \&
   71$^{\circ}$ , as blue, red and black dots
   respectively.\label{spread_opt_VVI_typ}}
\end{figure*}

As can be seen in Figure \ref{spread_opt_VVI_typ} current accretion in
a star and disc systems, for rates typical in the BD mass regime,
creates a scatter in our simulated photometry indicative of a much
larger isochronal age spread than 10 Myr. Indeed, for many BDD systems
even at nominal accretion rates of -11 or -12
log$(\frac{\dot{M}}{M_{\odot}}yr^{-1})$, for our simulations, the
colours of these stars move significantly blueward of the expected BD
locus in a $M_V$, $(V-I)_0$ CMD. As the dots in Figure
\ref{spread_opt_VVI_typ} are the simulated photometry of BDD systems
over a range of typical input parameters (see Section
\ref{par_space}), an observed coeval 1 Myr population could reasonably
be expected to show a similar scatter. The naked isochrones show that
this spread in simulated photometry, would lead to a spread in
isochronal age greater than the $\approx$10 Myr spread claimed for
higher mass stars in some star forming regions (for instance the
ONC). Furthermore, for the higher accretion rates of -9 or -10
log$(\frac{\dot{M}}{M_{\odot}}yr^{-1})$ the movement of the star
within the CMD will effectively move the star into the contamination
region expected for background CTTS or MS stars at a $(V-I)_0$ of
$\leq$1.5 and $M_V$ 12-10, and as such the star would not be included
in a photometrically selected BD sample. The solid green line, in the
\textit{top left panel}, showing the 1 Myr isochrone of
\cite{siess_2000} at a distance of 250 pc and extinction of
$A_V=2$ mags shows that the BDD systems with higher accretion rates
could easily be confused for a background CTTS or MS population (as MS
stars are simply less luminous at a roughly constant \textit{V-I} than
the CTTS counterparts). Indeed miss classification of a BDD system as
a CTTS system has already been revealed in
\cite{white_2003}. Furthermore, scatter, although somewhat
reduced, can be observed in the simulated photometry of the negligibly
accreting systems, for example in the \textit{top left panel} some red
dots, accreting at -12 or -11 log$(\frac{\dot{M}}{M_{\odot}}yr^{-1})$
(which will be negligible compared to the photospheric flux, see
Figure \ref{acc_dom}), still scatter significantly. These objects may
still be included in a `wide' photometric selection. This scatter is a
strong function of inclination, as shown in the \textit{bottom right
  panel}, where, as the inclination is increased the objects are
pushed lower in the CMD. Indeed, for the edge on cases some objects
have magnitudes fainter than show in Figure \ref{spread_opt_VVI_typ}
($M_V\approx 20$). This is expected as the star becomes obscured by
the flared disc, interestingly for these systems the \textit{bottom
  right panel} shows that this occurs for inclinations above around
71$^{\circ}$ (as found in Figure \ref{flare_inc}) in most cases.
However, even for the lower inclination angles some objects have very
faint magnitudes, this is due to the disc flaring leading to a smaller
opening angle and is discussed in Section \ref{disc_struct}. The
\textit{bottom left panel} shows that for naked systems accretion
rates of -9 log$(\frac{\dot{M}}{M_{\odot}}yr^{-1})$ lead to the stars
being scattered into the region occupied by background CTTS and
contamination (the lower accretion rates are all coincident with the
-12 log$(\frac{\dot{M}}{M_{\odot}}yr^{-1})$ for 1 Myr). Crucially, the
\textit{top left panel} shows that the scatter from the isochrone is,
generally, correlated with accretion rate. Effectively, as the
accretion rate increases the BDD system moves farther away from the
isochrone and is therefore less likely to be classified as a BDD
system and included in any target samples of such objects. The
\textit{top right panel} shows that the scatter, in $M_V$, in our
simulated photometry from the isochrones is significant for both areal
coverages. The \textit{bottom left panel}, shows significant scatter
in the simulated photometry for both adopted rotation periods. This
panel also shows that scatter in this diagram is not obviously
correlated with the rotation period and therefore inner edge position.
Overall, as found in \cite{walker_2004}, the dominant
scattering effect for BD disc systems in an optical CMD appears to be
caused by accretion rate and inclination, and therefore obscuration
effects of the disc on the star. This suggests that for a given
photometric survey of BDD systems accreting at typical accretion rates
and with an expected range of inclinations (centred on around
60$^{\circ}$) one would expect to exclude a significant fraction of
these objects from any isochrone based selection.

\begin{figure*}
  \vspace*{348pt}
  \special{psfile="Fig/spread_opt_VVI_ext.ps"
   hoffset=470 voffset=0 hscale=60 vscale=60 angle=90}
 \caption{As Figure \ref{spread_opt_VVI_typ} except for extreme
   accretors, $\dot{M}\geq$ -8
   log$(\frac{\dot{M}}{M_{\odot}}yr^{-1})$. In this Figure the solid
   black line now shows naked systems with an accretion rate of -6
   log$(\frac{\dot{M}}{M_{\odot}}yr^{-1})$ (as opposed to -12
   log$(\frac{\dot{M}}{M_{\odot}}yr^{-1})$, as in Figure
   \ref{spread_opt_VVI_typ}, the dashed black line is still for an
   accretion rate of -9
   log$(\frac{\dot{M}}{M_{\odot}}yr^{-1})$. \label{spread_opt_VVI_ext}}
\end{figure*}

Figure \ref{spread_opt_VVI_ext} shows that for extreme accretion rates
the BDD objects move significantly blueward and brighter than the
naked isochrone (the naked isochrone at -9
log$(\frac{\dot{M}}{M_{\odot}}yr^{-1})$ is shown for
comparison). Comparison of the \textit{four panels} again show that,
in general, the scatter is chiefly correlated with accretion rate and
inclination. Essentially, Figure \ref{spread_opt_VVI_ext} shows that
for, admittedly extreme, accretion rates BDD systems will not be
included in a photometrically (or indeed spectroscopically, from
Section \ref{sed_analysis}) selected sample.

\begin{figure*}
  \vspace*{348pt}
  \special{psfile="Fig/spread_opt_RRI_typ.ps"
   hoffset=470 voffset=0 hscale=60 vscale=60 angle=90}
 \caption{As Figure \ref{spread_opt_RRI_typ}, but in $M_R$, $(R-I)_0$,
   with the pre-MS isochrone omitted.\label{spread_opt_RRI_typ}}
\end{figure*}

Similar scattering, as observed in Figure \ref{spread_opt_VVI_typ},
occurs when using CMDs constructed using $M_R$, $(R-I)_0$ \citep[using
the photometric system of][]{bessell_1998}, except that the 1
Myr pre-MS isochrone has been omitted. Figure \ref{spread_opt_RRI_typ}
shows the $M_R$, $(R-I)_0$ CMD.

Similar consequences are apparent in CMDs best suited, and most used,
for derivation of masses using isochrones. Figure
\ref{spread_ir_JJK_typ} shows the same data as Figure
\ref{spread_opt_VVI_typ} in the same format except in this case the
pre-MS isochrone of \cite{siess_2000} has been smoothed,
purely for aesthetics and to remove the sharper features. Again Figure
\ref{spread_ir_JJK_ext} contains the same data in the same format as
Figure \ref{spread_opt_VVI_ext} except that the naked star isochrone
at an accretion rate of -6 log$(\frac{\dot{M}}{M_{\odot}}yr^{-1})$ is
not shown as it lies significantly blueward of the CMD.

\begin{figure*}
  \vspace*{348pt}
  \special{psfile="Fig/spread_ir_JJK_typ.ps"
   hoffset=470 voffset=0 hscale=60 vscale=60 angle=90}
 \caption{As Figure \ref{spread_opt_VVI_typ}, in $M_J$, $(J-K)_0$, in
   this Figure the pre-MS isochrone of \citet{siess_2000},
   has been smoothed.\label{spread_ir_JJK_typ}}
\end{figure*}

\begin{figure*}
  \vspace*{348pt}
  \special{psfile="Fig/spread_ir_JJK_ext.ps"
   hoffset=470 voffset=0 hscale=60 vscale=60 angle=90}
 \caption{As for Figure \ref{spread_opt_VVI_ext}, except that the -6
   log$(\frac{\dot{M}}{M_{\odot}}yr^{-1})$ isochrone is not shown (lies
   significantly blueward of locus of data), and the pre-MS isochrone
   has been smoothed.\label{spread_ir_JJK_ext}}
\end{figure*}

Figure \ref{spread_ir_JJK_typ} shows that for typical accretion rates
and expected ranges of the other input variables the spread in CMD
positions is large. Indeed. the \textit{top left panel} again shows
that, as in the $M_V$ $(V-I)_0$ case, some of the BDD objects will
appear as redenned background CTTS objects. Scrutiny of the
\textit{individual panels} shows no clear correlation of CMD position
with input variable except for the inclination. As shown in the
\textit{bottom right panel} as the inclination increases the systems
move to fainter magnitudes and redder colours, in general. The
\textit{top right panel} of Figure \ref{spread_ir_JJK_typ} shows that
for typical accretion rates there does not appear to be a correlation
of $(J-K)_0$ colour with rotation rate as expected from the modeling
of flat disc by \cite{meyer_1997}. It is clear that assigning
masses to accreting BDD systems using isochrones, for our model grid,
would result in the incorrect masses. In addition, photometric sample
selection for BD systems will also exclude the majority of our higher
accretion rate objects. Figure \ref{spread_ir_JJK_ext} shows that for
extreme accretors the scatter is much larger, and again correlated
only with inclination, and perhaps accretion rate to a lesser degree
(\textit{bottom right} and \textit{top left panels} of Figure
\ref{spread_ir_JJK_ext}). 

\subsubsection{Summary}

Therefore, if one adopts the range of input parameters we have used
(see Section \ref{par_space} for justification), our simulated
photometry shows, qualitatively, that a coeval 1 Myr population of
accreting BD stars and BDD systems, with typical accretion rates and
range of inclinations, will exhibit a significant scatter in apparent
isochronal age of ($>$10 Myr). Furthermore, objects with typical (and
extreme) accretion rates are scattered sufficiently in CMD space to
prohibit their identification as pre-MS BDs. Indeed, these objects
would not be included in a photometrically selected sample of BDs, and
as such are unlikely to be assigned the correct masses or ages. The
scatter from the naked 1 Myr systems, generally, increases with
increasing accretion rate. It is important to note at this point that
these conclusions are qualitative, and obviously based on our
assumptions. However, the repercussion for isochronal age derivation
and sample selection in the BD regime could be profound. Indeed this
study has only included the effects of current or ongoing accretion
from a disc, it has neglected any effects of accretion, both past and
present, on the evolution of the central star. This past accretion
could also act to reduce the stars radius, accelerating contraction
\citep{tout_1999,siess_1999} and introducing
additional scatter in a coeval population proportional to the range in
accretion rates \citep[see][for full discussion]{mayne_2008}.
Our findings support those of \cite{walker_2004}, showing that
a significant scatter is caused by obscuration of the central source
by the central disc. In addition we have shown that the typical
accretion rates may scatter BDD systems into the region of a CMD
occupied by the CTTS or background MS locus.

The fact that scatter in the CMD increases with increasing accretion
rate casts doubt on the veracity of the mass to accretion rate
relationship. For our model grid we have not assumed any such
relation, therefore, as our data would also show a similar relation it
suggests that the observed result may be caused by intrinsic
scattering. The relation, $\dot{M}\propto M_*^{2}$ suggests that their
is a dearth of lower mass stars accreting at higher rates. We have
shown that for accretion rates in the range -12 to -9
log$(\frac{\dot{M}}{M_{\odot}}yr^{-1})$ the BDD systems with higher
accretion rates would be preferentially missed using photometric or
isochronal selection.

Additionally, we have adopted the co-rotation radius as the initial
location of our inner wall. This leads to an initial proportionality
between the rotation rate and inner edge temperature. As discussed in
Section \ref{disc_struct} this correlation will be modified due to
dust sublimation effects, weakening the correlation. The derived
photometry however, at varying inclinations, does not show a strong
correlation in IR, $(J-K)_0$, colour with rotational rate, as proposed
by \cite{meyer_1997}.

\subsubsection{Disc fractions}
\label{disc_fractions_phot}

Disc fractions have been derived using infrared excesses previously in
\textit{JHK}, however recent works pre-dominantly use \textit{Spitzer}
IRAC magnitudes. Furthermore, MIPS magnitudes are used to identify
so-called debris discs, where IR excesses are not apparent at shorter
wavelengths. Finally, disc fractions have also been derived using the
$\alpha$ criteria, where $\alpha=\frac{dlog\lambda
  F\lambda}{dlogF\lambda}$ between two limiting wavelengths,
originally used to distinguish amongst Class I, II or II sources, but
now used to detect disc presence
\citep{lada_2006,kennedy_2009}. An $\alpha>$-2 is
used as a selection criterion for disc presence for TTS stars. We have
constructed the $\alpha$ values for our model grid by adopting the
limiting wavelengths of \cite{kennedy_2009}, namely 3.6 to
8.0\,$\mu$m.

As shown in Figure \ref{inner_temp}, Section \ref{disc_struct}, the
inner edge temperature for typical accretors retains some correlation
with rotation rate. In general the more slowly rotating objects having
cooler inner edges. For the more extreme rotators the inner edge
location is still weakly correlated with the inner edge
temperature. In the extreme accretors case the correlation is caused
by the radial fall in density, and therefore dust sublimation
temperature as the inner edge is ablated. Therefore, for the extreme
accretors the correlation between rotational period and inner edge
temperature is lost. The inner edge temperature is in fact correlated
with the flux and therefore accretion rate of the system. As shown in
Figures \ref{spread_ir_JJK_typ} and \ref{spread_ir_JJK_ext} (in
Section \ref{age_mass_pop}), there is no clear correlation in CMD
position, and $(J-K)_0$ colour with rotation rate for either typical
or extreme accretors. In fact any scatter in these Figures
(\ref{spread_ir_JJK_typ} and \ref{spread_ir_JJK_ext}) appears to be
most strongly correlated with accretion rate and inclination.

Figures \ref{disc_frac_JHJK_typ} and \ref{disc_frac_JHJK_ext} show the
$(J-H)_0$, $(J-K)_0$ CoCoDs for typical and extreme accretors
respectively. The dots are the simulate BDD systems for both ages (as
position in these CoCoDs is not a strong function of age) and masses,
and the black crosses the naked systems, again at both ages (and all
masses), for the accretion rates matching the shown BDD data. The
\textit{panels} of these Figures then separate the BDD systems by
accretion rate (\textit{top left}), areal spot coverage
(\textit{bottom left}), rotational period (\textit{top right}) and
inclination (\textit{bottom right}). The accretion rates of -9, -10
and -11 \& -12 log$(\frac{\dot{M}}{M_{\odot}}yr^{-1})$ are shown as
blue, black and red dots respectively in the \textit{top left panel}
of Figure \ref{disc_frac_JHJK_typ}. For Figure
\ref{disc_frac_JHJK_ext}, the \textit{top left panel} shows accretion
rates of -6, -7 and -8 log$(\frac{\dot{M}}{M_{\odot}}yr^{-1})$ are shown
as blue, black and red dots respectively. The \textit{bottom left
  panels} of both Figures \ref{disc_frac_JHJK_typ} and
\ref{disc_frac_JHJK_ext} show the systems with areal spot coverages of
1 and 10\% as blue and red dots respectively. The \textit{top right
  panels} of both Figures \ref{disc_frac_JHJK_typ} and
\ref{disc_frac_JHJK_ext} show the systems with rotational periods of
0.5 and 5 days as blue and red dots respectively. Finally, The
\textit{bottom right panels} of both Figures \ref{disc_frac_JHJK_typ}
and \ref{disc_frac_JHJK_ext} show the systems with inclinations of
$\theta \leq 48^{\circ}$ as blue dots, $\theta> 56$ \& $64^{\circ}$ as
black dots and $\theta \geq 71^{\circ}$ as red dots.


\begin{figure*}
  \vspace*{348pt}
  \special{psfile="Fig/disc_frac_JHJK_typ.ps"
   hoffset=470 voffset=0 hscale=60 vscale=60 angle=90}
 \caption{Figure showing $(J-H)_0$, $(J-K)_0$ colour-colour diagram
   for typical accretors ($\dot{M}<leq$ -9
   log$(\frac{\dot{M}}{M_{\odot}}yr^{-1})$). The dots are then all
   models (at 1 and 10 Myrs) with discs and the black crosses are the
   naked systems, in \textit{all panels}. The data are then separated
   by input variable in \textit{each panel}. The \textit{top left
     panel} shows accretion rates of -9, -10 and -11 \& -12
   log$(\frac{\dot{M}}{M_{\odot}}yr^{-1})$ as blue, black and red dots
   respectively. The \textit{top right panel} shows the systems with
   rotational periods of 0.5 and 5 days as blue and red dots
   respectively. The \textit{bottom left panel} shows systems with
   areal spot coverages of 1 and 10\% as blue and red dots
   respectively. Finally, the \textit{bottom right panel} separates
   the systems into three groups inclination, $\theta \leq$
   48$^{\circ}$, $\theta \geq$ 77$^{\circ}$ and $\theta=$ 56, 64 \&
   71$^{\circ}$ , as blue, red and black dots
   respectively.\label{disc_frac_JHJK_typ}}
\end{figure*}

\begin{figure*}
  \vspace*{348pt}
  \special{psfile="Fig/disc_frac_JHJK_ext.ps"
   hoffset=470 voffset=0 hscale=60 vscale=60 angle=90}
 \caption{As for Figure \ref{disc_frac_JHJK_typ} but for extreme
   accretors, $\dot{M} \geq$ -8 log$(\frac{\dot{M}}{M_{\odot}}yr^{-1})$
   .\label{disc_frac_JHJK_ext}}
\end{figure*}

Figures \ref{disc_frac_JHJK_typ} and \ref{disc_frac_JHJK_ext} show
that for typical and extreme accretion rates there is no clear
correlation of rotation rate, areal coverage or inclination with the
position in the CoCoD. Significantly no clear correlation between IR
colours and rotation rate can be observed for the typical accretors.
This suggests that the weak correlation found between inner edge
temperature and rotational period does not produce a matching
correlation with IR colours. For the extreme accretors, however, a
correlation between IR colours and accretion rate is clear. As the
accretion rate increases the systems move to redder CoCoDs. This is
caused, as discussed by the dust inner edge being sublimated and the
resulting sublimation decreasing, with density, radially. Therefore,
one can not extend the predictions of \cite{meyer_1997},
after adoption of a co-rotation radius to suggest that IR excesses may
be correlated with rotation rates, for typical or extreme accretors.
As can be seen for all the panels of Figures \ref{disc_frac_JHJK_typ}
and \ref{disc_frac_JHJK_ext} there is a slight overlap between the
naked or disc less objects (crosses) and the BDD systems (dots). This
overlap is small, only some of the lowest mass naked objects appear
redward of the main population. Generally, a well placed cut should
identify a large proportion of the disc candidates as is well known,
if there is little complication from variable extinction and a well
defined photometric system. However, the fact that an overlap exist at
all will lead to some confusion as to the position of any disc excess
cut. For real data a cut will be placed at a user identified paucity
or gap in the CoCoD. Therefore, in practice, if any other region of
the population appears equally, or more, sparse the cut is likely to
be miss-placed. In practice these disc fractions are known to be lower
limits of the true values, so ostensibly a miss placed cut, missing
some disc candidates is not problematic.

The analysis in Section \ref{age_mass_pop} and Figures
\ref{disc_frac_JHJK_typ} and \ref{disc_frac_JHJK_ext}, shows that
accretion rate and inclination are the dominant variables controlling
the photometric scatter, of the BDD systems. Therefore, for our
subsequent analysis we will separate the systems using only these
input variables.

Figures \ref{disc_frac_irac_typ} and \ref{disc_frac_irac_ext} show
CoCoDs for colours constructed using simulated IRAC photometry, for
the typical and extreme accreting systems respectively. These two
Figures contain the same data as Figures \ref{disc_frac_JHJK_typ} and
\ref{disc_frac_JHJK_ext}, respectively. The naked systems are again
shown as crosses and the BDD systems as dots. The \textit{left panels}
of these Figures show $([3.6]-[4.5])_0$, $([4.5]-[5.8])_0$ CoCoDs and
the \textit{right panels} $([3.6]-[4.5])_0$, $([5.8]-[8.0])_0$
CoCoDs. The \textit{top panels} then show the accretion rates of -9,
-10, -11 \& -12 log$(\frac{\dot{M}}{M_{\odot}}yr^{-1})$), for Figure
\ref{disc_frac_irac_typ}, and -6, -7 and -8
log$(\frac{\dot{M}}{M_{\odot}}yr^{-1})$), for Figure
\ref{disc_frac_irac_ext}, as blue, black and red dots
respectively. The \textit{bottom panels} of both Figures show the
systems with inclinations of $\theta \leq 48^{\circ}$ as blue dots,
$\theta> 56$ \& $64^{\circ}$ as black dots and $\theta \geq
71^{\circ}$ as red dots.

\begin{figure*}
  \vspace*{348pt}
  \special{psfile="Fig/disc_frac_irac_typ.ps"
   hoffset=470 voffset=0 hscale=60 vscale=60 angle=90}
 \caption{Figure showing typically accreting systems ($\dot{M}\leq$-9
   log$(\frac{\dot{M}}{M_{\odot}}yr^{-1})$), in irac photometric
   bands. The \textit{left panels} show the $([3.6]-[4.5])_0$ against
   $([4.5]-[5.8])_0$ CoCoD, and the \textit{right panels} show the
   $([3.6]-[4.5])_0$ against $([5.8]-[8.0])_0$ CoCoD. In \textit{all
     panels} the dots are disc systems and crosses naked systems. The
   \textit{top panels} then separate the systems by accretion rate
   with, -9, -10 and -11 \& -12 log$(\frac{\dot{M}}{M_{\odot}}yr^{-1})$
   as blue, black and red dots respectively. The \textit{bottom
     panels} separate the systems by inclination with $\theta \leq$
   48$^{\circ}$, $\theta \geq$ 77$^{\circ}$ and $\theta=$ 56, 64 \&
   71$^{\circ}$ , as blue, red and black dots
   respectively. \label{disc_frac_irac_typ}}
\end{figure*}

\begin{figure*}
  \vspace*{348pt}
  \special{psfile="Fig/disc_frac_irac_ext.ps"
   hoffset=470 voffset=0 hscale=60 vscale=60 angle=90}
 \caption{As for Figure \ref{disc_frac_irac_typ} but for extreme
   accretors, $\dot{M} \geq$ -8
   log$(\frac{\dot{M}}{M_{\odot}}yr^{-1})$.\label{disc_frac_irac_ext}}
\end{figure*}

Figure \ref{disc_frac_irac_typ} and \ref{disc_frac_irac_ext} show that
the naked and BDD systems are well separated. Supporting the view that
IRAC magnitudes are well suited for disc candidate identification. For
the typical accretors there is a possible correlation between
inclination and position in the $([3.6]-[4.5])_0$, $([5.8]-[8.0])_0$
CoCoD, (\textit{bottom right panel} of Figure
\ref{disc_frac_irac_typ}). The systems which are edge-on
(i.e. $\theta\geq$77$^{\circ}$) appear closer in colour to the naked
systems. In the case of extreme the accretors the separation between
the naked and BDD systems is correlated with accretion rate for both
CoCoDs (\textit{top panels} of Figure
\ref{disc_frac_irac_ext}). Effectively, as the accretion rate
increases, in general the system moves to redder colours.

Figure \ref{disc_frac_irac_mips_typ} and
\ref{disc_frac_irac_mips_ext}, show the same data as Figure
\ref{disc_frac_irac_typ} and \ref{disc_frac_irac_ext}, in the same
format and with the same notations. Figures
\ref{disc_frac_irac_mips_typ} and \ref{disc_frac_irac_mips_ext} show
CoCoDs of simulated photometry from IRAC and MIPS magnitudes. For both
Figures the \textit{left panels} show $([3.6]-[4.5])_0$,
$([8.0]-24)_0$ and the \textit{right panels} $([3.6]-[5.8])_0$,
$([8.0]-24)_0$ CoCoD, where 24, is the 24\,$\mu$m MIPS channel.

\begin{figure*}
  \vspace*{348pt}
  \special{psfile="Fig/disc_frac_irac_mips_typ.ps"
   hoffset=470 voffset=0 hscale=60 vscale=60 angle=90}
 \caption{Figure showing the $([3.6]-[4.5])$, $([8.0]-24)$ CMDs as the
   \textit{left panels} and $([3.6]-[5.8])$, $([8.0]-24)$ CMDs as the
   \textit{right panels}, combining IRAC and MIPS photometric
   channels. The systems are again separated by accretion rate and
   inclination angle in the \textit{top} and \textit{bottom} panels as
   in figure \ref{disc_frac_irac_typ}.\label{disc_frac_irac_mips_typ}}
\end{figure*}

\begin{figure*}
  \vspace*{348pt}
  \special{psfile="Fig/disc_frac_irac_mips_ext.ps"
   hoffset=470 voffset=0 hscale=60 vscale=60 angle=90}
 \caption{As for Figure \ref{disc_frac_irac_mips_typ} but for extreme
   accretors ($\dot{M} \geq$ -8
   log$(\frac{\dot{M}}{M_{\odot}}yr^{-1})$).\label{disc_frac_irac_mips_ext}}
\end{figure*}

Figures \ref{disc_frac_irac_mips_typ} and
\ref{disc_frac_irac_mips_ext} show that the BDD and naked systems are
very well separated in CoCoDs utilising IRAC and longer wavelength
MIPS data. For the systems with typical accretion rates, Figure
\ref{disc_frac_irac_mips_typ}, no correlation between CoCoD position
and accretion rate is obvious (\textit{top panels}). The \textit{lower
  panels} of this Figure show that, in general, the locus of BDD stars
is spread similarly for all but edge-on systems, which sometimes
appear a little redder in $([8.0]-24)_0$. For the extreme accretors,
Figure \ref{disc_frac_irac_mips_ext}, however, correlations in CoCoD
position with both inclination and accretion rate are clear. The
\textit{top panels} of Figure \ref{disc_frac_irac_mips_ext} show that
as the accretion rate increases, in general, the systems move to
redders colours in $([3.6]-[4.5])_0$ or $([3.6]-[5.8])_0$
(\textit{left} and \textit{right panels} respectively). This is to be
expected from our previous analysis of IRAC CoCoDs using this
photometry. The \textit{lower panels} of Figure
\ref{disc_frac_irac_mips_ext} show a correlation of inclination angle
with the colours in both axes. These two correlations essentially mean
that we can roughly identify systems, with regard to inclination and
accretion rate, by their position within this CoCoD. For instance,
extreme accretors with lower accretion rates seen face-on will occupy
the bottom left of the BDD locus, with edge-on systems with the
maximum accretion rate appearing to the top right. Comparing the
scales on Figures \ref{disc_frac_irac_mips_typ} and
\ref{disc_frac_irac_mips_ext} shows that these systems slightly
overlap, but in general the extreme accretors will lie redward of the
typical accretor locus in both colours.

Figure \ref{disc_frac_mips} shows CoCoDs constructed solely from the
longer wavelength MIPs photometry. Figure \ref{disc_frac_mips} shows
the BDD data shown in previous Figures (such as Figure
\ref{disc_frac_irac_typ} and \ref{disc_frac_irac_ext}) in $(24-70)_0$,
$(70-160)_0$ CoCoDs. The \textit{left} and \textit{right panels} show
the typically and extreme accreting systems, respectively. The
\textit{inset panels}, in \textit{both top panels}, show larger areas
to include the naked systems, as they are well removed from the BDD
locus.

\begin{figure*}
  \vspace*{348pt}
  \special{psfile="Fig/disc_frac_mips.ps"
   hoffset=470 voffset=0 hscale=60 vscale=60 angle=90}
 \caption{Figure showing CMDs of mips photometry for the typical
   ($\dot{M} \leq$ -9 log$(\frac{\dot{M}}{M_{\odot}}yr^{-1})$) and
   extreme ($\dot{M} \geq$ -8 log$(\frac{\dot{M}}{M_{\odot}}yr^{-1})$)
   accretors, as the \textit{left panels} and \textit{right panels}
   respectively. The \textit{top panels} separate the systems by
   accretion rate in the way described in Figures
   \ref{disc_frac_irac_typ} and \ref{disc_frac_irac_ext} in the
   \textit{left} and \textit{right panels} respectively. The
   \textit{bottom panels} then separate the systems by inclination as
   described in Figure \ref{disc_frac_irac_typ}. The \textit{inset
     panels} in the \textit{top panels} then show a larger region with
   the BDD systems as red dots and the naked systems as blue
   crosses.\label{disc_frac_mips}}
\end{figure*}

Figure \ref{disc_frac_mips} shows, in the \textit{inset panels}, that
BDD and naked systems are very well separated in CoCoDs constructed
using MIPs photometry. For the typical accretors, \textit{left panel},
the position within the CoCoD appears strongly correlated with
inclination only, with the edge on systems moving away from the main
BDD locus, as expected. For the extreme accretors, \textit{right
  panel}, significantly more structure is apparent. In this case as
the accretion rate increases the systems move redward in both colours.
Additionally, as the inclination increases the systems move to redder
$(24-70)_0$ colours. This suggests, as with Figure
\ref{disc_frac_irac_mips_ext}, for objects with extreme accretion
rates, CoCoDs of this sort may help distinguish the characteristic
inclination and accretion rate of the system. In practice, due to
saturation and reddening effects and small sample sizes this may prove
difficult. For our study this is interesting in that the fluxes for
wavelengths longward of 24\,$\mu$m come from regions at temperatures of
$~$120 K and below. This suggests that the outer disc flaring, as
discussed in Section \ref{disc_struct}, leads to significantly
structural changes in the outer, cooler disc, as a function of
accretion rate. In addition, observing the separation of the naked and
BDD systems within the CoCoDs it is clear that for systems within our
grid, the best disc indicators are at longer wavelengths, as is well
known from observations.

Disc fractions are also derived using the $\alpha$ value
\citep{lada_2006}, essentially a slope of the SED between
two wavelengths (at wavelengths longer than the stellar flux peak).
Figure \ref{disc_frac_alpha} shows the derived $\alpha$ values between
3.6 and 8.0\,$\mu$m for our entire model grid (against an arbitrary
model number), with dots showing BDD systems and crosses showing naked
systems. The \textit{left} and \textit{right panels} of Figure
\ref{disc_frac_alpha} show the typical and extreme accreting systems
respectively. For TTS $\alpha>-$2 distinguishes between systems with
and without discs \citep{kennedy_2009}. As can be seen in
Figure \ref{disc_frac_alpha} almost all the typical, and all of the
extreme accreting BDD systems would be successfully identified using
this criterion. Suggesting that the $\alpha$ value is a reliable disc
indicator for BD systems. The \textit{top panels} of Figure
\ref{disc_frac_alpha} show the alpha values separated by accretion
rates, with -9, -10 and -11 \& -12, or -6, -7 and -8
log$(\frac{\dot{M}}{M_{\odot}}yr^{-1})$ shown as blue, black and red
dots respectively. The \textit{bottom panels} differentiate systems
seen at different inclinations with $\theta \leq 48^{\circ}$ as blue
dots, $\theta> 56$ \& $64^{\circ}$ as black dots and $\theta \geq
71^{\circ}$ as red dots.

\begin{figure*}
  \vspace*{348pt}
  \special{psfile="Fig/disc_frac_alpha.ps"
   hoffset=470 voffset=0 hscale=60 vscale=60 angle=90}
 \caption{Figure showing $\alpha$ value \citep{lada_2006}
   (against an arbitrary model number) used
   \citet{kennedy_2009}, $\alpha=\frac{dlog(\lambda
     F_{\lambda})}{dlog(\lambda)}|^{3.6}_{8.0}$ ($\lambda$ in
   $\mu$m). Crosses show the simulated objects without discs and dots
   are those with circumstellar disc, the horizontal line is the
   $\alpha>-$2 cut used to identify disc candidates for solar type
   stars in \citet{kennedy_2009}. Note, that stars of
   $M_*<$0.01$M_{\odot}$ are not included due to unreliable flux
   estimates, in the atmosphere models, for longer wavelengths
   ($\lambda>$4\,$\mu$m). The \textit{left panels} show typical
   ($\dot{M} \leq$ -9 log$(\frac{\dot{M}}{M_{\odot}}yr^{-1})$) accretors
   and the \textit{right panels} the extreme ($\dot{M} \geq$ -8
   log$(\frac{\dot{M}}{M_{\odot}}yr^{-1})$) accretors. The \textit{top
     panels} separate the systems by accretion rate in the way
   described in Figures \ref{disc_frac_irac_typ} and
   \ref{disc_frac_irac_ext} in the \textit{left} and \textit{right
     panels} respectively. The \textit{bottom panels} then separate
   the systems by inclination as described in Figure
   \ref{disc_frac_irac_typ}.\label{disc_frac_alpha}}
\end{figure*}

Figure \ref{disc_frac_alpha} shows that except for a few systems
($~\frac{1}{50}$), all of which are typical accretors at edge-on
inclinations, a cut of $\alpha>-$2 would identify all of our BDD
systems. Furthermore, the \textit{top left} panel shows that for
typically accreting systems there would be no significant bias in
selection with accretion rate. The \textit{bottom left panel} shows
that the lowest $\alpha$ values are only apparent for the edge-on
systems. The \textit{right panels} of Figure \ref{disc_frac_alpha},
show that for extreme accretors the systems are well above the disc
discriminator line, and move to greater $\alpha$ values with
increasing accretion rate.

In summary, for our model grid, the best disc discriminators are MIPs
CoCoDs and the $\alpha$ values. With, in general, correlations in
position within the CoCoDs or the $\alpha$ values seen with accretion
rate only for the extreme accretors.

\subsubsection{Practical Observations}
\label{object}

Some recent studies of disc fractions for BD populations have used
data from the \textit{Spitzer} IRAC camera. Figures \ref{irac_cut},
\ref{irac_cut_2} and \ref{irac_cut_3} show the simulated photometry
for our complete model grid. Naked BDs are shown as crosses and BDD
systems as dots, with 1 and 10 Myrs data shown as blue and red dots
respectively. The dashed lines are cuts used in three recent
publications, Figure \ref{irac_cut} is from \cite{luhman_2005}
a study of IC348, Figure \ref{irac_cut_2} from
\cite{Luhman_2008} a study of $\sigma$ Orionis and Figure
\ref{irac_cut_3} a study of the Taurus region from
\cite{gutermuth_2008}. In these cases the effects of extinction
are either negligible in the plotted colours, with values of
$E([3.6]-[4.5])<0.04$ and $E([4.5]-[5.8])<0.02$ for IC348 and $A_V
\leq$4 mag \citep[which will be negligible in the IRAC
CoCoD,][]{allen_2004}, or the cuts have been placed in
intrinsic colour space as for $\sigma$ Orionis.

\begin{figure*}
  \vspace*{348pt}
  \special{psfile="Fig/irac_cut.ps"
   hoffset=470 voffset=0 hscale=60 vscale=60 angle=90}
 \caption{Figure showing all simulated photometry for the entire
   studied parameter range at ages 1Myr and 10Myr as dots (blue and
   red respectively), and the naked photometry as crosses. The dashed
   lines are a recent BD disc excess cuts used in
   \citet{luhman_2005} for IC348 at a nominal age of 3-4Myr
   \citet{mayne_2008}. It must be noted that the naked stars
   of $M=$0.01$M_{\odot}$ are not included in this figure as their
   SEDs do not extend far enough into the IRAC passbands to derive
   reliable colours. The \textit{left panel} shows the typical
   ($\dot{M} \leq$ -9 log$(\frac{\dot{M}}{M_{\odot}}yr^{-1})$) accretors
   and the \textit{right panel} the extreme ($\dot{M} \geq$ -8
   log$(\frac{\dot{M}}{M_{\odot}}yr^{-1})$) accretors.\label{irac_cut}}
\end{figure*}

\begin{figure*}
  \vspace*{348pt}
  \special{psfile="Fig/irac_cut_2.ps"
   hoffset=470 voffset=0 hscale=60 vscale=60 angle=90}
 \caption{Figure showing an observational cut used in
   \citet{Luhman_2008} for data of the $\sigma$ Orionis
   cluster. It must be noted that the naked stars of
   $M=$0.01$M_{\odot}$ are not included in this figure as their SEDs
   do not extend far enough into the IRAC passbands to derive reliable
   colours. The systems are separated by age as described in Figure
   \ref{irac_cut}. The \textit{left} and \textit{right panels} again
   show the typical ($\dot{M} \leq$ -9
   log$(\frac{\dot{M}}{M_{\odot}}yr^{-1})$) and extreme ($\dot{M} \geq$
   -8 log$(\frac{\dot{M}}{M_{\odot}}yr^{-1})$) accretors respectively as
   in Figure \ref{irac_cut}.\label{irac_cut_2}}
\end{figure*}

\begin{figure*}
  \vspace*{348pt}
  \special{psfile="Fig/irac_cut_3.ps"
   hoffset=470 voffset=0 hscale=60 vscale=60 angle=90}
 \caption{Figure showing an observational cut used in
   \citet{monin_2010} for data of the Taurus cloud. The
   criteria applied for BDD system selection are that of
   \citet{gutermuth_2008}. It must be noted that the naked stars
   of $M=$0.01$M_{\odot}$ are not included in this figure as their
   SEDs do not extend far enough into the IRAC passbands to derive
   reliable colours. The systems are separated by age as described in
   Figure \ref{irac_cut}. The \textit{left} and \textit{right panels}
   again show the typical ($\dot{M} \leq$ -9
   log$(\frac{\dot{M}}{M_{\odot}}yr^{-1})$) and extreme ($\dot{M} \geq$
   -8 log$(\frac{\dot{M}}{M_{\odot}}yr^{-1})$) accretors respectively as
   in Figure \ref{irac_cut}.\label{irac_cut_3}}
\end{figure*}

As can be seen from Figures \ref{irac_cut}, \ref{irac_cut_2} and
\ref{irac_cut_3} almost all of the disc systems from our simulated
photometry would be correctly identified using these cuts. It is
important to note that our conclusions so far have been drawn from
differential photometric arguments, in this case we are using
intrinsic colours and these values are extremely sensitive to changes
in zero point and photometric calibration. Figure \ref{irac_cut} shows
that other than a few typically accreting systems the observational
cut of \cite{luhman_2005} would select all of the BDD systems
in our grid. Figure \ref{irac_cut_2} shows that the cut of
\cite{Luhman_2008} would miss some typical and a very small
number of extreme accreting BDD systems. Figure \ref{irac_cut_3} shows
only very few BDD systems will be missed. The number of BDD systems
not identified for Figure \ref{irac_cut_2} and \ref{irac_cut_3} is
larger than that of \ref{irac_cut}, but these Figures show that there
is no preferential selection based on age. A comparison between Figure
\ref{irac_cut_2} and the \textit{bottom right panel} of Figure
\ref{disc_frac_irac_typ}, shows that all of the missed BDD systems are
edge-on systems. Given, that extreme reddening may move naked stars
into the BDD region selected, in general, dis fractions are usually
quoted as lower limits. Therefore, a small number of miss-identified
BDD system is a small effect. This allows us to conclude, that
ubiquitously used BDD cuts used in IRAC CoCoDs are reliable when
applied to our simulated dataset.

In a future publication we will release a fitting tool to be used in
conjunction with the online database of naked and BDD systems (for
both photometry and SED fitting). Therefore, we will be performing
fits to observed systems using our grid. In this work we simply pick
an example star and highlight its position in CMD and CoCoD space,
compared to models covering the expected range of parameters.
\cite{natta_2004} derive an accretion rate of -9
log$(\frac{\dot{M}}{M_{\odot}}yr^{-1})$ for $\rho$ Ophiucus 102. For
this object \cite{natta_2002} find a spectral type of M6,
$T_{\rm eff}$ of 2700\,K, an extinction of $A_V$=3.0, a mass of
$M_*$0.04-0.08$M_{\odot}$ and a distance of 150pc. Photometry for
$\rho$ Ophiucus 102 is also published in \cite{natta_2006},
in $J$, $H$ and $K$. Additionally, $R$ and $I$ photometry has been
published for this object in \cite{wilking_2005}. We have
adjusted our model spectra in the range $M_*$=0.04-0.08$M_{\odot}$,
with accretion rates of -9 log$(\frac{\dot{M}}{M_{\odot}}yr^{-1})$
(for all remaining input parameters) to the correct distance and
reddening. Figure \ref{rho_oph_102} shows, as dots, our BDD systems
with the parameters in the range quoted, with $\rho$ Ophiucus 102 is
shown as a black star. The \textit{panels} of Figure \ref{rho_oph_102}
then show example CMDs and CoCoDs. The \textit{top left panel} shows a
$J$, $J-H$ CMD, the \textit{bottom left panel} a $R$,
$R-I$ CMD, the \textit{top right panel} a $J$, $J-K$ CMD and
finally, the \textit{bottom right panel} a $J-K$, $J-H$ CoCoD.

\begin{figure*}
  \vspace*{348pt}
  \special{psfile="Fig/rho_oph_102.ps"
   hoffset=470 voffset=0 hscale=60 vscale=60 angle=90}
 \caption{Figure showing CMDs and CoCoDs of systems with matching
   parameters to $\rho$ Ophiucus 102 from
   \citet{natta_2002}. The object $\rho$ Ophiucus 102 is
   shown in \textit{all panels} as a black star, the BDD systems are
   the separated by age, with the 1 and 10 Myr systems appearing as
   blue and red dots respectively. The \textit{top left panel} shows a
   $J$, $J-H$ CMD, the \textit{top right panel} a $J$,
   $J-K$ CMD, the \textit{bottom left panel} a $R$, $R-I$
   CMD and the \textit{bottom right panel} shows a $J-H$,
   $J-K$ CoCoD.\label{rho_oph_102}}
\end{figure*}

We have not performed a formal fitting of the object $\rho$ Ophiucus
102. However, qualitatively, Figure \ref{rho_oph_102} shows that for
$\rho$ Ophiucus 102 the colours are well matched by our grid, yet the
magnitude is somewhat brighter. This could be an indication of
uncertainties in the distance to the object, reddening or magnitudes.
A full investigation is beyond the scope of this paper, but, as stated
in a future publication we will release an online fitting tool to be
used in conjunction with our grid.

\section{Conclusions}
\label{conclusions}

We have constructed a model grid of SEDs, and subsequently photometric
magnitudes and colours, for actively accreting BDs with or without an
associated accretion disc. We have modeled the photospheric flux from
these BDs by adopting (and interpolating) the interior `DUSTY00'
models of \cite{chabrier_2000} combined with the `AMES-Dusty',
atmospheric models of \cite{chabrier_2000}. We have then assumed
that accretion occurs from an inner edge of a magnetically truncated
accretion disc (truncated at the co-rotation radius). The accretion
flux is calculated using a simple blackbody emission, given the
derivation of a characteristic spot effective temperature. SEDs were
then produced for both naked BDs and BDD systems. For the BDD systems
we have modeled the disc using the TORUS radiative transfer code using
the Lucy radiative transfer algorithm and incorporating dust
sublimation and including a treatment of vertical hydrostatic
equilibrium (see Section \ref{model} for a discussion of the code). To
produce a `grid' of simulated systems we have varied several input
parameters namely: stellar mass, stellar age, stellar rotation rate,
accretion rate, the areal coverage of the accretion stream and the
system inclination (the disc mass was fixed). The ranges of these
variables were selected to represent and bound typical pre-MS BD
systems, justification is provided using evidence from observational
studies in Section \ref{par_space} and a final list of the values of
these variables can be found in Table \ref{par_space_table}.

Accepting our assumptions, parameter ranges and radiative transfer
code our resulting simulated dataset has allowed us to qualitatively
explore the effects of \emph{active} (current not past accretion)
accretion on disc structure. Furthermore through the simulation of
observations we have explored the effects of accretion, and disc
presence, on both the SEDs, and photometric colours and magnitudes of
these systems. 

As discussed in Section \ref{disc_struct} vertical hydrostatic
equilibrium, when applied to BDs, leads to increased flaring, when
compared to CTTS. This has previously been explored by
\cite{walker_2004}. However, in our study we have included a
simple treatment of accretion. This leads to increased flaring as more
flux reaches the outer disc, and subsequently lower opening angles for
BDD systems with higher accretion rates. Furthermore, the addition of
dust sublimation has shown that for BDD systems the inner disc
location, temperature and vertical size \& shape also varies with
accretion rate. The inner edge position is correlated with temperature
for the lower accreting models as suggested by
\cite{meyer_1997}. For the systems with higher accretion
rates the inner edge temperature is weakly correlated with
temperature, mainly due to the radial fall in density and therefore
dust sublimation temperature. The inner disc edge, initially
prescribed as a vertical wall, then becomes concave and finally convex
as dust sublimation is increased (with increasing flux from higher
rates of accretion).

Subsequently, the SEDs of BD systems with typical accretion rates and
associated discs, are changed significantly from the assumed
underlying photospheric model flux, and therefore become difficult to
classify. In Section \ref{sed_analysis} we have shown that the BD
photosphere becomes `swamped' or overwhelmed by the accretion flux for
rates of $\dot{M}>$-9 log$(\frac{\dot{M}}{M_{\odot}}yr^{-1})$. The outer
disc flaring observed in the BDD systems was shown to cause
occultation and a subsequent, sharp, fall in flux at an inclination
which decreases for more systems with higher accretion rates. The
thermal direct flux emanating from the disc inner edge was found to
fall less dramatically with inclination for the systems with a curved
inner boundary, as expected from the work of
\cite{tannirkulam_2007}. 

Subsequent derivation of photomertric magnitudes has allowed us to
demonstrate that, as expected, increased accretion without disc
presence, moves our naked systems to bluer and brighter magnitudes.
Once a disc is added the increase in accretion flux interacts with the
disc and does not necessarily lead to a simple motion toward brighter
magnitudes and bluer colours. The increased flaring and obscuration
present in BDD systems, over CTTS, leads to rapid falls in magnitude
with inclination as an accretion (or flaring) dependent inclination.
Furthermore, the disc inner edge leads to a shift redwards with
increasing accretion rate as more flux is intercepted by the inner
edge and the inner edge becomes convex and `puffed up'. For more
typically accreting BDD systems with vertical inner edges, the motion
in a CMD, with inclination, is more disjointed. 

In practice, however, most parameters for BDD systems are derived for
populations. We have shown, in Section \ref{photometry} that
derivation of an \emph{isochronal} (or photometric) age from our
simulated photometry of a coeval BD sample, with typical accretion
rates and associated circumstellar discs, would be inaccurate and
exceedingly difficult. Indeed, the resulting photometric colours and
magnitudes could be indicative of a more distant redenned CTTS
population. For more extreme accretion rates the scatter, in CMD
space, is significantly far from the pre-MS locus and as such these
stars have little chance of being selected as BDs. As discussed in
Section \ref{results} this does not include any effects due to past
accretion on the evolution of the central star, which acts to
accelerate the gravitational contraction and make the star appear
older \citep{tout_1999,siess_1999}, further
scattering the apparent age of a coeval population.

Concordantly, \emph{isochronal} derivations of mass and therefore
IMFs, for our simulated photometry, of a coeval population of
accreting BDs with associated discs, would be inaccurate and
problematic. Again caused by the changes in the SEDs as a result of
the accretion flux and increased occultation by the larger degree of
flaring seen in BD discs \citep[for the latter, as found
by][]{walker_2004}

We have also qualitatively explored the effects of accretion and disc
presence in our simulated dataset on disc fraction estimates. As is
currently well known, longer wavelength bandpasses are much more
reliable and suitable for disc identification. As shown in Section
\ref{results} the naked and BDD disc loci were much more clearly
separated in the CoCoD constructed using \textit{Spitzer} IRAC
magnitudes than the shorter wavelength CIT \textit{JHK} passbands. In
addition, we that the slope of the SED from 3.6 to 8.0\,$\mu$m, or
$\alpha$ value, is an effective disc indicator. We have also
tentatively shown that current observational cuts, when applied to our
simulated photometry (with its associated photometric system), results
in the reliable detection of disc candidates, for IRAC and MIPS
colours and $\alpha$ values, and therefore a robust lower limit disc
fraction.

A further, derivative area this study impacts on, and perhaps most
significantly, is the recent evidence for a stellar mass to accretion
rate correlation, of the form: $\dot{M_{acc}}\propto M_{*}^{~2}$
\citep{muzerolle_2003,natta_2004,natta_2006}.
This relationship has been extended into the BD mass regime in
\cite{natta_2006}. However, arguments based on selection and
detection thresholds have already cast this relation into doubt
\citep{clarke_2006}. As we have shown in Section \ref{results}
a relationship of this kind is self-reinforcing as lower mass objects
with higher accretion rates have little chance of being correctly
identified as such due to both the accretion flux and flared
associated disc. Essentially, at present it is unclear how many BD
stars are not included in this relationship due to misidentification.
As explained in \cite{walker_2004}, BD systems with a disc,
without including accretion effects, can have the characteristics of
higher mass CTTS stars, due to increased disc flaring from a reduced
surface gravity in the disc. The effects of accretion at typical or
larger rates further exacerbate the situation both spectroscopically,
as the photospheric flux essentially becomes swamped or completely
veiled, and photometrically as the resulting colours and magnitudes
are significantly shifted. Therefore, for our simulated dataset a
relationship of this sort may well be derived, if typical methods are
used to identify BD objects with discs and derive masses, ages and
accretion rates, even though it is not present.

Finally, although inner edge locations are correlated with their
temperature we do not find a resulting correlation with IR excess. As
our initial inner edge locations are placed at the co-rotation radius
one might expect a correlation between rotation rate and IR excess.
This in turn might suggest that studies of disc presence correlation
with slower rotation rates, exploring disc-locking, may have intrinsic
biases. However, for our systems with dust sublimation, vertical
flaring, accretion and view over a range of inclinations any
correlation is not apparent. The study of \cite{meyer_1997}
used analytical prescribed flat discs. Whether, any correlation is
apparent for such systems using our models is studied in an upcoming
paper where we introduce a fitting tool to be used with the model grid
and release a further set of models with analytic disc structures, and
an increased parameter space.

\subsection{Summary}
\label{conc_summary}

In summary, our simulated dataset shows that for typical parameter
ranges for BD stars and BDD systems, disc presence and accretion flux
lead to:

Difficulty deriving the following stellar parameters for a coeval
population:

\begin{itemize}
\item{Isochronal ages}
\item{Isochronal masses}
\item{IMFs}
\end{itemize}

And we have shown:
\begin{itemize}
\item{\textit{Spitzer} IRAC magnitudes are required for reliable disc
    identification}
\item{An expected correlation in inner disc edge with IR excess does
    not occur for systems with dust sublimation and vertical
    hydrostatic equilibrium view over a range of inclinations.}
\item{Low mass, high accretion rate systems are likely to be
    misidentified and therefore not included in any study relating
    $M_*\propto\dot{M}$}
\end{itemize}

\section[]{ACKNOWLEDGMENTS}

\bibliographystyle{mn2e}
\bibliography{references}
\appendix

\section{Consistency Checks and Problems}
\label{consistency}

Firstly, a check was made on the photospheric input flux and the
resulting stellar direct flux (tagged by {\sc torus}) after the radiative
transfer simulation. The resulting flux distributions should match
most closely for face-on configurations, and then match in shape only,
with the stellar direct flux level dropping towards higher
inclinations, as more photons are scattered and absorbed by the disc.

We also directly compared the magnitudes and colours of our naked BD
systems with the lowest accretion rate (-12
log$(\frac{\dot{M}}{M_{\odot}}yr^{-1})$ to those published in
\cite{chabrier_2000}, in the same photometric system
(\textit{CIT}). We found significant colour differences ($\delta
(J-K)\le$ 0.1), between our derived values and those of
\cite{chabrier_2000}. As a further check we derived the
magnitudes in the \cite{bessell_1998} system (by both adopting
the published zero points, and by using a Vega reference spectrum),
and applied conversions of \cite{leggett_1992}\footnote{We have
  also included the wavelength shift mention in
  \cite{stephens_2004}}, to the \textit{CIT} system. For each
method we failed to match the magnitudes and colours published in
\cite{chabrier_2000}. As a final test we passed the downloaded,
unaltered, atmospheric spectra directly through the filter response
program, without interpolation, for the closest matches in log(g) and
$T_{\rm eff}$ from the interior models published in
\cite{chabrier_2000}. These magnitudes and colours also failed
to match. Therefore, we must conclude that the most likely cause of
the mismatch is due to improvements in the model atmospheres available
online \footnote{http://perso.ens-lyon.fr/france.allard/} (this is
likely as the models available online, have a later time stamp,
$\approx$ 2005 compared to 2000). For the final published magnitudes,
for the naked systems, we have used a similar wavelength resolution as
in our BDD systems, i.e. 200 logarithmically spaced points. This means
that magnitudes derived from these spectra will differ slightly from
those derived from the full spectra, but this effect is negligible,
and increased resolution for only some of our model grid (i.e naked
stars) will hamper comparison between the models.

The final test of the derived colours and magnitudes was a comparison
of the naked systems with the results for the almost face-on BDD
systems. The results for the optical passbands should be similar and
an appraisal of the component SED, i.e. showing the stellar direct
flux, as can be seen in Figure \ref{disc_eg_173}.

We have adopted zeropoints derived using a Vega reference spectrum for
the optical and near-IR passbands. As a test we have compared our
derived zeropoints using the filter response of
\cite{bessell_1998} and the Vega reference spectrum against
those published in \cite{bessell_1998}. For our photometric
systems we integrate the summed number of photons counted by the
simulated telescope systems, however to test the zeropoints and match
the system of \cite{bessell_1998} we must integrate the summed
energy. The zeropoints we derived \citep[with the values of][in
parenthesis]{bessell_1998} were: $U=$20.977(20.94),
$B=$20.499(20.498), $V=$21.116(21.10), $R=$21.676(21.655),
$I=$22.376(22.371), $J=$23.735(23.755), $H=$24.989(24.860),
$K=$25.884(26.006) and $L=$27.809(27.875). Our derived zeropoints and
those of \cite{bessell_1998} match to within 0.05 mags (and
usually much closer) for all bands except the \textit{H,K} and
\textit{L} bands. This is probably due to the previously noted IR
excess (although detected at longer wavelengths) of the observed Vega
spectrum. \cite{bessell_1998} use a combined model spectrum of
Vega and Sirius as their reference spectrum. As a further test we also
used the synthetic A0V stellar spectrum of \cite{cohen_1993}
to derive zeropoints but were still unable to improve the match to the
\cite{bessell_1998} photometric system for the \textit{JHK}
colours. However, for these colours we have adopted the \textit{CIT}
system, but were unable to find published zeropoints, and therefore
used the Vega reference spectrum. Essentially this may mean there is a
small offset in our \textit{JHK} photometry, however as most of our
results are based on differential photometry this will not affect our
conclusions.

A further complication with our adopted photometric systems is the
range of zeropoints available for the \textit{Spitzer} IRAC
photometry. For this study, as stated, we have adopted zeropoints
calculated using the zero magnitude flux from the IRAC handbook
\footnote{http://ssc.spitzer.caltech.edu/documents/som/som8.0.irac.pdf}.
The resulting zeropoints were: Channel 1[3.6]=19.541, channel
2[4.5]=19.089, channel 3[5.8]=17.395 and channel 4[8.0]=17.966. The
corresponding zeropoints derived for the MIPS passbands where: channel
1[24]=2.139, channel 2[70]=-0.2726 and channel 3[160]=-1.990.

In summary, several careful consistency checks were performed to
confirm that the resulting SEDs and photometric magnitudes behaved as
expected and matched any available published results. A failure to
match the published zeropoints in the near-IR bands of the
\cite{bessell_1998} using a Vega reference spectrum was
probably due to an IR excess in our observed Vega spectrum. However as
in general most of the conclusions or implications of this study are
based on differential photometry, this should not affect them
adversely.  Furthermore, a failure to match the published magnitudes
(and colours) for the atmospheric models in
\cite{chabrier_2000}, even using published zeropoints for the
excellently defined system of \cite{bessell_1998}, and
subsequent conversions to the required \textit{CIT} system
\citep{leggett_1992}, was prescribed to an update in the model
atmospheres available online.

\section{Website}
\label{website}

**THIS SECTION NEEDS TO BE SORTED**

As stated throughout this paper the data presented are available from
a web page
\footnote{http://www.astro.ex.ac.uk/research/bd\textunderscore
  disc}. In this Appendix we briefly discuss the data included, and
the different ways of accessing or visualising these data on the
website.

\subsection{Available Data}
\label{web_data}

The magnitudes and colours presented in this paper are avaiable both
as individual magnitudes and as isochrones or mass tracks. Photometric
magnitudes have also been derived for several other systems and are
available online. These are Johnson, Cousins \textit{UBVRI(JHK)}
\citep{johnson_1966,bessell_2005}, \textit{Tycho} $V_t$
and $B_t$ \citep{bessell_2000}, Bessell \textit{UBVRIJHKL}
\citep{bessell_1988,bessell_1998}, SDSS \textit{UGRIZ}
\citep{fukugita_1996}, 2MASS $JHK_s$
\citep{cohen_2003,skrutskie_2006}, MKO \textit{JHK}
\citep{simons_2002,tokunaga_2002}, UKIRT \textit{ZYJHK}
\citep{hawarden_2001}, IRAS 12, 25, 60 and 100\,$\mu$m
\citep{neugebauer_1984} and SCUBA \textit{450WB} and
\textit{850WB} \citep{holland_1999}. For further information on
these magnitudes, such as the filter responses used and the adopted
zeropoints please refer to the website.

\textit{Monochromatic fluxes}

In addition to the magnitudes derived for each of these bands
monochromatic fluxes have also been derived for all bands listed
above. These have been derived following closely the methods of
\cite{robitaille_2006}, extended to further passbands. For details
of the assumed SED shape, central wavelengths and bandpasses adopted
please refer to the website. The derivations of these monochromatic
fluxes will be detailed in a coming paper, which details a fitting
tool associated with these data.

\subsection{Navigation}
\label{navigation}

The main page contains links to download the entire dataset in several
formats alongside files describing the format.

\begin{enumerate}
\item All SEDs (\AA and ergs s$^{-1}$ cm$^{-2} {\rm \AA} ^{-1}$) 
\item All SEDs ($\mu$m and mJy)
\item All Photometric Magnitudes for all BDD systems.
\item All Photometric Magnitudes for all Naked systems.
\item All Monochromatic Fluxes for all Naked systems ($\mu$m and ergs
  s$^{-1}$ cm$^{-2} {\rm \AA} ^{-1}$)
\item All Monochromatic Fluxes for all Naked systems ($\mu$m and mJy)
\item All Monochromatic Fluxes for all BDD systems ($\mu$m and ergs
  s$^{-1}$ cm$^{-2} {\rm \AA} ^{-1}$)
\item All Monochromatic Fluxes for all BDD systems ($\mu$m and mJy)
\end{enumerate}

Also included is a link to the calibration information listing all the
filter response sources, adopted zero points, central wavelengths,
bandwidths and assumed SED shapes (for monochromatic flux derivation).
A brief scientific overview is also available explaining, in general
terms, the dataset and model.

\subsection{Browsing Tools}
\label{tools}

For users who wish to investigate the dataset in a more specific or
interactive fashion two browsing tools are included.

The isochrones, mass tracks and individual stars magnitudes and
colours generated and presented in the work can be queried using the
``Isochrone and Mass Track Tool''. This allows the user to select the
parameters of the model required and retrieve the specific data.

Secondly, an interactive tool is included allowing a user to select a
given set of parameters and download the ``SEDs, monochromatic fluxes
or magnitudes''. In addition this tool plots the SEDs for the Naked
system or the three inclinations, in the case of BDD systems and
allows users to select filters sets, whose monochromatic flux values
will be overlaid on the displayed SEDs.

\label{lastpage}
\end{document}
