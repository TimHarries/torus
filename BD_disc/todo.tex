\begin{abstract}

   1) All graphs of SEDS in log log units.

  1a) Add in the new data and write in using more masses.

  1b) Write in the extra inclinations-check from incs.lis.

  1c) Write in that using same points for photosphere.dat

  1) BOUY ET AL 2008, NEUGEBAUER 1984 AND SKRUTSKIE 2006 TRUNCATED AUTHOR LIST.

  2) CHANGE SYMBOLS OF NAKED STARS IN IR EXCESS AND IR EXCESS 2 TO
  MAKE CLEARER. 

  3) REWRITE CONCLUSIONS WITH MORE REFERENCE TO
  \cite{2004MNRAS.351..607W}, \cite{2006A&A...452..245N} AND
  \cite{2006MNRAS.370L..10C}.

  4) REDO \& CHECK ALL FIGURES.
 
  5) COMPARE FINAL INNER RADIUS, AFTER SUBLIMATION, TO COROTATION
  RADIUS-HOT GAS STUFF.

  6) ADD IN DISCUSSION FOR MIPS AND CHECK ZEROPOINTS

  7) NEED A DISCUSSION OF THE DISC SCALEHEIGHT ETC (PROBABLY IN MODEL
  OR PAR SPACE): $\rho \propto R^{-\alpha}$, $\alpha =
  2.25$. $h\propto R^{\beta}$, $\beta = 1.25$. $\sigma = R^{(\beta
    -\alpha)}$, AND IS CONSERVED. $R_{\rm outer}=300$ AU. DISC SCALEHEIGHT
  AT 100 AU $H(100)=12.5$ AU

  8) ADD IN DISCUSSION OF THE TIME TAKEN FOR EACH MODEL AND
  SED-JUSTIFY SMALL NUMBER OF GRID. EACH MODEL (FOR ALL INCLINATIONS)
  =** HOURS OF SUPERCOMPUTER TIME ON ** CORES
     
  9) WHAT ABOUT EFFECT OF VERTICAL WALL FOR LONGER PERIOD OBJECTS (NO
  DUST SUBLIMATION) VERSUS SHORTER PERIOD AND CURVED WALL
 
  10) PUT IN INNER BOUNDARY PICTURES
  
  11) SORT OUT PLACEMENT OF CLEARPAGE.

  12) SORT RESULTS BIT ABOUT IRAC FLUX, THIS IS FROM REGIONS BEHIND
  THE INNER EDGE SO WHY IS THE TREND REVERSED, IT IS NOT COMING FROM
  THE OUTER FLARED REGIONS OF THE DISC BUT FROM JUST BEHIND THE PUFFED
  UP INNER EGDE. THEREFORE, FOR LONGER PERIOD OBJECTS THE INNER EDGE
  IS SMALLER AND THE DISC FLARED SO REGIONS JUST BEHIND CAN EMIT MORE
  FLUX. FOR SHORTER PERIOD OBJECTS, THE INNER EDGE IS PUFFED UP AND
  DOMINATES THE SED.

  13) TO PROVE THE POINT ABOVE PRODUCE AN IMAGE BATCH  FILE TO PRODUCE
  IMAGES FOR A TEST CASE I.E. CLOSE INNER EDGE DISTANT ONE ETC.

  14) Add in reference to the Mike Meyer paper about changes due to
  inner radius.

15) Reference to bouy et al 2008 showing that the outer radius has a
negligible effect

16) IMAGES.

17) Add in the Herschel filters.

18) Disc mass fraction.

19) IMF implications.

20) ADD IN RHO OPH 102 STUFF- and add in comparison to stars without accretion

21) Write in a bit about do we expect to see the correlation with
rotation rate.

22) Colours of 0.01Msol and 0.01Gyr models are very different.-REMOVE

23) Notes abnout isong 200 pts for SEDS naked in consisitency checks
used full resolution

24) NOTE ABOUT THE FACT THAT 0.01Msol STARS ARE OFF THE BOTTOM OF SOME
PLOTS AS THEY DIVE DOWN VERY ABRUPTLY.ASK ISABELLE ABOUT THIS

25) Add in comments to mention that due to numerics some models have
inner edges up to 100K above the 1700K mean dust sublimation temps.

27) For detections add in piece about the overlapping stars i.e what
would be missed.

34) Note about r=300 but outer radius doesnt matter largey see bouy
and vitell 2008 or something.

   \end{abstract}

\section{Conclusions}
\label{conclusions}

**THIS CLEARLY NEEDS REWRITING AFTER THE NEW DATA ARE ANALYSED**

We have constructed a model grid of SEDs, and subsequently photometric
magnitudes and colours, for actively accreting BDs with or without an
associated accretion disc. We have modeled the photospheric flux from
these BDs by adopting (and interpolating) the interior `DUSTY00'
models of \cite{2000ApJ...542..464C} combined with the `AMES-Dusty',
atmospheric models of \cite{2000ApJ...542..464C}. We have then assumed
that accretion occurs from an inner edge of a magnetically truncated
accretion disc (truncated at the co-rotation radius). The accretion
flux is calculated using a simple blackbody emission, given the
derivation of a characteristic spot effective temperature. SEDs were
then produced for both naked BDs and BDD systems. For the BDD systems
we have modeled the disc using the TORUS radiative transfer code using
the Lucy radiative transfer algorithm and incorporating dust
sublimation and including including a treatment of vertical
hydrostatic equilibrium (see Section \ref{model} for a discussion of
the code). To produce a `grid' of simulated systems we have varied
several input parameters namely: stellar mass, stellar age, stellar
rotation rate, accretion rate, the areal coverage of the accretion
stream and the system inclination (the disc mass was fixed). The
ranges of these variables were selected to represent and bound typical
pre-MS BD systems, justification is provided using evidence from
observational studies in Section \ref{par_space} and a final list of
the values of these variables can be found in Table
\ref{par_space_table}.

Accepting our assumptions, parameter ranges and radiative transfer
code our resulting simulated dataset has allowed us to qualitatively
explore the effects of \emph{active} (current not past accretion)
accretion, and disc presence, on both the SEDs, and photometric
colours and magnitudes of these systems. This analysis (described in
Section \ref{results}) has important implications in two main areas.

\textbf{Effect of accretion and disc on derived parameters:}

Firstly, as disc presence and active accretion change the
characteristics of BD stars, both in terms of its spectrum and
photometric magnitudes, they also, subsequently, imply changes in any
derived parameters from these primary quantities.

\textit{Age.}

As discussed in Section \ref{results}, the SEDs of BD systems with
typical accretion rates and associated discs, are changed significantly
from the assumed underlying photospheric model flux, and therefore
become difficult to classify. We have shown that derivation of an
\emph{isochronal} (or photometric) age from our simulated photometry
of a coeval BD sample, with typical accretion rates and associated
circumstellar discs, would be inaccurate and exceedingly difficult. As
discussed in Section \ref{results} this does not include any effects
due to past accretion on the evolution of the central star, which acts
to accelerate the gravitational contraction and make the star appear
older \citep{1999MNRAS.310..360T,1999A&A...342..480S}, further
scattering the apparent age of a coeval population.

\textit{Mass, therefore IMF.}

Concordantly, \emph{isochronal} derivations of mass and therefore
IMFs, for our simulated photometry, of a coeval population of
accreting BDs with associated discs, would be inaccurate and
problematic. Again caused by the changes in the SEDs as a result of
the accretion flux and increased occultation by the larger degree of
flaring seen in BD discs \citep[for the latter, as found
by][]{2004MNRAS.351..607W}

\textit{Disc fraction-Spitzer best.}

We have also qualitatively explored the effects of accretion and disc
presence in our simulated dataset on disc fraction estimates.As is
currently well known, longer wavelength bandpasses are much more
reliable and suitable for disc identification. As shown in Section
\ref{results} the naked and BDD disc loci were much more clearly
separated in the CoCoD constructed using \textit{Spitzer} IRAC
magnitudes than the shorter wavelength CIT \textit{JHK} passbands. In
addition, we that the slope of the SED from 3.6 to 8.0$\mu$m, or
$\alpha$ value, is an effective disc indicator. We have also
tentatively shown that current observational cuts, when applied to our
simulated photometry (with its associated photometric system), results
in the reliable detection of disc candidates, for IRAC colours and
$\alpha$ values, and therefore a robust lower limit disc fraction.

\textbf{Effect of accretion and disc presence on emergent theories or
  secondary parameters.}

Secondly, the results of our simulated dataset have implications for
secondary derived correlations and physical theories, which utilise
the primary parameters discussed. It is clear that changes in derived
ages and masses of a BD sample, due to typical accretion and disc
presence, has serious implications in a range areas. Inaccuracies in
age derivations for pre-MS populations are well known, especially for
the BD mass regime. However, the combined effects of accretion and
disc presence on BD SEDs, and therefore age and mass derivation, have
not been previously explored. Therefore, some of the uncovered biases
and problems in this paper may bring new light to a few areas of BD
evolution.

\textbf{`Imperfect' disc locking perhaps due to biases.}

\textit{Disc timescales: Rotation rates affect disc detection as function of
  wavelength.}

Firstly, we have found a wavelength dependant bias of naked to BD disc
loci separation within CoCoDs. As we have shown in Section
\ref{results}, for shorter wavelength data (\textit{JHK}), objects
with a longer period lie closer to the naked BD locus, with shorter
period objects further removed. This bias is reversed for IRAC data,
where shorter period objects lie closer to the naked locus (although
still well separated) and longer period objects farther away, in
colour-colour space. Whilst, for the longer wavelength
\textit{Spitzer} data the clearer separation between the naked and BDD
disc loci means that this bias may well be minimised (especially in
regions without significant variable extinction), for shorter
wavelength data these populations (naked and BDD disc) overlap. As
disc excess cuts are usually placed empirically based on perceived
paucity's in stellar density, within colour-colour space, this overlap
and associated bias for \textit{JHK} data could lead to a significant
bias in any sample of BD stars with an inferred discs. This wavelength
dependant bias is essentially due to the change in inner radius of the
circumstellar disc, as the wavelength of emitted flux is dependant on
temperature and therefore radial position within the disc plane. For
shorter periods the disc is truncated closer to the central star and
the disc inner boundary intercepts more flux at all inclinations
leading to a hotter inner wall and less incident flux on the outer
flared edges of the disc. This means that for \textit{JHK} bandpasses,
which are sensitive to flux from the inner boundary, shorter period
objects appear more different to naked objects than longer, and
therefore cooler inner boundary systems.

The effect of this bias has not previously been accounted for in disc
fraction, and therefore disc timescale derivation studies. Our model
does not include angular momentum loss only changes in the disc inner
radius through a disc locking mechanism (and subsequent dust
sublimation). If disc-locking to a central star is true and slower
rotators are therefore more likely to have associated discs, any
correlation between disc presence and rotation rates will be weakened
for \textit{JHK} data and strengthened for \textit{Spitzer} IRAC data
due to the biases we observe, for BDs. These biases may well aid
understanding of early claims for a weaker or `imperfect' disc locking
mechanism for lower mass stars. The use of \textit{JHK} magnitudes
would lead to a less clear correlation between slow rotators and disc
presence for a BD sample.

These biases may also apply to higher mass objects, as would be
expected albeit at a weaker level due to the less significant flaring
seen in the circumstellar disc of TTS. This, in turn, may explain
early difficulties in finding any correlation between rotation rate
and disc presence \citep[amongst other issues such as non-coevality,
see][for discussion]{2007prpl.conf..297H}. Once data at longer
wavelengths was used this correlation was found
\citep{2007ApJ...671..605C}, which may have been enhanced by the
revealed bias. It is important to note however, that this bias is not
strong enough (given the clear separation of our simulated naked and
BDD disc systems in IRAC colour-colour space) to cast any
doubt over this result.  Finally, as we have stated inner holes in
circumstellar disc may well be created by other period-independent
mechanisms such as clearing from gas accretion onto a planet or due to
photoevaporation from UV flux, the latter of which would be increased
at high accretion rates, which we have not included.

\textbf{Accretion rate to mass relationship in doubt for BD regime.}

\textit{Mass to accretion rate relationship-high accretors of low mass
  missed-literature high mass low accretors missed.}

The second area which this study impacts on, and perhaps most
significantly, is the recent evidence for a stellar mass to accretion
rate correlation, of the form: $\dot{M_{acc}}\propto M_{*}^{~2}$
\citep{2003ApJ...592..266M,2004A&A...424..603N,2006A&A...452..245N}.
This relationship has been extended into the BD mass regime in
\cite{2006A&A...452..245N}. However, arguments based on selection and
detection thresholds have already cast this relation into doubt
\citep{2006MNRAS.370L..10C}. As we have shown in Section \ref{results}
a relationship of this kind is self-reinforcing as lower mass objects
with higher accretion rates have little chance of being correctly
identified as such due to both the accretion flux and flared
associated disc. Essentially, at present it is unclear how many BD
stars are not included in this relationship due to misidentification.
As explained in \cite{2004MNRAS.351..607W}, BD systems with a disc,
without including accretion effects, can have the characteristics of
higher mass CTTS stars, due to increased disc flaring from a reduced
surface gravity in the disc. The effects of accretion at typical or
larger rates further exacerbate the situation both spectroscopically,
as the photospheric flux essentially becomes swamped or completely
veiled, and photometrically as the resulting colours and magnitudes
are significantly shifted. Therefore, for our simulated dataset a
relationship of this sort may well be derived, if typical methods are
used to identify BD objects with discs and derive masses, ages and
accretion rates, even though it is not present.

\textbf{Link to deriving accretion rates-paper II}.

This work has argued qualitatively, largely using positions within
CMDs or CoCoDs, highlighting problems with the proposed stellar mass
to accretion rate relationship. A further quantitative examination of
methods used to derive accretion rates and their results when applied
to our simulated dataset is certainly required, ***and this will be
the subject of a future publication***, to confirm our conjectures.

\textbf{SUMMARY}

**WRITE AT THE VERY END**

In summary, our simulated dataset shows that for typical parameter
ranges for BD stars and BDD systems, disc presence and accretion flux
lead to:

Difficulty deriving the following stellar parameters for a coeval population:
\begin{itemize}
\item{Isochronal ages}
\item{Isochronal masses}
\item{IMFs}
\end{itemize}

And we have shown:
\begin{itemize}
\item{\textit{Spitzer} IRAC magnitudes are required for reliable disc
    identification}
\item{A correlation between naked to BDD disc system separation (in a
    CoCoD, as a function of wavelength.}
\item{Low mass, high accretion rate systems are likely to be
    misidentified and therefore not included in any study relating
    $M_*\propto\dot{M}$}
\end{itemize}
