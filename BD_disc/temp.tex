
The resulting converged disc structures, as discussed in Section
\ref{seds} are then used to create simulated SEDs. In this section we
explore features in these derived SEDs and link them back to the
physical structures within the disc discussed in Section
\ref{disc_struct}. For analysis purposes our SEDs will comprise
contributions from four main luminosity sources. Firstly, the
accretion luminosity ($L_{\rm acc}$) and the intrinsic stellar
luminosity ($L_{*}$), will effectively be combined and provide flux at
the shorter wavelength section of the SED. For our parameter space the
temperature of the photospheres has a range of $~$3000\,K$<T_{\rm
  eff}<$1600\,K, this will comprise flux contributions peaking in the
range, 0.97 to 1.8\,$\mu$m ($\lambda_{\rm peak}\propto \frac{1}{T_{\rm
    acc}}$). The accretion spot temperatures, for typical accretors,
will range from $~$340$>T_{\rm acc}>$9400\,K, emitting in the range
0.3 to 8.5\,$\mu$m. For the extreme accretors temperatures which can
reach as high as 53\,000\,K, provide flux to $~$0.05\,$\mu$m. The
second pair of luminosity sources are from constituent parts of the
circumstellar disc. The inner edge, ($L_{\rm wall}$), at temperatures
of $~$370 to $~$1500\,K, or flux contributions from $~$1.9 to
7.8\,$\mu$m \citep[nominally 2 to 3\,$\mu$m][]{dullemond_2001}, and
the outer disc ($L_{\rm disc}$), at temperatures of $~$300\,K or less
contributing flux longward of $~$10\,$\mu$m. Disentangling the
accretion and stellar photospheroc flux becomes difficult as $T_{\rm
  acc} \to T_{\rm eff}$, which for an acretion rate of -9 \logmdot,
occurs at a fractional coverages of 20 and 10\%, for rotation periods
of 5 and 0.5 days (with $M_*=0.04M_{\odot}$ and an age of 1 Myr). As
discussed in Section \ref{seds} the {\sc torus} radiative transfer
code tags the photon packets before they reach the simulated observer
in one of four ways, stellar and thermal, either direct or
scattered. These components allow us to analyse the contributions from
the main flux contributors discussed, i.e. $L_{\rm acc}$ and $L_*$
equals stellar scattered or direct, with thermal emission comprising
$L_{\rm wall}$ and $L_{\rm disc}$. 

Figure \ref{disc_eg_173} shows a typically accreting BDD system with
the component photon contributions highlighted, with
$M_*=0.04M_{\odot}$ at 1 Myr with an areal coverage of 10\%, accretion
rate of \logmdot=$-10$ and rotation period of 0.5 days. The solid
black line is the total simulated SED, with the stellar photons shown
as blue lines and the thermal as red. The scattered and direct
contributions are then shown as dashed and solid lines
respectively. The horizontal and vertical dotted lines then show the
approximate ranges of the flux contributions from the (from left to
right), $L_{\rm acc}$ (blue), $L_*$ (black), $L_{\rm wall}$ (red) and
$L_{\rm disc}$ (black). The vertical dashed lines at the bottom of
Figure \ref{disc_eg_173} show the approximate ranges of the $V$, $I$,
$J$, and $K$ photometric bands we have adopted.

\begin{figure}
  \vspace*{174pt}
  \special{psfile="Fig/disc_eg_173.ps"
   hoffset=250 voffset=-20 hscale=36 vscale=36 angle=90}
 \caption{Figure showing the SED log($\lambda \mu$m) against
   log(flux), flux in ergs$/s/cm^2/{\rm \AA}$) for $M_*=0.04M_{\odot}$
   at 1 Myr with an areal coverage of 10\%, accretion rate of \logmdot
   = $-10$ and rotation period of 0.5 days. The solid black line is
   the total simulated SED. The stellar components of the flux are
   shown in blue and the thermal in red, with direct emission
   appearing as solid lines and scattered photons as dashed lines. The
   vertical dotted bars linked by horizontal bars show the approximate
   (log) wavelength ranges of the luminosity components. With the blue
   lines denoting the accretion luminosity ($L_{\rm acc}$) and the
   first (left to right) black lines showing the photospheric
   contribution ($L_*$). The final two sets of lines show the inner
   edge contribution ($L_{\rm wall}$) in red and the outer disc
   ($L_{\rm disc}$) in black. The vertical and horizontal dashed lines
   (in the \textit{lower panels}) denote the approximate sensitivity
   ranges of our chosen $V$, $I$, $J$ and $K$
   filters. \label{disc_eg_173}}
\end{figure}

Figure \ref{disc_eg_173} shows that the photosphere and accretion
luminosities dominate the SED at wavelengths shortward of
$~$1$\,\mu$m. After this, for this model, the SED is dominated by
contributions from the disc, at shorter wavelengths, the inner edge
and then at larger wavelengths the outer disc. As shown in Figure
\ref{disc_eg_173} the intrinsic stellar photosphere and accretion
luminosities overlap, in terms of peak wavelength of emission. This
overlap essentially means that the combined stellar flux ($L_*+L_{\rm
  acc}$) will eventually become dominated by the accretion flux, as
one increases the accretion rate. With the features of the intrinsic
stellar photosphere or atmosphere eventually become veiled by the
accretion flux. This will lead to difficulties in using spectral
features to identify BDD systems for wavelengths shorter than $~$1
$\mu$m, and, of course, for wavelengths longer than this the SED
becomes dominated by disc emission. Therefore, once the SED of a BDD
system, at shorter wavelengths, becomes dominated by accretion
luminosity spectral classification of these systems (or at least their
central stars) will become unreliable. 

\subsubsection{Accretion dominance}
\label{acc_dominance}

Figure \ref{acc_dom} shows the affect of increasing the accretion
blackbody flux (for increasing accretion rates) for a BD star,
$M=$0.04$M_{\odot}$, at 1 Myr. Whilst the photons originating from the
star (both from $L_*$ and $L_{\rm acc}$) will be tagged as stellar by
{\sc torus} we can separate these flux contributions simply by
observing the naked star system. The panels in Figure \ref{acc_dom}
show the flux from a naked system, with no treatment of the disc. This
enables us to view the effect of increasing accretion rate on the
photospheric flux in isolation. The accretion rates included in all
panels are \logmdot = $-8$, $-9$ and $-12$ (blue, black and red lines
respectively). The \textit{bottom panels} show the systems with a
rotation period of 5 days and \textit{top panels} for those with a
rotation period of 0.5 days. Given our assumption that accretion
occurs from the co-rotation radius, decreasing the rotational period
moves this accretion radius closer to the star, $R_{\rm inner}\propto
\tau^{2/3}$ (see Equation \ref{inner_eq}). As the accretion radius
moves father from the star the potential energy released by the
accreted material is reduced (XXXX WHAT?). This effect can be seen
when comparing the \textit{top} and \textit{bottom panels}, although
the effect is marginal for all but the highest displayed accretion
rates.
\begin{figure*}
  \vspace*{348pt}
  \special{psfile="Fig/acc_dom.ps"
   hoffset=470 voffset=0 hscale=60 vscale=60 angle=90}
 \caption{Figure showing the photospheric flux (log(ergs$/s/cm^2/{\rm
     \AA}$)) against $\lambda$ (log($\mu$m)) of a Brown Dwarf with
   $M=$0.04$M_{\odot}$, at 1 Myr. No disc is included, but blackbody
   fluxes from an accretion stream at the rates of \logmdot $=-8$,
   $-9$ and $-12$ are shown as blue, black and red lines
   respectively. The \textit{bottom panels} show accretion for a star
   rotating at 5 days, with the \textit{top panels} showing that of
   0.5 days. The \textit{left panels} systems with an areal coverage
   of 10\% and the \textit{right panels} has 1\%. The vertical and
   horizontal dashed lines (in the \textit{lower panels}) denote the
   approximate sensitivity ranges of our chosen $V$, $I$, $J$ and $K$
   filters.\label{acc_dom}}
\end{figure*}

The \textit{left panels} show accretion streams with an areal coverage
of 10\% and the \textit{left panels} 1\%. As the areal coverage
reduces the effective temperature of the accretion hot spot increases,
resulting in an increase in accretion flux, and resulting shift to
bluer wavelengths of the peak flux. This can be seen clearly by
comparing the \textit{left} and \textit{right panels} of Figure
\ref{acc_dom}. Perhaps the most important, albeit qualitative, result
shown is Figure \ref{acc_dom} is an insight into the accretion rate at
which the accretion blackbody flux dominates over the photospheric
flux. Figure \ref{acc_dom} shows that as the accretion rate raises
above \logmdot = $-9$ for systems with 1 or 10\% areal coverage, the
accretion flux dominates the emergent SED at both periods. Therefore
for reasonable coverages (1--10\%) and rotational periods (0.5--5 days)
the photospheric flux is effectively swamped by accretion flux for
accretion rates \logmdot$>-9$. 


As is well known the shorter wavelength component of the SED for
extreme accretors is dominated by the accretion luminosity. For the
typical accretors the intrinsic photospheric emission is important and
the flux levels of this component are dependent on the stellar
parameters of mass and age, through evolution of the stellar radius
and temperature, as is well understood. In general, the bluer (short
wavelength) components of the SEDs are therefore controlled by the
stellar input parameters (mass and age) and the accretion input
parameters (accretion rate and areal coverage, and to a lesser extent
rotation rate).

\subsubsection{Flaring and obscuration}
\label{flare_obscure}

As shown in Figures \ref{flare_-12_183} and \ref{flare_-7_193} and
discussed in Section \ref{disc_flaring} BDD systems under vertical
hydrostatic equilibrium have highly flared discs. As discussed in
\cite{walker_2004} this increased flaring (when compared to
CTTS stars) leads to occultation at the star at lower inclination
angles and, therefore, significant changes to the SED. Increases in
the inclination angle, for these systems, quickly lead a significant
proportion of the stellar flux being intercepted, and reprocessed, by
the highly flared disc. This reprocessing will lead to a change in the
flux levels at the shorter, bluer, wavelengths as more stellar flux is
reprocessed and intercepted by the disc. It will also lead to
significant changes in the flux reaching the observer from the inner
and outer regions of the disc. 

Figure \ref{flare_sed_fast} shows the SEDs for a range of inclinations
for BDD systems with two accretion rates. The BDD system has a central
stellar mass of $0.04M_{\odot}$, an age of 1Myr, a rotation period of
0.5 days and an areal coverage of the accretion stream of 10\%. The
\textit{top panels} show the SEDs for a star with an accretion rate of
\logmdot$=-12$ (the lowest in our typical
accretion range). The \textit{bottom panels} then show SEDs for
systems with an accretion rate of \logmdot$=-8$ (classed as an extreme
accretor). From left to right the \textit{panels} then show the naked
stellar flux then the fluxes from BDD systems seen at inclination
angles of 0$^{\circ}$, 64$^{\circ}$ and 90$^{\circ}$, or face-on, the
approximate expectation value of inclination and edge-on. For each of
the \textit{second} to the \textit{fourth panels} the flux components
are then shown as in Figure \ref{disc_eg_173}, i.e. the total SED
shown as a black line, then stellar light shown in blue and thermal in
red, with direct shown as bold and scattered as dashed lines. The
inset graph in the \textit{bottom right} panel of Figure
\ref{flare_sed_fast} simply shows a lower flux scale, as the flux
levels have fallen to the lowest shown division in the main Figure.

\begin{figure*}
  \vspace*{348pt}
  \special{psfile="Fig/flare_sed_fast.ps"
   hoffset=470 voffset=0 hscale=60 vscale=60 angle=90}
 \caption{Figure showing SEDs of a system with a central stellar mass
   of $0.04M_{\odot}$, an age of 1Myr, a rotation period of 0.5 days
   and an areal coverage of the accretion stream of 10\%. The
   \textit{panels}, both \textit{top} and \textit{bottom}, (from left
   to right) show the naked stellar flux (\textit{first panel}), then
   constituent fluxes for SEDs at inclination angles of 0$^{\circ}$,
   64$^{\circ}$ and 90$^{\circ}$. The \textit{second} to
   \textit{fourth panels} on each line then show the total flux as a
   black line, with the component fluxes presented as explained Figure
   \ref{disc_eg_173}. The \textit{top panels} are for systems with an
   accretion rate of -12 log$(\frac{\dot{M}}{M_{\odot}}yr^{-1})$ and
   the \textit{lower panels} are for an accretion rate of -8
   log$(\frac{\dot{M}}{M_{\odot}}yr^{-1})$. The \textit{bottom right
     panel} inset Figure shows an enlargement of the lower flux levels
   of this SED. The vertical and horizontal dashed lines (in the
   \textit{third panels} across) denote the approximate sensitivity
   ranges of our chosen
   $V$, $I$, $J$ and $K$ filters. \label{flare_sed_fast}}
\end{figure*}

Figure \ref{flare_sed_fast} shows, by comparing the \textit{top
  panels} that as a disc is added to the system and the inclination of
the BDD system is increased more stellar (and accretion flux) is
reprocessed by the disc, for a negligibly accreting system. As one
moves from inclinations of 0 to 64$^{\circ}$ the stellar flux has
reduced as has the thermal component. At edge-on inclinations the
system is only visible through scattered light, and some (slight)
thermal emission from the outer disc at wavelengths $\lambda>30\,\mu$m
(indicative of temperatures of $~$100\,K. Increasing the accretion rate
from -12 to -8 log$(\frac{\dot{M}}{M_{\odot}}yr^{-1})$ leads to a
significant increase in the naked stellar flux (compare \textit{top
  left} and \textit{bottom left panels}). As we add a disc and
increase the viewing inclination for the extreme accreting system
(\textit{bottom panels}), we also lose stellar flux. In the extreme
accretor case however the fall in flux contribution to the total SED
from the photosphere (and accretion) is much more significant. Indeed,
at edge-on inclinations the system becomes only visible in the thermal
($~$100\,K) disc regime, with the scattered light significantly reduced.

As shown in Section \ref{disc_struct}, and more specifically, Figures
\ref{flare_-12_183} and \ref{flare_-7_193} as one increases the
accretion rate for BDD systems, in general, the outer disc structure
increases in vertical height. Effectively, the greater flux levels
increase the temperature of the outer disc and the condition of
vertical hydrostatic equilibrium leads to a vertical expansion. This
vertical expansion leads to a significant change in the angle at which
photons emitted from the stellar photosphere are intercepted by the
outer disc. This is an extension of the effected found in
\cite{walker_2004}, where it was found that this effect alone
can shift the SED, and therefore colours and magnitudes, of inclined
BDD systems to those indicative of higher mass CTTS. The actual
changes in flux are complicated by dust sublimation and shape changes
of the inner edge (discussed later). As the inclination of the BDD
system is increased we see the total flux drop significantly once we
exceed the opening angle of the BDD system. This opening angle will
decrease with vertical scaleheight of the outer disc. Figure
\ref{flare_inc} shows the total SEDs for the BDD system of Figures
\ref{flare_-12_183} and \ref{flare_-7_193}, with the corresponding
accretion rates as \textit{left} and \textit{right panels}
respectively. The solid lines of \textit{both panels} of Figure
\ref{flare_inc} show the total SED for each of our ten simulated
inclinations, namely, 0, 27, 39, 48, 56, 64, 71, 77, 84 and
90$^{\circ}$ (with the flux decreasing with increasing
inclination). The dashed line then highlights the inclination at which
the flux observed drops significantly, at 71 and 56$^{\circ}$ for the
accretion rates of -12 and -7 (or \textit{left} and \textit{right
  panels}), respectively. Whilst this effect is probably artificially
enhanced for such extreme accretion rates it is illustrative of the
general trend, notwithstanding complications from inner edge shape.

\subsubsection{SED summary}
\label{sed_summary}

The shape of the combined SED, $L_*+L_{\rm acc}+L_{\rm wall}+L_{\rm
  disc}$, is controlled by the combined stellar flux (intrinsic
stellar temperature and luminosity, and accretion stream temperature
and luminosity), and the disc characteristics (inner edge location and
temperature, and outer disc flaring). The stellar flux and temperature
both increase with increasing accretion rates and reducing areal
coverages. The stellar flux (at constant temperature) also reduces as
a pre-MS star contracts with age (or reduces in mass). Furthermore,
the inner disc flux increases as the inner wall becomes hotter, larger
and more visible with increasing incident flux. The outer disc
component also increases as the outer disc flares. The flaring and
temperature of the disc are controlled by the stellar flux and the
rotation rate. Finally, and most importantly, the shape of the SED
will be dominated by the inclination of the system, especially for
extremely flared systems. This results in the observable SED being a
major function of the input variables, age (and mass), accretion rate
(and areal coverage), rotation period and system inclination.

As we have discussed in this Section and Section \ref{disc_struct} the
underlying disc structures are caused by complicated interplays
between the sublimation physics, vertical hydrostatic equilibrium and
emission of the disc components. However, several underlying trends
have been observed in the physical characteristics of the systems.
Most notably the correlation, to different strengths, of the disc
inner wall location and its resulting temperature, as found in
\cite{meyer_1997}. However, \cite{meyer_1997} go on
to predict a correlation of IR excess with inner edge location. Given
that the initial inner edge location is dependent (chiefly) on the
rotation period one might expect a correlation of rotation rate with
IR excess. This has been shown to be unlikely due to complications in
the inner edge location due to dust sublimation, and obscuration of
the inner disc edge at higher inclination angles. In addition, for
large groups of stars there is evidence for a correlation in stellar
mass with accretion rates. Although we have shown that the observable
SEDs of BDD systems at higher accretion rates are difficult to
classify as such, most surveys supporting this relationship use
photometric data. Therefore, we must examine whether these problems
translate into the broadband photometric magnitudes.
