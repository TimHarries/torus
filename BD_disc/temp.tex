
Data for rotation rates, from periodic variability surveys, are widely
available for a range of different age clusters of TTS. However, fewer
studies exist on the rotation rates of BD mass objects. To select a
feasible range of periods for our model grid we have collated the
results from some recent studies of BD variability. Rotation period
data for $\sigma$ Ori, at an age of $\approx$ 3 Myrs
\citep{2008MNRAS.386..261M}, is studied in \cite{2004A&A...419..249S},
where periods are found over the range 5.78$-$74.4 hours ($\approx$
0.24$-$3.1 days) for BD mass objects. In some cases the periodic
variability is irregular and assumed to come from active accretion hot
spots on the BD surface \citep[see][for discussion of variability
causes]{1995A&A...299...89B,2007prpl.conf..297H}. Infrared excesses
and evidence of accretion are also found to correlate with slower
rotating TTS, which the authors construe as evidence for a disc
locking scenario. \cite{2004A&A...419..249S} also find a $mass\propto
period$ relationship extending into the BD mass regime, this faster
rotation in younger mass objects is taken as evidence for a decreasing
effectiveness of the angular momentum removal of the disc locking
mechanism, i.e. imperfect disc locking. In addition,
\cite{2004A&A...419..249S} argue that when compared with data from
younger clusters disc lifetimes can be seen to decrease with central
object mass.

\cite{2005A&A...429.1007S} study rotation period data for stars in the
vicinity of $\epsilon$ Ori, with an assumed age of $\approx$3 Myrs
\citep{2002A&A...384..937Z} and the ONC at an age of $\approx$2 Myrs
\citep{2008MNRAS.386..261M}. Rotation periods, from photometric
variability, in the range 4.7$-$87.6 hours ($\approx$ 0.2$-$3.65 days)
for BD mass stars are found. Again, \cite{2005A&A...429.1007S} find
some BD objects with irregular variability indicative of accretion hot
spots on the stellar surface. \cite{2005A&A...429.1007S} then analyse
the mass-period relationship of these data in comparison with data
from $\sigma$ Ori \citep[3 Myr,][]{2004A&A...419..249S}, NGC2264
\citep[2 Myr,][]{2004A&A...417..557L} and then ONC \citep[1
Myr][]{2002A&A...396..513H}. They conclude that either the $\epsilon$
Ori sample has an age of around 2$-$3 Myrs or there is still
significant disc locking (concluded using the slope of the mass-period
relation). More specifically for the BD mass regime, they find
\citep[in agreement with][]{2004A&A...419..249S} an age independent BD
lower period limit of 2$-$4 hours, therefore at any age a fraction of
BD stars rotate at short and constant period. Given that these objects
should spin up as they hydrostatically contract they invoke rotational
braking in the BD mass regime as a mechanism of removing significant
angular momentum.

\cite{2003ApJ...594..971J} study the rotational periods of BD (and
very low mass stars) in the Chameleon I region. This region is $\le$ 1
Myrs old, and the authors find rotation periods of 2.19, 3.376 and
3.21 days for their BD mass counterparts. Using supplementary data
from the literature \cite{2003ApJ...594..971J} propose that a shorter
lifetime of $\approx$5 Myrs for accretions discs around BDs is more
probable and is inferred from the derivation of a shorter locking
timescale. However, as discussed in the previous studies, an imperfect
disc locking mechanisms is also hypothesised as responsible for the
less significant loss in angular momentum out to ages of 10 Myrs. The
data mentioned above and other datasets are summarised and discussed
in \cite{2007prpl.conf..297H}, where a strong argument for the
mass-period relationship can be found, shown in their Fig 6, alongside
further support for either a shorter disc lifetime or imperfect disc
locking mechanism for BD stars.

Therefore, in summary, it is currently unclear if any disc locking
mechanisms efficiency or disc lifetimes are a function of mass, with
an associated phase change over the TTS to BD mass limit. Rotation
rate and disc presence data are being used to explore issue.
Therefore, to create a set of useful models to help contextualise the
observational constraints we must adopt a realistic, and bounding,
range of rotation rates. For our model grid, and associated age range
($<$10 Myrs), we have selected 0.5 and 5 days.  The lower limit is
placed at around the approximate median for the faster rotating
samples, providing a useful benchmark for exploring possible detection
limits of faster rotators with discs. The longer period is set at the
approximate edge of the slow rotating tail, providing a limiting case
for the slower rotating, disc-locked, candidates.
