% mn2esample.tex
%
% v2.1 released 22nd May 2002 (G. Hutton)
%
% The mnsample.tex file has been amended to highlight
% the proper use of LaTeX2e code with the class file
% and using natbib cross-referencing. These changes
% do not reflect the original paper by A. V. Raveendran.
%
% Previous versions of this sample document were
% compatible with the LaTeX 2.09 style file mn.sty'
% v1.2 released 5th September 1994 (M. Reed)
% v1.1 released 18th July 1994
% v1.0 released 28th January 1994

\documentclass[useAMS,usenatbib]{mn2e}

% If your system does not have the AMS fonts version 2.0 installed, then
% remove the useAMS option.  
% useAMS allows you to obtain upright Greek characters.
% e.g. \umu, \upi etc.  See the section on "Upright Greek characters" in
% this guide for further information.
%
% If you are using AMS 2.0 fonts, bold math letters/symbols are available
% at a larger range of sizes for NFSS release 1 and 2 (using \boldmath or
% preferably \bmath).
%
% The usenatbib command allows the use of Patrick Daly's natbib.sty for
% cross-referencing.
%
% If you wish to typeset the paper in Times font (if you do not have the
% PostScript Type 1 Computer Modern fonts you will need to do this to get
% smoother fonts in a PDF file) then uncomment the next line
% \usepackage{Times}
\usepackage{lscape}
\usepackage{graphicx}

%%%%% AUTHORS - PLACE YOUR OWN MACROS HERE %%%%%
\newcommand\mnras{MNRAS}
\newcommand\pasp {PASP}
\newcommand\aj   {AJ}
\newcommand\apj  {ApJ}
\newcommand\apjl  {ApJL}
\newcommand\apjs  {ApJS}
\newcommand\aap  {A\&A}
\newcommand\aaps  {A\&AS}
\newcommand\apss  {AP\&SS}
\newcommand\gca  {Geochim. Cosmochim. Acta}
\newcommand\araa {ARA\&A}
\usepackage{multirow}

%%%%%%%%%%%%%%%%%%%%%%%%%%%%%%%%%%%%%%%%%%%%%%%%

\title[Accreting Brown Dwarfs]{Problems with the characterisation of
  accreting Brown Dwarfs: A public grid of simulated SEDs and
  photometry}  \author[N.J.  Mayne and Tim
Harries]{N.J. Mayne$^{1}$\thanks{E-mail: nathan@astro.ex.ac.uk
    (NJM)} and Tim Harries$^{1}$\\
  $^{1}$ School of Physics,
  University of Exeter, Stocker Road, Exeter, EX4 4QL.\\
}
 \begin{document}


\date{Accepted ?. Received ?; in
  original form ?}

\pagerange{\pageref{firstpage}--\pageref{lastpage}} \pubyear{2009}

\maketitle

\label{firstpage}

\begin{abstract}

   1) All graphs of SEDS in log log units.

  1a) Add in the new data and write in using more masses.

  1b) Write in the extra inclinations-check from incs.lis.

  1c) Write in that using same points for photosphere.dat

  1) BOUY ET AL 2008, NEUGEBAUER 1984 AND SKRUTSKIE 2006 TRUNCATED AUTHOR LIST.

  2) CHANGE SYMBOLS OF NAKED STARS IN IR EXCESS AND IR EXCESS 2 TO
  MAKE CLEARER. 

  3) REWRITE CONCLUSIONS WITH MORE REFERENCE TO
  \cite{2004MNRAS.351..607W}, \cite{2006A&A...452..245N} AND
  \cite{2006MNRAS.370L..10C}.

  4) REDO \& CHECK ALL FIGURES.
 
  5) COMPARE FINAL INNER RADIUS, AFTER SUBLIMATION, TO COROTATION
  RADIUS-HOT GAS STUFF.

  6) ADD IN DISCUSSION FOR MIPS AND CHECK ZEROPOINTS

  7) NEED A DISCUSSION OF THE DISC SCALEHEIGHT ETC (PROBABLY IN MODEL
  OR PAR SPACE): $\rho \propto R^{-\alpha}$, $\alpha =
  2.25$. $h\propto R^{\beta}$, $\beta = 1.25$. $\sigma = R^{(\beta
    -\alpha)}$, AND IS CONSERVED. $R_{\rm outer}=300$ AU. DISC SCALEHEIGHT
  AT 100 AU $H(100)=12.5$ AU

  8) ADD IN DISCUSSION OF THE TIME TAKEN FOR EACH MODEL AND
  SED-JUSTIFY SMALL NUMBER OF GRID. EACH MODEL (FOR ALL INCLINATIONS)
  =** HOURS OF SUPERCOMPUTER TIME ON ** CORES
     
  9) WHAT ABOUT EFFECT OF VERTICAL WALL FOR LONGER PERIOD OBJECTS (NO
  DUST SUBLIMATION) VERSUS SHORTER PERIOD AND CURVED WALL
 
  10) PUT IN INNER BOUNDARY PICTURES
  
  11) SORT OUT PLACEMENT OF CLEARPAGE.

  12) SORT RESULTS BIT ABOUT IRAC FLUX, THIS IS FROM REGIONS BEHIND
  THE INNER EDGE SO WHY IS THE TREND REVERSED, IT IS NOT COMING FROM
  THE OUTER FLARED REGIONS OF THE DISC BUT FROM JUST BEHIND THE PUFFED
  UP INNER EGDE. THEREFORE, FOR LONGER PERIOD OBJECTS THE INNER EDGE
  IS SMALLER AND THE DISC FLARED SO REGIONS JUST BEHIND CAN EMIT MORE
  FLUX. FOR SHORTER PERIOD OBJECTS, THE INNER EDGE IS PUFFED UP AND
  DOMINATES THE SED.

  13) TO PROVE THE POINT ABOVE PRODUCE AN IMAGE BATCH  FILE TO PRODUCE
  IMAGES FOR A TEST CASE I.E. CLOSE INNER EDGE DISTANT ONE ETC.

  14) Add in reference to the Mike Meyer paper about changes due to
  inner radius.

15) Reference to bouy et al 2008 showing that the outer radius has a
negligible effect

16) IMAGES.

17) Add in the Herschel filters.

18) Disc mass fraction.

19) IMF implications.

20) ADD IN RHO OPH 102 STUFF- and add in comparison to stars without accretion

21) Write in a bit about do we expect to see the correlation with
rotation rate.

22) Colours of 0.01Msol and 0.01Gyr models are very different.-REMOVE

23) Notes abnout isong 200 pts for SEDS naked in consisitency checks
used full resolution

24) NOTE ABOUT THE FACT THAT 0.01Msol STARS ARE OFF THE BOTTOM OF SOME
PLOTS AS THEY DIVE DOWN VERY ABRUPTLY.ASK ISABELLE ABOUT THIS

25) Add in comments to mention that due to numerics some models have
inner edges up to 100K above the 1700K mean dust sublimation temps.

26) Note about colour colour plots cut in magnitude for J and removed
the 0.01Msol stars from the naked

27) For detections add in piece about the overlapping stars i.e what
would be missed.

28) missing 0.01 in alpha graph also.

29) Adjust size of the smaller figures (i.e non-four) to fit.

30) Change argument around to say rotation period correlation not
present.

31) Plots of inner temp against dust sublimation inner radius etc

32) inclination against magnitude for different accretion rates.-one mass

33) Sot out JJK isochrone

34) Note about r=300 but outer radius doesnt matter largey see bouy
and vitell 2008 or something.

   \end{abstract}

\begin{keywords}
  stars:evolution -- stars:formation -- stars: pre-main-sequence --
  techniques: photometric -- catalogues -- (stars) Hertzsprung-Russell
  H-R diagram
\end{keywords}



\section{Results: Analysis and Discussion}
\label{results}

\textbf{Intro-Questions to answer:IMF, Formation, Disc timescales and
  $\dot{M}\propto M_*^{\approx 2}$=Ages, Masses and Disc fractions}

Uncertainty remains over the formation mechanisms, resulting IMFs and
subsequent interaction mechanisms with associated circumstellar disc,
for BDs. Furthermore, no current theoretical explanation can be
demonstrated for the suspected stellar mass-accretion rate
relationship \citep[$\dot{M}\propto
M_*^2$,][]{2003ApJ...592..266M,2004A&A...424..603N,2006A&A...452..245N}
derived from observations and doubt remains over the existence of a
disc-locking mechanism in BDD systems. Theoretical models and are
built upon constraints, such as ages, masses and disc fractions, and
correlations, such as the accretion rate to stellar mass relation,
derived from the data. Critically, these resulting constraints and
relationships involve both the application of existing models, such as
isochrones to derive ages, and the selection of bias-minimised
samples, for instance removal of foreground or background contaminant
stars. Clearly, if one is to adopt a new theoretical model or use an
observed correlation as a constraint one must explore the suspected
intrinsic biases and the reliability of the selection criteria.

In this Section we explore the effects of disc presence and accretion
(across the described representative ranges, Section \ref{par_space})
for our model grid. Essentially the premise is that if one accepts the
assumptions and physical processes within our model grid would one be
able to derive robust parameters using existing methods and
furthermore, are the correlations found in the data intrinsic to the
grid. 

Practically, most parameters for young pre-MS stars are derived from
surveys of populations, usually open clusters, using broadband
photometry and subsequently constructed colour-magnitude and
colour-colour diagrams (CMDs and CoCoDs respectively,
hereafter). Therefore, in this Section, firstly we demonstrate the
prohibitive effects on the broadband photometry of varying our input
parameters (Section \ref{photometry}). Secondly, we explore the
underlying causes of the photometric scatters and intrinsic
correlations using the SEDs and structure of individual BDD systems
(Section \ref{sed_causes}).

\subsection{Photometry}
\label{photometry}

\subsubsection{Ages and Masses}
\label{age_mass}

For the derivation of ages optical CMDs, in particular in \textit{V,
  V-I}, are most often used, and indeed most suitable. Whereas, IR
CMDs, such as a \textit{J, J-K} CMD, are most suitable for mass
derivation \cite[see references and discussions in][ and Section
\ref{derived}]{2007MNRAS.375.1220M,2008MNRAS.386..261M}. 

To delineate the effects of the accretion rate and circumstellar discs
we have subdivided the grid into two groups, those with accretion
rates typical for higher mass CTTS objects, defined as
$\dot{M}=$10$^{-12}M_{\odot}yr^{-1}$ (negligible) to
$\dot{M}=$10$^{-9}M_{\odot}yr^{-1}$ and those with elevated accretion
rates, where $\dot{M}>$10$^{-9}M_{\odot}yr^{-1}$. For several of the
plots in this section the magnitude and colours for the
$M_*=0.01M_{\odot}$ at high inclinations become extremely faint and
red, respectively. In some cases these objects appear slightly without
the scale of the diagram the axes where limited in this way to be able
to show the changes in colour an magnitude for the majority of stars
better. The colours and magnitudes for these lowest mass stars are not
necessarily unreliable but simple hinder the aesthetics of the plots,
and as they are at the limits of our grid we have decided to omit them
from some Figures by trimming the axes. Additionally, stars at the
highest inclinations, i.e edge on disc systems, are often omitted from
the Figures due to their extremely faint magnitudes, meaning they
would not be practically observable.

\textit{Derivation of Ages}

Figure \ref{spread_opt_VVI} shows four CMDs in \textit{V, V-I} with
isochrones constructed from our data (comprising all masses from Table
\ref{par_space_table}, excluding $M_*=0.01M_{\odot}$ for 10 Myr) at
ages of 1 Myr and 10 Myrs, solid and dashed lines respectively. These
isochrones are produced using the naked BD photometry at an accretion
rate of -12 log$(\frac{\dot{M}}{M_{\odot}}yr^{-1})$ and a areal spot
coverage of 10\%. The solid green line in the \textit{bottom left
  panel} only is the 1Myr isochrone for naked systems at an accretion
rate of -9 log$(\frac{\dot{M}}{M_{\odot}}yr^{-1})$ and a areal spot
coverage of 10\%. The solid green line in the \textit{top left panel}
only is the 1 Myr from \cite{2000A&A...358..593S} adjusted to a
distance of 250pc and an extinction of $A_V=2$ mag, simulating a
background population of CTTS stars. The remaining dots are the
simulated photometry for the BD disc systems with accretion rates
classed as typical (-9 to -12 log$(\frac{\dot{M}}{M_{\odot}}yr^{-1})$)
at 1 Myr (only) for all areal coverages, periods and inclinations (see
Table \ref{par_space_table}). The four panels then highlight the
scatter in CMD space caused by variation across our input parameter
space. The \textit{top left panel} splits the accretion rates into
three, -11 \& -12 log$(\frac{\dot{M}}{M_{\odot}}yr^{-1})$) as blue
dots, -10 log$(\frac{\dot{M}}{M_{\odot}}yr^{-1})$ as black dots and -9
log$(\frac{\dot{M}}{M_{\odot}}yr^{-1})$ as red dots. The
\textit{bottom left panel} shows the scatter due to period changes,
with systems with periods of 5 days shown as blue dots and 0.5 days as
red dots. The \textit{top right panel} shows the scatter caused by
changes in areal coverage, with blue dots showing photometry from 1\%
coverage systems and red dots for 10\% coverage systems. Finally, the
\textit{bottom right panel} shows the results separated by inclination
angles. The inclinations have been separated into three groups for the
purposes of CMD analysis. These groups are $\theta \leq 48^{\circ}$ as
blue dots (classed as face-on systems), $\theta> 56$ \& $64^{\circ}$
(classed as the expected systems, as the expectation value of
$cos(\theta)=60^{\circ}$) as black dots and $\theta \geq 71^{\circ}$
(classed as edge-on systems) as red dots.

\begin{figure*}
  \vspace*{348pt}
  \special{psfile="Fig/spread_opt_VVI.ps"
   hoffset=470 voffset=0 hscale=60 vscale=60 angle=90}
 \caption{CMDs in $M_V, (V-I)_0$ showing,as lines, the isochrones for
   all masses included in Table \ref{par_space_table} for ages 1 Myr
   (solid) and 10 Myr (dashed). The isochrones are for naked BDs
   without a disc, for the accretion rate -12
   log$(\frac{\dot{M}}{M_{\odot}}yr^{-1})$ (shown as in black), with
   an areal coverage of 10\%. The solid green line in the \textit{top
     left panel} only is the 1 Myr from \citet{2000A&A...358..593S}
   adjusted to a distance of 250pc and an extinction of $A_V=2$ mag,
   simulating a background population of CTTS stars. The dots are the
   simulated photometry, of the star and disc system, for all models
   at an age of 1 Myr with the accretion rates classed as typical (-9
   to -12 log$(\frac{\dot{M}}{M_{\odot}}yr^{-1})$). \textit{Top left
     panel} shows the accretion rates separated into three groups -12
   \& -11, -10 and -9 log$(\frac{\dot{M}}{M_{\odot}}yr^{-1})$ as blue,
   black and red dots respectively. \textit{Bottom left panel} shows
   BDD systems with a rotation period of 5 days as blue dots and 0.5
   days as red dots. \textit{Top right panel} shows those simulations
   with an areal coverage of 1\% as blue dots and 10\% as red
   dots. Finally, the \textit{bottom right panel} shows the systems in
   three groups of inclinations, $\theta \leq 48^{\circ}$, as blue
   dots, $\theta \geq 71^{\circ}$ as red dots and $\theta = 56$ \&
   $64^{\circ}$ as black dots. \label{spread_opt_VVI}}
\end{figure*}

As can be seen in Figure \ref{spread_opt_VVI} current accretion in a
star and disc system, for rates typical in the BD mass regime, creates
a scatter in our simulated photometry indicative of a much larger
isochronal age spread than 10 Myr. Indeed, for many BDD systems even
at nominal accretion rates of -11 or -12
log$(\frac{\dot{M}}{M_{\odot}}yr^{-1})$, for our simulations, the
colours of these stars move significantly blueward of the expected BD
locus in a\textit{V, V-I} CMD. As the dots in Figure
\ref{spread_opt_VVI} are the simulated photometry of BDD systems over
a range of typical input parameters (see Section \ref{par_space}), an
observed coeval 1 Myr population could reasonably be expected to show
a similar scatter. The naked isochrones show that this spread in
simulated photometry, would lead to a spread in isochronal age greater
than the $\approx$10 Myr spread claimed for higher mass stars in some
star forming regions (for instance the ONC). Furthermore, for the
higher accretion rates of -9 or -10
log$(\frac{\dot{M}}{M_{\odot}}yr^{-1})$ the movement of the star
within the CMD will effectively move the star into the contamination
region expected for background CTTS or MS stars at a \textit{V-I} of
$\leq$1.5 and $M_V 12-10$, and as such the star would not be included
in a photometrically selected BD sample. The solid green line, in the
\textit{top left panel}, showing the 1 Myr isochrone of
\cite{2000A&A...358..593S} at a distance of 250 pc and extinction of
$A_V=2$ mags shows that the BDD systems with higher accretion rates
could easily be confused for a background CTTS or MS population (as MS
stars are simply less luminous at a roughly constant \textit{V-I} than
the CTTS counterparts). Furthermore, scatter, although somewhat
reduced, can be observed in the simulated photometry in the negligibly
accreting systems, for example in the \textit{top left panel} some red
dots, accreting at -12 or -11 log$(\frac{\dot{M}}{M_{\odot}}yr^{-1})$
(which will be negligible compared to the photospheric flux, see
Figure \ref{acc_dom}), still scatter below the 10 Myr. These objects
may still be included in a `wide' photometric selection. This scatter
is a strong function of inclination, as shown in the \textit{bottom
  right panel}, where, as the inclination is increased the objects are
pushed lower in the CMD. Indeed, for the edge on cases some objects
have magnitudes fainter than show in Figure \ref{spread_opt_VVI}
($M_V\approx 30$). This is expected as the star becomes obscured by
the flared disc, interestingly for these systems the \textit{bottom
  right panel} shows that this occurs for inclinations above around
71$^{\circ}$ in most cases yet even for the lower inclination angles
some objects have very faint magnitudes, this is due to the disc
flaring leading to a smaller opening angle and is discussed later in
Section \ref{disc_struct}. The \textit{top right panel} shows that the
scatter, in \textit{V}, in our simulated photometry from the
isochrones is significant for both areal coverages. However,
significant scatter blueward in colour is slightly more evident for
the lower areal coverage and therefore hotter accretion spot
temperature, as is expected given that the higher spot temperature
contributes more flux blueward of the photosphere. Finally, the
\textit{bottom left panel}, shows significant scatter in the simulated
photometry for both adopted rotation periods. This panel also shows
that scatter in this diagram is not obviously correlated with the
rotation period and therefore inner edge position. Additionally, the
\textit{bottom left panel} shows that for naked systems accretion
rates of -9 log$(\frac{\dot{M}}{M_{\odot}}yr^{-1})$ lead to the stars
being scattered into the region occupied by background CTTS and
contamination (the lower accretion rates are all coincident with the
-12 log$(\frac{\dot{M}}{M_{\odot}}yr^{-1})$ for 1 Myr). Overall, as
found in \cite{2004MNRAS.351..607W}, the dominant scattering effect
for BD disc systems in an optical CMD appears to be caused by
inclination and therefore obscuration effects of the disc on the
star. This suggests that for a given photometric survey of BDD systems
accreting at typical accretion rates and with an expected range of
inclinations (centred on around 60$^{\circ}$) one would expect to
exclude a significant fraction of these objects from any isochrone
based selection.

Similar scattering occurs when using CMDs constructed using \textit{R,
  R-I} \citep[using the photometric system
of][]{1998A&A...333..231B}. Figure \ref{spread_opt_RRI} shows the
\textit{R, R-I} CMD.

\begin{figure*}
  \vspace*{348pt}
  \special{psfile="Fig/spread_opt_RRI.ps"
   hoffset=470 voffset=0 hscale=60 vscale=60 angle=90}
 \caption{As for Figure \ref{spread_opt_VVI} but the 1 Myr isochrone
   of \citet{2000A&A...358..593S} is omitted and the CMDs are for
   $M_R, (R-I)_0$. \label{spread_opt_RRI}}
\end{figure*}

Therefore, if one adopts the range of input parameters we have used
(see Section \ref{par_space} for justification), our simulated
photometry shows, qualitatively, that a coeval 1 Myr population of
accreting BD stars and BDD systems, with typical accretion rates and
range of inclinations, will exhibit a scatter in apparent isochronal
age of $>$10 Myr. Furthermore, some objects will be scattered below
and blueward of a 10 Myr isochrone. This has been shown for CMDs in
\textit{V, V-I} and \textit{R, R-I}. It is important to note at this
point that these conclusions are qualitative, and obviously based on
our assumptions. However, the repercussion for isochronal age
derivation and sample selection in the BD regime could be
profound. Indeed this study has only included the effects of current
or ongoing accretion from a disc, it has totally neglected any effects
of accretion, both past and present, on the evolution of the central
star. This past accretion could also act to reduce the stars radius,
accelerating contraction
\citep{1999MNRAS.310..360T,1999A&A...342..480S} and introducing
additional scatter in a coeval population proportional to the range in
accretion rates \citep[See][for full
discussion]{2008MNRAS.386..261M}. Our findings support those of
\cite{2004MNRAS.351..607W}, showing that a significant scatter is
caused by obscuration of the central source by the central disc. In
addition we have shown that the scatter blueward, into the CTTS or
background MS locus, is caused by active accretion at levels found in
CTTS stars, but currently not regularly recorded in stars classified
or selected as BDs.

\textbf{Derivation of Masses}

Similar consequences are apparent when attempting to derive masses
from isochrones using CMDs. We have constructed CMDs using the
simulated photometry of our model grid in the \textit{JK}
magnitudes. Figure \ref{spread_ir_JJK} shows CMDs in \textit{J, J-K} for
the same data as Figure \ref{spread_opt_VVI}. The symbols have the
same meanings as in Figure \ref{spread_opt_VVI}.

\begin{figure*}
  \vspace*{348pt}
  \special{psfile="Fig/spread_ir_JJK.ps"
   hoffset=470 voffset=0 hscale=60 vscale=60 angle=90}
 \caption{As for Figure \ref{spread_opt_VVI} but the CMDs is for $M_J,
   (J-K)_0$.\label{spread_ir_JJK}}
\end{figure*}

As is shown by Figure \ref{spread_ir_JJK} a similar scatter is
observed in our simulated photometry in the IR CMD as is evident in
its optical counterparts. Once again, as shown in the \textit{top left
  panel}, scatter is seen in both the nominal and higher accretion
rates (-9 to -12 log$(\frac{\dot{M}}{M_{\odot}}yr^{-1})$), meaning
derivation of masses for objects using IR isochronal fitting of these
objects would be very difficult. However, in this CMD the scatter is
generally redward of the isochrone, this is still however into the
region where background CTTS stars would be expected. The \textit{top
  right panel} shows no correlation with areal coverage of the hot
spot. Interestingly the \textit{bottom left panel} shows no clear
evidence for a correlation in CMD position, and therefore distance
from the isochrone, with rotation period of the system as found in
\cite{1997AJ....114..288M}, this is discussed in more detail in
Section \ref{disc_struct}. Additionally, \textit{bottom left panel}
agains shows significant scatter away from the isochrone for the 1 Myr
naked stars at an accretion rate of -9
$(\frac{\dot{M}}{M_{\odot}}yr^{-1})$, however in this case the scatter
is in the opposite sense (blueward as opposed to redward) to that
caused by accretion and disc presence in the BDD systems. Again,
scatter is still caused by changes in inclinations, as shown in the
\textit{bottom right panel}, with fainter magnitudes for more inclined
systems, as expected however, this effect causes a smaller change in
magnitude in the IR due to less of the disc appearing optically thick
at these wavelengths.

Therefore, as for derivations of age, given our assumptions, and the
adopted range of input parameters, a 1 Myr coeval population scatters
significantly in a \textit{J, J-K} CMD. The \textit{J, J-K} CMD is
representative of CMDs constructed from the \textit{JHK}
passbands. So, qualitatively at least, derivation of masses using
isochronal fitting of our simulated photometry would not produce
reliable or accurate results. Again, as discussed for optical CMDs,
these simulations do not include any of the effects of past or current
accretion on the evolution of the central object, which is expected to
significantly exacerbate the observed scatter. This is also true for
the naked systems with small increases in accretion rate leading to
significant scatters in the CMDs. Additionally, given the evidence
discussed in \cite{2004MNRAS.351..607W}, supported by the evidence
from our own simulated photometry shown in Figures
\ref{spread_opt_VVI}, \ref{spread_RI} and \ref{spread_ir_JJK}, it is
clear that young, accreting, BDD systems are likely to be
misidentified when using photometric colours or magnitudes, even at
generally canonical accretion rates. This casts doubt on the veracity
of the mass to accretion rate relationship. For our model grid we have
not assumed any such relation, therefore, as our data would also show
a similar relation it suggests that the observed result may be caused
by intrinsic scattering. The relation, $\dot{M}\propto M_*^{2}$
suggests that their is a dearth of lower mass stars accreting at
higher rates. We have shown that for accretion rates in the range -12
to -9 log$(\frac{\dot{M}}{M_{\odot}}yr^{-1})$ the BDD systems with
higher accretion rates would be preferentially missed using
photometric or isochronal selection.

\subsubsection{Extreme accretion rates}
\label{elevated_accn}

\textbf{CMDs}
 
BDD and naked BD systems with elevated or extreme accretion rates are
also likely to be miss-classified, or not observable in the same
dynamic magnitude range as a BD survey.  Figure \ref{high_acc_opt}
shows our isochrones for the naked BD systems with accretion rates of
-6, -7 \& -8 log$(\frac{\dot{M}}{M_{\odot}}yr^{-1})$ at 1 Myr (solid
lines in red) with areal coverages of 10\%, and the pre-MS isochrone,
at 1 Myr, of \cite{2000A&A...358..593S} for higher mass stars, down to
$M_*=$0.1$M_{\odot}$ (solid black line). The remaining dots are the
simulated photometry from our model grid for stars with an age of 1
Myr and accretion rates of -6, -7 \& -8
log$(\frac{\dot{M}}{M_{\odot}}yr^{-1})$, and the complete range of all
other input parameters (see Table \ref{par_space_table}. As can be
seen by comparing Figure \ref{spread_opt_VVI} with Figure
\ref{high_acc_opt} the simulated photometry for the naked systems
changes dramatically with increasing accretion rate, meaning that even
without the effects of a disc individual highly accreting BDs
dramatically change position in an optical CMD. The simulated
photometry of the disc systems is then separated in each panel. The
\textit{top left panel} shows systems with accretion rates of -6, -7
and -8 log$(\frac{\dot{M}}{M_{\odot}}yr^{-1})$ as red, black and blue
dots respectively. The \textit{top right panel} distinguishes the
areal coverages of 1 and 10\% as red and blue dots respectively. The
\textit{bottom left panel} separates the two rotational periods of 5
and 0.5 days into blue and red dots respectively. Finally, the
\textit{bottom right panel} plots the three adopted inclinations
groups $\theta \leq 48^{\circ}$, $\theta> 56$ \& $64^{\circ}$,
$cos(\theta)=60^{\circ}$) and $\theta \geq 71^{\circ}$, as blue, black
and red dots respectively.

The \textit{top left panel} shows that these objects are generally
scattered to an apparently higher (isochronal) mass and older
(isochronal) age, with the scatter increasing with accretion
rates. This means that as the accretion rate increases the star is
pushed to bluer and brighter colours and magnitudes, effectively, to
higher stellar masses. The \textit{top right panel} shows that this
effect is apparent for both areal coverages, it is perhaps surprising
that in general there does not appear to be a reduction in the
magnitude of the scatter when one moves to the larger areal
coverage. The \textit{bottom left panel} shows again that there is no
correlation in CMD position with rotation rate, and therefore
co-rotation radius (again this is discussed further in Section
\ref{disc_struct}. Finally, the \textit{bottom right panel} again
shows that the system inclination scatters the resulting position
within the CMD, with greater inclination meaning greater obscuration
of the central star, by the disc, and therefore fainter magnitudes and
redder colours, and it is clear that photometrically a highly
obscured, highly accreting BDD system could be photometrically
classified as contamination or a higher mass CTTS system.

Once again, given our assumptions, qualitatively, young, pre-MS BD
disc systems accreting at higher rates would not be identifiable as
such from optical CMDs and isochronal fitting. Meaning it is likely
that our data would indicate a stellar mass to accretion rate
relation, even though we have not adopted such a relationship.

\begin{figure*}
  \vspace*{348pt}
  \special{psfile="Fig/high_acc_opt.ps"
   hoffset=470 voffset=0 hscale=60 vscale=60 angle=90}
 \caption{The optical CMD, $M_V, (V-I)_0$ showing the effects of
   extreme accretion rates. The solid black line is the pre-MS
   isochrones of \citet{2000A&A...358..593S} at 1 Myr for masses down
   to 0.1 $M_{\odot}$, the solid red lines are our derived isochrones
   for the naked photospheric models at accretion rates of -6, -7 \&
   -8 log$(\frac{\dot{M}}{M_{\odot}}yr^{-1})$, colours of red, black
   and blue respectively, using an areal coverage of 10\% (the coolest
   and therefore lowest accretion luminosity). The dots are the
   simulated photometry for the BDD systems for the three highest
   accretion rates (as for the isochrones). \textit{Top left panel}
   shows the accretion rates separated into three groups -6, -7 and -6
   log$(\frac{\dot{M}}{M_{\odot}}yr^{-1})$ as red, black and blue dots
   respectively. The data in the remaining panels is separated into
   the groups used in Figure \ref{spread_opt_VVI} using the same
   symbols and colours.\label{high_acc_opt}}
\end{figure*}

A similar situation is apparent for the IR passbands, where isochrones
are often used to derive stellar masses. Figure \ref{high_acc_ir} is
the IR version of Figure \ref{high_acc_opt}, the only difference
between the figures are the colours and magnitudes, the conventions,
symbols and models are the same. Figure \ref{high_acc_ir} shows the
CMDs using \textit{J, J-K}, also using the same conventions for each
of the \textit{four panels}.

\begin{figure*}
  \vspace*{348pt}
  \special{psfile="Fig/high_acc_ir.ps"
   hoffset=470 voffset=0 hscale=60 vscale=60 angle=90}
 \caption{As above but for $M_J, (J-K)_0$ colours. Here the effects
   of high accretion can shift the stars into higher or lower mass bins.
   \label{high_acc_ir}}
\end{figure*}

As can be seen by comparing Figure \ref{spread_ir_JJK} and Figure
\ref{high_acc_ir}, accreting BD systems without discs are shifted
brighter and bluer resulting in a large error in isochronal mass
derivation. The simulated photometry for BDD systems however, shows
large magnitudes of scatter redward and towards slightly brighter
magnitudes. Given that the reddening vector is approximately
horizontal in a \textit{J, J-K} CMD many of these objects will appear
as heavily obscured objects of higher mass. The \textit{top left
  panel} shows that the scatter does increase for higher accretion
rates, however, even at the lower of the accretion rates (-8
log$(\frac{\dot{M}}{M_{\odot}}yr^{-1})$), significant scatter is
evident. The \textit{top right panel} shows that the scatter is
roughly the same for both areal coverages. The \textit{bottom left
  panel} shows, again, no correlation between rotation rate and
scatter within the CMD. This is unsurprising as at these elvated
accretion rates the increase in flux will lead to significant levels
of dust sublimation of the inner edge, as discussed in more detail in
Section \ref{disc_struct}. Finally, the \textit{bottom right panel}
shows systems with a larger inclination having, generally lower
magnitudes, and therefore lower masses, as expected.

Therefore, again as for the optical case, given our assumptions and
parameter space, BDD systems accreting at higher rates would not be
identified with the correct mass in most cases using IR CMDs.

\subsubsection{Disc fractions}
\label{derive_frac}

To distinguish between the proposed formation mechanisms for BDs, and
to derive disc lifetimes, accurate derivations of disc fractions are
required (as well as stellar masses).

\textbf{Derivation of disc fractions.}

Disc fractions have been derived using infrared excesses previously in
\textit{JHK}, however recent works pre-dominantly use \textit{Spitzer}
IRAC magnitudes. Furthermore, MIPS magnitudes are used to identify
so-called debris discs, where IR excesses are not apparent at shorter
wavelengths. Finally, disc fractions have also been derived using the
$\alpha$ criteria, where $\alpha=\frac{dlog\lambda
  F\lambda}{dlogF\lambda}$ between two limiting wavelengths,
originally used to distinguish amongst Class I, II or II sources, but
now used to detect disc presence
\citep{2006AJ....131.1574L,2009arXiv0901.2603K}. An $\alpha>$-2 is
used as a selection criterion for disc presence for TTS stars. We have
constructed the $\alpha$ values for our model grid by adopting the
limiting wavelengths of \cite{2009arXiv0901.2603K}, namely 3.6 to
8.0$\mu$m.

\subsubsection{Colour-colour plots}
\label{col_col_plots}

Figure \ref{ir_excess} shows CoCoDs in \textit{J-H, J-K} for our
simulated photometry for all accretion rates, masses, inclinations,
coverages and periods, at 1 Myr only. The black crosses are the
simulated photometry of the naked BD systems, with the dots the
simulated photometry of the BDD systems, for 1 Myr only.  The
\textit{top left panel} then separates the accretion rates with -12,
-11, and -10 log$(\frac{\dot{M}}{M_{\odot}}yr^{-1})$ shown as red dots
and -9 log$(\frac{\dot{M}}{M_{\odot}}yr^{-1})$ shown as green dots and
-8, -7, and -6 log$(\frac{\dot{M}}{M_{\odot}}yr^{-1})$ shown as blue
dots. The \textit{top right panel} distinguishes those models with
areal accretion stream coverages of 1 and 10\% as blue and red dots
respectively. The \textit{bottom left panel} separates the rotational
periods with 0.5 days shown as red dots and 5 days as blue dots. The
\textit{bottom right panel} shows each of the three inclination groups
$\theta \leq 48^{\circ}$ as blue dots, $\theta> 56$ \& $64^{\circ}$ as
green dots and $\theta \geq 71^{\circ}$ as red dots.

As can be seen for all the panels of Figure \ref{ir_excess} there is a
slight overlap between the naked or disc less objects and the BDD
systems. This overlap is small, only some of the smallest mass naked
objects appear redward of the main population. Generally, a well
placed cut should identify a large proportion of the disc candidates
as is well known, if there is little complication from variable
extinction and a well defined photometric system. However, the fact
that an overlap exist at all will lead to some confusion as to the
position of any disc excess cut. For real data a cut will be placed at
a user identified paucity or gap in the CoCoD. Therefore, in practice,
if any other region of the population appears equally, or more, sparse
the cut is likely to be miss-placed. In practice these disc fractions
are known to be lower limits of the true values, so ostensibly a miss
placed cut, essentially missing some disc candidates is not
problematic. On further inspection however it does introduce a crucial
bias. In all panels of Figure \ref{ir_excess} potential gaps or drops
in the number of stars can be seen at $J-K \approx$ 1.5 \& 2.5. The
\textit{top left panel} shows that changes in accretion rate do not
appear, qualitatively, to correlate with CoCoD position, indeed
changes in the different areal coverages (\textit{top left panel}),
inclinations (\textit{bottom right panel}) and rotation rate
(\textit{bottom left panel}) also appear to produce random scatter.



Critically a strong correlation can be
observed between CoCoD position and rotation period of the system in
the \textit{bottom left panel}. This panel shows that longer period
objects, in general, lie closer to the naked BD locus and are
therefore more likely to be classed as disc less stars. Conversely,
almost all of the short period objects lie well separated from the
naked BD locus, and one could also argue for a separation between the
two populations of different rotation periods at $J-K \approx$2. It is
therefore, likely, that early surveys of BD (or perhaps TTS) stars
attempting to find a correlation between disc presence and slower
rotation rates, to support disc locking, were heavily affected by this
bias. Essentially, unless the data are free from variable extinction
and photometric errors are small, manually placing a disc excess cut
based on population densities is likely to preferentially miss slower
rotators and include faster rotators. This leads to an
anti-correlation between disc presence and slow rotation period.

\begin{figure*}
  \vspace*{348pt}
  \special{psfile="Fig/ir_excess.ps"
   hoffset=470 voffset=0 hscale=60 vscale=60 angle=90}
 \caption{Figure showing relative positions of accreting BDs with
   (naked) and without discs in $(J-H)_0, (J-K)_0$ CoCoDs. \textit{Top
     left panel}, this shows all studied accretion rates as green, red
   blue and black dots for -6, -7, -8 and -9, -10, -11, -12 (together)
   respectively log$(\frac{\dot{M}}{M_{\odot}}yr^{-1})$.
   \textit{Lower left panel}, here blue dots are star disc systems
   with a rotation period of 5 days and red for 0.5
   days. \textit{Upper right panel}, blue dots are those simulations
   with an areal coverage of 1\% and red dots have 10\%. Finally, the
   \textit{lower right panel}, the systems at different inclinations
   are shown with 0, 60 and 90$^{\circ}$ shown as black, blue and
   red dots respectively.
   \label{ir_excess}}
\end{figure*}

More recent studies, and indeed those studies where strong support for
a disc locking mechanism is found, use data from the IRAC camera on
the \textit{Spitzer} space telescope \citep[as for TTS,
see][]{2007ApJ...671..605C}. Therefore, using our simulated photometry
in these passbands we are able to investigate any possible expected
biases in observations of this type. Figures \ref{irac_excess} and
\ref{irac_excess_2} are similar to Figure \ref{ir_excess} showing the
same models and using the sames conventions and symbol meanings, only
showing constructed CoCoDs using IRAC channels. It is important to
note that objects, either naked or with discs, of mass,
$M_*=$0.01$M_{\odot}$ are not included in Figures \ref{irac_excess} and
\ref{irac_excess_2}, as their flux becomes unreliable and noise
dominated at these longer wavelengths. Figure \ref{irac_excess} shows
[4.5]-[5.8] against [3.6]-[4.5] and Figure \ref{irac_excess_2}
shows [5.8]-[8.0] against [3.6]-[4.5], two commonly used disc
identification CoCoDs. In both these figures the naked (disc less) BD
population is very well separated from the main population, meaning in
a region free of variable extinction and with reliable photometry disc
excess cuts can be placed with a high degree of confidence. As seen
for Figure \ref{ir_excess} little correlation is found between CoCoD
position and accretion rate, areal coverage or inclination, however, a
clear correlation is again found with rotational period.

The effect of rotational period on the position of the BDD systems in
the IRAC CoCoDs is the reverse of that found for \textit{JHK}. Longer
period objects will be observed as lying further from the naked BD
locus, resulting in any uncertainty in the cut position preferentially
including shorter period systems. This may introduce a slight bias in
disc fraction surveys carried out using IRAC data, perhaps falsely
increasing the apparent strength of a disc-locking type correlation
between inferred disc presence and slower rotators. However, if a
similar effect is apparent in TTS (as would be expected albeit
slightly reduced due to less significant flaring of the outer disc),
the much clearer separation between naked and disc systems means that
this bias will be much easier to avoid.  Therefore, our results do not
affect the recent evidence for a disc-locking mechanism in TTS.

\begin{figure*}
  \vspace*{348pt}
  \special{psfile="Fig/irac_excess.ps"
   hoffset=470 voffset=0 hscale=60 vscale=60 angle=90}
 \caption{As for figure \ref{ir_excess} but in IRAC colours
   [3.6]-[4.5] against [4.5]-[5.8]. 
   \label{irac_excess}}
\end{figure*}

\begin{figure*}
  \vspace*{348pt}
  \special{psfile="Fig/irac_excess_2.ps"
   hoffset=470 voffset=0 hscale=60 vscale=60 angle=90}
 \caption{As for figure \ref{irac_excess} but in IRAC colours
   [3.6]-[4.5] against [5.8]-[8.0]. 
   \label{irac_excess_2}}
\end{figure*}

\textbf{IRAC and MIPS diagrams}

**NOTE ABOUT $M_*=$0.01$M_{\odot}$ CURRENTLY NOT REMOVED**

Figures \ref{irac_mips_excess} and \ref{irac_mips_excess_2} show
CoCoDs for colours constructed using IRAC and MIPS data. These Figures
include the same models as Figures \ref{irac_excess} and
\ref{irac_excess_2}, presented using the same symbols and colour
conventions. The longer wavelength MIPS data provides better
sensitivity to the outer disc emission. As with the IRAC data
alone the separation between the BDD and naked systems is large and
the clearest correlation between input parameter and colour-colour
space position is rotation period. A evident in the \textit{bottom
  left panels} of Figures \ref{irac_mips_excess} and
\ref{irac_mips_excess_2}, the longer period systems generally lie
redward in both [3.6]-[4.5] and [8.0]-[24.0] than the shorter period
systems. Some correlation of colour-colour position and inclination is
evident from the \textit{bottom right panels} of Figures
\ref{irac_mips_excess} and \ref{irac_mips_excess_2}. In this case as
the inclination increases the systems generally move redward in
[8.0]-[24.0].

\begin{figure*}
  \vspace*{348pt}
  \special{psfile="Fig/irac_mips_excess.ps"
   hoffset=470 voffset=0 hscale=60 vscale=60 angle=90}
 \caption{A CoCoD constructed using both IRAC and MIPS data showing
   [3.6]-[4.5] against [8.0]-[24.0]. The symbols are as described in
   As for figure \ref{ir_excess}.
   \label{irac_mips_excess}}
\end{figure*}

\begin{figure*}
  \vspace*{348pt}
  \special{psfile="Fig/irac_mips_excess_2.ps"
   hoffset=470 voffset=0 hscale=60 vscale=60 angle=90}
 \caption{As for figure \ref{irac_mips_excess} but using colours
   [3.6]-[5.8] against [8.0]-[24.0].
   \label{irac_mips_excess_2}}
\end{figure*}

Figure \ref{mips_four} shows CoCoDs constructed using only MIPS data,
plotting the colours 24$-$70$\mu$m against 70$-$160$\mu$m for the 1 Myr
naked and BDD systems. Figure \ref{mips_four} shows that, again, the
naked and BDD systems are well separated. Figure \ref{mips_four} also
shows the correlations in scatter for each input variable. For
\textit{all panels} of Figure \ref{mips_four} the crosses show the
naked systems and the dots are the BDD systems. The \textit{upper left
  panel} separates the two accretion rates with -8 and -12
log$(\frac{\dot{M}}{M_{\odot}}yr^{-1})$ plotted as blue and red dots
respectively. In the \textit{upper right panel} the BDD systems with 1
and 10\% areal coverages are plotted as blue and red dots
respectively. The \textit{lower left panel} plots the periods of 5 and
0.5 days as blue and red dots respectively. \textit{lower right panel}
separates the BDD systems by inclination with 0, 60 and 90$^{\circ}$
plotted as black, blue and red dots respectively. As can be see in
Figure \ref{mips_four} the inclination correlates strongly with CoCoD
position with the almost edge-on systems appearing much redder than
the naked or face on BDD systems. Furthermore, using the \textit{lower
  panels} it can be seen that for each inclination the CoCoD position
is then separated by rotation period, with two separated populations
for each inclination group.

\begin{figure*}
  \vspace*{348pt}
  \special{psfile="Fig/mips_four.ps"
   hoffset=470 voffset=0 hscale=60 vscale=60 angle=90}
 \caption{As for figure \ref{irac_excess} but in MIPS colours
   [24]-[70]($\mu$m) against [70]-[160]($\mu$m). 
   \label{mips_four}}
\end{figure*}

\subsubsection{Observational cuts}
\label{obs_cuts}

Some recent studies of disc fractions for BD populations have used
data from the \textit{Spitzer} IRAC camera. Figures
\ref{irac_cut} and \ref{irac_cut_2} show the simulated photometry for
our complete model grid. Naked BDs are shown as crosses and BDD
systems as dots, with 1 and 10 Myrs data shown as blue and red dots
respectively. The dashed lines are cuts used in two recent
publications, Figure \ref{irac_cut} is from \cite{2005ApJ...631L..69L}
a study of IC348 and Figure \ref{irac_cut_2} from
\cite{2008ApJ...688..362L} a study of $\sigma$ Orionis. In both cases
the effects of extinction are either negligible in the plotted
colours, with values of $E([3.6]-[4.5])<0.04$ \& $E([4.5]-[5.8])<0.02$
for IC348, or the cuts have been placed in intrinsic colour space as
for $\sigma$ Orionis. As can be seen from Figures \ref{irac_cut} and
\ref{irac_cut_2} most of the disc systems from our simulated
photometry would be correctly identified using these cuts. It is
important to note that as for Figures \ref{irac_excess} and
\ref{irac_excess_2} stars of mass, $M_*=$0.01$M_{\odot}$, are not
included due to unreliable flux levels at these
wavelengths. Additionally, our conclusions so far have been drawn from
differential photometric arguments, in this case we are using
intrinsic colours and these values are extremely sensitive to changes
in zero point and photometric calibration.

\begin{figure}
  \vspace*{174pt}
  \special{psfile="Fig/irac_cut.ps"
   hoffset=235 voffset=0 hscale=30 vscale=30 angle=90}
 \caption{This shows all simulated photometry for the entire studied
   parameter range at ages 1Myr and 10Myr as dots (blue and red
   respectively), and the naked photometry as crosses. The dashed
   lines are a recent BD disc excess
   cuts used in \citet{2005ApJ...631L..69L} for IC348 at a nominal age
   of 3-4Myr \citet{2008MNRAS.386..261M}. It must be noted that the naked
   stars of $M=$0.01$M_{\odot}$ are not included in this figure as their
   SEDs do not extend far enough into the IRAC passbands to derive
   reliable colours.
   \label{irac_cut}}
\end{figure}

\begin{figure}
  \vspace*{174pt}
  \special{psfile="Fig/irac_cut_2.ps"
   hoffset=235 voffset=0 hscale=30 vscale=30 angle=90}
 \caption{This figure shows an observational cut used in
   \citet{2008ApJ...688..362L} for data of the $\sigma$ Orionis
   cluster. It must be noted that the naked
   stars of $M=$0.01$M_{\odot}$ are not included in this figure as their
   SEDs do not extend far enough into the IRAC passbands to derive
   reliable colours.
   \label{irac_cut_2}}
\end{figure}

\textbf{alpha}

Disc fractions are also derived using the $\alpha$ value
\citep{2006AJ....131.1574L}, essentially a slope of the SED between
two wavelengths (at wavelengths longer than the stellar flux
peak). Figure \ref{alpha_zoom} shows the derived $\alpha$ values
between 3.6 and 8.0$\mu$m for our entire model grid (against an
arbitrary model number), with dots showing BDD systems and crosses
showing naked systems. As shown for TTS $\alpha>-$2 distinguishes
systems with discs \citep{2009arXiv0901.2603K}. As can be seen in
Figure \ref{alpha_zoom} almost all (except $\frac{3}{1120}$) disc
systems would be successfully identified using this
criterion. Suggesting that the $\alpha$ value is a reliable disc
indicator for BD systems. Figure \ref{alpha_four} show the same Figure
(Figure \ref{alpha_zoom} in \textit{four panels} with each panel
showing spread caused by a particular input variable. The
\textit{upper left panel} shows the alpha values for two different
accretion rates, with -12 and -8
log$(\frac{\dot{M}}{M_{\odot}}yr^{-1})$ shown as blue and red dots
respectively. The \textit{upper right panel} differentiates
populations with different areal coverages of the accretion stream,
where 1 and 10\% are shown as blue and red dots respectively. The
\textit{lower left panel} shows scatter caused by rotational period
changes, with periods of 5 and 0.5 days shown as blue and red dots
respectively. Finally, the \textit{lower right panel}, the systems at
different inclinations are shown with 0, 60 and 90$^{\circ}$ shown as
black, blue and red dots respectively. It is important to note that
the lowest mass objects ($M_*=$0.01$M_{\odot}$) do not appear in Figures
\ref{alpha_zoom} or \ref{alpha_four} due to unreliable flux estimates
for the stellar atmospheres at the longer wavelengths ($\lambda >$4$\mu$m).

Figure \ref{alpha_four} shows, as one would expect a similar
scattering pattern to the IRAC photometry. This is as the $\alpha$
value has been derived over a similar wavelength range, namely, 3.6
to 8.0$\mu$m. Essentially, the $\alpha$ values shows the scatter is
uncorrelated with coverage and accretion rate, as expected for the
shorter wavelength accretion flux contribution. As the inclination
angle increases the $\alpha$ value generally reduces. Finally, the
shorter period objects lie closer to the naked BD locus in $\alpha$
space.

\begin{figure}
  \vspace*{174pt}
  \special{psfile="Fig/alpha_zoom.ps"
   hoffset=235 voffset=0 hscale=30 vscale=30 angle=90}
 \caption{This Figure shows the $\alpha$ value
   \citep{2006AJ....131.1574L} (against an arbitrary
   model number)  used
   \citet{2009arXiv0901.2603K}, $\alpha=\frac{dlog(\lambda
     F_{\lambda})}{dlog(\lambda)}|^{3.6}_{8.0}$ ($\lambda$ in
   $\mu$m). Crosses show the simulated objects without discs and dots
   are
   those with circumstellar disc, the horizontal line is the
   $\alpha>-$2 cut used to identify disc candidates for solar type
   stars in \citet{2009arXiv0901.2603K}. Note, that stars of
   $M_*<$0.01$M_{\odot}$ are not included due to unreliable flux
   estimates, in the atmosphere models, for longer wavelengths
   ($\lambda>$4$\mu$m).
   \label{alpha_zoom}}
\end{figure}

\begin{figure*}
  \vspace*{348pt}
  \special{psfile="Fig/alpha_four.ps"
   hoffset=470 voffset=0 hscale=60 vscale=60 angle=90}
 \caption{As Figure \ref{alpha_zoom} except each of the \textit{four
     panels} shows the underlying scatter separated by input
   variable. The \textit{upper left panel} shows the alpha values for
   two different accretion rates, with -12 and -8
   log$(\frac{\dot{M}}{M_{\odot}}yr^{-1})$ shown as blue and red dots
   respectively. The \textit{upper right panel} differentiates
   populations with different areal coverages of the accretion stream,
   where 1 and 10\% are shown as blue and red dots respectively. The
   \textit{lower left panel} shows scatter caused by rotational period
   changes, with periods of 5 and 0.5 days shown as blue and red dots
   respectively. Finally, the \textit{lower right panel}, the systems
   at different inclinations are shown with 0, 60 and 90$^{\circ}$
   shown as black, blue and red dots respectively. Note, that stars of
   $M_*<$0.01$M_{\odot}$ are not included due to unreliable flux
   estimates, in the atmosphere models, for longer wavelengths
   ($\lambda>$4$\mu$m).
   \label{alpha_four}}
\end{figure*}

\subsection{Physical causes of scatters and correlations}
\label{scatter_causes}
USE THIS BIT FOR THE SEDS SECTION
**ADD IN TEMPERATURE RANGES FOR DISCUSSION HERE**

The scatters observed in our photometry, caused by varying our input
parameters, are obviously caused by changes in the simulated
SEDs. These changes are in turn due to changes in the physical
structure of the system. In this Section we examine the simulated SEDs
and resulting temperature and density structure of the discs to
explain the photometric scatters.

As discussed, Figure \ref{spread_opt_VVI} shows that for the optical CMD
(\textit{V, V-I}), the scatter in colour appears correlated with areal
coverage and accretion rate of the accretion stream. Objects with
higher accretion rates and smaller areal coverages and, therefore,
higher fluxes or accretion luminosities scatter blueward of the
isochrone. There is also perhaps evidence of a weak correlation with
rotation period and colour scatter in Figure
\ref{spread_opt_VVI}. Finally, the scatter in the magnitude (\textit{V}),
and to some extent colour (\textit{V-I}), present in the data in
Figure \ref{spread_opt_VVI} is clearly correlated with inclination. For
the IR CMD (\textit{J, J-K}), once again, scatter in magnitude appears
strongly correlated with inclination. There is also a clear
correlation of scatter in colour with rotation period, but the
remaining two variables appear to show random scatter only.

To explore the causes of the shifts we must relate the changes in the
resulting SEDs to the relevant physical parameters. Essentially, the
summed flux is comprised of four main components, the stellar
photosphere (luminosity $L_*$), the accretion stream (luminosity
$L_{\rm acc}$), the disc inner boundary or wall (luminosity $L_{\rm
  wall}$), and the outer regions of the, highly-flared, circumstellar
disc (luminosity $L_{\rm disc}$). For a given mass and age, the radius
and intrinsic stellar photospheric flux are fixed. For our model grid
the fluxes are produced at effective temperatures of 3000K$<T_{\rm
  eff}$1600K.

The flux from the accretion stream is dominant at shorter wavelengths,
within our range, peaking below a wavelength of 1$\mu$m (see Figure
\ref{acc_dom}). This is as the accretion hotspot can reach much higher
temperatures than the stellar photosphere, up to 16700K. The accretion
luminosity is calculated using equation \ref{Lacc} (in Section
\ref{physics}), and the resulting flux levels are computed using an
effective temperature (Equation \ref{Tacc}, Section \ref{physics}) to
generate a blackbody emission curve. As the accretion flux is
calculated using a blackbody law, the peak wavelength is a function of
the effective temperature of the accretion spot, $\lambda_{\rm
  peak}\propto \frac{1}{T_{\rm acc}}$. Therefore, the effects of the
accretion flux component on the resulting SED, in terms of total flux
and peak wavelength, are a strong function of the input variables,
accretion rate and areal coverage. Essentially, high accretion rates
and low areal coverages lead to an increased influence of accretion
stream on the final SED. The increased accretion flux creates a blue
excess, easily dominating the cooler BD photospheres at shorter
wavelengths. The converse of this is also true i.e. lower accretion
rates with larger hot spots results in a reduction of the influence of
the accretion flux on the SED.

The flux from the inner boundary is a function of the temperature of
the inner wall. This is dependent on its proximity to the central
object. The initial inner wall radius was set as the magnetic
co-rotation radius (see Section \ref{physics}). This distance is a
strong function of rotational period, with $R_{\rm inner}\propto
\tau^{2/3}$ (see Equation \ref{inner_eq}). During the radiative
transfer simulations the inclusion of dust sublimation effects will
result in a shift of the inner wall radius for the hotter wall
temperatures. The differences in the initial and final inner wall
radius are discussed in Section \ref{inner_edge_position}. However,
they do not show significant differences between initial and final
inner wall radii. Therefore, for our purposes the inner disc radius
can still be considered proportional to the rotation period. This
inner boundary will clearly be cooler than the stellar surface, or the
accretion hot spot, and will therefore contribute flux at longer
wavelengths. The resulting flux component, for the inner wall,
dominates at longer wavelengths, peaking at between 2 and 3$\mu$m
\citep{2001ApJ...560..957D}. This is due to its cooler temperature,
which is neccesarily less than the dust sublimation ($\approx$1400K).
 
**CHECK THIS EMISSION FROM IRAC ACTUALLY FROM JUST BEHIND THE INNER
EDGE-THEREFORE ARGUMENT STILL STANDS BUT MUST ADD IN A PUFFING UP OF
THE INNER EDGE EFFECT**

Finally, the flux from the outer flared region of the disc is simply a
function of the temperature of the outer disc. This, in turn, is a
function of the stellar flux reaching these regions of the disc. This
is a function of the rotational period. For closer inner boundaries
more flux is intercepted by the inner boundary and therefore less
reaches the outer regions, for longer periods the farther inner wall
radii means more flux reaches the outer disc regions (see Section
\ref{disc_period_relation}). These regions of the disc will be cooler
than the inner boundary and, therefore only have a significant flux
contribution at wavelengths longer than 3$\mu$m.

Therefore, the shape of the resulting SED, for a star-disc system of
at fixed stellar mass and age, will be dependent on the dominance of
the hot short wavelength accretion component, the cooler
mid-wavelength component of the inner boundary, and the even cooler,
longer wavelength component from the outer regions of the disc.
Practically, for our model grid, the shape is therefore a function of:
Accretion rate and coverage ($L_{\rm acc}$) and the rotation period
($L_{\rm wall}$ \& $L_{\rm disc}$).

The final, and most important variable affecting the shape of the SED
is inclination. This variable affects the balance of the photospheric
flux escaping ($L_*$ \& $L_{\rm acc}$) and that observed from the disc
($L_{\rm wall}$ \& $L_{\rm disc}$). Essentially, the flux for all wavelengths
will reduce as the inclination increases. However, this reduction,
with inclination, will reduce toward longer wavelengths. This is as
the opacity of the disc will reduce with increasing wavelength.

So, to relate these changes to the scatter we observe in the CMDs of
Figures \ref{spread_opt_VVI} and \ref{spread_ir_JJK}, we have plotted the CMDs
for a star of $M=$0.04$M_{\odot}$, at an age of 1 Myr, for a range of
parameters in optical and IR colours in Figure \ref{spread_cmd}.
Figure \ref{spread_cmd} shows the \textit{V, V-I} (\textit{left
  panel}) and \textit{J, J-K} (\textit{right panel}) CMDs. For
\textit{both panels} the naked BD systems, with negligible accretion
flux (-12 log$(\frac{\dot{M}}{M_{\odot}}yr^{-1})$ and 10\% coverage)
are shown as black crosses. For \textit{both panels} the red coloured
symbols are those which have the inner boundary component maximised,
i.e. with the shorter rotation period of 0.5 days, and the blue
symbols denote systems with the longer period of 5 days, and therefore
minimised inner boundary flux. The circles represent those systems
with the accretion flux component maximised, i.e. an accretion rate of
-8 log$(\frac{\dot{M}}{M_{\odot}}yr^{-1})$ with 1\% areal coverage,
and the triangles show systems with the accretion component minimised,
i.e. an accretion rate of -12 log$(\frac{\dot{M}}{M_{\odot}}yr^{-1})$
with 10\% areal coverage. Finally, the open symbols (circles or
triangles in both red and blue) show systems at an inclination of
face-on, 0$^{\circ}$, and the filled symbols at edge on, 90$^{\circ}$.

**CHECK AND REWRITE**

Figure \ref{spread_cmd} shows, that increasing the accretion flux
component moves the star blueward significantly in the optical
colours, \textit{left panel}, and much less in the IR, \textit{right
  panel} (compare the circles to triangles of the same colour and
type, open or filled.). In addition, maximising the inner boundary
component moves the system redward significantly in both CMDs,
although to a greater extent in the IR, \textit{left panel} (compare
different colour symbols of the same type). Finally, the inclination
as shown in \cite{2004MNRAS.351..607W}, both reduces the magnitude and
shifts the object in colour (compare open and filled symbols of same
colour and type). Comparing the \textit{two panels} shows that the
disc component has a much more significant affect in both CMDs,
optical and IR, than the accretion component. Interestingly, the
effect of a disc is still significant for the optical \textit{V, V-I}
CMD. The accretion component has a much more significant effect in the
optical \textit{V, V-I} CMD than the IR \textit{J, J-K}, and indeed is
probably negligible in the latter. The effect of the two variables is
then essentially additive in CMD space, weighted by their dominance in
either CMD regime. This means a blue-shifted, accretion dominated
object in a \textit{V, V-I} CMD will be less blue-shifted if the inner
disc boundary component is also maximised. Whereas, for the \textit{J,
  J-K} CMD, the inner boundary has a more significant effect and
maximising the accretion component will only lessen the red-shifting
from the maximised disc. It is perhaps surprising that for BDD systems
both accretion and disc presence, due to the inner boundary, can
scatter the star in an optical CMD.

\begin{figure}
  \vspace*{174pt}
  \special{psfile="Fig/spread_cmd.ps"
   hoffset=235 voffset=0 hscale=30 vscale=30 angle=90}
 \caption{CMDs showing the colours and magnitudes of a BD with
   $M_*=$0.04$M_{\odot}$. For \textit{both panels} a BD with negligible
   accretion flux, -12 log$(\frac{\dot{M}}{M_{\odot}}yr^{-1})$ and
   10\% coverage without a disc is plotted as a black
   cross. Triangles, in \textit{both panels}, are BDD systems with
   negligible accretion flux, i.e. -12
   log$(\frac{\dot{M}}{M_{\odot}}yr^{-1})$ and 10\% coverage. Circles,
   in \textit{both panels}, show BDD systems with maximised accretion
   flux, i.e. -8 log$(\frac{\dot{M}}{M_{\odot}}yr^{-1})$ and 1\%
   coverage. The red symbols then have a maximised inner boundary
   component with a period of 0.5 days and blue symbols a minimised
   inner boundary component with a period of 5 days. Finally, for
   \textit{both panels} inclinations of 0 \& 90$^{\circ}$ are shown
   as open and filled symbols respectively. The \textit{left panel}
   shows the $M_V, (V-I)_0$ CMD and \textit{right panel} the $M_J,
   (J-K)_0$ CMD.
   \label{spread_cmd}}
\end{figure}
\clearpage

The TORUS code identifies and tags photons as they pass through the
simulation in one of four ways: Stellar direct, stellar scattered and
thermal direct or thermal scattered, as described in Section
\ref{seds}. This allows us to plot the component flux distributions of
the objects, with discs, shown in Figure \ref{spread_cmd} and examine
their flux components. The convenient components discussed above,
namely: stellar photosphere ($L_*$), accretion component ($L_{\rm acc}$),
disc inner wall component ($L_{\rm wall}$) and outer disc component
($L_{\rm disc}$), will have contributions from differently tagged
photons. The stellar flux will comprise the accretion and intrinsic
stellar photospheric components, with the direct flux reaching the
observer without interaction with the disc. Photons scattering through
the disc will be dominated by photons from the inner boundary for
systems with a small inner disc radius. The inner boundary component
and outer flared disc components will also include the direct and
scattered thermal photons.

Figures \ref{spread_spec_lowacc} and \ref{spread_spec_highacc} show
the total and component fluxes for the BDD systems shown in Figure
\ref{spread_cmd}. For \textit{all panels}, in both Figures, total flux
is shown as the solid black line, with the stellar and thermal fluxes
shown as blue and red lines respectively. Also, in both Figures, the
direct and scattered components of the stellar and thermal flux are
plotted as solid and dashed lines respectively. In addition for both
Figures, the \textit{left panels} shows the minimised inner boundary
component, with a period of 5 days, and the \textit{left panels} the
maximised inner disc boundary component with a period of 0.5 days. The
vertical lines show the approximate response limits in wavelength of
the \textit{VI} and \textit{JK} filters as dashed and solid lines
respectively. The \textit{lower panels} show systems at an edge-on
inclination, of 90$^{\circ}$, and the \textit{upper panels} show
face-on systems with an inclination of 0$^{\circ}$.  Figures
\ref{spread_spec_lowacc} and \ref{spread_spec_highacc} show the
minimised and maximised accretion flux component systems respectively,
i.e. -12 log$(\frac{\dot{M}}{M_{\odot}}yr^{-1})$ with 10\% areal
coverage for Figure \ref{spread_spec_lowacc} and -8
log$(\frac{\dot{M}}{M_{\odot}}yr^{-1})$ with 1\% areal coverage for
Figure \ref{spread_spec_highacc}.

Comparing the \textit{left \& right panels} of Figures
\ref{spread_spec_lowacc} and \ref{spread_spec_highacc} shows that as
the period shortens the proportion of flux reprocessed by the disc,
and inner wall, increases dramatically. This affects the flux and flux
slopes in both the filter response regimes, however, it is much more
significant in the IR regime. This effect causes the reddening seen in
Figure \ref{spread_cmd}, and its increase in magnitude when comparing
the \textit{left} to the \textit{right panel}, or optical to IR CMD.
Comparing the \textit{upper \& lower panels} of Figures
\ref{spread_spec_lowacc} and \ref{spread_spec_highacc} shows the flux
becoming dominated by the scattered components, which causes the
reduction in magnitude and slight reddening of the systems as the
inclination increases. Finally, comparing Figure
\ref{spread_spec_lowacc} to Figure \ref{spread_spec_highacc} shows
that as the accretion is maximised there is a significant increase in
the flux and flux gradient, across the optical filter bandpasses, for
the longer period. Whereas, there is a less significant change in the
optical CMD. This is caused by a higher proportion of the stellar
intrinsic and accretion flux being intercepted by the inner disc
boundary.

\begin{figure*}
  \vspace*{348pt}
  \special{psfile="Fig/spread_spec_lowacc.ps"
   hoffset=470 voffset=0 hscale=60 vscale=60 angle=90}
 \caption{This Figure shows SEDs for the minimised accretion flux
   component systems (-12 log$(\frac{\dot{M}}{M_{\odot}}yr^{-1})$ and
   10\% coverage), from Figure \ref{spread_cmd}, which are shown as
   triangles. The total flux is shown as a black line, with stellar
   flux in blue and thermal flux in red, separated into direct (solid
   lines) and scattered (dashed lines) components. The inner boundary
   component is maximised on the \textit{left} and minimised on the
   \textit{right panels}, with periods of 0.5 and 5 days
   respectively. The vertical lines are the approximate filter
   response limits for the \textit{VI}, dashed, and \textit{JK},
   solid, bands. Two inclinations of 0 \& 90$^{\circ}$ are shown as
   the \textit{upper} and \textit{lower panels}, respectively.
   \ref{spread_cmd}.
   \label{spread_spec_lowacc}}
\end{figure*}

\begin{figure*}
  \vspace*{348pt}
  \special{psfile="Fig/spread_spec_lowacc_log.ps"
   hoffset=470 voffset=0 hscale=60 vscale=60 angle=90}
 \caption{As Figure \ref{spread_spec_lowacc} but in log space.
   \label{spread_spec_lowacc_log}}
\end{figure*}

\begin{figure*}
  \vspace*{348pt}
  \special{psfile="Fig/spread_spec_highacc.ps"
   hoffset=470 voffset=0 hscale=60 vscale=60 angle=90}
 \caption{As for Figure \ref{spread_spec_lowacc} but for a maximised
   accretion flux component (-8
   log$(\frac{\dot{M}}{M_{\odot}}yr^{-1})$ and
   1\% coverage), shown as circles in Figure \ref{spread_cmd}.
   \label{spread_spec_highacc}}
\end{figure*}

\begin{figure*}
  \vspace*{348pt}
  \special{psfile="Fig/spread_spec_highacc_log.ps"
   hoffset=470 voffset=0 hscale=60 vscale=60 angle=90}
 \caption{As for Figure \ref{spread_spec_highacc} but in log space.
   \label{spread_spec_highacc_log}}
\end{figure*}

\subsubsection{Accretion Dominance: $L_{\rm acc}$}
\label{accn_dominance}
 
Figure \ref{acc_dom} shows the affect of the increasing accretion
blackbody flux (for increasing accretion rates) for a BD star,
$M=$0.04$M_{\odot}$, at 1 Myr. The panels in this figure also show the
flux from a naked system, with no treatment of the disc. This enables
us to view the effect of increasing accretion rate on the photospheric
flux in isolation. The accretion rates included in all panels are -6,
-7 and -8 log$(\frac{\dot{M}}{M_{\odot}}yr^{-1})$ (blue, red and black
lines respectively). The \textit{lower panels} show the systems with a
rotation period of 0.5 days and \textit{upper panels} for those with a
rotation period of 5 days. Given our assumption that accretion occurs
from the co-rotation radius, decreasing the rotational period moves
this accretion radius closer to the star. As the accretion radius
moves closer to the star the potential energy released by the accreted
material is reduced. This effect can be seen when comparing the
\textit{upper} and \textit{lower panels}, although the effect is
marginal for all but the highest accretion rates. The \textit{left
  panels} show accretion streams with an areal coverage of 1\% and the
\textit{left panels} 10\%. As the areal coverage reduces the effective
temperature of the accretion hot spot increases, resulting in an
increase in accretion flux, and resulting shift to bluer wavelengths
of the peak flux. This can be seen clearly by comparing the
\textit{left} and \textit{right panels} of Figure
\ref{acc_dom}. Perhaps the most important, albeit qualitative, result
shown is Figure \ref{acc_dom} is an insight into the accretion rate at
which the accretion blackbody flux dominates over the photospheric
flux. Figure \ref{acc_dom} shows that as the accretion rate raises
above -8 log$(\frac{\dot{M}}{M_{\odot}}yr^{-1})$ for systems with 1\%
areal coverage, the accretion flux dominates the emergent SED at both
periods. For the larger coverage the accretion flux begins to dominate
between -7 to -6 log$(\frac{\dot{M}}{M_{\odot}}yr^{-1})$. Therefore
for reasonable coverages (1-10\%) and rotational periods (0.5-5 days)
the photospheric flux is effectively swamped by accretion flux for
accretion rates log$(\frac{\dot{M}}{M_{\odot}}yr^{-1})>-7$ , and for
the most conservative case (1\% coverage with 5 day period) at
accretion rates much lower than this
(log$(\frac{\dot{M}}{M_{\odot}}yr^{-1})>-$8). This suggests that
accreting BD systems with accretion rates as low as -8
log$(\frac{\dot{M}}{M_{\odot}}yr^{-1})$ could be difficult to identify
from SEDs as BD stars, and may well be classified as higher mass
stars.

\begin{figure*}
  \vspace*{348pt}
  \special{psfile="Fig/acc_dom.ps"
   hoffset=470 voffset=0 hscale=60 vscale=60 angle=90}
 \caption{This figure shows the photospheric flux against $\lambda$
   ($\mu$m) of a Brown Dwarf with $M=$0.04$M_{\odot}$, at 1 Myr. No
   disc is included, but blackbody fluxes from an accretion stream at
   the rates of -6, -7 and -8 log$(\frac{\dot{M}}{M_{\odot}}yr^{-1})$
   are shown as blue, red and black lines respectively. The
   \textit{lower panels} show accretion for a star rotating at 0.5
   days, with the \textit{upper panels} showing that of 5 days. The
   \textit{left panels} systems with an areal coverage of 1\% and the
   \textit{right panels} has 10\%. CONSTRUCTED USING GRAPHPS**
   \label{acc_dom}}
\end{figure*}

\begin{figure*}
  \vspace*{348pt}
  \special{psfile="Fig/acc_dom_log.ps"
   hoffset=470 voffset=0 hscale=60 vscale=60 angle=90}
 \caption{As Figure \ref{acc_dom} but in log space.
   \label{acc_dom_log}}
\end{figure*}

Therefore, one can see that for higher accretion spot temperatures,
flux shortward of the photospheric flux is markedly increased. This
results, in general, in a brighter and bluer magnitude and colour for
the combined system.

\subsubsection{Disc effects: Flaring, $L_{\rm disc}$}
\label{flaring}

**SOME IMAGES TO PROVE POINT**
**ADD MIPS DISCUSSION HERE AND SCALE HEIGHT STUFF**

Scatter in optical CMDs is dominated by disc inclination (see Figures
\ref{spread_opt_VVI}, \ref{spread_RI} and \ref{spread_VB}), and, even at
longer wavelengths, scatter in IR CMDs is still correlated with
inclination (see Figure \ref{spread_ir_JJK}). \cite{2004MNRAS.351..607W}
have already shown that increased disc inclination lead to significant
changes in the spectroscopic, and photometric properties of BDD
systems. The scatter in optical and IR CMDs (in Section
\ref{derive_primary}), for our model grid, appears strongly correlated
with system inclination. This is a similar effect as found in
\cite{2004MNRAS.351..607W}, where increases in the inclination angle
quickly lead a significant proportion of the stellar flux being
intercepted, and reprocessed, by the highly flared disc.

Figure \ref{disc_eg} shows examples of our derived spectra for a BD of
$M=$0.04$M_{\odot}$ (the middle of our mass range), at 1 Myr, with a
rotation period of 0.5 days, and an accretion stream covering 1\% of
the stellar surface accreting at a negligible rate of -12
log$(\frac{\dot{M}}{M_{\odot}}yr^{-1})$. The \textit{upper left panel}
shows the SED for the naked BD system with the remaining panels
showing the SEDs from the three modeled inclinations, after a TORUS
simulation of a disc with $M_{disc}=$0.01$M_*$. The \textit{lower left,
  upper right and lower right panels} show inclinations of 0, 60
and 90$^{\circ}$ respectively. This figure plots, for the disc
systems, the TORUS tagged component fluxes from the star (solid lines)
and the disc or thermal flux (dashed lines), for both direct flux
(black) and scattered flux (red). As the inclination of the system
increases, the stellar direct flux can be seen to decrease, and the
thermal and scattered components increase. This is simply caused by
occultation of the star by the disc. The condition of vertical
hydrostatic equilibrium applied during the TORUS simulation results in
the disc flaring (see Section \ref{derive_primary}). This flaring is
more pronounced for BD when compared to higher mass stars meaning a
higher fraction of the stellar flux is intercepted by the disc at
earlier inclinations. \cite{2004MNRAS.351..607W} have previously
modeled this effect for a range of disc masses and found that this
alone can shift the SED, and therefore colours and magnitudes, of
inclined BDD systems to those indicative of higher mass CTTS.

\begin{figure*}
  \vspace*{348pt}
  \special{psfile="Fig/disc_eg.ps"
   hoffset=470 voffset=0 hscale=60 vscale=60 angle=90}
 \caption{Figure of the SEDs of a BDD of $M=$0.04$M_{\odot}$, at 1 Myr,
   with a disc mass of $M_{\rm disc}=$0.01$M_{\rm star}$. The BD is rotating
   with a period of 0.5 days and is accreting through a stream, with
   an areal coverage of 1\%, at a rate of -12
   log$(\frac{\dot{M}}{M_{\odot}}yr^{-1})$. The \textit{upper left
     panel} shows the photospheric spectrum of the naked star (naked stars have
   spherically symmetric flux distributions, therefore the SED is not
   a function of inclination), the total flux is shown as a bold
   line. The \textit{lower left, upper right and lower right panels}
   show the SEDs of the BDD systems at inclinations of 0,
   60 and 90$^{\circ}$. For the BDD SEDs, the solid lines
   are stellar flux and dashed are the thermal disc flux, with the
   black denoting direct flux and red for scattered flux, i.e. black
   solid line is direct stellar flux.
   \label{disc_eg}}
\end{figure*}

\begin{figure*}
  \vspace*{348pt}
  \special{psfile="Fig/disc_eg_log.ps"
   hoffset=470 voffset=0 hscale=60 vscale=60 angle=90}
 \caption{As Figure \ref{disc_eg} but in log space.
   \label{disc_eg_log}}
\end{figure*}

The resulting density structure of the highly flared disc, for the
system used to construct Figure \ref{disc_eg}, is shown in Figure
\ref{flare_eg}. Figure \ref{flare_eg} shows that for a typical BDD
system the resulting disc is highly flared and therefore, the opening
angle for stellar radiation is much smaller. As discussed the flaring
for BDD systems was found to be larger than CTTS systems by
\cite{2004MNRAS.351..607W}. \cite{2004MNRAS.351..607W}, state that the
degree of disc flaring depends on the disc temperature structure and
the mass of the central star, with the disc scaleheight $h\propto
\frac{T_{\rm disc}}{M_*}^{1/2}$ \citep{1973A&A....24..337S}. Recently,
\cite{2009arXiv0901.4445E} suggest the inverse relation of flaring
with stellar mass, i.e. $h\propto M_*$. This suggestion was based on
evidence from \cite{2006ApJ...644..364A}, where SEDs for 17 systems in
the mass range 6$M_{\rm Jup}<M_*<$350$M_{\rm Jup}$ were fit with flared or flat
disc models. In general, \cite{2006ApJ...644..364A} find that lower
mass objects achieve better fits with the flat disc models and higher
mass objects with the flared discs.

The results of \cite{2006ApJ...644..364A} show that above a mass of
50$M_{\rm Jup}$ all objects (6/17) are better fit with flared
discs. Whilst at masses below 50$M_{\rm Jup}$ only one object is better
fit by the flared disc model, with the remaining objects (10/17)
better fit with flat models. Whether, this result is statistically
significant enough to assert a $h\propto M_*$ is doubtful as the
fitting process contains, presumably two fixed scaleheight
distributions. Therefore, for our study we continue to assume that our
flared BDD systems will have larger characteristic scaleheights than
typical CTTS systems.

\begin{figure}
  \vspace*{174pt}
  \special{psfile="Fig/28_rho.ps"
   hoffset=30 voffset=0 hscale=30 vscale=25 angle=0}
 \caption{A Figure of the density for the most flared
   disc. $M_*=$0.01$M_{\odot}$, $\tau=$5 days, $\dot{M}=$-6
   log$(\frac{\dot{M}}{M_{\odot}}yr^{-1})$, areal coverage of 1\% at 1
   Myr **RUN NUMBER028**.
   \label{flare_eg}}
   \end{figure}

\subsubsection{Disc effects: Inner wall, $L_{\rm wall}$}
\label{inner_wall}

Therefore, as the scattering in optical CMDs is dominated by system
inclination, due to disc flaring, what dominates scattering in the IR?
Clearly, system inclination still has a significant correlation with
the scatter CMD position. However, as shown in Section
\ref{derive_primary} IR CMD scatter appears strongly correlated with
rotational period. A similar correlation between scatter and rotation
period is evident for IR and IRAC (and perhaps MIPS) CoCoDs (although
the trend is reversed), as shown in Section \ref{derive_frac}. The
cause of this scatter is the relative flux contribution of the inner
disc boundary, as discussed in Section \ref{scatter_causes}. It is
also clear in Section \ref{derive_primary} that the longer wavelength
photometric colours are better disc discriminators. Furthermore,
interestingly, the $\alpha$ value, in our case the slope of the SED
between 3.6 and 8.0$\mu$m \citep[chosen to
match][]{2009arXiv0901.4120D}, is a very effective disc disriminator.

The position of a given BDD system, in an IR CoCoD or CMD, is
dominated by rotation period, and it is clear that this is due to the
radial position of the inner wall (See discussion in Section
\ref{scatter_causes}. As the rotation period decreases the inner disc
co-rotation radius decreases, meaning a closer, hotter, inner disc
wall. However, we have also included a treatment of dust sublimation
which, during the radiative transfer modeling, will change the radial
position for the higher temperature inner walls \citep[where
$T_{\rm wall}$ exceeds $\approx$1400K][]{1994ApJ...421..615P}.

\subsubsection{Disc structure and the inner edge}
\label{inner_edge_position}
**SHOULD BE PRINTED FROM TORUS**pooooo
\textbf{Initial inner wall radius}

Given the strong dependence of IR flux on the inner wall temperature,
and therefore radial position, it is important to examine the final
structure of the inner wall after the radiative transfer simulation.

Table \ref{inner_edge} list the initial inner edge radius (from the
central star), set at the co-rotation radius, in stellar radii for each
mass increment. It is clear that the inner edge position is a strong
function of rotation period and a much weaker function of stellar
mass. This is obvious from our calculation of the co-rotation radius
for which we use equation \ref{inner_eq}. Equation \ref{inner_eq}
clearly shows the weak dependence on stellar mass ($R_{\rm inner}\propto
M_*$) and strong dependence on rotation period ($R_{\rm inner}\propto \tau
^2$). Table \ref{inner_edge_au} shows the same data as Table
\ref{inner_edge} but converted using the stellar radius into AU, which
reveals a magnetic truncation radius of between 0.0026$-$0.026 AU for
1 Myr BDD systems.

\begin{table}
\begin{tabular}{|l|l|l|l|l|}
Age (Myr)&\multicolumn{2}{|c|}{1}&\multicolumn{2}{|c|}{10}\\
\hline
Period (days)&0.5&5&0.5&5\\
\hline
Mass ($M_{\odot}$)&\multicolumn{4}{|c|}{Inner boundary radius ($R_*$)}\\
0.01&2.18&10.16&3.52&16.36\\
0.02&2.42&11.24&3.33&15.47\\
0.04&1.90&8.86&3.26&15.15\\
0.06&1.57&7.32&3.13&14.53\\
0.08&1.37&6.39&2.96&13.74\\
\end{tabular}
\caption{Table showing the initial inner edge radius (in $R_*$) for all models.
  \label{inner_edge}}
\end{table}

\begin{table}
\begin{tabular}{|l|l|l|l|l|}
Age (Myr)&\multicolumn{2}{|c|}{1}&\multicolumn{2}{|c|}{10}\\
\hline
Period (days)&0.5&5&0.5&5\\
\hline
Mass ($M_{\odot}$)&\multicolumn{4}{|c|}{Inner boundary radius (AU)}\\
0.01&0.00264&0.0123&0.00265&0.0123\\
0.02&0.00333&0.0155&0.00375&0.0174\\
0.04&0.00403&0.0188&0.00411&0.0191\\
0.06&0.00479&0.0195&0.00395&0.0183\\
0.08&0.00527&0.0246&0.00501&0.0233\\
\end{tabular}
\caption{Table showing the initial inner edge radius (in $R_*$) for all models.
 \label{inner_edge_au}}
\end{table}

\textbf{Dust sublimation radius}

During the simulations of radiative transfer, implementation of dust
sublimation, may result in changes in inner radius. For systems where
the temperature of dust at the inner disc boundary exceeds
$\approx$1400K \citep{1994ApJ...421..615P}, the inner disc will be
destroyed and the inner radius increased. Table \ref{inner_edge_dust}
shows the resulting radii for our model grid in stellar radii with the
corresponding values in AU presented in Table
\ref{inner_edge_dust_au}.**THIS TABLE MUST BE CHANGED TO INCLUDE
DIFFERENT FLUXES**

\begin{table}
\begin{tabular}{|l|l|l|l|l|}
Age (Myr)&\multicolumn{2}{|c|}{1}&\multicolumn{2}{|c|}{10}\\
\hline
Period (days)&0.5&5&0.5&5\\
\hline
Mass ($M_{\odot}$)&\multicolumn{4}{|c|}{Inner boundary radius ($R_*$)}\\
0.01&&&&\\
0.02&&&&\\
0.04&&&&\\
0.06&&&&\\
0.08&&&&\\
\end{tabular}
\caption{Table showing the final inner edge radius (in $R_*$) for all models.
 \label{inner_edge_dust}}
\end{table}

\begin{table}
\begin{tabular}{|l|l|l|l|l|}
Age (Myr)&\multicolumn{2}{|c|}{1}&\multicolumn{2}{|c|}{10}\\
\hline
Period (days)&0.5&5&0.5&5\\
\hline
Mass ($M_{\odot}$)&\multicolumn{4}{|c|}{Inner boundary radius (AU)}\\
0.01&&&&\\
0.02&&&&\\
0.04&&&&\\
0.06&&&&\\
0.08&&&&\\
\end{tabular}
\caption{Table showing the final inner edge radius (in $R_*$) for all models.
 \label{inner_edge_dust_au}}
\end{table}

\textbf{Compare $R_{\rm co}$ to $R_{\rm dust}$}

**NEEDS TO BE DONE AFTER NEW DATA**

Comparing Table \ref{inner_edge_dust_au} to Table \ref{inner_edge_au}
shows that the inner disc radius is changed little from the initial
co-rotation radius (QUOTE MAX CHANGE**) due to dust sublimation
effects. The resulting inner radii are still, broadly, proportional to
the rotation period. Therefore, for subsequent analysis, we assume an
inner radius to rotation period relationship.

Our final inner radii range from ** to ** AU. **COMPARE TO
OTHER RESULTS?**

\textbf{Dust sublimation regions}

**AGAIN NEEDS TO BE DONE AFTER NEW DATA**

In some cases however the inner disc radius has been changed by the
implementation of dust sublimation. For these systems the resulting
shape of the inner disc boundary has also changed. Evidence that real
inner disc boundaries (for CTTS) are not vertical walls comes from a
lack of dependence of the derived IR excess on system inclination
\citep{2007ApJ...661..374T}. As discussed in Section \ref{dust_edge}
the disc inner hole can be created by several mechanisms, however, for
this study we have only included a treatment of magnetic truncation
(initially) and subsequent dust sublimation. The effects of dust
sublimation have been shown to produce curved inner walls due to the
temperature and density dependence of the dust sublimation temperature
\citep{2005A&A...438..899I}. The curved boundary in this case is
concave, as density increases towards the midplane of the disc, the
dust sublimation temperature increases, meaning the destruction radius
moves closer to the star. A curved inner boundary is also found to
result from dust sublimation in \cite{2007ApJ...661..374T}. In
\cite{2007ApJ...661..374T} two populations of dust grains where
included with different scaleheight distributions. This is essentially
a simple model of grain growth and subsequent midplane
settling. \cite{2007ApJ...661..374T}, found a convex inner disc
boundary, this is due to the larger grains cooling more
efficiently. These larger grains dominate the dust towards smaller
vertical distances from the midplane, and therefore the increased
cooling means that the dust sublimation temperature is reached at
closer radii.

For our dust distribution and dust sublimation implementation we
produce a curved inner boundary, as can be seen in
Figure \ref{curve_inner}. This is due to the density dependent dust
sublimation temperature (as discussed in Section \ref{dust_edge}).

\begin{figure}
  \vspace*{174pt}
  \special{psfile="Fig/238_tau.ps"
   hoffset=30 voffset=0 hscale=30 vscale=25 angle=0}
 \caption{A Figure of the opacity (scaled $1<\tau <0$) for the most
   extreme case of a dust sublimated inner
   boundary. $M_*=$0.08$M_{\odot}$, $\tau=$0.5 days, $\dot{M}=$-6
   log$(\frac{\dot{M}}{M_{\odot}}yr^{-1})$, areal coverage of 1\% at 1
   Myr **RUN NUMBER238**.
   \label{curve_inner}}
\end{figure}

**CONCLUDE**

\subsubsection{Disc effects: Inner wall to Outer edge balance,
  $L_{\rm wall}$ to $L_{\rm disc}$}
\label{disc_period_relation}

\textbf{Why do inner disc radius changes cause photometric scatter?}

**ADD SOME IMAGES**
**PERHAPS ADD A FIGURE HER J-K, AND IRAC COLOUR VERSUS FINAL RADIUS**

Physically, as the inner disc radius changes the flux intercepted by
the inner wall is changed. At first glance it may appear more
instructive to plot the \textit{bottom left panel} as a function of
inner disc radius, to strengthen the correlation and remove the
scatter within the Figure. However, intrinsic scatter will also be
included due to inclination and accretion rate, the latter perhaps
being less intuitive. Even though accretion flux is generally blueward
of these passbands, interception and subsequent reprocessing of this
additional flux for higher accretion rates increase scatter in a
CoCoDs of this sort. Therefore, plotting the simulated photometry as a
function of inner disc radius does not reduce the
scatter. Furthermore, separating the populations using the rotational
period is perhaps more useful for observational studies as this is
often the most ubiquitous and precise data available. It is important
at this stage to re-iterate that magnetic truncation and dust
sublimation as we have assumed occur, is not the only method of
producing an inner hole, for instance planet formation or
photoevaporation can create significant inner holes, which are not
necessarily correlated, in size, with rotation period.

\textbf{SED change caused by inner radius}

Figures \ref{rot_effect}, \ref{rot_effect_log}, \ref{rot_comp} and
\ref{rot_comp_-7} show the effects of the changes in inner boundary
radius due to rotational period changes on a BD, $M_*=$0.04$M_{\odot}$,
at 1 Myr. For all these figures the \textit{left panel} shows the SEDs
for a period of 0.5 days and the \textit{right panel} shows the SED
for a period of 5 days. We have chosen to plot the SEDs as a function
of rotational period for two reason. Firstly, rotational period will
be the most easily, and precisely, observed variable. Secondly, the
inner disc radius varies most strongly as a function of the rotational
period (see Equation \ref{inner_eq}).The final inner radii for the
systems shown in the \textit{left} and \textit{right panels} are **
and ** AU, respectively (see Table \ref{inner_edge_dust_au}).

Figures \ref{rot_effect} and \ref{rot_effect_log} show the total flux
(in normal and log space respectively) for both canonical accretion
rates (-8 \& -7 log$(\frac{\dot{M}}{M_{\odot}}yr^{-1})$, bold and
dashed lines respectively). The SEDs are for systems with the smallest
areal coverage (1\%) and therefore highest accretion flux, for each of
the three inclinations (0, 60 and 90$^{\circ}$, black, red and
blue respectively).

Figures \ref{rot_comp} and \ref{rot_comp_-7} show the component flux
for two inclinations, $0$ and $60^{\circ}$, \textit{upper} and
\textit{lower panels} respectively. The component flux is separated
into total (black lines), stellar (blue lines) and thermal or disc
(red lines) flux. The stellar and thermal flux is then separated into
direct (bold lines) or scattered (dashed lines) flux. The vertical
bold lines on these figures show the approximate filter ranges for the
IRAC channels ([3.6], [4.5], [5.8] \& [8.0]), with the vertical dashed
lines showing the limits of the JHK filters of the CIT system. Figure
\ref{rot_comp} shows the SEDs for a system with an accretion rate of
-8 log$(\frac{\dot{M}}{M_{\odot}}yr^{-1})$ and Figure
\ref{rot_comp_-7} an accretion rate of -7
log$(\frac{\dot{M}}{M_{\odot}}yr^{-1})$.

Figures \ref{rot_effect} and \ref{rot_effect_log} show that for
systems with a typical accretion rate and shorter rotational periods,
as the inclination increases the flux and slope of the SED decreases
dramatically, for wavelengths shorter than $\approx$2$\mu$m. In this
case the inner boundary is close to the central star. Conversely, the
flux, and slope of the SED, at wavelengths longer than
$\approx$2$\mu$m decreases less swiftly with increasing inclination.
Effectively shorter period systems show a large change in flux and
flux ratios in the $<$2$\mu$m range and a more constant flux or flux
ratio in the $>$2$\mu$m range. This trend is clearly reversed for
longer period objects. The objects, where the disc inner edge is
farther from the star, show a large change in flux and flux ratio, as
a function of inclination, for wavelengths $>$2$\mu$m and a smaller
change for wavelengths $<$2$\mu$m. In addition, the total SED is
dominated by the photosphere at shorter wavelengths.

\begin{figure}
  \vspace*{174pt}
  \special{psfile="Fig/rot_effect.ps"
   hoffset=235 voffset=0 hscale=30 vscale=30 angle=90}
 \caption{Figure showing flux against wavelength for a
   $M_*=$0.04$M_{\odot}$ BD at 1 Myr, for two accretion rates -7
   (dashed line) and -8 log$(\frac{\dot{M}}{M_{\odot}}yr^{-1})$ (bold
   line) for the three modeled inclinations, 0 (black), 60 (red) and
   90$^{\circ}$ (blue). The \textit{left panel} shows the SEDs for the
   shorter rotation period system, with a closer disc inner
   boundary. The \textit{right panel} shows the longer period system
   with farther disc inner boundary.**USE GRAPHPS**
   \label{rot_effect}}
\end{figure}

\begin{figure}
  \vspace*{174pt}
  \special{psfile="Fig/rot_effect_log.ps"
   hoffset=235 voffset=0 hscale=30 vscale=30 angle=90}
 \caption{A Figure \ref{rot_effect} but in log space.
   \label{rot_effect_log}}
\end{figure}

The wavelength dependent variations in flux, and SED slope, with
inclination are particularly important when considering photometric or
broadband magnitudes and colours. Figures \ref{rot_comp} and
\ref{rot_comp_-7} show that the variations evident in Figures
\ref{rot_effect} and \ref{rot_effect_log} have serious implications
over the filter responses of the CIT \textit{JHK} and IRAC
systems. Essentially, Figures \ref{rot_comp} and \ref{rot_comp_-7}
show that for increasing inclinations discs have a much more apparent
effect on the SED over the IRAC range for longer period systems, and
in the CIT \textit{JHK} range for shorter period systems. Thereby
explaining the opposite correlations in disc scatter and rotation
period, found for IR and IRAC colours in Section \ref{derive_frac}.

\begin{figure*}
  \vspace*{348pt}
  \special{psfile="Fig/rot_comp.ps"
   hoffset=470 voffset=0 hscale=60 vscale=60 angle=90}
 \caption{This Figure shows flux against wavelength for a
   $M_*=$0.04$M_{\odot}$ BD at 1 Myr, with a disc ($M_{\rm disc}=$0.01$M_*$)
   and associated accretion rate of -8
   log$(\frac{\dot{M}}{M_{\odot}}yr^{-1})$ over a hot spot coverage of
   1\%. The \textit{left panels} show systems with a stellar rotation
   period of 0.5 days and the \textit{right panel} those with a
   stellar rotation period of 5 days. The \textit{upper panels} show
   systems observed at an inclination of $0^{\circ}$ and the
   \textit{lower panel} those observed at an inclination of
   $60^{\circ}$. For \textit{all panels} the total SED is shown
   (solid, black line), alongside the components flux from the star
   (blue) and thermal or disc (red) flux. These components are also
   separated into direct flux (solid lines) or scattered flux (dashed
   lines) from the star or disc. The extents of the filter responses
   for the IRAC (including channels [3.6], [4.5], [5.8] \&
   [8.0]) (solid vertical line) and the CIT (\textit{JHK})
   (dashed vertical lines) bands are shown. 
   \label{rot_comp}}
\end{figure*}

\begin{figure*}
  \vspace*{348pt}
  \special{psfile="Fig/rot_comp_log.ps"
   hoffset=470 voffset=0 hscale=60 vscale=60 angle=90}
 \caption{As Figure \ref{rot_comp} but in log space.
   \label{rot_comp_log}}
\end{figure*}

\begin{figure*}
  \vspace*{348pt}
  \special{psfile="Fig/rot_comp_-7.ps"
   hoffset=470 voffset=0 hscale=60 vscale=60 angle=90}
 \caption{As above but for higher canonical accretion rate -7
   log$(\frac{\dot{M}}{M_{\odot}}yr^{-1})$.
   \label{rot_comp_-7}}
\end{figure*}

\begin{figure*}
  \vspace*{348pt}
  \special{psfile="Fig/rot_comp_-7_log.ps"
   hoffset=470 voffset=0 hscale=60 vscale=60 angle=90}
 \caption{As Figure \ref{rot_comp_-7} but in log space.
   \label{rot_comp_-7_log}}
\end{figure*}

**NEED TO GET VALUES ETC FOR THIS SECTION AFTER NEW DATA**

As the temperature of the surface of the disc generally decrease
towards larger radii the longer the wavelength of the flux the further
from the central object its origin will be.  Effectively, more flux is
being intercepted by the disc inner wall for the shorter period
systems and consequently less flux is impinging on the outer flared
surface of this disc, creating larger changes as a function of
inclination for shorter wavelengths. In the case of the longer periods
more flux is escaping to the outer flared edge of the disc and greater
change is seen as a function of inclination angle at longer
wavelengths. Figures \ref{far_rho}, \ref{far_rho_bw}, \ref{close_rho},
and \ref{close_rho_bw}, show the density structure for the inner
regions of the disc ($R<**AU$) as a function of radius and scaleheight
(10$^{10}$ cms). The inner edge location can be seen to move further
out when moving from short period systems to long period systems, in
Figures \ref{far_rho}/\ref{far_rho_bw} and
\ref{close_rho}/\ref{close_rho_bw}, from ** to **AU in fact. Dust
sublimation effects have shifted the shorter period inner wall to a
slightly larger radius (see Section \ref{inner_edge_position}), but
this is ,however, still closer than longer period counterpart. The
final temperature distribution in the disc inner regions ($R<**AU$)
can be seen in Figures \ref{far_temp}, \ref{far_temp_bw},
\ref{close_temp} and \ref{close_temp_bw}. As can be seen by comparing
Figure \ref{far_temp}/\ref{far_temp_bw} to Figure
\ref{close_temp}/\ref{close_temp_bw}, the inner wall reaches
significantly higher temperatures for the shorter period systems. For
the closer systems, the curved inner edge is due to some of the disc
regions exceeding the dust sublimation temperature, the remaining wall
reaches temperatures of **K. Whereas, for the longer period object the
inner wall reaches a temperature of **K. This produces the correlation
of IR excess caused by the inner wall with rotation period seen for IR
CoCoDs in Section \ref{derive_frac}. In addition, comparing the
temperature structure at a farther disc radius, indicative of fluxes
within the IRAC passbands, one finds that the disc temperature is
larger for the longer period systems, $**$ compared to $**$K at $**$
AU. This causes the resulting correlation, where larger IRAC excesses
are found for longer period BDD systems.

\begin{figure}
  \vspace*{174pt}
  \special{psfile="Fig/far_rho0000.ps"
   hoffset=30 voffset=0 hscale=30 vscale=25 angle=0}
 \caption{CHANGE THESE AXES IN 10$^{10}$ cms*****
   \label{far_rho}}
\end{figure}

\begin{figure}
  \vspace*{174pt}
  \special{psfile="Fig/far_rho_bw0000.ps"
   hoffset=30 voffset=0 hscale=30 vscale=25 angle=0}
 \caption{
   \label{far_rho_bw}}
\end{figure}

\begin{figure}
  \vspace*{174pt}
  \special{psfile="Fig/close_rho0000.ps"
   hoffset=30 voffset=0 hscale=30 vscale=25 angle=0}
 \caption{
   \label{close_rho}}
\end{figure}

\begin{figure}
  \vspace*{174pt}
  \special{psfile="Fig/close_rho_bw0000.ps"
   hoffset=30 voffset=0 hscale=30 vscale=25 angle=0}
 \caption{
   \label{close_rho_bw}}
\end{figure}

\begin{figure}
  \vspace*{174pt}
  \special{psfile="Fig/far_temp0000.ps"
   hoffset=30 voffset=0 hscale=30 vscale=25 angle=0}
 \caption{
   \label{far_temp}}
\end{figure}

\begin{figure}
  \vspace*{174pt}
  \special{psfile="Fig/far_temp_bw0000.ps"
   hoffset=30 voffset=0 hscale=30 vscale=25 angle=0}
 \caption{
   \label{far_temp_bw}}
\end{figure}

\begin{figure}
  \vspace*{174pt}
  \special{psfile="Fig/close_temp0000.ps"
   hoffset=30 voffset=0 hscale=30 vscale=25 angle=0}
 \caption{
   \label{close_temp}}
\end{figure}

\begin{figure}
  \vspace*{174pt}
  \special{psfile="Fig/close_temp_bw0000.ps"
   hoffset=30 voffset=0 hscale=30 vscale=25 angle=0}
 \caption{
   \label{close_temp_bw}}
\end{figure}
 
\textbf{Short conclusion, alpha}

As can be seen the effects of a circumstellar disc on the observed SED
are most constant for longer wavelengths. Wavelengths generally longer
than 3$\mu$m are little effected by inclination, accretion rate,
stellar mass etc. This explains, result that BDD systems are well
selected by IRAC photometry and the $\alpha$ value. Essentially, as
the wavelength limits used to construct Figure \ref{alpha_zoom} are
3.6 and 8.0$\mu$m, it follows that the separation between disc-less
and BDD systems should be at least similar in clarity to the IRAC
CoCoDs.

\section{Conclusions}
\label{conclusions}

**THIS CLEARLY NEEDS REWRITING AFTER THE NEW DATA ARE ANALYSED**

We have constructed a model grid of SEDs, and subsequently photometric
magnitudes and colours, for actively accreting BDs with or without an
associated accretion disc. We have modeled the photospheric flux from
these BDs by adopting (and interpolating) the interior `DUSTY00'
models of \cite{2000ApJ...542..464C} combined with the `AMES-Dusty',
atmospheric models of \cite{2000ApJ...542..464C}. We have then assumed
that accretion occurs from an inner edge of a magnetically truncated
accretion disc (truncated at the co-rotation radius). The accretion
flux is calculated using a simple blackbody emission, given the
derivation of a characteristic spot effective temperature. SEDs were
then produced for both naked BDs and BDD systems. For the BDD systems
we have modeled the disc using the TORUS radiative transfer code using
the Lucy radiative transfer algorithm and incorporating dust
sublimation and including including a treatment of vertical
hydrostatic equilibrium (see Section \ref{model} for a discussion of
the code). To produce a `grid' of simulated systems we have varied
several input parameters namely: stellar mass, stellar age, stellar
rotation rate, accretion rate, the areal coverage of the accretion
stream and the system inclination (the disc mass was fixed). The
ranges of these variables were selected to represent and bound typical
pre-MS BD systems, justification is provided using evidence from
observational studies in Section \ref{par_space} and a final list of
the values of these variables can be found in Table
\ref{par_space_table}.

Accepting our assumptions, parameter ranges and radiative transfer
code our resulting simulated dataset has allowed us to qualitatively
explore the effects of \emph{active} (current not past accretion)
accretion, and disc presence, on both the SEDs, and photometric
colours and magnitudes of these systems. This analysis (described in
Section \ref{results}) has important implications in two main areas.

\textbf{Effect of accretion and disc on derived parameters:}

Firstly, as disc presence and active accretion change the
characteristics of BD stars, both in terms of its spectrum and
photometric magnitudes, they also, subsequently, imply changes in any
derived parameters from these primary quantities.

\textit{Age.}

As discussed in Section \ref{results}, the SEDs of BD systems with
typical accretion rates and associated discs, are changed significantly
from the assumed underlying photospheric model flux, and therefore
become difficult to classify. We have shown that derivation of an
\emph{isochronal} (or photometric) age from our simulated photometry
of a coeval BD sample, with typical accretion rates and associated
circumstellar discs, would be inaccurate and exceedingly difficult. As
discussed in Section \ref{results} this does not include any effects
due to past accretion on the evolution of the central star, which acts
to accelerate the gravitational contraction and make the star appear
older \citep{1999MNRAS.310..360T,1999A&A...342..480S}, further
scattering the apparent age of a coeval population.

\textit{Mass, therefore IMF.}

Concordantly, \emph{isochronal} derivations of mass and therefore
IMFs, for our simulated photometry, of a coeval population of
accreting BDs with associated discs, would be inaccurate and
problematic. Again caused by the changes in the SEDs as a result of
the accretion flux and increased occultation by the larger degree of
flaring seen in BD discs \citep[for the latter, as found
by][]{2004MNRAS.351..607W}

\textit{Disc fraction-Spitzer best.}

We have also qualitatively explored the effects of accretion and disc
presence in our simulated dataset on disc fraction estimates.As is
currently well known, longer wavelength bandpasses are much more
reliable and suitable for disc identification. As shown in Section
\ref{results} the naked and BDD disc loci were much more clearly
separated in the CoCoD constructed using \textit{Spitzer} IRAC
magnitudes than the shorter wavelength CIT \textit{JHK} passbands. In
addition, we that the slope of the SED from 3.6 to 8.0$\mu$m, or
$\alpha$ value, is an effective disc indicator. We have also
tentatively shown that current observational cuts, when applied to our
simulated photometry (with its associated photometric system), results
in the reliable detection of disc candidates, for IRAC colours and
$\alpha$ values, and therefore a robust lower limit disc fraction.

\textbf{Effect of accretion and disc presence on emergent theories or
  secondary parameters.}

Secondly, the results of our simulated dataset have implications for
secondary derived correlations and physical theories, which utilise
the primary parameters discussed. It is clear that changes in derived
ages and masses of a BD sample, due to typical accretion and disc
presence, has serious implications in a range areas. Inaccuracies in
age derivations for pre-MS populations are well known, especially for
the BD mass regime. However, the combined effects of accretion and
disc presence on BD SEDs, and therefore age and mass derivation, have
not been previously explored. Therefore, some of the uncovered biases
and problems in this paper may bring new light to a few areas of BD
evolution.

\textbf{`Imperfect' disc locking perhaps due to biases.}

\textit{Disc timescales: Rotation rates affect disc detection as function of
  wavelength.}

Firstly, we have found a wavelength dependant bias of naked to BD disc
loci separation within CoCoDs. As we have shown in Section
\ref{results}, for shorter wavelength data (\textit{JHK}), objects
with a longer period lie closer to the naked BD locus, with shorter
period objects further removed. This bias is reversed for IRAC data,
where shorter period objects lie closer to the naked locus (although
still well separated) and longer period objects farther away, in
colour-colour space. Whilst, for the longer wavelength
\textit{Spitzer} data the clearer separation between the naked and BDD
disc loci means that this bias may well be minimised (especially in
regions without significant variable extinction), for shorter
wavelength data these populations (naked and BDD disc) overlap. As
disc excess cuts are usually placed empirically based on perceived
paucity's in stellar density, within colour-colour space, this overlap
and associated bias for \textit{JHK} data could lead to a significant
bias in any sample of BD stars with an inferred discs. This wavelength
dependant bias is essentially due to the change in inner radius of the
circumstellar disc, as the wavelength of emitted flux is dependant on
temperature and therefore radial position within the disc plane. For
shorter periods the disc is truncated closer to the central star and
the disc inner boundary intercepts more flux at all inclinations
leading to a hotter inner wall and less incident flux on the outer
flared edges of the disc. This means that for \textit{JHK} bandpasses,
which are sensitive to flux from the inner boundary, shorter period
objects appear more different to naked objects than longer, and
therefore cooler inner boundary systems.

The effect of this bias has not previously been accounted for in disc
fraction, and therefore disc timescale derivation studies. Our model
does not include angular momentum loss only changes in the disc inner
radius through a disc locking mechanism (and subsequent dust
sublimation). If disc-locking to a central star is true and slower
rotators are therefore more likely to have associated discs, any
correlation between disc presence and rotation rates will be weakened
for \textit{JHK} data and strengthened for \textit{Spitzer} IRAC data
due to the biases we observe, for BDs. These biases may well aid
understanding of early claims for a weaker or `imperfect' disc locking
mechanism for lower mass stars. The use of \textit{JHK} magnitudes
would lead to a less clear correlation between slow rotators and disc
presence for a BD sample.

These biases may also apply to higher mass objects, as would be
expected albeit at a weaker level due to the less significant flaring
seen in the circumstellar disc of TTS. This, in turn, may explain
early difficulties in finding any correlation between rotation rate
and disc presence \citep[amongst other issues such as non-coevality,
see][for discussion]{2007prpl.conf..297H}. Once data at longer
wavelengths was used this correlation was found
\citep{2007ApJ...671..605C}, which may have been enhanced by the
revealed bias. It is important to note however, that this bias is not
strong enough (given the clear separation of our simulated naked and
BDD disc systems in IRAC colour-colour space) to cast any
doubt over this result.  Finally, as we have stated inner holes in
circumstellar disc may well be created by other period-independent
mechanisms such as clearing from gas accretion onto a planet or due to
photoevaporation from UV flux, the latter of which would be increased
at high accretion rates, which we have not included.

\textbf{Accretion rate to mass relationship in doubt for BD regime.}

\textit{Mass to accretion rate relationship-high accretors of low mass
  missed-literature high mass low accretors missed.}

The second area which this study impacts on, and perhaps most
significantly, is the recent evidence for a stellar mass to accretion
rate correlation, of the form: $\dot{M_{acc}}\propto M_{*}^{~2}$
\citep{2003ApJ...592..266M,2004A&A...424..603N,2006A&A...452..245N}.
This relationship has been extended into the BD mass regime in
\cite{2006A&A...452..245N}. However, arguments based on selection and
detection thresholds have already cast this relation into doubt
\citep{2006MNRAS.370L..10C}. As we have shown in Section \ref{results}
a relationship of this kind is self-reinforcing as lower mass objects
with higher accretion rates have little chance of being correctly
identified as such due to both the accretion flux and flared
associated disc. Essentially, at present it is unclear how many BD
stars are not included in this relationship due to misidentification.
As explained in \cite{2004MNRAS.351..607W}, BD systems with a disc,
without including accretion effects, can have the characteristics of
higher mass CTTS stars, due to increased disc flaring from a reduced
surface gravity in the disc. The effects of accretion at typical or
larger rates further exacerbate the situation both spectroscopically,
as the photospheric flux essentially becomes swamped or completely
veiled, and photometrically as the resulting colours and magnitudes
are significantly shifted. Therefore, for our simulated dataset a
relationship of this sort may well be derived, if typical methods are
used to identify BD objects with discs and derive masses, ages and
accretion rates, even though it is not present.

\textbf{Link to deriving accretion rates-paper II}.

This work has argued qualitatively, largely using positions within
CMDs or CoCoDs, highlighting problems with the proposed stellar mass
to accretion rate relationship. A further quantitative examination of
methods used to derive accretion rates and their results when applied
to our simulated dataset is certainly required, ***and this will be
the subject of a future publication***, to confirm our conjectures.

\textbf{SUMMARY}

**WRITE AT THE VERY END**

In summary, our simulated dataset shows that for typical parameter
ranges for BD stars and BDD systems, disc presence and accretion flux
lead to:

Difficulty deriving the following stellar parameters for a coeval population:
\begin{itemize}
\item{Isochronal ages}
\item{Isochronal masses}
\item{IMFs}
\end{itemize}

And we have shown:
\begin{itemize}
\item{\textit{Spitzer} IRAC magnitudes are required for reliable disc
    identification}
\item{A correlation between naked to BDD disc system separation (in a
    CoCoD, as a function of wavelength.}
\item{Low mass, high accretion rate systems are likely to be
    misidentified and therefore not included in any study relating
    $M_*\propto\dot{M}$}
\end{itemize}

\section[]{ACKNOWLEDGMENTS}
The computations reported here were performed using the
University of Exeter Supercomputer.

\bibliographystyle{mn2e}
\bibliography{references}
\appendix

\section{Website}
\label{website}

**NEEDS LOTS OF WORK**

As stated throughout this paper the data presented are available from
a web page
\footnote{http://www.astro.ex.ac.uk/research/bd\textunderscore
  disc}. In this Appendix we briefly discuss the data included, and
the different ways of accessing or visualising these data on the
website.

\subsection{Available Data}
\label{web_data}

The magnitudes and colours presented in this paper are avaiable both
as individual magnitudes and as isochrones or mass tracks. Photometric
magnitudes have also been derived for several other systems and are
available online.

\textit{Full list of filters}

These are Johnson, Cousins \textit{UBVRI(JHK)}
\citep{1966ARA&A...4..193J,2005ARA&A..43..293B}, \textit{Tycho} $V_t$
and $B_t$ \citep{2000PASP..112..961B}, Bessell \textit{UBVRIJHKL}
\citep{1988PASP..100.1134B,1998A&A...333..231B}, SDSS \textit{UGRIZ}
\citep{1996AJ....111.1748F}, 2MASS $JHK_s$
\citep{2003AJ....126.1090C,2006AJ....131.1163S}, MKO \textit{JHK}
\citep{2002PASP..114..169S,2002PASP..114..180T}, UKIRT \textit{ZYJHK}
\citep{2001MNRAS.325..563H}, IRAS 12, 25, 60 and 100 $\mu$m
\citep{1984ApJ...278L...1N} and SCUBA \textit{450WB} and
\textit{850WB} \citep{1999MNRAS.303..659H}. For further information on
these magnitudes, such as the filter responses used and the adopted
zeropoints please refer to the website.

\textit{Monochromatic fluxes}

In addition to the magnitudes derived for each of these bands
monochromatic fluxes have also been derived for all bands listed
above. These have been derived following closely the methods of
\cite{2006ApJS..167..256R}, extended to further passbands. For details
of the assumed SED shape, central wavelengths and bandpasses adopted
please refer to the website.

**PERHAPS INCLUDE FORMULA FOR DERIVATION**

\subsection{Navigation}
\label{navigation}

The main page contains links to download the entire dataset in several
formats alonside files describing the format.

\begin{enumerate}
\item All SEDs ($\AA$ and ergs s$^{-1}$ cm$^{-2} \AA ^{-1}$) 
\item All SEDs ($\mu$m and mJy)
\item All Photometric Magnitudes for all BDD systems.
\item All Photometric Magnitudes for all Naked systems.
\item All Monochromatic Fluxes for all Naked systems ($\mu$m and ergs
  s$^{-1}$ cm$^{-2} \AA ^{-1}$)
\item All Monochromatic Fluxes for all Naked systems ($\mu$m and mJy)
\item All Monochromatic Fluxes for all BDD systems ($\mu$m and ergs
  s$^{-1}$ cm$^{-2} \AA ^{-1}$)
\item All Monochromatic Fluxes for all BDD systems ($\mu$m and mJy)
\end{enumerate}

Also included is a link to the calibration information listing all the
filtersponse sources, adopted zero points, central wavelengths,
bandwidths and assumed SED shapes (for monochromatic flux
derivation). A brief scientific overview is also available explaining,
in general terms, the dataset and model.

\subsection{Browsing Tools}
\label{tools}

For users who wish to investigate the dataset in a more specific or
interactive fashion two browsing tools are included.

The isochrones, mass tracks and individual stars magnitudes and
colours generated and presented in the work can be queried using the
``Isochrone and Mass Track Tool''. This allows the user to select the
parameters of the model required and retrieve the specific data.

Secondly, an iteractive tool is included allowing a user to select a
given set of parameters and download the ``SEDs, monochromatic fluxes
or magnitudes''. In addition this tool plots the SEDs for the Naked
system or the three inclinations, in the case of BDD systems and
allows users to select filters sets, whose monochromatic flux values
will be overlaid on the displayed SEDs.

\label{lastpage}
\end{document}
