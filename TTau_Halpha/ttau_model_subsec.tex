\subsection{An example model fit of T~Tau}

\label{sub:ttau-model}

As a demonstrative purpose, we present a model fit of H$\alpha$ observation
of a classical T Tauri star (T~Tau) obtained by \citet{vink:2005}.
T~Tau is known to be a triple system (e.g.~\citealt{koresko:2000}).
One of the star in the system T~Tau~N (a prototype for T~Tauri
stars) is dominating in optical (e.g.~\citealt{calvet:1994}). The
rotational velocity and the rotational period of the star found by
\citet{herbst:1987} are $v\,\sin i=19.5\pm2.5\,\mathrm{km\, s^{-1}}$
and $2.80\,\mathrm{d}$ respectively. The inclination of the disc
is known to be relatively small (e.g.~$i=29^{\circ}$, \citealt{akeson:2002}).
There is a rather large scatter in the previously estimated the accretion
rates e.g.~$3\times10^{-7}\,\mathrm{M_{\sun}\, yr^{-1}}$ \citep{johns-krull:2000}
and $\left(3.1-5.7\right)\times10^{-8}\,\mathrm{M_{\sun}\, yr^{-1}}$
\citep{calvet:2004}.

For the photospheric model, the following parameters from \citet{calvet:2004}
are adopted: $T_{\mathrm{eff}}=5500\,\mathrm{K}$, $M_{*}=1.9\,\mathrm{M_{\sun}}$
and $R_{*}=2.9\, R_{*}$. The standard geometry for the magnetosphere
($R_{\mathrm{mi}}=2.2\, R_{*}$ and $R_{\mathrm{mo}}=3.0\, R_{*}$)
and the bipolar wind configuration (Table~\ref{tab:std_parameters})
are used for simplicity. Again, our purpose here is not to obtain
the best fit parameters, but rather it is to examine if the model
is capable of reasonably reproducing the data from a real object,
in this case, T~Tau~N. The main model parameters here are (1)~the temperature
of the magnetospherical accretion flow ($T_{\mathrm{max}}$), (2)~the
mass-accretion rate ($\dot{M}_{\mathrm{acc}}$), (3)~the temperature
of the wind ($T_{\mathrm{wind}}$), and (4)~the wind mass-loss rate
($\dot{M}_{\mathrm{wind}}$). The inclination angle is simplify adopted
from \citet{akeson:2002}, i.e.~$i=29^{\circ}$. 

Figure~\ref{fig:ttau_model} shows the result of a model fit. The
amount of veiling near H$\alpha$ is very small ($<0.01$) for T~Tau,
according to \citet{basri:1990}; hence, the observation is not corrected
for veiling. The model consists of the magnetosphere, the bipolar
wind, and the accretion disc (with a large hole i.e. $\sim5$\, AU. The
parameters used for the model fit are:
$T_{\mathrm{max}}=22000\,\mathrm{K}$,
$T_{\mathrm{wind}}=8000\,\mathrm{K}$,
$\dot{M}_{\mathrm{acc}}=1.1\times10^{-8}\,\mathrm{M_{\sun}\,
yr^{-1}}$, and
$\dot{M}_{\mathrm{wind}}=2.0\times10^{-9}\,\mathrm{M_{\sun}\,
yr^{-1}}$.  The temperature of the magentosphere is relatively high
compared to the standard cases
(e.g.~Figure~\ref{fig:atlas_acc_models}; also
\citealt{muzerolle:2001}). Note that the estimated value of
$T_{\mathrm{max}}$ depends on the adopted values of $L_{*}$ (or
$M_{*}$ and $T_{\mathrm{eff}}$) for the photospheric continuum
model. If a lower value of $L_{*}$ is adopted, $T_{\mathrm{max}}$ value
would become smaller in order to match the line strength which is
mainly determined by the emission from the magnetosphere, in this
case. For example, if the photospheric continuum is 3 times smaller
than the one used here, one finds
$T_{\mathrm{max}}\sim9600\,\mathrm{K}$ which is more similar to that
of the standard cases. The figure shows that the agreement between the
model and the observation is quite reasonable although the line wings
are slightly stronger in the models.  In this example, the main
profile feature is defined by the emission from the magnetosphere, but
the blue side of the profile is affected by the absorption by the
wind. The mass-loss to mass-accretion rate $\mu=0.18$ which is similar
to the value predicted by observations and MHD simulations
i.e.~$\sim0.1$ \citep{koenigl:2000}. Compared our mass-accretion rate
estimate is much less than that of \citep{johns-krull:2000}, and
slightly smaller than that of \citep{calvet:2004} (see above).

Although not shown here for clarity, the peak flux near the line centre
is also affected by the wind absorption (causing about 10 per cent
reduction) compared to the profile computer with magnetosphere only
model (similar effect also seen in Figure~\ref{fig:atlas_disc_wind_acc_model}).
If the fit was performed by ignoring the wind, the temperature and
the mass-accretion rate of the magnetospherical accretion flow will
be underestimated. A better approach to a model fit is a simultaneous
fitting of multiple lines. For example, the near-infrared lines such
as Pa$\beta$ and Br$\gamma$ should be used to determined the mass-loss
rate since these lines are much less affected by the wind. Using the
mass-loss rate from these lines, H$\alpha$ can be used to determine
the wind mass-loss rate.

%%%%%%%%%%%%%%%%%%%%%%%%%%%%%%%%%%%%%%%%%%%%%%%%%%%%%%%%%%%%%%%%%%%

\begin{figure}

\begin{center}

\includegraphics[%
  clip,
  scale=0.45]{figures/ttau_compare.eps}

\end{center}

\caption{Comparison of the observed H$\alpha$ profile of T~Tau and the
profile computed with the hybrid model, which consist of the
magnetosphere, the colliminated wind, and the accretion disc (with a
large hole i.e. $\sim 5$~AU). The inclination angle $29^\circ$ is
used.  The maximum temperature of the magnetosphere
($T_{\mathrm{max}}$) and the isothermal wind temperature are
22000~K and 8000~K respetively. The mass-accretion rate
($\dot{M}_{\mathrm{acc}}$) and the mass-loss rate
($\dot{M}_{\mathrm{wind}}$) are $1.1 \times
10^{-8}\,\mathrm{M_{\sun}\,yr^{-1}}$ and $2.0 \times
10^{-9}\,\mathrm{M_{\sun}\,yr^{-1}}$ resplectively
($\mu=\dot{M}_{\mathrm{wind}}/\dot{M}_{\mathrm{acc}}=0.18$)}

\label{fig:ttau_model}

\end{figure}

%%%%%%%%%%%%%%%%%%%%%%%%%%%%%%%%%%%%%%%%%%%%%%%%%%%%%%%%%%%%%%%%%%%
